%%%%%%%%------------------------------------------------------------------------
% This program is free software: you can redistribute it and/or modify
% it under the terms of the GNU General Public License as published by
% the Free Software Foundation, either version 3 of the License, or
% (at your option) any later version.
% 
% This program is distributed in the hope that it will be useful,
% but WITHOUT ANY WARRANTY; without even the implied warranty of
% MERCHANTABILITY or FITNESS FOR A PARTICULAR PURPOSE.  See the
% GNU General Public License for more details.
% 
% You should have received a copy of the GNU General Public License
% along with this program.  If not, see <http://www.gnu.org/licenses/>.

%%%%%%%%------------------------------------------------------------------------
%%%% 导言区 
%% 文档类型为article
\documentclass[a4paper, 10pt]{book}
%1m = 39.4 inch
%大18开 (18.5cm * 23cm)
%\usepackage[left=3.25cm, right=3.25cm, top=2.3cm,bottom=1.4cm]{geometry}
\usepackage{geometry}
\geometry{left=3.75cm,right=3.25cm,top=3cm,bottom=2.5cm}

%% en_preamble包含基本的宏包配置
%%%%%%%%------------------------------------------------------------------------
%%%% 日常所用宏包

%%设置行间距
\usepackage{setspace}

%% 控制项目列表
\usepackage{enumerate}

%% 多栏显示
\usepackage{multicol}

%% hyperref宏包,生成可定位点击的超链接,并且会生成pdf书签
\usepackage[%
    pdfstartview=FitH,%
    CJKbookmarks=true,%
    bookmarks=true,%
    bookmarksnumbered=true,%
    bookmarksopen=true,%
    colorlinks=true,%
    citecolor=blue,%
    linkcolor=blue,%
    anchorcolor=green,%
    urlcolor=blue%
]{hyperref}

%% 控制标题
\usepackage{titlesec}

%% 控制表格样式
\usepackage{booktabs}

%% 控制目录
\usepackage{titletoc}

%% 控制字体大小
\usepackage{type1cm}

%% 首行缩进,用\noindent取消某段缩进
\usepackage{indentfirst}

%% 支持彩色文本、底色、文本框等
\usepackage{color,xcolor}

%% AMS LaTeX宏包
\usepackage{amsmath}
\usepackage{amssymb}

%% 一些特殊符号
% \usepackage{bbding}

%% 支持引用
% \usepackage{cite}

%% LaTeX一些特殊符号宏包
% \usepackage{latexsym}

%% 数学公式中的黑斜体
% \usepackage{bm}

%% 调整公式字体大小:\mathsmaller, \mathlarger
% \usepackage{relsize}

%% 生成索引
% \makeindex

%%%% 基本插图方法
%% 图形宏包
\usepackage{graphicx}
\usepackage{float}
%% 如果插入的图片没有指定扩展名,那么依次搜索下面的扩展名所对应的文件
\DeclareGraphicsExtensions{.pdf,.eps,.png,.jpg}
%% 让 latex 从 .bb 中读取 Bounding Box 信息
%\DeclareGraphicsRule{.jpg}{eps}{.bb}{}
%\DeclareGraphicsRule{.png}{eps}{.bb}{}
%\DeclareGraphicsRule{.pdf}{eps}{.bb}{}

%% 多个图形并排,参加lnotes.pdf
%\usepackage{subfig}
\usepackage{subfigure}


\usepackage{caption}
\captionsetup{font={sf, scriptsize}, labelfont={bf}, skip=15pt}
\DeclareCaptionLabelSeparator{colon}{~~}

\usepackage[perpage,stable]{footmisc}

\usepackage{longtable}
% \begin{figure}[htbp]               %% 控制插图位置
%   \setlength{\abovecaptionskip}{0pt}
%   \setlength{\belowcaptionskip}{10pt}
                                     %% 控制图形和上下文的距离
%   \centering                       %% 使图形居中显示
%   \includegraphics[width=0.8\textwidth]{CTeXLive2008.jpg}
                                     %% 控制图形显示宽度为0.8\textwidth
%   \caption{CTeXLive2008安装过程} \label{fig:CTeXLive2008}
                                     %% 图形题目和交叉引用标签
% \end{figure}
%%%% 基本插图方法结束

%%%% pgf/tikz绘图宏包设置
\usepackage{pgf,tikz}
\usetikzlibrary{shapes,automata,snakes,backgrounds,arrows}
\usetikzlibrary{mindmap, trees,  calendar}
\usetikzlibrary{positioning}
\usepackage{pgf-umlsd}
%% 可以直接在latex文档中使用graphviz/dot语言,
%% 也可以用dot2tex工具将dot文件转换成tex文件再include进来
%% \usepackage[shell,pgf,outputdir={docgraphs/}]{dot2texi}
%%%% pgf/tikz设置结束


%%%% fancyhdr设置页眉页脚
%% 页眉页脚宏包
\usepackage{fancyhdr}

%% 页眉页脚风格
\pagestyle{fancy}

%%这两行代码是修改\leftmark和\rightmark的经典模式
\renewcommand{\chaptermark}[1]{\markboth{{\hei {第\thechapter{}章}}\hspace 1  #1}{}}
\renewcommand{\sectionmark}[1]{\markright{\thesection{} #1}}

%% 清空当前页眉页脚的默认设置
\fancyhf{}

%\fancyhead[L]{\scriptsize \fangsong \ascii{ZTE}中兴}
%\fancyhead[R]{\scriptsize \fangsong 内部公开}

%\fancyhead[CE]{\scriptsize \fangsong \leftmark}
%\fancyhead[CO]{\scriptsize \fangsong \rightmark}

%\fancyfoot[RO, LE]{\scriptsize \thepage}
%\fancyfoot[C]{\scriptsize \fangsong 本文中的所有信息均为中兴通讯股份有限公司内部信息,不得向外传播}

\renewcommand{\headrulewidth}{0.4pt}
\renewcommand{\footrulewidth}{0.4pt}

%第{\couriernew\thechapter{}}章
%%下面开始修改页眉和页脚
\fancyhead[RE]{\fangsong \leftmark}
\fancyhead[LO]{\fangsong \rightmark}
\fancyhead[RO, LE]{\small \thepage}
\fancypagestyle{plain}{%
  \fancyhead{} % get rid of headers
  \renewcommand{\headrulewidth}{0pt} % and the line.
}

%%定义空白页面
\makeatletter
\def\cleardoublepage{\clearpage\if@twoside \ifodd\c@page\else
  \hbox{}
  \vspace*{\fill}
  \begin{center}
   {\sffamily\large}
   \end{center}
   \vspace{\fill}
   \thispagestyle{empty}
   \newpage
   \if@twocolumn\hbox{}\newpage\fi\fi\fi}
\makeatother

\makeatletter
\def\cleardedicatepage{\clearpage
  \hbox{}
  \vspace*{\fill}
  \begin{center}
   {\sffamily\Large 献给我的女儿刘楚溪}
   \end{center}
   \vspace{\fill}
   \thispagestyle{empty}
   \newpage
   \if@twocolumn\hbox{}\newpage\fi}
\makeatother

%% 有时会出现\headheight too small的warning
\setlength{\headheight}{15pt}
%%%% fancyhdr设置结束

%%设置行间距离
\usepackage{framed}  
%%%% listings宏包设置结束

%%%% 附录设置
\usepackage[title,titletoc,header]{appendix}
%%%% 附录设置结束

%%%% 日常宏包设置结束
%%%%%%%%------------------------------------------------------------------------

%%%%%%%%------------------------------------------------------------------------
%%%% 英文字体设置结束
%% 这里可以加入自己的英文字体设置
%%%%%%%%------------------------------------------------------------------------

%%%%%%%%------------------------------------------------------------------------
%%%% 设置常用字体字号,与MS Word相对应

%% 一号, 1.4倍行距
\newcommand{\yihao}{\fontsize{26pt}{36pt}\selectfont}
%% 二号, 1.25倍行距
\newcommand{\erhao}{\fontsize{22pt}{28pt}\selectfont}
%% 小二, 单倍行距
\newcommand{\xiaoer}{\fontsize{18pt}{18pt}\selectfont}
%% 三号, 1.5倍行距
\newcommand{\sanhao}{\fontsize{16pt}{24pt}\selectfont}
%% 小三, 1.5倍行距
\newcommand{\xiaosan}{\fontsize{15pt}{22pt}\selectfont}
%% 四号, 1.5倍行距
\newcommand{\sihao}{\fontsize{14pt}{21pt}\selectfont}
%% 半四, 1.5倍行距
\newcommand{\bansi}{\fontsize{13pt}{19.5pt}\selectfont}
%% 小四, 1.5倍行距
\newcommand{\xiaosi}{\fontsize{12pt}{18pt}\selectfont}
%% 大五, 单倍行距
\newcommand{\dawu}{\fontsize{11pt}{11pt}\selectfont}
%% 五号, 单倍行距
\newcommand{\wuhao}{\fontsize{10.5pt}{10.5pt}\selectfont}
%%%%%%%%------------------------------------------------------------------------

%%%%%%%%------------------------------------------------------------------------
%%%% 一些个性设置

%% 设定页码方式,包括arabic、roman等方式
%% \pagenumbering{arabic}

%% 有时LaTeX无从断行,产生overfull的错误,这条命令降低LaTeX断行标准
%% \sloppy

%% 设定目录显示深度\tableofcontents
%% \setcounter{tocdepth}{2}
%% 设定\listoftables显示深度
%% \setcounter{lotdepth}{2}
%% 设定\listoffigures显示深度
%% \setcounter{lofdepth}{2}

%% 中文破折号,据说来自清华模板
\newcommand{\pozhehao}{\kern0.3ex\rule[0.8ex]{2em}{0.1ex}\kern0.3ex}

%% 设定itemize环境item的符号
\renewcommand{\labelitemi}{$\bullet$}

%\makeatletter
%\@addtoreset{lstlisting}{section} 
%\makeatother

\newenvironment{enum}
{
  \begin{spacing}{1.2}
  \begin{enumerate}[1.]
    \setlength{\itemsep}{0pt} 
    \setlength{\itemindent}{2em}
    %\setlength{\listparindent}{2em}
}{%
  \end{enumerate}
  \end{spacing}
}

\newcommand{\suggest}[1]{
\tikzstyle{mybox} = [draw=black, very thick,
rectangle, rounded corners, inner sep=9pt, inner ysep=20pt]
\tikzstyle{fancytitle} =[fill=white, text=black, ellipse]
\noindent
\begin{tikzpicture}
\node [mybox] (box){%
\begin{minipage}{\textwidth}
\fangsong
#1
\end{minipage}
};
\node[fancytitle, right=10pt] at (box.north west) {\emph{建议}};
% \node[fancytitle, rounded corners] at (box.east) {$\clubsuit$};
\end{tikzpicture}
}

\newcommand{\notice}[1]{
\tikzstyle{mybox} = [draw=black, very thick,
rectangle, rounded corners, inner sep=9pt, inner ysep=20pt]
\tikzstyle{fancytitle} =[fill=white, text=black]
\noindent
\begin{tikzpicture}
\node [mybox] (box){%
\begin{minipage}{\textwidth}
\fangsong
#1
\end{minipage}
};
\node[fancytitle, right=10pt] at (box.north west) {\emph{注意}};
%\node[fancytitle, rounded corners] at (box.east) {$\clubsuit$};
\end{tikzpicture}
}

\newcommand{\tip}[1]{
\tikzstyle{mybox} = [draw=black, very thick,
rectangle, rounded corners, inner sep=9pt, inner ysep=20pt]
\tikzstyle{fancytitle} =[fill=white, text=black]
\noindent
\begin{tikzpicture}
\node [mybox] (box){%
\begin{minipage}{\textwidth}
\fangsong
#1
\end{minipage}
};
\node[fancytitle, right=10pt] at (box.north west) {\emph{提示}};
%\node[fancytitle, rounded corners] at (box.east) {$\clubsuit$};
\end{tikzpicture}
}

\newcommand\refch[1]{\ascii{第\ref{ch:#1}章(\nameref{ch:#1})}}
\newcommand\refsec[1]{\ascii{\ref{sec:#1}节(\nameref{sec:#1})}}

\newcommand\eitem[1]{\item {\itshape {#1}}}
\newcommand\cpp{\ascii{C\nobreak+\nobreak+}}
\newcommand\clang{\ascii{C}}

\newcommand\quo[1]{“#1”}

\newcommand\percent[1]{\ascii{#1\%}}

\newcommand{\trans}{\emph{事务}}
\newcommand{\act}{\emph{操作}}
\newcommand{\seqact}{\emph{顺序操作}}
\newcommand{\conact}{\emph{并发操作}}
\newcommand{\atomact}{\emph{基本操作}}
\newcommand{\syncact}{\emph{同步操作}}
\newcommand{\asynact}{\emph{异步操作}}
\newcommand{\action}[1]{\emph{\ascii{\itshape\_\_#1}}}
\newcommand{\sigwait}{\action{sig\_wait}}
\newcommand{\sigsync}{\action{sig\_sync}}
\newcommand{\sigreply}{\action{sig\_reply}}
\newcommand{\timerprot}{\action{timer\_prot}}
\newcommand{\unknownevet}{\ascii{UNKNOWN\_EVENT}}
\newcommand{\transdsl}{\ascii{Transaction DSL}}
\newcommand{\oper}[1]{\ascii{Action#1}}
\newcommand{\protproc}{\ascii{prot\_procedure}}

\newcommand{\code}[1]{\ascii{\scriptsize{\texttt{#1}}}}

%\newcommand{\Email}{\begingroup \def\UrlLeft{<}\def\UrlRight{>} \urlstyle{tt}\Url}
%\def\mailto|#1|{\href{mailto:#1}{Email|#1|}}
\newcommand{\contrib}[2]{#1\quad{\small\quad\textit{#2}}\\[1ex]}
%% 设定正文字体大小
% \renewcommand{\normalsize}{\sihao}

%%%% 个性设置结束
%%%%%%%%------------------------------------------------------------------------


%%%%%%%%------------------------------------------------------------------------
%%%% bibtex设置

%% 设定参考文献显示风格

%%%% bibtex设置结束
%%%%%%%%------------------------------------------------------------------------


%% 如果不写中文的话就不需要引用xecjk_preamble里面的配置
%%%%%%%%------------------------------------------------------------------------
%%%% xeCJK相关宏包

\usepackage{xltxtra,fontspec,xunicode}

%% \CJKsetecglue{\hskip 0.15em plus 0.05em minus 0.05em}
%% slanfont: 允许斜体
%% boldfont: 允许粗体
%% CJKnormalspaces: 仅忽略汉字之间的空白,但保留中英文之间的空白。 
%% CJKchecksingle: 避免单个汉字单独占一行。
\usepackage[slantfont, boldfont]{xeCJK} 
% \usepackage{ctex}

%% 针对中文进行断行
\XeTeXlinebreaklocale "zh"             

%% 给予TeX断行一定自由度
\XeTeXlinebreakskip = 0pt plus 1pt minus 0.1pt

%%%% xeCJK设置结束                                       
%%%%%%%%------------------------------------------------------------------------

%%%%%%%%------------------------------------------------------------------------
%%%% xeCJK字体设置

%% 设置中文标点样式,支持quanjiao、banjiao、kaiming等多种方式
\punctstyle{quanjiao}                                        
                                                     
%% 设置缺省中文字体
\setCJKmainfont[BoldFont={Adobe Heiti Std}, ItalicFont={Adobe Kaiti Std}]{Adobe Song Std}   %  FZBaoSongZ04
%% 设置中文无衬线字体
\setCJKsansfont[BoldFont={Adobe Heiti Std}, ItalicFont={Adobe Kaiti Std}]{Adobe Kaiti Std}  
%% 设置等宽字体
\setCJKmonofont{Adobe Heiti Std}                            
%\setCJKmonofont{Monaco}                            

%% 英文衬线字体
\setmainfont{Lucida Bright}                                  
%% 英文等宽字体
%\setmonofont{Courier}
\setmonofont{Monaco}                             
%\setmonofont{Consolas}                              
%% 英文无衬线字体
\setsansfont{Optima}                                   

%% 定义新字体
\setCJKfamilyfont{song}{Adobe Song Std}                     
\setCJKfamilyfont{kai}{Adobe Kaiti Std}
\setCJKfamilyfont{hei}{Adobe Heiti Std}
\setCJKfamilyfont{fangsong}{Adobe Song Std}
\setCJKfamilyfont{lisu}{LiShu}
\setCJKfamilyfont{youyuan}{Adobe Kaiti Std}

%%自定义英文字体
\newfontfamily\couriernew{Lucida Grande}
\newfontfamily\optima{Optima}
\newfontfamily\lucida{Lucida Bright}

\newcommand{\ascii}[1]{{\sffamily #1}}
\newcommand{\speak}[1]{{\itshape #1}}
\renewcommand{\emph}[1]{{\hei #1}}

%% 自定义宋体
\newcommand{\song}{\CJKfamily{song}}                       
%% 自定义楷体
\newcommand{\kai}{\CJKfamily{kai}}                         
%% 自定义黑体
\newcommand{\hei}{\CJKfamily{hei}}                         
%% 自定义仿宋体
\newcommand{\fangsong}{\CJKfamily{fangsong}}               
%% 自定义隶书
\newcommand{\lisu}{\CJKfamily{lisu}}                       
%% 自定义幼圆
\newcommand{\youyuan}{\CJKfamily{youyuan}}                 

%%%% xeCJK字体设置结束
%%%%%%%%------------------------------------------------------------------------

%%%%%%%%------------------------------------------------------------------------
%%%% 一些关于中文文档的重定义

%% 数学公式定理的重定义

\newtheorem{example}{例}[section]                                   
\newtheorem{algorithm}{算法}
%% 按section编号
\newtheorem{theorem}{定理}[section]                         
\newtheorem{definition}{定义}
\newtheorem{axiom}{公理}
\newtheorem{property}{性质}
\newtheorem{proposition}{命题}
\newtheorem{lemma}{引理}
\newtheorem{corollary}{推论}
\newtheorem{condition}{条件}
\newtheorem{conclusion}{结论}
\newtheorem{assumption}{假设}

\newtheorem{principle}{原则}[section]
\newtheorem{regulation}{规则}[section]
\newtheorem{advise}{建议}[section]
\newtheorem{concept}{概念}[section]

\usepackage{titlesec}

\renewcommand{\partname}{}
\renewcommand{\thepart}{第\Roman{part}部分}

%% 章节等名称重定义
\renewcommand{\contentsname}{目录}
%\renewcommand{\abstractname}{摘要}
\renewcommand{\indexname}{索引}
\renewcommand{\listfigurename}{插图目录}
\renewcommand{\listtablename}{表格目录}
\renewcommand{\figurename}{图}
\renewcommand{\tablename}{表}
\renewcommand{\appendixname}{附录}
\renewcommand{\appendixpagename}{附录}
\renewcommand{\appendixtocname}{附录}
%\renewcommand\refname{参考文献} 

%%设置内容环境
\newenvironment{content}{%
  \setlength{\parskip}{0.5\baselineskip}
  \begin{spacing}{1.5}
}{%
  \end{spacing}
  \setlength{\parskip}{-0.5\baselineskip}
  \vskip -0.5\baselineskip
}

% 插入小段故事的语法:
% \begin{story}
%   \begin{center}
%     \inlinetitle{分水岭}
%   \end{center}
% \end{story}
\newenvironment{story}
{
  \setlength{\parskip}{0.5\baselineskip}
  \hbox to \textwidth{\hfil\rule{\linewidth}{0.5mm}\hfil}
  \begin{spacing}{1.5}
}{%
  \end{spacing}
  \hbox to \textwidth{\hfil\rule{\linewidth}{0.5mm}\hfil}
  \setlength{\parskip}{-0.5\baselineskip}
  \vskip -0.5\baselineskip
}

%% 设置chapter、section与subsection的格式
%\titleformat{\chapter}[display]{\flushright\yihao}{\thechapter{}}{1em}{\textbf}
\titleformat{\section}[block]{\flushleft\sanhao}{\optima{\thesection}}{1em}{\textbf}
\titleformat{\subsection}{\sihao}{\optima{\thesubsection}}{0.5em}{\textbf}
\titleformat{\subsubsection}{\xiaosi}{\thesubsubsection}{0.5em}{\textbf}

%\titlespacing{\chapter}{0pt}{0pt}{-\baselineskip}
\titlespacing{\section}{0pt}{0pt}{0\baselineskip}
\titlespacing{\subsection}{0pt}{0.5\baselineskip}{0\baselineskip}

%% 设置章格式
\usepackage{quotchap}

\renewcommand\chapterheadstartvskip{
   \vspace*{-5\baselineskip}
}

\renewcommand\chapterheadendvskip{
   \vspace*{0.5\baselineskip}
}

\usepackage{helvet}
\renewcommand\sectfont{\rmfamily\bfseries}

\newcommand\refig[1]{{\itshape \figurename\ascii{\ref{fig:#1}(第\pageref{fig:#1}页)}}}
\newcommand\reftbl[1]{{\itshape \tablename\ascii{\ref{tbl:#1}(第\pageref{tbl:#1}页)}}}

\renewcommand{\footnoterule}{\vspace*{3pt}%
  \hrule width 0.382\textwidth height 0.4pt \vspace*{2.6pt}}

% Remark
\newenvironment{remark}{\par\vskip10pt\footnotesize\itshape % Vertical white space above the remark and smaller font size
\begin{list}{}{
\leftmargin=35pt % Indentation on the left
\rightmargin=25pt}\item\ignorespaces % Indentation on the right
\makebox[-2.5pt]{\begin{tikzpicture}[overlay]
\node[draw=red!60,line width=1pt,circle,fill=red!25,font=\sffamily\bfseries,inner sep=2pt,outer sep=0pt] at (-15pt,0pt){\textcolor{red}{R}};\end{tikzpicture}}
\advance\baselineskip -1pt}{\end{list}\vskip5pt}

%%%% 中文重定义结束
%%%%%%%%------------------------------------------------------------------------


%%%% 设置listings宏包用来粘贴源代码
%% 方便粘贴源代码,部分代码高亮功能
\usepackage{listings}
\usepackage{color}

\DeclareCaptionFont{red}{\color{red}}

%% 所要粘贴代码的编程语言
\lstloadlanguages{{[LaTeX]TeX}, {[ISO]C++}, {Java}, {Ruby}, {Python}}

%% 设置listings宏包的一些全局样式
%% 参考http://hi.baidu.com/shawpinlee/blog/item/9ec431cbae28e41cbe09e6e4.html
\lstset{
numberbychapter=true,
breakatwhitespace=true,
showstringspaces=false,              %% 设定是否显示代码之间的空格符号
basicstyle=\scriptsize\ttfamily,           %% 设定字体大小\tiny, \small, \Large等等
keywordstyle=\bfseries,
commentstyle=\color{red!50!green!50!blue!50},                           
escapechar=`,                        %% 中文逃逸字符,用于中英混排
xleftmargin=1.5em,xrightmargin=0em, aboveskip=1em,
breaklines,                          %% 这条命令可以让LaTeX自动将长的代码行换行排版
extendedchars=false,                 %% 这一条命令可以解决代码跨页时,章节标题,页眉等汉字不显示的问题
frameround=fttt,
captionpos=top,
belowcaptionskip=1em
}

\lstdefinestyle{numbers}{
   numbers=left,
   numberstyle=\tiny,
   stepnumber=1,
   numbersep=1em
}

\lstdefinestyle{C++}{
   language=C++,
   texcl=true,
   prebreak=\textbackslash,
   breakindent=1em,
   keywordstyle=\bfseries, %% 关键字高亮
   morekeywords={alignas, alignof, char16_t, char32_t, constexpr, decltype, noexcept, nullptr, static_assert, thread_local, override, OVERRIDE, INTERFACE, ABSTRACT, DEFINE_ROLE, ROLE, HAS_ROLE, USE_ROLE}
   style=numbers,
   %frame=leftline,                     %% 给代码加框
   %framerule=2pt,
   %rulesep=5pt
}

\lstnewenvironment{c++}[1][]
  {\setstretch{1}
  \lstset{style=C++, #1}}
  {}

%\captionsetup[lstlisting]{textfont=red}
%{labelfont=bf, singlelinecheck=off, labelsep=space, textfont=red}

\lstdefinestyle{Java}{
   language=Java,
   texcl=true,
   prebreak=\textbackslash,
   breakindent=1em,
   keywordstyle=\bfseries, %% 关键字高亮
   morekeywords={}
   style=numbers,
   %frame=leftline,                     %% 给代码加框
   %framerule=2pt,
   %rulesep=5pt
}

\lstnewenvironment{java}[1][]
  {\setstretch{1}
  \lstset{style=Java, #1}}
  {}

\lstdefinestyle{Ruby}{
   language=Java,
   texcl=true,
   prebreak=\textbackslash,
   breakindent=1em,
   keywordstyle=\bfseries, %% 关键字高亮
   morekeywords={}
   style=numbers,
   %frame=leftline,                     %% 给代码加框
   %framerule=2pt,
   %rulesep=5pt
}

\lstnewenvironment{ruby}[1][]
  {\setstretch{1}
  \lstset{style=Ruby, #1}}
  {}

\lstdefinestyle{Python}{
   language=Python,
   texcl=true,
   prebreak=\textbackslash,
   breakindent=1em,
   keywordstyle=\bfseries, %% 关键字高亮
   morekeywords={}
   style=numbers,
   %frame=leftline,                     %% 给代码加框
   %framerule=2pt,
   %rulesep=5pt
}

\lstnewenvironment{python}[1][]
  {\setstretch{1}
  \lstset{style=Python, #1}}
  {}

\renewcommand{\lstlistingname}{示例代码}
\renewcommand\thefigure{\thechapter-\arabic{figure}}

\newcommand\refcode[1]{{\itshape \lstlistingname\ascii{\ref{code:#1}(第\pageref{code:#1}页)}}}


% \usepackage[
%   placement=center,
%   angle=45,
%   scale=20,
%   color=black!40,
%   %hshift=60,
%   %vshift=-5
% ]{background}

% \backgroundsetup{contents={样章}}
% \backgroundsetup{contents={\includegraphics[width=0.2\textwidth]{figures/cock.jpg}}}

\newcommand{\myclearpage}{\clearpage{\pagestyle{empty}\cleardoublepage}}
\newcommand{\mydedicate}{\clearpage{\pagestyle{empty}\cleardedicatepage}}

%%%% 导言区结束
%%%%%%%%------------------------------------------------------------------------

%%%%%%%%------------------------------------------------------------------------
%%%% 正文部分

\begin{document}

\frontmatter
\pagestyle{empty}

%%自定义封面
\def\titlename{TensorFlow内核剖析}
\def\subtitle{TensorFlow Internals}
\def\authors{刘光聪\ 著}
% \def\orgnization{\ascii{}}


\newlength{\centeroffset}
\setlength{\centeroffset}{-0.5\oddsidemargin}
\addtolength{\centeroffset}{0.5\evensidemargin}
\thispagestyle{empty}

% \begin{tikzpicture}
%   \path[mindmap,concept color=black,text=white]
%     node[concept] {\ascii{Programming}}
%     [clockwise from=0]
%     child[concept color=green!50!black] {
%       node[concept] {\ascii{XP}}
%       [clockwise from=90]
%       child { node[concept] {\ascii{Simple Design}} }
%       child { node[concept] {\ascii{TDD}} }
%       child { node[concept] {\ascii{Refactoring}} }
%       child { node[concept] {\ascii{Pair Programming}} }
%     }
%     child[concept color=blue] {
%       node[concept] {\ascii{Oriented-Object}}
%       [clockwise from=-30]
%       child { node[concept] {\ascii{C++}} }
%       child { node[concept] {\ascii{Java}} }
%     }
%     child[concept color=red]
%     { node[concept] {\ascii{Evolutionary Design}} }
%     child[concept color=orange]
%     { node[concept] {\ascii{Clean Code}} };
%  \end{tikzpicture}

%title and subtitle
\vspace*{\stretch{1}}
\noindent\hspace*{\centeroffset}\makebox[0pt][l]{\begin{minipage}{\textwidth}
\flushright
{\Huge \hei \bfseries \titlename}
\noindent\rule[-1ex]{\textwidth}{5pt}\\[2.5ex]
\hfill\emph{\subtitle}
\end{minipage}}

%author
\vspace{\stretch{2}}
\noindent\hspace*{\centeroffset}\makebox[0pt][l]{\begin{minipage}{\textwidth}
\center
{\bfseries \authors} \\[1.5ex]
{\bfseries \orgnization}
\end{minipage}}

\vspace{\stretch{1}}

\pagebreak

\myclearpage

\mydedicate

\input{contents/foreword}
\myclearpage

\chapter{前言} 
\label{ch:preface}

\begin{content}

\ascii{C++ Programming Style}是一份关于\ascii{C++}的代码规范,为\ascii{C++}程序员提供程序设计的参考和建议。

规范总结了我们长期在编码实践中遇到的一些宝贵经验,其中部分规则、原则和建议摘自社区软件设计大师的论文、报告、书籍和博客,并引用了一些典型的源代码。

\end{content}

\section*{约定}

\begin{content}

\begin{table}[!htb]
\resizebox{0.95\textwidth}{!} {
\begin{tabular*}{1.2\textwidth}{@{}ll@{}}
\toprule
\ascii{约定} & \ascii{说明} \\
\midrule
\ascii{原则} & \ascii{坚持遵守的指导思想} \\
\ascii{规则} & \ascii{必须遵守的约定} \\ 
\ascii{建议} & \ascii{可以考虑的约定} \\ 
\ascii{正例} & \ascii{原则、规则、建议所给出的正确例子} \\ 
\ascii{反例} & \ascii{原则、规则、建议所给出的错误例子} \\ 
\bottomrule
\end{tabular*}
}
\caption{规范约定}
\label{tbl:regulation-tbl}
\end{table}

\end{content}

\section*{宏定义}

\begin{content}

为了提高代码的表现力,规范中使用了一部分实用的宏定义。为了避免歧义,这里列出了部分较为重要的宏定义供参考和查阅。

\begin{leftbar}
\begin{c++}[caption={\ttfamily{cub/base/Default.h}}]
#ifndef GKOQWPRT_1038935_NCVBNMZHJS_8909603
#define GKOQWPRT_1038935_NCVBNMZHJS_8909603

namespace details
{
   template <typename T>
   struct DefaultValue
   {
      static T value()
      {
         return T();
      }
   };

   template <typename T>
   struct DefaultValue<T*>
   {
       static T* value()
       {
           return 0;
       }
   };

   template <typename T>
   struct DefaultValue<const T*>
   {
       static T* value()
       {
           return 0;
       }
   };

   template <>
   struct DefaultValue<void>
   {
      static void value()
      {
      }
   };
}

#define DEFAULT(type, method)  \
    virtual type method { return ::details::DefaultValue<type>::value(); }

#endif
\end{c++}
\end{leftbar}

\ascii{DEFAULT}对于定义空实现的\ascii{virtual}函数非常方便。需要注意的是,\ascii{DEFAULT}所有计算都是发生在编译时,并没有任何运行时成本。

\begin{leftbar}
\begin{python}[caption={\ttfamily{cub/base/Keywords.h}}]
def func():
  return 0
\end{python}
\end{leftbar}

\begin{leftbar}
\begin{c++}[caption={\ttfamily{cub/base/Keywords.h}}]
#ifndef H16274882_9153_4DB2_A2E2_F23D4CCB9381
#define H16274882_9153_4DB2_A2E2_F23D4CCB9381

#include <cub/base/Config.h>
#include <cub/base/Default.h>

#define ABSTRACT(...) virtual __VA_ARGS__ = 0

#if __SUPPORT_VIRTUAL_OVERRIDE
#   define OVERRIDE(...) virtual __VA_ARGS__ override
#else
#   define OVERRIDE(...) virtual __VA_ARGS__
#endif

#define EXTENDS(...) , ##__VA_ARGS__
#define IMPLEMENTS(...) EXTENDS(__VA_ARGS__)

#endif
\end{c++}
\end{leftbar}

\ascii{Config.h}提供了编译器支持\ascii{C++11}特性的配置信息。\ascii{ABSTRACT, OVERRIDE, EXTENDS, IMPLEMENTS}等关键字,使得其他阵营的程序员,也能看懂规范中大部分\ascii{C++}的代码,极大地改善了\ascii{C++}的表现力。

\begin{leftbar}
\begin{c++}[caption={\ttfamily{cub/base/Role.h}}]
#ifndef HF95EF112_D6C6_4DB0_8C1A_BE5A6CF8E3F1
#define HF95EF112_D6C6_4DB0_8C1A_BE5A6CF8E3F1

#include <cub/base/Keywords.h>

namespace details
{
   template <typename T>
   struct Role
   {
      virtual ~Role() {}
   };
}

#define DEFINE_ROLE(type) struct type : ::details::Role<type>

#endif
\end{c++}
\end{leftbar}

通过\ascii{DEFINE\_ROLE}的宏定义来实现对接口的定义,可以消除子类对虚拟析构函数的重复实现,消除代码重复。

\end{content}

\section*{作者}

\begin{content}

本规范主要由刘光聪,王博,尉刚强,赵永刚撰写,并由袁英杰审校。由于时间的仓促,规范错误再所难免。如果您发现了错误,或者有更好的意见,请第一时间联系\ascii{COCK(C++ Optimal Construction Kit)},我们将非常感激!

\end{content}

\myclearpage

\tableofcontents
\myclearpage

\def\thelstlisting{\thechapter-\arabic{lstlisting}}
%% 中文习惯是设定首行缩进为2em。
%% 注意此设置一定要在document环境之中,这可能与\setlength作用范围相关
\setlength{\parindent}{2em}

%%%%%%%%%%%%%%%%%%%%%%
%%开始正文,页面计数从正文开始
\mainmatter
\setcounter{page}{1}
\pagestyle{fancy}

\begin{savequote}[45mm]
\ascii{Any fool can write code that a computer can understand. Good programmers write code that humans can understand.}
\qauthor{\ascii{- Martin Flower}}
\end{savequote}

\chapter{线性模型} 
\label{ch:linear-model}

\section{逻辑回归}

\begin{content}

在逻辑回归(\ascii{Logistic Regression})中,$y$是一个实数值,且$y \in \{ 0,1\}$。但是,逻辑回归解决的是二分类问题。一般地,如果$y \ge 0.5$则判定为正类;否则判定为负类。

\[c = \left\{ \begin{gathered}
  1,y \ge 0.5 \hfill \\
  0,y < 0.5 \hfill \\ 
\end{gathered}  \right.\]

另外,常常使用\emph{指示函数}表达上式。

\[
c = \mathbb I(y \ge 0.5)
\]

其中,$\mathbb I(true) = 1$;否则$\mathbb I(false) = 0$。

\subsection{符号定义}

为了形式化地描述逻辑回归问题,此处定义了一些常用符号。需要注意的是,为了区分预测值与标签值的符号,因此使用$y, t$分别表示它们。

 \begin{itemize}
   \item \ascii{训练样本集}: $ S = \{ ({x^{(i)}},{t^{(i)}});i = 1,2,...,m\} $
   \item \ascii{第$i$个训练样本}: $ ({x^{(i)}},{t^{(i)}}) $
   \item \ascii{样本输入}: $ x = ({x_1},{x_2},...,{x_n})^{T}  \in {\mathbb{R}^n} $
   \item \ascii{样本标签}: $ t \in {\mathbb{R}} $
   \item \ascii{预测值}: $ y \in {\mathbb{R}} $   
   \item \ascii{误差值}: $ e = y - t \in {\mathbb{R}} $
   \item \ascii{权重}: $ w \in {\mathbb{R}^{n}} $   
   \item \ascii{偏置}: $ b \in {\mathbb{R}} $
   \item \ascii{线性加权和}: $ z = w^Tx + b \in {\mathbb{R}} $   
   \item \ascii{激活函数}: $ f(z) = \frac{1}{{1 + {e^{ - z}}}} \in {[0, 1]} $
   \item \ascii{正则项}: $ R(w) \in {\mathbb{R}} $
   \item \ascii{L2正则项}: $ \parallel w\parallel _2^2 \in {\mathbb{R}} $
   \item \ascii{损失函数(无正则项)}: $ L(w, b) \in {\mathbb{R}} $
   \item \ascii{损失函数(含正则项)}: $ J(w, b) = L(w, b) + \lambda R(w)\in {\mathbb{R}} $
   \item \ascii{梯度函数}: $ {\nabla _w}L( {w,b}) \in {\mathbb{R}^n} $
 \end{itemize}

\subsection{模型定义}

如\refig{logistic-regression-nn}所示,逻辑回归等价于仅具有一个神经元的网络模型。

\begin{figure}[H]
\centering
\includegraphics[width=0.6\textwidth]{figures/logistic-regression-nn.png}
\caption{逻辑回归:神经元模型}
 \label{fig:logistic-regression-nn}
\end{figure}

首先,逻辑回归完成$z = {w^T}x + b$的线性加权和;然后,使用\ascii{sigmoid}的激活函数,完成非线性变换。

\[\begin{aligned}
  y =  & {h_{w,b}}(x) \\ 
   =  & f({w^T}x + b) \\ 
   =  & \frac{1}{{1 + {e^{ - {w^T}x + b}}}} \\ 
\end{aligned} \]

$w$表示模型的权重,它是一个$n$维的向量;$b$表示神经元的偏置,它是一个标量。

\[\begin{gathered}
  w = \left[ {\begin{array}{*{20}{c}}
  {{w_1}} \\ 
  {{w_2}} \\ 
   \vdots  \\ 
  {{w_n}} 
\end{array}} \right] \in {\mathbb{R}^n} \\ 
  b \in \mathbb{R} \\ 
\end{gathered} \]

$x$表示模型的输入,它是一个$n$维的向量;其中,$\forall {x_i} \in x,i = 1,2,...,n$,它表示输入向量$x$的一个特征值;$y$表示模型的输出,它是一个标量。

\[\begin{gathered}
  x = \left[ {\begin{array}{*{20}{c}}
  {{x_1}} \\ 
  {{x_2}} \\ 
   \vdots  \\ 
  {{x_n}} 
\end{array}} \right] \in {\mathbb{R}^n} \\ 
  {\text{y}} \in {\mathbb{R}} \\ 
\end{gathered}\]

$f(z)$是神经元的激活函数,此处使用\ascii{sigmoid}函数。

\subsection{Sigmoid函数}

\ascii{sigmoid}函数将无穷的定义域空间压缩为$[0, 1]$的值域空间,因此具有天然的概率解释意义。

\[
f(z) = \frac{1}{{1 + {e^{ - z}}}}
\]


\subsubsection{软饱和性}

如\refig{sigmoid}所示,\ascii{sigmoid}函数具有软饱和性,即

\[\begin{gathered}
  \mathop {\lim }\limits_{z \to  + \infty } f(z) = 1 \hfill \\
  \mathop {\lim }\limits_{z \to  - \infty } f(z) = 0 \hfill \\ 
\end{gathered} \]

\begin{figure}[H]
\centering
\includegraphics[width=0.6\textwidth]{figures/sigmoid.png}
\caption{sigmoid函数}
 \label{fig:sigmoid}
\end{figure}

\subsubsection{导数}

\ascii{sigmoid}的导数具有特殊的表现形式,其导函数是$y$的二次函数。令

\[
y = \frac{1}{{1 + {e^{ - z}}}}
\]

根据链式求导法则,可以很容易地推导出\ascii{sigmoid}的导数。

\[\begin{aligned}
  y' =  & \frac{d}{{dz}}\left( {\frac{1}{{1 + {e^{ - z}}}}} \right) \\ 
   =  & \frac{d}{{dz}}{\left( {1 + {e^{ - z}}} \right)^{ - 1}} \\ 
   =  &  - {\left( {1 + {e^{ - z}}} \right)^{ - 2}}\frac{d}{{dz}}\left( {{e^{ - z}}} \right) \\ 
   =  &  - {\left( {1 + {e^{ - z}}} \right)^{ - 2}}{e^{ - z}}\frac{d}{{dz}}\left( { - z} \right) \\ 
   =  &  - {\left( {1 + {e^{ - z}}} \right)^{ - 2}}{e^{ - z}}\left( { - 1} \right) \\ 
   =  & \frac{{{e^{ - z}}}}{{{{\left( {1 + {e^{ - z}}} \right)}^2}}} \\ 
   =  & \frac{1}{{1 + {e^{ - z}}}}\left( {1 - \frac{1}{{1 + {e^{ - z}}}}} \right) \\ 
   =  & y(1 - y) \\ 
\end{aligned} \]

\subsection{概率解释}

对于任意的$(x,t) \in S$,根据\ascii{Sigmoid}的函数特性,假设$t|x$服从$Bernoulli(y)$的概率分布。

\[\begin{aligned}
  P(t = 1|x;w,b) = & y \\ 
  P(t = 0|x;w,b) = & 1 - y \\ 
\end{aligned} \]

众所周知,$Bernoulli(y)$概率分布可以描述为如下更为简洁的表达方式。

\[P(t|x;w,b) = {y^t}{(1 - y)^{1 - t}}\]

推而广之,假如存在$m$个独立同分布的训练样本$\{ ({x^{(i)}},{t^{(i)}});i = 1,2,...,m\}$,以$(w,b)$为参数的似然函数可表达为:

\[\begin{aligned}
  L(w,b) =  & P\left( {\vec y|X} \right) \\ 
   =  & \prod\limits_{i = 1}^m {P\left( {{t^{(i)}}|{x^{(i)}};w,b} \right)}  \\ 
   =  & \prod\limits_{i = 1}^m {{{\left( {{y^{(i)}}} \right)}^{{t^{(i)}}}}{{\left( {1 - {y^{(i)}}} \right)}^{1 - {t^{(i)}}}}}  \\ 
\end{aligned} \]

其中,
${y^{(i)}} = {h_{w,b}}({x^{(i)}}), i=1,2,...,m $。一般地,根据对数函数的单调递增性,将其变换为对数似然函数,使得连乘运算转换为连加运算,简化了问题的复杂度。

\[l(w,b) = \log L(w,b) = \sum\limits_{i = 1}^m {{t^{(i)}}\log {y^{(i)}} + \left( {1 - {t^{(i)}}} \right)\log \left( {1 - {y^{(i)}}} \right)} \]

因此,逻辑回归问题可以转换为最大化对数似然函数的参数估计问题。

\[\hat w,\hat b = \arg \mathop {\max }\limits_{w,b} \sum\limits_{i = 1}^m {{t^{(i)}}\log {y^{(i)}} + \left( {1 - {t^{(i)}}} \right)\log \left( {1 - {y^{(i)}}} \right)} \]

可以使用梯度上升的优化方法,迭代求取最优的参数$(\hat w,\hat b)$。

\subsection{交叉熵损失函数}

在机器学习领域中,常常使用\emph{损失函数},并使用\emph{梯度下降}的优化算法,迭代求取最优的参数$(\hat w,\hat b)$。根据上一节的推导,给定一个训练样本$(x, t)$,逻辑回归可以使用如下定义的损失函数,它是交叉熵损失函数在二分类问题下的特殊表现形式。

\[L(w,b;x,t) =  - \left\{ {t\log y + \left( {1 - t} \right)\log \left( {1 - y} \right)} \right\}\]

推而广之,给定训练数据集$ S = \{ ({x^{(i)}},{t^{(i)}});i = 1,2,...,m\} $,逻辑回归的损失函数可以表达为:

\[L(w,b; S) =  - \sum\limits_{i = 1}^m {{t^{(i)}}\log {y^{(i)}} + \left( {1 - {t^{(i)}}} \right)\log \left( {1 - {y^{(i)}}} \right)} \]

因此,逻辑回归的学习问题便转换为最小化损失函数$L(w,b)$的优化问题。

\[\hat w,\hat b = \arg \mathop {\min }\limits_{w,b} L(w, b)\]

\subsection{正则项}

为了控制模型的复杂度,可以增加$L2$的正则项,它等于向量$w$的内积。

\[R(w) = \lambda \parallel w\parallel _2^2 = {w^T}w \]

其中,$\lambda$称为正则项因子,用于控制正则项的影响程度;它是模型的超参数,可以通过交叉验证试验获取最佳的数值。增加了正则项后,模型的损失函数可以表达为:

\[J(w,b) = L(w,b) + \lambda \parallel w\parallel _2^2\]

其优化问题也随之表达为:

\[\hat w,\hat b = \arg \mathop {\min }\limits_{w,b} J(w, b)\]

\subsection{梯度下降}

可以使用梯度下降算法,迭代求取最优的$(\hat w,\hat b)$。其中,$\alpha$表示学习速率,表示参数更新的大小。

\[\begin{aligned}
  w \leftarrow & w - \alpha {\nabla _w}L(w,b) \\ 
  b \leftarrow & b - \alpha {\nabla _b}L(w,b) \\ 
\end{aligned} \]

其中,$w,b$的梯度分别定义为:

\[\begin{aligned}
  {\nabla _w}L(w,b) = & \frac{{\partial L}}{{\partial w}} \\ 
  {\nabla _b}L(w,b) = & \frac{{\partial L}}{{\partial b}} \\ 
\end{aligned} \]

如\refig{mnist-gd}所示,可以将损失函数比作一座山,登山者试图寻找最佳的行动方案达到山谷。登山者站在某个山坡上环顾四周,决定沿梯度的反方向向下走一小步,直到获得更优的解。当实施梯度下降更新算法时,初始点不同,获得的最小值可能也不同。因此梯度下降求得的只是局部最小值。

\begin{figure}[H]
\centering
\includegraphics[width=0.8\textwidth]{figures/mnist-gd.jpeg}
\caption{梯度下降算法}
 \label{fig:mnist-gd}
\end{figure}

梯度下降的步伐大小非常重要,如果太小,则找到函数最小值的速度就很慢;如果太大,则可能会越过极值点。一般地,在模型训练初期,因为离模型收敛的目标还比较远,因此将其学习速率调得大一些;随着迭代次数增加,学习速率调得小一些。因此,学习速率$\alpha$是自适应的,目前理论上存在很多学习速率$\alpha$随时间衰减的优化算法,例如\ascii{Adagrad}等。因此,优化问题的关键在于梯度如何计算。

\subsection{计算梯度}

根据链式求导法则,给定任意一个样本$ (x,t) \in S $,可以推导出$ L(y, t) $相对于$ w_i, i=1,2,...,n $的梯度公式。

\[\begin{aligned}
  \frac{{\partial L}}{{\partial {w_i}}}
   =  & \frac{{\partial L}}{{\partial y}}\frac{{\partial y}}{{\partial z}}\frac{{\partial z}}{{\partial {w_i}}} \\ 
   =  & \left( {\frac{{1 - t}}{{1 - y}} - \frac{t}{y}} \right)y(1 - y){x_i} \\ 
   =  & (y - t){x_i} \\ 
\end{aligned} \]

同理,可以推导出$ L(W,b) $相对于$ b $的梯度公式。

\[\begin{aligned}
  \frac{{\partial L}}{{\partial b}} 
   =  & \frac{{\partial L}}{{\partial y}}\frac{{\partial y}}{{\partial z}}\frac{{\partial z}}{{\partial b}} \\ 
   =  & \left( {\frac{{1 - t}}{{1 - y}} - \frac{t}{y}} \right)y(1 - y) \\ 
   =  & y - t \\ 
\end{aligned} \]

进一步,通过矢量化得到$w,b$的梯度计算公式。其中,$\nabla _w}L( {w,b;x,t}) \in {\mathbb{R}^{n}}$,$\nabla _b}L( {w,b;x,t}) \in {\mathbb{R}}$。

\[\begin{gathered}
  {\nabla _w}L( {w,b;x,t}) =  \left({y - t} \right)x \\ 
  {\nabla _b}L( {w,b;x,t) =   y - t  \\ 
\end{gathered} \]

推而广之,给定训练数据集$ S = \{ ({x^{(i)}},{t^{(i)}});i = 1,2,...,m\} $,得到$w,b$的矢量化的梯度计算公式为:

\[\begin{gathered}
  {\nabla _w}L(w,b;S) = \left( {\frac{1}{m}\sum\limits_{i = 1}^m {\left( {{y^{(i)}} - {t^{(i)}}} \right)} } \right)x \\ 
  {\nabla _b}L(w,b;S) = \frac{1}{m}\sum\limits_{i = 1}^m {\left( {{y^{(i)}} - {t^{(i)}}} \right)}  \\ 
\end{gathered} \]

\subsection{参数更新}

参数更新存在三种基本的算法。对于给定一个训练样本$ ({x^{(i)}},{t^{(i)}}) $,根据$w, b$的梯度公式,完成本次迭代的参数更新,该算法常称为\emph{随机梯度下降法}(\ascii{SGD})。因为更新一次参数,\ascii{SGD}只需读取一个训练样本,而单个样本并不能代表普遍性,可能存在很多噪声,因此模型的收敛情况较为抖动。但是,\ascii{SGD}可以实现模型的快速收敛,效率较为高效。

\[\begin{gathered}
  w \leftarrow w - \alpha \left( {{y^{(i)}} - {t^{(i)}}} \right)x \\ 
  b \leftarrow b - \alpha \left( {{y^{(i)}} - {t^{(i)}}} \right) \\ 
\end{gathered} \]

另外一种极端,给定整个训练样本数据$ S $,根据$w, b$的梯度公式,完成本次迭代的参数更新,该算法常称为\emph{批式梯度下降法}(\ascii{BGD})。因为更新一次参数,需要遍历整个训练数据集,因此模型收敛较为平稳。但是,由于计算量大,模型收敛速度相对\ascii{SGD}较为缓慢。

\[\begin{gathered}
  w \leftarrow w - \alpha \left( {\frac{1}{m}\sum\limits_{i = 1}^m {\left( {{y^{(i)}} - {t^{(i)}}} \right)} } \right)x \\ 
  b \leftarrow b - \alpha \left( {\frac{1}{m}\sum\limits_{i = 1}^m {\left( {{y^{(i)}} - {t^{(i)}}} \right)} } \right) \\ 
\end{gathered} \]

而在现实应用中,既不会使用\ascii{BGD},也不会使用\ascii{SGD},而使用\ascii{MiniBatch}的\ascii{SGD}算法。更为甚者,由于\ascii{MiniBatch}的\ascii{SGD}的大规模应用,因此常简称\ascii{MiniBatch}的\ascii{SGD}为\ascii{SGD}算法。

\[\begin{aligned}
  w \leftarrow w - \alpha \left( {\frac{1}{{batch\_zie}}\sum\limits_{i = 1}^{batch\_size} {\left( {{y^{(i)}} - {t^{(i)}}} \right)} } \right)x \\ 
  b \leftarrow b - \alpha \left( {\frac{1}{{batch\_size}}\sum\limits_{i = 1}^{batch\_size} {\left( {{y^{(i)}} - {t^{(i)}}} \right)} } \right) \\ 
\end{aligned} \]

相对于\ascii{BGD},\ascii{MiniBatch}的\ascii{SGD}每次跟新参数不用遍历真个训练数据集,只需遍历一个\ascii{batch\_size}个训练样本;因此,相对\ascii{BGD},\ascii{MiniBatch}的\ascii{SGD}的收敛速度更快。而相对于\ascii{SGD},\ascii{MiniBatch}的\ascii{SGD}每次跟新参数,读取携带更多普遍性的\ascii{batch\_size}个训练样本;因此,相对\ascii{SGD},\ascii{MiniBatch}的\ascii{SGD}的收敛更为平稳。

\subsection{计算图}

一般地,当使用神经网络表示学习模型时,其训练过程都包括两个基本步骤:

\begin{itemize}
  \item \ascii{正向计算}: 计算损失
  \item \ascii{反向传播}: 计算梯度
\end{itemize}

为了实现正向的损失计算,及其反向的梯度计算,可以将网络翻译为等价的计算图。在正向的计算图中,节点描述了某种抽象的数学运算,例如矩阵乘法,向量内积,激活函数等;边表示函数的输出值,并作为下游节点函数的输入。



如\refig{logistic-bp}所示,逻辑回归的正向

\begin{figure}[H]
\centering
\includegraphics[width=0.8\textwidth]{figures/logistic-bp.png}
\caption{逻辑回归:前向计算与反向传播}
 \label{fig:logistic-bp}
\end{figure}


\end{content}

\section{单层感知器}

\begin{content}

首先,尝试构建\ascii{10}个神经元的单层感知器。如\refig{mnist-slp}所示,对于诸如手写数字识别的多分类问题,理论上常使用\ascii{softmax}的激活函数。

\begin{figure}[H]
\centering
\includegraphics[width=0.8\textwidth]{figures/mnist-slp.png}
\caption{单层感知器}
 \label{fig:mnist-slp}
\end{figure}

\subsection{理论基础}

理论上,\ascii{softmax}回归是\ascii{logistic}回归的广义扩展。其中,\ascii{logistic}回归是为了解决二分类问题,即$y \in \{ 0,1\}$;而\ascii{softmax}回归是为了解决$ k $分类问题,即$y \in \{ 1,2,...,k\}$。

\subsubsection{符号定义}

为了形式化地描述\ascii{softmax}回归问题,此处定义了一些常用符号。

 \begin{itemize}
   \item \ascii{训练样本集}: $ S = \{ ({x^{(i)}},{y^{(i)}});i = 1,2,...,m\} $
   \item \ascii{第$i$个训练样本}: $ ({x^{(i)}},{y^{(i)}}) $
   \item \ascii{样本输入}: $ x = ({x_1},{x_2},...,{x_n})^{T}  \in {\mathbb{R}^n} $
   \item \ascii{样本标签(one-hot)}: $ y = ({y_1},{y_2},...,{y_k})^{T} \in {\mathbb{R}^k} $
   \item \ascii{权重}: $ W \in {\mathbb{R}^{n \times k}} $   
   \item \ascii{偏置}: $ b \in {\mathbb{R}^k} $   
   \item \ascii{softmax函数}: $ 
softmax {(z_i)} = \tfrac{{{e^{{z_i}}}}}{{\sum\limits_{j = 1}^k {{e^{{z_j}}}} }}  \quad i = 1,2,...,k
$
 \end{itemize}

\subsubsection{softmax函数}

如\refig{softmax}所示,模型先求取线性加权和$z$,然后求取$e^z$,最后再实施归一化操作。

\begin{figure}[H]
\centering
\includegraphics[width=0.8\textwidth]{figures/softmax.png}
\caption{softmax函数}
 \label{fig:softmax}
\end{figure}

\subsubsection{权重与偏置}

权重$W$为一个$n \times k$的二维矩阵。

\[
W = \left( {{W_1},{W_2},...,{W_k}} \right) = \left( {\begin{array}{*{20}{c}}
  {{w_{11}}}& \ldots &{{w_{1k}}} \\ 
   \vdots & \ddots & \vdots  \\ 
  {{w_{n1}}}& \cdots &{{w_{nk}}} 
\end{array}} \right) \in {\mathbb{R}^{n \times k}}
\]

其中,$W_j$是一个长度为$n$的向量。

\[
{W_j} = {\left( {{w_{1j}},{w_{2j}},...,{w_{nj}}} \right)^T} \in {\mathbb{R}^n}, j = 1,2,...,k \\
\]

而偏置$b$是一个长度为$k$的\ascii{\quo{one-hot}}向量。

\[
b = {({b_1},{b_2},...,{b_k})^T} \in {\mathbb{R}^k}
\]

\subsubsection{模型定义}

多分类问题的单层感知器模型,使用\ascii{softmax}激活函数,可以如此定义。

\[\begin{aligned}
  y =  & {h_{W,b}}(x) = softmax (z) = softmax ({W^T}x + b) \\ 
   =  & {\left( {{y_1},{y_2},...,{y_k}} \right)^T} \\ 
   =  & \frac{1}{{\sum\limits_{j = 1}^k {{e^{{z_j}}}} }}{\left( {{e^{{z_1}}},{e^{{z_2}}},...,{e^{{z_k}}}} \right)^T} \\ 
   =  & \frac{1}{{\sum\limits_{j = 1}^k {{e^{W_j^Tx + {b_j}}}} }}{\left( {{e^{W_1^Tx + {b_1}}},{e^{W_2^Tx + {b_2}}},...,{e^{W_k^Tx + {b_k}}}} \right)^T} \ 
\end{aligned} \]

其中,对于任意给定的样本$ (x, y) \in S $,$ z_i $表示$W_i^Tx+b_i$的线性加权和,而$y_i(i=1,2,...,k)$表示将其划归为类$i$的概率。

\[\begin{gathered}
  P\left( {y = i|x;W,b} \right) = \frac{{{e^{W_i^Tx + b_i}}}}{{\sum\limits_{j = 1}^k {{e^{W_j^Tx + b_j}}} }} \hfill \\
  i = 1,2,...,k \hfill \\ 
\end{gathered} \]


\subsubsection{交叉熵函数}

基于样本数据集$ S = \{ ({x^{(i)}},{y^{(i)}});i = 1,2,...,m\} $,交叉熵损失函数可以如此定义。

\[\begin{aligned}
  J(W,b) =  &  - \frac{1}{m}\sum\limits_{i = 1}^m {{y^{(i)}}\log \left( {{{\widehat y}^{(i)}}} \right)}  \\ 
   =  &  - \frac{1}{m}\sum\limits_{i = 1}^m {\sum\limits_{j = 1}^k {y_j^{(i)}\log \left( {\widehat y_j^{(i)}} \right)} }  \\
\end{aligned} \]

\ascii{softmax}多分类问题,就是求取最优解$(W^*,b^*)$,使得

\[W^*,b^* = \mathop {\arg \min }\limits_{W,b} J(W,b)\]

\subsubsection{计算梯度}

对于任意一个样本$ (x,y) \in S $,可以推导出$ J(W,b) $相对于$ W $与$ b $的梯度公式。

\[\begin{aligned}
  {\nabla _W}J\left( {W,b;x,y} \right) =  & \left( {\widehat y - y} \right)x \\ 
  {\nabla _b}J\left( {W,b;{x^{(i)}},{y^{(i)}}} \right) =  & \left( {\widehat y - y} \right) \\ 
\end{aligned} \]


\subsubsection{参数更新}

对于训练样本数据$ S $,根据$W, b$的梯度公式,完成本次迭代的参数更新。

\[\begin{aligned}
  W \leftarrow  & W - \alpha \frac{{\sum\limits_{i = 1}^m {{\nabla _W}J\left( {W,b;{x^{(i)}},{y^{(i)}}} \right)} }}{m} \\ 
  b \leftarrow  & b - \alpha \frac{{\sum\limits_{i = 1}^m {{\nabla _b}J\left( {W,b;{x^{(i)}},{y^{(i)}}} \right)} }}{m} \\ 
\end{aligned} \]

\subsection{定义模型}

接下来,使用\tf{}完成该模型的搭建和训练。需要注意的是,理论上的公式与\tf{}具体实现存在微妙的差异。理论上,公式中的$x$常表示一个样本,但\tf{}中的\code{x}常表示一个\ascii{mini-batch}的样本数据集。因此,使用\tf{}设计网络模型时,需要特别关注各个张量大小的变化是否符合预期。

\subsubsection{输入和标签}

首先,使用\code{tf.placeholder}分别定义训练样本的输入和标签。

\begin{leftbar}
\begin{python}
x = tf.placeholder(tf.float32, [None, 28, 28, 1])
t = tf.placeholder(tf.float32, [None, 10])
\end{python}
\end{leftbar}

\code{tf.placeholder}定义了一个占位的\ascii{OP}。\code{None}表示未确定的样本数目,此处表示\code{batch\_size}的大小;当\code{Session.run}时,将通过\code{feed\_dict}的字典提供一个\ascii{mini-batch}的样本数据集,从而自动推导出\code{tf.placeholder}的大小。

另外,每张图片使用$ 28 \times 28 \times 1 $的三维数据表示(灰度为\ascii{1})。为了简化问题,此处将输入的样本数据扁平化,将其变换为长度为\ascii{784}的一维向量。其中,\ascii{-1}表示\ascii{mini-batch}的样本数目,由运行时自动推演其大小。

\begin{leftbar}
\begin{python}
x = tf.reshape(x, [-1, 784])
\end{python}
\end{leftbar}

\subsubsection{定义变量}

然后,使用\code{tf.Variable}定义模型参数。定义训练参数时,必须指定参数的初始化值;训练参数将根据初始值,自动推演数据的类型,及其大小。

\begin{leftbar}
\begin{python}
w = tf.Variable(tf.zeros([784, 10]))
b = tf.Variable(tf.zeros([10]))
\end{python}
\end{leftbar}

此外,变量在使用之前,必须完成初始化。此处,\code{init\_op}将初始化所有全局的训练参数。

\begin{leftbar}
\begin{python}
init_op = tf.global_variables_initializer()
\end{python}
\end{leftbar}

\subsubsection{模型定义}

接下来,便可以很容易地得到多分类问题的单层感知器模型。

\begin{leftbar}
\begin{python}
y = tf.nn.softmax(tf.matmul(x, w) + b)
\end{python}
\end{leftbar}

如\refig{mnist-linear-sum}所示,首先计算\code{x}与\code{w}的矩阵乘法,让后将\code{b}广播(\ascii{broadcast})到矩阵的每一行相加,最终得到训练参数的线性加权和。

\begin{figure}[H]
\centering
\includegraphics[width=0.8\textwidth]{figures/mnist-linear-sum.png}
\caption{线性加权和}
 \label{fig:mnist-linear-sum}
\end{figure}

如\refig{mnist-softmax}所示,\ascii{softmax}将逐行实施运算,最终,\code{y}的大小为\code{[100, 10]}。

\begin{figure}[H]
\centering
\includegraphics[width=0.8\textwidth]{figures/mnist-softmax.png}
\caption{激活函数:softmax}
 \label{fig:mnist-softmax}
\end{figure}

\subsubsection{损失函数}

对于多分类问题,可以使用交叉熵的损失函数。

\begin{leftbar}
\begin{python}
cross_entropy = -tf.reduce_sum(t * tf.log(y))
\end{python}
\end{leftbar}

如\refig{mnist-cross-entropy}所示,\code{t}和\code{y}的大小都为\code{[100, 10]};特殊地,\code{t}的每一行都是一个\quo{\ascii{one-hot}}向量。

对\code{y}实施\code{tf.log}操作,也将得到一个大小为\code{[100, 10]}的矩阵。然后,\code{t}与\code{tf.log(y)}逐位相乘(并非矩阵相乘),也将得到大小为\code{[100, 10]}的矩阵。最终,\code{tf.reduce\_sum}将矩阵中所有元素相加,得到一个标量(\ascii{scalar})值。

\begin{figure}[H]
\centering
\includegraphics[width=0.8\textwidth]{figures/mnist-cross-entropy.png}
\caption{交叉熵损失函数}
 \label{fig:mnist-cross-entropy}
\end{figure}

\subsubsection{精度}

\code{tf.argmax(y,1)}将按第\ascii{1}个维度计算最大值的索引。既按照$ y_{100 \times 10} $的每一行,计算得到在每一行中最大值的的索引值。因此,\code{tf.argmax(y,1)}将得到大小为\code{[100, 1]}的矩阵,或大小为\ascii{100}的向量。同样地,\code{tf.argmax(t,1)}也是一个大小为\ascii{100}的向量。

然后,使用\code{tf.equal}将它们逐元素(\ascii{element-wise})进行相等性比较,得到大小为\ascii{100}的布尔向量。为了计算精度,先将布尔向量转别为数值向量,最终求取该数值向量的均值。

\begin{leftbar}
\begin{python}
is_correct = tf.equal(tf.argmax(y,1), tf.argmax(t,1))
accuracy = tf.reduce_mean(tf.cast(is_correct, tf.float32))
\end{python}
\end{leftbar}

\subsection{优化算法}

接下来,使用梯度下降算法实现交叉熵损失函数的最小化。其中,\code{learning\_rate}表示学习速率,描述参数更新的快慢和步伐大小,是一个典型的超参。

\begin{leftbar}
\begin{python}
optimizer = tf.train.GradientDescentOptimizer(learning_rate=0.003)
train_step = optimizer.minimize(cross_entropy)
\end{python}
\end{leftbar}

\subsection{训练模型}

在此之前,\tf{}仅构造计算图,并没有启动计算图的执行。接下来,客户端创建一个会话,建立与本地或远端计算设备集的通道,启动计算图的执行过程。

首先,完成训练参数的初始化。通过运行模型参数的初始化子图,并发地执行各个训练参数的初始化器,将初始值就地修改到相应的训练参数内。

\begin{leftbar}
\begin{python}
with tf.Session() as sess:
  sess.run(init_op)
\end{python}
\end{leftbar}

然后,开始迭代地执行\code{train\_step},完成模型的一次迭代训练。其中,每\ascii{100}次迭代,计算当前模型在训练数据集及测试数据集的精度和损失。

\begin{leftbar}
\begin{python}
with tf.Session() as sess:
  for step in range(1000):
    batch_xs, batch_ys = mnist.train.next_batch(100)        
    sess.run(train_step, feed_dict={x: batch_xs, t: batch_ys})
    
    if step % 100 == 0:
      acc, loss = sess.run([accuracy, cross_entropy], 
        feed_dict={x: batch_xs, t: batch_ys})
      acc, loss = sess.run([accuracy, cross_entropy], 
        feed_dict={x: mnist.test.images, t: mnist.test.labels}) 
\end{python}
\end{leftbar}

据统计,经过\ascii{1000}次迭代,可得到大约\percent{92}的精度。

\begin{figure}[H]
\centering
\includegraphics[width=0.8\textwidth]{figures/mnist-slp-accuracy.png}
\caption{可视化:单层感知器,运行1000次step}
 \label{fig:mnist-slp-accuracy}
\end{figure}

\end{content}

\part{基础知识}
\begin{savequote}[45mm]
\ascii{Any fool can write code that a computer can understand. Good programmers write code that humans can understand.}
\qauthor{\ascii{- Martin Flower}}
\end{savequote}

\chapter{介绍} 
\label{ch:introduction}

\begin{content}

\ascii{TensorFlow}是一个支持大规模和异构环境的机器学习系统。它使用\emph{数据流图}\ascii{(Dataflow Graph)}表示计算过程和共享状态,使用节点表示抽象计算,使用边表示数据流。\upcite{tf-white-paper}

数据流图的节点被映射在集群中的多个机器,在一个机器内被映射在多个计算设备\ascii{(Device)}上,包括\ascii{CPU, GPU, TPU}。\ascii{TensorFlow}灵活的架构支持多种计算平台,包括台式机,服务器,移动终端等。

\tf{}最初由\ascii{Google Brain}的研究员和工程师们开发出来,用于开展机器学习和深度神经网络方面的研究,但\tf{}优异的通用性使其也可广泛用于其他领域的数值计算。

\end{content}

\section{前世今生}

\begin{content}

\tf{}是\ascii{DistBelief}的后继者,它站在巨人的肩膀上,革命性地重新设计架构设计,使得\tf{}在机器学习领域一鸣惊人,在社区中产生了重大的影响。

为了更好地理解\tf{}系统架构的优越性,得先从\ascii{DistBelief}谈起。

\end{content}

\subsection{DistBelief}

\begin{content}

\ascii{DistBelief}是一个用于训练大规模神经网络的的分布式系统,是\ascii{Google}第一代分布式机器学习框架。自\ascii{2011}年以来,在\ascii{Google}内部大量使用\ascii{DistBelief}训练大规模的神经网络。

\end{content}

\subsubsection{编程模型}

\begin{content}

\ascii{DistBelief}的编程模型是基于层\ascii{(Layer)}的\ascii{DAG(Directed Acyclic Graph)}图。层可以看做是一种组合多个运算操作符的复合运算符,它完成特定的计算任务。

例如,全连接层完成$f({W^T}x + b)$的复合计算,包括输入与权重的矩阵乘法,随后再与偏置相加,最后在线性加权值的基础上实施非线性变换。

\end{content}

\subsubsection{架构}

\begin{content}

\ascii{DistBelief}使用参数服务器\ascii{(Parameter Server)}的系统架构,训练作业包括两个分离的进程:无状态的\ascii{Worker}进程,用于模型训练的计算;有状态的\ascii{PS(Parameter Server)}进程,用于维护模型参数。

如\refig{parameter-server}所示,在分布式训练过程中,各个模型备份\ascii{(Model Relica)}异步地从\ascii{PS}上拉取\ascii{(Fetch)}训练参数$w$,当完成一步迭代运算后,推送\ascii{(Push)}参数的梯度$ \nabla w $到\ascii{PS}上去,并完成参数更新。

\begin{figure}[!htbp]
\centering
\includegraphics[width=0.6\textwidth]{figures/parameter-server.png}
\caption{DistBelief: Parameter Server架构}
 \label{fig:parameter-server}
\end{figure}

\end{content}

\subsubsection{缺陷}

\begin{content}


但是,对于高级用户,\ascii{DistBelief}的编程模型,及其\ascii{Parameter Server}的系统架构,缺乏如下几个方面的扩展性。

\begin{enum}
  \eitem{优化算法:添加新的优化算法,必须修改\ascii{Parameter Server}的实现;\code{get(), put()}的抽象方法,对某些优化算法并不高效;}
  \eitem{训练算法:支持非前馈的神经网络具有很大的挑战性,例如包含循环的\ascii{RNN},交替训练的对抗网络,及其损失函数由分离的代理完成增强学习模型;} 
  \eitem{加速设备:\ascii{DistBelief}设计之初仅支持多核\ascii{CPU},并不支持\ascii{GPU};遗留的系统架构对支持新的计算设备缺乏弹性空间。}
\end{enum}

\end{content}

\subsection{TensorFlow}

\begin{content}

正因为\ascii{DistBelief}遗留的架构和设计,不再满足潜在的深度学习与日俱增的需求,\ascii{Google}毅然决定在\ascii{DistBelief}基础上做全新的架构设计,从而诞生了\ascii{TensorFlow}。

\end{content}

\subsubsection{编程模型}

\begin{content}

\ascii{TensorFlow}使用数据流图\ascii{(Dataflow Graph)}表示计算过程和共享状态,使用节点表示抽象计算,使用边表示数据流。如\refig{tf-dataflow}所示,展示了\ascii{mnist}手写识别应用的数据流图。

在该模型中,前向子图使用了\ascii{2}层全连接网络,分别为\ascii{ReLU}层和\ascii{Softmax}层;随后,由\ascii{Gradients}构建了与前向子图对应的反向子图,用于训练参数的梯度计算;最后,使用`SGD`的优化算法,构造参数更新子图,完成参数的更新。

\begin{figure}[!htbp]
\centering
\includegraphics[width=0.4\textwidth]{figures/tf-dataflow.png}
\caption{TensorFlow数据流图}
 \label{fig:tf-dataflow}
\end{figure}

\end{content}

\subsubsection{设计原则}

\begin{content}

\begin{enum}
  \eitem{延迟计算:图的构造与执行分离,并推迟计算图的执行过程;}
  \eitem{原子\ascii{OP}:\ascii{OP}是最小的抽象计算单元,支持构造复杂的网络模型;} 
  \eitem{抽象设备:支持\ascii{CPU, GPU, TPU}多种异构计算设备类型;}
  \eitem{抽象任务:基于\ascii{Task}的\ascii{PS}任务,对新的优化算法和网络模型具有良好的可扩展性。}  
\end{enum}

\end{content}

\subsubsection{优势}

\begin{content}

相对于其他机器学习框架,\ascii{TensorFlow}具有如下方面的优势。

\begin{enum}
  \eitem{跨平台:支持多\ascii{CPU/GPU/TPU}运算;支持台式机/服务器/移动设备;支持\ascii{Windows,Linux,MacOS};}
  \eitem{分布式:支持本地和分布式的模型训练和推理;}
  \eitem{多语言:支持\ascii{Python, C++, Java, Go}等多种程序设计语言的\ascii{API};}  
  \eitem{通用性:支持各种复杂的网络模型的设计和实现;}
  \eitem{可扩展:支持\ascii{OP}扩展,\ascii{Kernel}扩展,\ascii{Device}扩展;}
  \eitem{可视化:使用\ascii{TensorBoard}可视化整个训练过程,包括计算图。}
\end{enum}

\end{content}

\section{社区发展}

\begin{content}

\tf{}是目前最为流行的机器学习框架。自开源以来,\tf{}社区相当活跃。来自众多的非\ascii{Google}员工拥有数万次代码提交,并且每周拥有近百个\ascii{Issue}被提交;在\ascii{Stack Overflow}上也拥有上万个关于\tf{}的问题被回答;在各类技术大会上,\tf{}也是一颗闪亮的明星,得到众多开发者的青睐。

\end{content}

\subsection{开源}

\begin{content}

\ascii{2015.11},\ascii{Google Research}发布文章:\href{https://research.googleblog.com/2015/11/tensorflow-googles-latest-machine\_9.html}{TensorFlow: Google's latest machine learning system, open sourced for everyone},正式宣布新一代机器学习系统\ascii{TensorFlow}开源。

随后,\ascii{TensorFlow}在\ascii{Github}上代码仓库短时间内获得了大量的\ascii{Star}和\ascii{Fork}。如\refig{tf-commits}所示,\ascii{TensorFlow}的社区活跃度已远远超过其他竞争对手,逐渐成为目前最为炙手可热的机器学习和深度学习框架,已然成为事实上的工业标准。

\begin{figure}[!htbp]
\centering
\includegraphics[width=1.0\textwidth]{figures/tf-commits.png}
\caption{TensorFlow社区活跃度}
 \label{fig:tf-commits}
\end{figure}

毫无疑问,\ascii{TensorFlow}的开源对学术界和工业界产生了巨大的影响,其极大地降低了深度学习在各个行业中应用的难度。众多的学者,工程师,企业,组织纷纷地投入到了\ascii{TensorFlow}社区,并一起完善和改进\ascii{TensorFlow},推动其不断地向前演进和发展。

\end{content}

\subsection{里程碑}

\begin{content}

\tf{}自\ascii{2015.11}开源依赖,平均一个多月发布一个版本。如\refig{tf-versions}所示,展示了\tf{}几个重要特性的发布时间。

\begin{figure}[!htbp]
\centering
\includegraphics[width=1.0\textwidth]{figures/tf-versions.png}
\caption{TensorFlow重要里程碑}
 \label{fig:tf-versions}
\end{figure}

\end{content}

\subsection{工业应用}

\begin{content}

\ascii{TensorFlow}自开源发展一年多以来,在生产环境中被大量应用使用。在医疗方面,使用\ascii{TensorFlow}构建机器学习模型,帮助医生预测皮肤癌;在音乐、绘画领域,使用\ascii{TensorFlow}构建深度学习模型,帮助人类更好地理解艺术;在移动端,多款移动设备搭载\ascii{TensorFlow}训练的机器学习模型,用于翻译等工作。

如\refig{tf-google-apps}所示,\ascii{TensorFlow}在\ascii{Google}内部项目应用的增长也十分迅速,多个产品都有相关应用,包括:\ascii{Search, Gmail, Translate,  Maps}等等。

\begin{figure}[!htbp]
\centering
\includegraphics[width=1.0\textwidth]{figures/tf-google-apps.png}
\caption{TensorFlow在Google内部使用情况}
 \label{fig:tf-google-apps}
\end{figure}


\end{content}
\begin{savequote}[45mm]
\ascii{Any fool can write code that a computer can understand. Good programmers write code that humans can understand.}
\qauthor{\ascii{- Martin Flower}}
\end{savequote}

\chapter{编程环境} 
\label{ch:prog-env}

\begin{content}

为了实现\tf{}的快速入门,本章将介绍\tf{}的编程环境,包括代码结构,工程构建,以便对\tf{}的系统架构建立基本的感性认识。

\end{content}

\section{代码结构}

\begin{content}

\subsection{克隆源码}

首先,从\ascii{Github}上克隆\tf{}的源代码。


\begin{leftbar}
\begin{python}
$ git clone git@github.com:tensorflow/tensorflow.git
\end{python}
\end{leftbar}

然后,切换到最新的稳定分支上。例如,\code{r1.4}分支。

\begin{leftbar}
\begin{python}
$ cd tensorflow
$ git checkout r1.4
\end{python}
\end{leftbar}

\subsection{源码结构}

运行如下命令,打印出\tf{}源码的组织结构。

\begin{leftbar}
\begin{python}[]
$ tree -d -L 1 ./tensorflow
\end{python}
\end{leftbar}

其中,本书将重点关注\code{core, python}组件,部分涉及\code{c, cc, stream\_executor}组件。

\begin{leftbar}
\begin{c++}[caption={TensorFlow源码结构}]
./tensorflow
├── c
├── cc
├── compiler
├── contrib
├── core
├── docs_src
├── examples
├── g3doc
├── go
├── java
├── python
├── stream_executor
├── tools
└── user_ops
\end{c++}
\end{leftbar}

截止当前最新发布的\ascii{1.4}版本,\tf{}代码库拥有大约\ascii{100}万代码。其中,包括\ascii{53}万行\ascii{C/C++}代码,\ascii{37}万行\ascii{Python}代码,而且代码规模在不断膨胀之中。其中,\ascii{Python}提供的\ascii{API}是最完善的;相比之下,其他编程语言的\ascii{API}尚未成熟,甚至处于起步阶段。

\begin{leftbar}
\begin{python}[caption={TensorFlow代码统计}]
-------------------------------------------------------
Language             files    blank    comment    code
-------------------------------------------------------
C++                   2238    77610     68275    443099
Python                1881    92018    151807    369399
C/C++ Header          1175    27392     46215     86691
Markdown               218     8859         2     30925
CMake                   50     2183       986     16398
Go                      28     1779     13290     15003
Java                    72     1789      3111      7779
Bourne Shell           103     1487      3105      6074
Protocol Buffers        87     1313      3339      3452
Objective C++            9      227       181      1201
C                        8      157       130       941
make                     4      105       136       612
XML                     25      135       265       315
Groovy                   3       46        74       246
Maven                    5       21         4       245
DOS Batch                9       30         0       143
Dockerfile               7       41        69       133
Perl                     2       29        38       130
Bourne Again Shell       3       24        63       111
JSON                     3        0         0        23
Objective C              1       10        13        21
YAML                     1        3        24        15
-------------------------------------------------------
SUM:                  5932   215258    291127    982956
-------------------------------------------------------
\end{python}
\end{leftbar}

\subsection{Core}

内核的源码结构如下所示,主要包括平台,实用函数库,基础框架,\ascii{Protobuf}定义,本地运行时,分布式运行时,图操作,\ascii{OP}定义,以及\ascii{Kernel}实现等组成,这是本书重点剖析的组件之一,将重点挖掘基础框架中隐藏的领域模型,追踪整个运行时环境的生命周期和图操作的详细过程,并揭示常见\ascii{OP}的\ascii{Kernel}实现原理和运行机制。

\begin{leftbar}
\begin{c++}[caption={Core源码结构}]
./tensorflow/core
├── common_runtime
├── debug
├── distributed_runtime
├── example
├── framework
├── graph
├── grappler
├── kernels
├── lib
├── ops
├── platform
├── profiler
├── protobuf
├── public
├── user_ops
└── util
\end{c++}
\end{leftbar}

其中,\code{core}主要由\code{C++}实现,大约拥有\ascii{26}万行代码。

\begin{leftbar}
\begin{python}[caption={Core代码统计}]
-------------------------------------------------------
Language             files    blank   comment      code
-------------------------------------------------------
C++                   1368    44791     38968    259289
C/C++ Header           653    15040     24474     50506
Protocol Buffers        57      736      2371      1806
Markdown                11      327         0      1285
JSON                     2        0         0        18
-------------------------------------------------------
SUM:                  2091    60894     65813    312904
-------------------------------------------------------
\end{python}
\end{leftbar}

\subsection{Python}

\ascii{Python}定义和实现了\tf{}的编程模型,并对外开放\ascii{API}供程序员使用,其源码结构如下所示,这也是本书重点剖析的部分。

\begin{leftbar}
\begin{c++}[caption={Python源码结构}]
./tensorflow/python
├── client
├── debug
├── estimator
├── feature_column
├── framework
├── grappler
├── kernel_tests
├── layers
├── lib
├── ops
├── platform
├── profiler
├── saved_model
├── summary
├── tools
├── training
├── user_ops
└── util
\end{c++}
\end{leftbar}

其中,该组件由\code{Python}实现,大约有\ascii{18}万行代码。

\begin{leftbar}
\begin{python}[caption={Python代码统计}]
-------------------------------------------------------
Language            files     blank   comment      code
-------------------------------------------------------
Python                714     45769     69407    179565
C++                    20       496       506      3658
C/C++ Header           15       207       387       363
Markdown                4        48         0       200
Protocol Buffers        3        16        10        71
Bourne Shell            1        13        28        68
-------------------------------------------------------
SUM:                  757     46549     70338    183925
-------------------------------------------------------
\end{python}
\end{leftbar}

\subsection{Contrib}

\code{contrib}是第三方贡献的编程库,它也是\tf{}标准化之前的实验性编程接口,犹如\ascii{Boost}社区与\ascii{C++}标准之间的关系。当\code{contrib}的接口成熟后,便会被\tf{}标准化,并从\code{contrib}中搬迁至\code{core, python}中,并正式对外发布。

\begin{leftbar}
\begin{python}[caption={Contrib源码结构}]
./tensorflow/contrib
├── android
├── batching
├── bayesflow
├── benchmark_tools
├── boosted_trees
├── cloud
├── cluster_resolver
├── cmake
├── compiler
├── copy_graph
├── crf
├── cudnn_rnn
├── data
├── decision_trees
├── deprecated
├── distributions
├── eager
├── factorization
├── ffmpeg
├── framework
├── fused_conv
├── gdr
├── graph_editor
├── grid_rnn
├── hooks
├── hvx
├── image
├── imperative
├── input_pipeline
├── integrate
├── keras
├── kernel_methods
├── labeled_tensor
├── layers
├── learn
├── legacy_seq2seq
├── linalg
├── linear_optimizer
├── lookup
├── losses
├── makefile
├── memory_stats
├── meta_graph_transform
├── metrics
├── mpi
├── nccl
├── ndlstm
├── nearest_neighbor
├── nn
├── opt
├── pi_examples
├── predictor
├── quantization
├── reduce_slice_ops
├── remote_fused_graph
├── resampler
├── rnn
├── saved_model
├── seq2seq
├── session_bundle
├── signal
├── slim
├── solvers
├── sparsemax
├── specs
├── staging
├── stat_summarizer
├── stateless
├── tensor_forest
├── tensorboard
├── testing
├── text
├── tfprof
├── timeseries
├── tpu
├── training
├── util
├── verbs
└── xla_tf_graph
\end{python}
\end{leftbar}

由于\tf{}社区相当活跃,\code{contrib}的变更相当频繁,截止\ascii{1.4}版本,大约有\ascii{23}万行代码,主要由\ascii{Python}设计和实现的编程接口,部分运行时由\ascii{C++}实现。

\begin{leftbar}
\begin{python}[caption={Contrib代码统计}]
-------------------------------------------------------
Language            files     blank   comment      code
-------------------------------------------------------
Python               1007     41436     75096    170355
C++                   201      5500      5313     32944
CMake                  48      2172       955     16358
C/C++ Header           99      1875      2867      6583
Markdown               46      1108         0      3485
Bourne Shell           18       232       386      1272
C                       7       151       118       931
Protocol Buffers       20       224       454       680
make                    4       105       136       612
Java                    2        77       209       335
Groovy                  1        10        20        75
Bourne Again Shell      1         6        15        59
Dockerfile              1         2         1        14
XML                     2         3         0         9
-------------------------------------------------------
SUM:                 1457     52901     85570    233712
-------------------------------------------------------
\end{python}
\end{leftbar}

\subsection{StreamExecutor}

\ascii{StreamExecutor}是\ascii{Google}另一个开源组件库,它提供了主机端(\ascii{host-side})的编程模型和运行时环境,实现了\ascii{CUDA}和\ascii{OpenCL}的统一封装。使得在主机端的代码中,可以将\ascii{Kernel}函数无缝地部署在\code{CUDA}或\code{OpenCL}的计算设备上执行。

目前,\ascii{StreamExecutor}被大量应用于\ascii{Google}内部\ascii{GPGPU}应用程序的运行时。其中,\tf{}运行时也包含了一个\ascii{StreamExecutor}的快照版本,用于封装\ascii{CUDA}和\code{OpenCL}的运行时。本书将简单介绍\ascii{CUDA}的编程模型和线程模型,并详细介绍\ascii{StreamExecutor}的系统架构与工作原理,揭示\ascii{Kernel}函数的实现模式和习惯用法。

\begin{leftbar}
\begin{c++}[caption={StreamExecutor源码结构}]
./tensorflow/stream_executor
├── cuda
├── host
├── lib
└── platform
\end{c++}
\end{leftbar}

其中,\ascii{StreamExecutor}使用\code{C++}实现,大约有\ascii{2.5}万行代码。

\begin{leftbar}
\begin{python}[caption={StreamExecutor代码统计}]
-------------------------------------------------------
Language            files     blank   comment      code
-------------------------------------------------------
C++                    43      2440      1196     16577
C/C++ Header           81      2322      5080      8625
-------------------------------------------------------
SUM:                  124      4762      6276     25202
-------------------------------------------------------
\end{python}
\end{leftbar}

\subsection{Compiler}

众所周知,灵活性是\tf{}最重要的设计目标和核心优势,因此\tf{}的系统架构具有良好的可扩展性。\tf{}可用于定义任意图结构,并使用异构的计算设备有效地执行。但是,熊掌与鱼翅不可兼得,当低级\ascii{OP}组合为计算子图时,并期望在\ascii{GPU}上有效执行时,运行时将启动更多的\ascii{Kernel}的运算。

因此,\tf{}分解和组合\ascii{OP}的方法,在运行时并不能保证以最有效的方式运行。此时,\ascii{XLA}技术孕育而生,它使用\ascii{JIT}编译技术来分析运行时的计算图,它将多个\ascii{OP}融合在一起,并生成更高效的本地机器代码,提升计算图的执行效率。

\begin{leftbar}
\begin{python}[caption={Compiler源码结构}]
./tensorflow/compiler
├── aot
├── jit
├── plugin
├── tests
├── tf2xla
└── xla
\end{python}
\end{leftbar}

\ascii{XLA}技术目前处于初级的研发阶段,是\tf{}社区较为活跃研究方向,截止目前代码规模大约为\ascii{12.5}万行,主要使用\ascii{C++}实现。

\begin{leftbar}
\begin{python}[caption={Compiler代码统计}]
-------------------------------------------------------
Language            files     blank   comment      code
-------------------------------------------------------
C++                   455     19010     18334    102537
C/C++ Header          250      5939     10323     15510
Python                 37      1255      1416      6446
Protocol Buffers        5       312       501       781
Markdown                2         0         0         3
-------------------------------------------------------
SUM:                  749     26516     30574    125277
-------------------------------------------------------
\end{python}
\end{leftbar}

\end{content}

\section{工程构建}

\begin{content}

在开始之前,尝试\tf{}源码的构建过程,了解\tf{}的基本构建方式、工具,及其依赖的组件库、第三方工具包,对于理解\tf{}工作原理具有很大的帮助。但是,因篇幅受限,本章仅以\ascii{Mac OS}系统为例,讲述\tf{}的源码编译、安装、及其验证过程。其他操作系统,请查阅\tf{}发布的官方文档。

\subsection{环境准备}

\ascii{TensorFlow}的前端是一个支持多语言的编程接口。因此,编译\ascii{TensorFlow}源代码之前,需要事先安装相关的编译器、解释器、及其运行时环境。例如,使用\ascii{Python}作为编程接口,需要事先安装\ascii{Python}解释器。其次,在构建系统之前,也需要事先安装\ascii{GCC}或\ascii{Clang}等\ascii{C++}编译器,用于编译后端系统实现。因为\ascii{TensorFlow}使用\ascii{C++11}语法实现,所以要保证安装\ascii{C++}编译器要支持\ascii{C++11}。另外,\ascii{TensorFlow}使用\ascii{Bazel}的构建工具,可以将其视为更高抽象的\ascii{Make}工具。不幸的是,\ascii{Bazel}使用\ascii{Java8}实现,其依赖于\ascii{JDK}。因此在安装\ascii{Bazel}之前,还得需要事先安装\ascii{1.8}及以上版本的\ascii{JDK}。

\subsubsection{安装JDK}

推荐从\ascii{Oracle}官网上下载\ascii{1.8}版本的\ascii{JDK},然后创建相关的环境变量,并将其添加到\code{~/.bashrc}的配置文件中。

\begin{leftbar}
\begin{python}
$ echo 'export JAVA_HOME=$(/usr/libexec/java_home)' >> ~/.bashrc
$ echo 'export PATH="$JAVA_HOME/bin:$PATH"' >> ~/.bashrc
\end{python}
\end{leftbar}

\subsubsection{安装Bazel}

在\ascii{Mac OS}上,可以使用\ascii{brew}安装\ascii{Bazel}。

\begin{leftbar}
\begin{python}
$ brew install bazel
\end{python}
\end{leftbar}

如果系统未安装\ascii{brew},可以执行如下命令先安装\ascii{brew}。当然,安装\ascii{brew}需要事先安装\ascii{Ruby}解释器,在此不再冗述。

\begin{leftbar}
\begin{python}
$ ruby -e "$(curl -fsSL https://raw.githubusercontent.com/Homebrew/install/master/install)"
\end{python}
\end{leftbar}

\subsubsection{安装Swig}

\ascii{TensorFlow}使用\ascii{Swig}构建多语言的编程环境,自动生成相关编程语言的包装器。因此,在构建之前需要事先安装\ascii{Swig}的工具包。

\begin{leftbar}
\begin{python}
$ brew install swig
\end{python}
\end{leftbar}

\subsubsection{安装Python依赖包}

使用\ascii{pip}安装\ascii{TensorFlow}所依赖的\ascii{Python}包。

\begin{leftbar}
\begin{python}
$ sudo pip install six numpy wheel autograd
\end{python}
\end{leftbar}

如果系统未安装\ascii{pip},则可以使用\ascii{brew}预先安装\ascii{pip}。

\begin{leftbar}
\begin{python}
$ brew install pip
\end{python}
\end{leftbar}

\subsubsection{安装CUDA工具包}

当系统安装了\ascii{CUDA}计算兼容性大于或等于\ascii{3.0}的\ascii{GPU}卡时,则需要安装\ascii{CUDA}工具包,及其\ascii{cuDNN},实现\tf{}运行时的\ascii{GPU}加速。推荐从\ascii{NVIDIA}官网上下载\ascii{CUDA Toolkit 8}及以上版本,并安装到系统中,配置相关环境变量。

\begin{leftbar}
\begin{python}
$ echo 'export CUDA_HOME=/usr/local/cuda' >> ~/.bashrc
$ echo 'export LD_LIBRARY_PATH=$CUDA_HOME/lib:$LD_LIBRARY_PATH' >> ~/.bashrc
\end{python}
\end{leftbar}

然后,再下载\ascii{cuDNN 5.1}及以上版本,并将其解压至\code{CUDA\_HOME}相关系统目录中。

\begin{leftbar}
\begin{python}
$ sudo tar -xvf cudnn-8.0-macos-x64-v5.1.tgz -C /usr/local
\end{python}
\end{leftbar}

\subsection{配置}

至此,编译环境一切就绪,执行\code{./configure}配置\ascii{TensorFlow}的编译环境了。当系统支持\ascii{GPU},则需要配置相关的\ascii{CUDA/cuDNN}编译环境。

\begin{leftbar}
\begin{python}
$ ./configure
\end{python}
\end{leftbar}

\subsection{构建}

当配置成功后,使用\ascii{Bazel}启动\ascii{TensorFlow}的编译。在编译启动之前,会尝试从代码仓库中下载相关依赖库的源代码,包括\ascii{gRPC, Protobuf, Eigen}等,并自动完成编译。

\begin{leftbar}
\begin{python}
$ bazel build --config=opt //tensorflow/tools/pip_package:build_pip_package
\end{python}
\end{leftbar}

当支持\ascii{GPU}计算时,添加\code{--config=cuda}编译选项。

\begin{leftbar}
\begin{python}
$ bazel build -c opt --config=cuda //tensorflow/tools/pip_package:build_pip_package
\end{python}
\end{leftbar}

编译成功后,便可以构建\ascii{TensorFlow}的\ascii{Wheel}包。

\begin{leftbar}
\begin{python}
$ bazel-bin/tensorflow/tools/pip_package/build_pip_package /tmp/tensorflow_pkg
\end{python}
\end{leftbar}

\subsection{安装}

当\ascii{Whell}包构建成功后,使用\ascii{pip}安装\ascii{TensorFlow}到系统中。

\begin{leftbar}
\begin{python}
$ sudo pip install /tmp/tensorflow_pkg/tensorflow-1.4.0-py2-none-any.whl
\end{python}
\end{leftbar}

\subsection{验证}

启动\ascii{Python}解释器,验证\ascii{TensorFlow}安装是否成功。

\begin{leftbar}
\begin{python}
$ python
>>> import tensorflow as tf
>>> hello = tf.constant('Hello, TensorFlow!')
>>> sess = tf.Session()
>>> print(sess.run(hello))
Hello, TensorFlow!
\end{python}
\end{leftbar}

\subsection{IDE}

在阅读代码之前,选择一个适宜的\ascii{IDE}可以改善代码阅读的质量和速度。推荐使用\ascii{Eclipse CDT}阅读\ascii{C++}代码,安装\code{PyDev}的插件阅读\ascii{Python}代码。同时,也推荐\ascii{JetBrains}出品的\ascii{Clion}阅读\ascii{C++},\ascii{PyCharm}阅读\ascii{Python}。但是,当阅读\ascii{C++}代码时,需要配置\ascii{TensorFlow, CUDA, Eigen3}头文件的搜索目录,并添加相关预定义的宏,以便\code{IDE}正确解析代码中的符号。本章以\ascii{Eclipse CDT}为例讲述相关配置方法。

\subsubsection{创建Eclipse工程}

创建一个\ascii{Eclipse C++}工程,如\refig{setup-eclipse}所示。确定唯一的项目名称,手动地指定\ascii{TensorFlow}源代码的根目录,并选择\ascii{Makefile}的空工程。然后,按照\ascii{Properties > C/C++ General > Paths and Symbols > Includes}配置头文件的搜索目录。

\begin{table}[!htbl]
\resizebox{0.95\textwidth}{!} {
\begin{tabular*}{1.2\textwidth}{@{}ll@{}}
\toprule
\ascii{配置项} & \ascii{目录} \\
\midrule
\ascii{TensorFlow} & \code{/usr/local/lib/python2.7/site-packages/tensorflow/include} \\
\ascii{CUDA} & \code{/usr/local/cuda/include} \\ 
\bottomrule
\end{tabular*}
}
\caption{头文件搜索目录}
\label{tbl:tf-includes}
\end{table}

\begin{figure}[!htbl]
\centering
\includegraphics[width=0.75\textwidth]{figures/setup-eclipse.png}
\caption{创建Eclipse C++工程}}
 \label{fig:setup-eclipse}
\end{figure}

\subsubsection{配置Eigen}

不幸的是,\ascii{Eigen}对外公开的头文件缺少\code{.h}的后缀名,\ascii{CDT}无法解析相关的符号。请参阅\code{\href{http://eigen.tuxfamily.org/index.php?title=IDEs}{http://eigen.tuxfamily.org/index.php?title=IDEs}}相关说明,按照\ascii{Preferences > C/C++ > Coding Style > Organize Includes > Header Substitution}导入\code{eigen-header-substitution.xml}文件,如\refig{eclipse-eigen3}所示。

\begin{figure}[!htbl]
\centering
\includegraphics[width=0.75\textwidth]{figures/eclipse-eigen3.png}
\caption{替换\ascii{Eigen}的头文件}
 \label{fig:eclipse-eigen3}
\end{figure}

\end{content}

\section{代码生成}

\begin{content}

在构建\ascii{TensorFlow}系统时,\ascii{Bazel}或\ascii{CMake}会自动生成部分源代码。理解代码生成器的输出结果,可以加深理解系统的行为模式。

\end{content}

\section{技术栈}

\begin{content}

如\refig{tf-stack}所示,按照系统的层次结构展示了\tf{}的技术栈,构成了\tf{}生态系统的核心。

\begin{figure}[H]
\centering
\includegraphics[width=0.7\textwidth]{figures/tf-stack.png}
\caption{TensorFlow技术栈}}
 \label{fig:tf-stack}
\end{figure}

\end{content}

\begin{savequote}[45mm]
\ascii{Any fool can write code that a computer can understand. Good programmers write code that humans can understand.}
\qauthor{\ascii{- Martin Flower}}
\end{savequote}

\chapter{破冰之旅} 
\label{ch:ice-breaker}

\begin{content}

在开始探究\tf{}内核之前,亲自动手实践模型的训练,熟悉模型训练的基本方法和调优技术,对于理解后续章节将大有裨益。\footnote{ 本章内容摘自\ascii{Martin G\"{o}rner}在\ascii{Codelabs}上发表的文章:\href{https://codelabs.developers.google.com/codelabs/cloud-tensorflow-mnist}{Tensorflow and deep learning, without a PhD},经由\ascii{Martin G\"{o}rner}同意,授权该文章在本书中发表。}

通过本文学习和实践,将了解到如何构建并训练出能够识别手写数字的神经网络。最终,将模型的准确率提升至\ascii{99%}。

\end{content}

\section{问题提出}

\begin{content}

此处使用\ascii{MNIST}数据集,它包含了\ascii{60000}个训练样本数据,及其\ascii{10000}个测试样本数据。

如\refig{mnist-x}所示,对于任意一个样本数据$x$,使用$28 \times 28$像素的矩阵表示。为了简化,将$28 \times 28$的矩阵实施扁平化处理,得到长度为\ascii{784}的一维向量。

\begin{figure}[H]
\centering
\includegraphics[width=0.9\textwidth]{figures/mnist-x.png}
\caption{MNIST样本数据表示}
 \label{fig:mnist-x}
\end{figure}

因此,在\ascii{MNIST}训练数据集中,\code{mnist.train.images}是一个\code{[60000, 784]}的二维矩阵。其中,矩阵中每一个元素,表示图片中某个像素的强度值,其值介于0和1之间。如\refig{mnist-train-xs}所示。

\begin{figure}[H]
\centering
\includegraphics[width=0.9\textwidth]{figures/mnist-train-xs.png}
\caption{MNIST训练数据集:输入数据集}
 \label{fig:mnist-train-xs}
\end{figure}

相对应地,\ascii{MNIST}数据集的标签是介于\ascii{0}到\ascii{9}的数字。此处,样本的标签使用\ascii{\quo{one-hot}}的向量表示。因此,\code{mnist.train.labels}是一个\code{[60000, 10]}的二维矩阵。如\refig{mnist-train-ys}所示。

\begin{figure}[H]
\centering
\includegraphics[width=0.9\textwidth]{figures/mnist-train-ys.png}
\caption{MNIST训练数据集:标签数据集}
 \label{fig:mnist-train-ys}
\end{figure}

\end{content}

\section{图示说明}

\begin{content}

为了更好地可视化整个训练过程,使用了\ascii{5}种面板。如\refig{mnist-training-digits}所示,表示\ascii{batch\_size}为\ascii{100}时,一个批次的训练样本。其中,白色背景表示数字被正确识别;而红色背景表示数字被误分类,手写数字的左侧标识正确的标签值,而右侧表示错误的预测值。

\ascii{MNIST}拥有\ascii{50000}个训练样本,如果\ascii{batch\_size}为\ascii{100},则需要迭代\ascii{500}次将完整地遍历一次训练样本的数据集,常称为一个\ascii{epoch}。

\begin{figure}[H]
\centering
\includegraphics[width=0.6\textwidth]{figures/mnist-training-digits.jpeg}
\caption{一次mini-batch的训练样本数据集,其中batch\_size=100}
 \label{fig:mnist-training-digits}
\end{figure}

如\refig{mnist-test-digits}所示,\ascii{MNIST}使用了规模为\ascii{10000}的测试样本数据集测试模型的当前精度。其中,左侧表示目前模型的大致精度;同样地,白色背景表示数字被正确识别;而红色背景表示数字被误分类,手写数字的左侧标识正确的标签值,而右侧表示错误的预测值。

\begin{figure}[H]
\centering
\includegraphics[width=0.6\textwidth]{figures/mnist-test-digits.jpeg}
\caption{当前的模型精度:基于测试样本数据集}
 \label{fig:mnist-test-digits}
\end{figure}

如\refig{mnist-cross-entropy-loss-fig}所示,使用交叉熵函数量化预测值与标签值之前的误差。其中,\ascii{x}轴表示迭代的次数,\ascii{y}轴表示损失值。

\begin{figure}[H]
\centering
\includegraphics[width=0.6\textwidth]{figures/mnist-cross-entropy-loss-fig.jpeg}
\caption{训练和测试的交叉熵损失}
 \label{fig:mnist-cross-entropy-loss-fig}
\end{figure}

如\refig{mnist-accuracy-fig}所示,可以实时计算得到模型在当前训练数据集和测试集上的精度。其中,\ascii{x}轴表示迭代的次数,\ascii{y}轴表示精度值。

\begin{figure}[H]
\centering
\includegraphics[width=0.6\textwidth]{figures/mnist-accuracy-fig.jpeg}
\caption{训练和测试的精度}
 \label{fig:mnist-accuracy-fig}
\end{figure}

如\refig{mnist-weight-fig.png}所示,对于模型的每一个训练参数(包括偏置),可以统计得到其对应的数值分布图。

\begin{figure}[H]
\centering
\includegraphics[width=0.6\textwidth]{figures/mnist-weight-fig.png}
\caption{权重分布图}
 \label{fig:mnist-weight-fig}
\end{figure}

\end{content}

\section{单层感知器}

\begin{content}

首先,尝试构建\ascii{10}个神经元的单层感知器。对于诸如手写数字识别的多分类问题,常常使用\ascii{softmax}的激活函数。

如\refig{mnist-slp}所示,对于任意一个输出神经元$n$,其线性加权和为$L_n$;然后,对$L_n$实施\ascii{softmax}的激活函数。


\begin{figure}[H]
\centering
\includegraphics[width=0.8\textwidth]{figures/mnist-slp.png}
\caption{单层感知器}
 \label{fig:mnist-slp}
\end{figure}

\subsection{定义模型}

接下来,使用\tf{}完成该模型的搭建和训练。

\subsubsection{输入和标签}

首先,使用\code{tf.placeholder}分别定义一个\ascii{mini-batch}的训练样本数据集。其中,\code{tf.placeholder}定义了一个占位的\ascii{OP};\code{None}表示未确定的样本数目,此处表示\code{batch\_size}的大小;
\code{Session.run}时,将通过\code{feed\_dict}的字典提供一个\ascii{mini-batch}的样本数据集。

\begin{leftbar}
\begin{python}
x = tf.placeholder(tf.float32, [None, 784]) 
t = tf.placeholder(tf.float32, [None, 10])
\end{python}
\end{leftbar}

\subsubsection{定义变量}

然后,使用\code{tf.Variable}定义模型参数。定义训练参数时,必须指定参数的初始化值;训练参数将根据初始化值,自动推演数据的类型,及其维度。

此外,参数在使用之前,必须完成初始化。此处,\code{init\_op}将初始化所有全局的训练参数。

\begin{leftbar}
\begin{python}
w = tf.Variable(tf.zeros([784, 10]))
b = tf.Variable(tf.zeros([10]))

init_op = tf.global_variables_initializer()
\end{python}
\end{leftbar}

\subsubsection{模型定义}

接下来,便可以很容易地得到多分类问题的单层感知器模型。

\begin{leftbar}
\begin{python}
y = tf.nn.softmax(tf.matmul(x, w) + b)
\end{python}
\end{leftbar}

如\refig{mnist-linear-sum}所示,首先计算\code{x}与\code{w}的矩阵乘法,让后将\code{b}广播(\ascii{broadcast})到矩阵的每一行相加,最终得到训练参数的线性加权和。

\begin{figure}[H]
\centering
\includegraphics[width=0.8\textwidth]{figures/mnist-linear-sum.png}
\caption{线性和}
 \label{fig:mnist-linear-sum}
\end{figure}

如\refig{mnist-softmax}所示,\ascii{softmax}将逐行实施运算,最终,\code{y}的大小为\code{[100, 10]}。

\begin{figure}[H]
\centering
\includegraphics[width=0.8\textwidth]{figures/mnist-softmax.png}
\caption{激活函数:softmax}
 \label{fig:mnist-softmax}
\end{figure}

\subsubsection{损失函数}

对于多分类问题,可以使用交叉熵的损失函数。

\begin{leftbar}
\begin{python}
cross_entropy = -tf.reduce_sum(t * tf.log(y))
\end{python}
\end{leftbar}

如\refig{mnist-cross-entropy}所示,\code{t}和\code{y}的大小都为\code{[100, 10]};特殊地,\code{t}的每一行都是一个\quo{\ascii{one-hot}}向量。

对\code{y}实施\code{tf.log}操作,也将得到一个大小为\code{[100, 10]}的矩阵。最终,\code{tf.reduce\_sum}将矩阵中所有元素相加,得到一个标量(\ascii{scalar})值。

\begin{figure}[H]
\centering
\includegraphics[width=0.8\textwidth]{figures/mnist-cross-entropy.png}
\caption{交叉熵损失函数}
 \label{fig:mnist-cross-entropy}
\end{figure}

\subsubsection{精度}

\code{tf.argmax(y,1)}将按第\ascii{1}个维度计算最大值的索引。既按照${{\text{y}}_{{\text{100}} \times {\text{10}}}}$的每一行,计算得到在每一行中最大值的的索引值。因此,\code{tf.argmax(y,1)}将得到大小为\code{[100, 1]}的矩阵,或大小为\ascii{100}的向量。同样地,\code{tf.argmax(t,1)}也是一个大小为\ascii{100}的向量。

然后,使用\code{tf.equal}将它们逐元素(\ascii{element-wise})进行相等性比较,得到大小为\ascii{100}的布尔向量。为了计算精度,先将布尔向量转别为数值向量,最终求取该数值向量的均值。

\begin{leftbar}
\begin{python}
is_correct = tf.equal(tf.argmax(y,1), tf.argmax(t,1))
accuracy = tf.reduce_mean(tf.cast(is_correct, tf.float32))
\end{python}
\end{leftbar}

\subsection{优化算法}

接下来,使用梯度下降算法实现交叉熵损失函数的最小化。其中,\code{learning\_rate}表示学习速率,描述参数更新的快慢和步伐大小,是一个典型的超参。如\refig{mnist-gd}所示。

\begin{leftbar}
\begin{python}
optimizer = tf.train.GradientDescentOptimizer(learning_rate=0.003)
train_step = optimizer.minimize(cross_entropy)
\end{python}
\end{leftbar}

\begin{figure}[H]
\centering
\includegraphics[width=0.8\textwidth]{figures/mnist-gd.jpeg}
\caption{梯度下降算法}
 \label{fig:mnist-gd}
\end{figure}

\subsection{训练模型}

在此之前,\tf{}仅构造计算图,并没有启动计算图的执行。接下来,客户端创建一个会话,建立与本地或远端计算设备集的通道,启动计算图的执行过程。

首先,完成训练参数的初始化。通过运行模型参数的初始化子图,并发地执行各个训练参数的初始化器,将初始值就地修改到相应的训练参数内。

\begin{leftbar}
\begin{python}
with tf.Session() as sess:
  sess.run(init_op)
\end{python}
\end{leftbar}

然后,开始迭代地执行\code{train\_op},完成模型的训练。其中,每\ascii{100}次迭代,计算当前模型在训练数据集及测试数据集的精度和损失。

\begin{leftbar}
\begin{python}
with tf.Session() as sess:
  for step in range(1000):
    batch_xs, batch_ys = mnist.train.next_batch(100)        
    sess.run(train_step, feed_dict={x: batch_xs, t: batch_ys})
    
    if step % 100 == 0:
      acc, loss = sess.run([accuracy, cross_entropy], 
        feed_dict={x: batch_xs, t: batch_ys})

      acc, loss = sess.run([accuracy, cross_entropy], 
        feed_dict={x: mnist.test.images, t: mnist.test.labels}) 
\end{python}
\end{leftbar}

通过\ascii{1000}次迭代,可得到大约\percent{92}的精度。

\begin{figure}[H]
\centering
\includegraphics[width=0.8\textwidth]{figures/mnist-slp-accuracy.png}
\caption{可视化:单层感知器,运行1000次step}
 \label{fig:mnist-slp-accuracy}
\end{figure}

\end{content}

\section{多层感知器}

\begin{content}

为了进一步提高精度,接下来尝试搭建\ascii{5}层的多层感知器模型。

\begin{figure}[H]
\centering
\includegraphics[width=0.8\textwidth]{figures/mnist-5-layer.png}
\caption{5层感知器}
 \label{fig:mnist-5-layer}
\end{figure}

\subsection{定义模型}

相对于上一节中尝试的单层感知器,此处定义每一个隐式层的权重时,并没有使用常量表示的初始值,而使用满足某种数据分布特征的随机值。

\begin{leftbar}
\begin{python}
K = 200
L = 100
M = 60
N = 30

w1 = tf.Variable(tf.truncated_normal([28*28, K] ,stddev=0.1)) 
b1 = tf.Variable(tf.zeros([K]))

w2 = tf.Variable(tf.truncated_normal([K, L], stddev=0.1))
b2 = tf.Variable(tf.zeros([L]))

w3 = tf.Variable(tf.truncated_normal([L, M], stddev=0.1)) 
b3 = tf.Variable(tf.zeros([M]))

w4 = tf.Variable(tf.truncated_normal([M, N], stddev=0.1)) 
b4 = tf.Variable(tf.zeros([N]))

w5 = tf.Variable(tf.truncated_normal([N, 10], stddev=0.1)) 
b5 = tf.Variable(tf.zeros([10]))
\end{python}
\end{leftbar}

在定义每一个隐式层时,采用\ascii{sigmoid}的激活函数。最最后的输出层,采用\ascii{softmax}的激活函数。

\begin{leftbar}
\begin{python}
y1 = tf.nn.sigmoid(tf.matmul(x,  w1) + b1)
y2 = tf.nn.sigmoid(tf.matmul(y1, w2) + b2)
y3 = tf.nn.sigmoid(tf.matmul(y2, w3) + b3)
y4 = tf.nn.sigmoid(tf.matmul(y3, w4) + b4)
y  = tf.nn.softmax(tf.matmul(y4, w5) + b5)
\end{python}
\end{leftbar}

\subsection{训练模型}

\end{content}

\section{优化技术}

\begin{content}

\subsection{激活函数:RELU}

在深度模型中,不适合使用\ascii{sigmoid}激活函数。它将把所有的值都挤到了\ascii{0}到\ascii{1}之间;随着网络层次的增加,导致梯度消失。

\begin{figure}[H]
\centering
\includegraphics[width=0.8\textwidth]{figures/mnist-relu.png}
\caption{ReLU激活函数}
 \label{fig:mnist-relu}
\end{figure}

可以使用\ascii{ReLU}激活函数,不仅避免了\ascii{sigmoid}导致的一些问题,而且能够加快更快的初始的收敛速度。如\refig{mnist-sigmoid-to-relu}所示,前300次迭代,使用\ascii{ReLU}相对于使用\ascii{sigmoid},其初始收敛速度提升显著。

\begin{figure}[H]
\centering
\includegraphics[width=0.8\textwidth]{figures/mnist-sigmoid-to-relu.png}
\caption{应用ReLU激活函数:初始收敛速度提升显著}
 \label{fig:mnist-sigmoid-to-relu}
\end{figure}

\subsection{学习速率衰减}

随着网路层次的增加,及其应用相关优化技术后,模型的精度能够得到\percent{98}左右,但很难得到一个稳定的精度。如\refig{mnist-lr-too-larger}所示,精度和损失抖动相当明显。

\begin{figure}[H]
\centering
\includegraphics[width=0.8\textwidth]{figures/mnist-lr-too-larger.png}
\caption{噪声抖动:学习速率过大}
 \label{fig:mnist-lr-too-larger}
\end{figure}

可以采用更好的优化算法,例如\code{AdamOptimizer},随着迭代过程的次数,学习速率指数衰减,可以得到一个更稳定的精度。如\refig{mnist-apply-learning-rate-decay}所示。

\begin{figure}[H]
\centering
\includegraphics[width=0.8\textwidth]{figures/mnist-apply-learning-rate-decay.png}
\caption{应用Adam优化算法后,精度趋于稳定}
 \label{fig:mnist-apply-learning-rate-decay.png}
\end{figure}

\subsection{应用Dropout}

但是,训练集与测试集上的损失曲线出现分离,出现明显的过拟合现象。

\begin{figure}[H]
\centering
\includegraphics[width=0.8\textwidth]{figures/mnist-overfitting.png}
\caption{过拟合}
 \label{fig:mnist-overfitting}
\end{figure}

如\refig{mnist-dropout}所示,可以在训练时,对每一层的输出实施\ascii{dropout}操作,而在推理时丢弃\ascii{dropout},减低过拟合的问题。

\begin{figure}[H]
\centering
\includegraphics[width=0.8\textwidth]{figures/mnist-dropout.png}
\caption{Dropout方法}
 \label{fig:mnist-dropout}
\end{figure}


\begin{figure}[H]
\centering
\includegraphics[width=0.8\textwidth]{figures/mnist-apply-dropout-result.png}
\caption{实施Dropout后,训练集与测试集的损失曲线再次重合}
 \label{fig:mnist-apply-dropout-result}
\end{figure}

\subsection{过拟合}

\end{content}


\part{系统架构}
\begin{savequote}[45mm]
\ascii{Any fool can write code that a computer can understand. Good programmers write code that humans can understand.}
\qauthor{\ascii{- Martin Flower}}
\end{savequote}

\chapter{系统架构} 
\label{ch:architecture}

\begin{content}

本章将阐述\tf{}的系统架构,并一个简单的例子,讲述图结构的变换过程;最后,通过挖掘会话管理的工作机制,加深理解\tf{}运行时的工作机理。

\end{content}

\section{系统架构}
	
\begin{content}

\tf{}的系统结构以\emph{\ascii{C API}}为界,\footnote{事实上,后端系统中也存在\ascii{Client}的代码,并常称它为前端系统在后端系统实现中的代理\ascii{Client}。在后面的章节,将详细地讨论这个问题。}将整个系统分为\emph{前端}和\emph{后端}两个子系统:

\begin{enum}
  \eitem{前端系统:提供编程模型,负责构造计算图;}
  \eitem{后端系统:提供运行时环境,负责执行计算图。} 
\end{enum}

\begin{figure}[!htbp]
\centering
\includegraphics[width=0.9\textwidth]{figures/tf-architecture.png}
\caption{TensorFlow系统架构}
 \label{fig:tf-architecture}
\end{figure}

如\refig{tf-architecture}所示,重点关注系统中如下\ascii{4}个基本组件,它们是系统分布式运行时的核心。

\subsection{Client}

\ascii{Client}是前端系统的主要组成部分,它是一个支持多语言的编程环境。\ascii{Client}基于\ascii{TensorFlow}的编程接口,构造计算图。

目前,\ascii{TensorFlow}支持\ascii{Python}和\ascii{C++}的编程接口较为完善,尤其对\ascii{Python}的\ascii{API}支持最为全面。并且,对其他编程语言的\ascii{API}支持日益完善。

此时,\ascii{TensorFlow}并未执行任何的图计算,直至与后台计算引擎建立\ascii{Session},并以\ascii{Session}为桥梁,建立\ascii{Client}与\ascii{Master}之间的通道,将\ascii{Protobuf}格式的\ascii{GraphDef}序列化后发送至\ascii{Master},启动计算图的执行过程。

\subsection{Master}

在分布式的运行时环境中,\ascii{Client}根据\code{Session.run}传递整个计算图给后端的\ascii{Master};此时,计算图是完整的,常称为\emph{\ascii{Full Graph}}。

随后,\ascii{Master}根据\ascii{Client}通过\code{Session.run}传递\code{fetches}参数列表,反向遍历\ascii{Full Graph},并按照依赖关系,对其实施剪枝,最终计算得到最小的依赖子图,常称为\ascii{Client Graph}。

随后,\ascii{Master}负责将\ascii{Client Graph}按照任务的名称分裂\ascii{(split-by-task)}为多个子图片段,常称为\ascii{(Graph Partition)};其中,每个\ascii{Worker}对应一个\ascii{Graph Partition}。

随后,\ascii{Master}将\ascii{Graph Partition}分别注册到相应的\ascii{Worker}上,以便在不同的\ascii{Worker}上并发执行这些子图片段。

最后,\ascii{Master}将通知所有\ascii{Work}启动相应子图片段的执行;其中,\ascii{Work}之间可能存在数据交互,\ascii{Master}不参与两者之间的数据交换,它们两两互相通信,独立地完成交换数据,直至完成所有计算。

\subsection{Worker}

对于每一个任务,\tf{}都将启动一个\ascii{Worker}实例。\ascii{Worker}主要负责如下\ascii{3}个方面的职责:

\begin{enum}
  \eitem{处理来自\ascii{Master}的请求;}
  \eitem{按照拓扑排序算法执行本地子图,并调度\ascii{OP}的\ascii{Kernel}实现;} 
  \eitem{协同任务之间的数据通信。}
\end{enum}

首先,\ascii{Worker}收到\ascii{Master}发送过来的图执行命令,此时的计算图相对于\ascii{Worker}是完整的,也称为\ascii{Full Graph},它对应于\ascii{Master}的一个\ascii{Graph Partition}。随后,\ascii{Worker}也会执行图剪枝,得到最小依赖的\ascii{Client Graph}。

随后,\ascii{Worker}根据当前可用的硬件环境,包括\ascii{(GPU/CPU)}资源,按照\ascii{OP}设备的约束规范,再将\ascii{Client Graph}分裂\ascii{(split-by-device)}为多个\ascii{Graph Partition};其中,每个计算设备对应一个\ascii{Graph Partition};随后,\ascii{Worker}启动所有的\ascii{Graph Partition}的执行。

最后,对于每一个计算设备,\ascii{Worker}将按照计算图中节点之间的依赖关系执行拓扑排序算法,并依次调用\ascii{OP}的\ascii{Kernel}实现,完成\ascii{OP}的运算(一种典型的多态实现技术)。其中,\ascii{Worker}还要负责将\ascii{OP}运算的结果发送到其他的\ascii{Worker}上去;或者接受来自其他\ascii{Worker}发送给它的运算结果,以便实现\ascii{Worker}之间的数据交互。

\subsection{Kernel}

\ascii{Kernel}是\ascii{OP}在某种硬件设备的特定实现,它负责执行\ascii{OP}的具体运算。目前,\ascii{TensorFlow}系统中包含\ascii{200}多个标准的\ascii{OP},包括数值计算,多维数组操作,控制流,状态管理等。

一般每一个\ascii{OP}根据设备类型都会存在一个优化了的\ascii{Kernel}实现。在运行时,运行时根据\ascii{OP}的设备约束规范,及其本地设备的类型,为\ascii{OP}选择特定的\ascii{Kernel}实现,完成该\ascii{OP}的计算。

其中,大多数\ascii{Kernel}基于\code{Eigen::Tensor}实现。\code{Eigen::Tensor}是一个使用\ascii{C++}模板技术,为多核\ascii{CPU/GPU}生成高效的并发代码。但是,\ascii{TensorFlow}也可以灵活地直接使用\ascii{cuDNN}实现更高效的\ascii{Kernel}。

此外,\ascii{TensorFlow}实现了矢量化技术,在高吞吐量、以数据为中心的应用需求中,及其移动设备中,实现更高效的推理。如果对于复合\ascii{OP}的子计算过程很难表示,或执行效率低下,\ascii{TensorFlow}甚至支持更高效的\ascii{Kernel}注册,其扩展性表现非常优越。

\end{content}

\section{图控制}

\begin{content}

通过一个最简单的例子,进一步抽丝剥茧,逐渐挖掘出\tf{}计算图的控制与运行机制。

\subsection{组建集群}

如\refig{tf-1ps-1worker}所示。假如存在一个简单的分布式环境:\ascii{1 PS + 1 Worker},并将其划分为两个任务:

\begin{enum}
  \eitem{\ascii{ps0}: 使用\code{/job:ps/task:0}标记,负责模型参数的存储和更新;}
  \eitem{\ascii{worker0}: \code{/job:worker/task:0}标记,负责模型的训练。} 
\end{enum}

\begin{figure}[!htbp]
\centering
\includegraphics[width=0.9\textwidth]{figures/tf-1ps-1worker.png}
\caption{TensorFlow集群:\ascii{1 PS + 1 Worker}}
 \label{fig:tf-1ps-1worker}
\end{figure}

\subsection{图构造}

如\refig{tf-graph-construction}所示。\ascii{Client}构建了一个简单计算图;首先,将$w$与$x$进行矩阵相乘,再与截距$b$按位相加,最后更新至$s$中。

\begin{figure}[!htbp]
\centering
\includegraphics[width=0.9\textwidth]{figures/tf-graph-construction.png}
\caption{图构造}}
 \label{fig:tf-graph-construction}
\end{figure}

\subsection{图执行}

如\refig{tf-graph-execution}所示。首先,\ascii{Client}创建一个\code{Session}实例,建立与\ascii{Master}之间的通道;接着,\ascii{Client}通过调用\code{Session.run}将计算图传递给\ascii{Master}。

随后,\ascii{Master}便开始启动一次\ascii{Step}的图计算过程。在执行之前,\ascii{Master}会实施一系列优化技术,例如\emph{公共表达式消除},\emph{常量折叠}等。最后,\ascii{Master}负责任务之间的协同,执行优化后的计算图。

\begin{figure}[!htbp]
\centering
\includegraphics[width=0.9\textwidth]{figures/tf-graph-execution.png}
\caption{图执行}}
 \label{fig:tf-graph-execution}
\end{figure}

\subsubsection{图分裂}

如\refig{tf-graph-split-by-task}所示,存在一种合理的图划分算法。\ascii{Master}将模型参数相关的\ascii{OP}划分为一组,并放置在\ascii{ps0}任务上;其他\ascii{OP}划分为另外一组,放置在\ascii{worker0}任务上执行。

\begin{figure}[!htbp]
\centering
\includegraphics[width=1.0\textwidth]{figures/tf-graph-split-by-task.png}
\caption{图分裂:按任务划分}}
 \label{fig:tf-graph-split-by-task}
\end{figure}

\subsubsection{子图注册}

如\refig{tf-register-graph}所示。在图分裂过程中,如果计算图的边跨越节点或设备,\ascii{Master}将该边实施分裂,在两个节点或设备之间插入\ascii{Send}和\ascii{Recv}节点,实现数据的传递。

其中,\code{Send}和\code{Recv}节点也是\ascii{OP},只不过它们是两个特殊的\ascii{OP},由内部运行时管理和控制,对用户不可见;并且,它们仅用于数据的通信,并没有任何数据计算的逻辑。

最后,\ascii{Master}通过调用\code{RegisterGraph}接口,将子图注册给相应的\ascii{Worker}上,并由相应的\ascii{Worker}负责执行运算。

\begin{figure}[!htbp]
\centering
\includegraphics[width=1.0\textwidth]{figures/tf-register-graph.png}
\caption{子图注册:插入Send和Recv节点}}
 \label{fig:tf-register-graph}
\end{figure}

\subsubsection{子图运算}

如\refig{tf-run-graph}所示。\ascii{Master}通过调用\code{RunGraph}接口,通知所有\ascii{Worker}执行子图运算。其中,\ascii{Worker}之间可以通过调用\code{RecvTensor}接口,完成数据的交换。

\begin{figure}[!htbp]
\centering
\includegraphics[width=1.0\textwidth]{figures/tf-run-graph.png}
\caption{子图执行}}
 \label{fig:tf-run-graph}
\end{figure}

\end{content}

\section{会话管理}
	
\begin{content}

接下来,通过概述会话的整个生命周期过程,及其与图控制之间的关联关系,进一步揭开运行时的内部运行机制。

\subsection{创建会话}

首先,\ascii{Client}\emph{首次}执行\code{tf.Session.run}时,会将整个图序列化后,通过\ascii{GRPC}发送\code{CreateSessionRequest}消息,将图传递给\ascii{Master}。

随后,\ascii{Master}创建一个\code{MasterSession}实例,并用全局唯一的\code{handle}标识,最终通过\code{CreateSessionResponse}返回给\ascii{Client}。如\refig{tf-create-session-overview}所示。

\begin{figure}[!h]
\centering
\includegraphics[width=0.9\textwidth]{figures/tf-create-session-overview.png}
\caption{创建会话}}
 \label{fig:tf-create-session-overview}
\end{figure}

\subsection{迭代运行}

随后,\ascii{Client}会启动迭代执行的过程,并且称每次迭代为一次\ascii{Step}。此时,\ascii{Client}发送\code{RunStepRequest}消息给\ascii{Master};并且消息携带\code{handle}标识,用于\ascii{Master}索引相应的\code{MasterSession}实例。如\refig{tf-run-step-overview}所示。

\begin{figure}[!h]
\centering
\includegraphics[width=1.0\textwidth]{figures/tf-run-step-overview.png}
\caption{迭代执行}}
 \label{fig:tf-run-step-overview}
\end{figure}

\subsubsection{注册子图}

\ascii{Master}收到\code{RunStepRequest}消息后,将执行图剪枝,分裂,优化等操作;最终按照任务\ascii{(Task)},将图划分为多个子图片段\ascii{(Graph Partition)}。

随后,\ascii{Master}向各个\ascii{Worker}发送\code{RegisterGraphRequest}消息,将子图片段依次注册到各个\ascii{Worker}节点上。

当\ascii{Worker}收到\code{RegisterGraphRequest}消息后,再次执行图剪枝,分裂操作;最终按照设备\ascii{(Device)},将图划分为多个子图片段\ascii{(Graph Partition)}。\footnote{在分布式运行时,图分裂经过两级分裂过程。在\ascii{Master}上按照任务分裂,而在\ascii{Worker}按照设备分裂;因此得到结果都称为子图片段,它们仅存在范围,及其大小的差异。}

当\ascii{Worker}完成子图注册后,通过返回\code{RegisterGraphReponse}消息,并携带\code{graph\_handle}标识。这是因为\ascii{Worker}可以并发注册并运行多个子图,每个子图使用\code{graph\_handle}唯一标识。

\subsubsection{运行子图}

\ascii{Master}完成子图注册后,将广播所有\ascii{Worker}并发执行所有子图。这个过程是通过\ascii{Master}发送\code{RunGraphRequest}消息给\ascii{Worker}完成的;其中,消息中携带\code{graph\_handle}标识,用于\ascii{Worker}索引相应的子图。

\ascii{Worker}收到消息\code{RunGraphRequest}消息后,按照如下形式化了的算法执行子图的运算。

\begin{leftbar}
  \begin{python}
def run_partitions(rendezvous, execs_and_parts, inputs, outputs):
  rendezvous.send(inputs)
  for (exec, part) in execs_and_parts: 
    exec.run(part)
  rendezvous.recv(outputs)
  \end{python}
\end{leftbar}

首先,\ascii{Worker}根据\code{graph\_handle}索引相应的子图;然后,并发执行所包含的所有子图片段。其中,每个子图片段放置在单独的\code{Executor}中执行,\code{Executor}将按照拓扑排序算法完成子图片段的计算。

\subsubsection{交换数据}

如果两个设备之间需要交换数据,则通过插入\ascii{Send/Recv}节点完成的。特殊地,如果两个\ascii{Worker}之间需要交换数据,则需要涉及跨主机,或跨进程的通信。

此时,需要通过接收端主动发送\code{RecvTensorRequest}消息到发送方,再从发送方的信箱里取出对应的\ascii{Tensor},并通过\code{RecvTensorResponse}返回。如\refig{tf-recv-tensor-overview}所示。


\begin{figure}[!h]
\centering
\includegraphics[width=0.7\textwidth]{figures/tf-recv-tensor-overview.png}
\caption{Worker之间的数据交换}}
 \label{fig:tf-recv-tensor-overview}
\end{figure}

\subsection{关闭会话}

经过许多次迭代执行后,\ascii{Client}向\ascii{Master}发送\code{CloseSessionReq}消息;\ascii{Master}收到消息后,开始释放\code{MasterSession}所持有的所有资源。如\refig{tf-close-session-overview}所示。

\begin{figure}[!h]
\centering
\includegraphics[width=0.7\textwidth]{figures/tf-close-session-overview.png}
\caption{关闭会话}}
 \label{fig:tf-close-session-overview}
\end{figure}

\end{content}

\begin{savequote}[45mm]
\ascii{Any fool can write code that a computer can understand. Good programmers write code that humans can understand.}
\qauthor{\ascii{- Martin Flower}}
\end{savequote}

\chapter{C API:分水岭} 
\label{ch:c-api}

\begin{content}

本章通过客户端\ascii{Session}生命周期的实现为例,揭示前端\ascii{Python}与后端\cpp{}系统的实现通道,揭示\ascii{TensorFlow}多语言编程的奥秘。

\end{content}

\section{Swig:幕后英雄}

\begin{content}

前端多语言编程环境与后端\cpp{}实现系统的通道归功于\ascii{Swig}的包装器。\ascii{TensorFlow}使用\ascii{Bazel}的构建工具,在系统编译之前启动\ascii{Swig}的代码生成过程,通过\code{tensorflow.i}自动生成了两个适配(\ascii{Wrapper})文件:

\begin{enum}
  \eitem{\code{pywrap\_tensorflow\_internal.py}:  负责对接上层\ascii{Python}调用;}
  \eitem{\code{pywrap\_tensorflow\_internal.cc}: 负责对接下层\ascii{C API}调用。}
\end{enum}

如\refig{swig}所示,\code{pywrap\_tensorflow\_internal.py}模块首次被导入时,自动地加载\code{\_pywrap\_tensorflow\_internal.so}的动态链接库;其中,\code{\_pywrap\_tensorflow\_internal.so}包含了整个\tf{}运行时的所有符号。在\code{pywrap\_tensorflow\_internal.cc}的实现中,静态注册了一个函数符号表,实现了\ascii{Python}函数名到\ascii{C}函数名的二元关系。在运行时,按照\ascii{Python}的函数名称,匹配找到对应的\ascii{C}函数实现,最终实现到\code{c\_api.c}具体实现的调用关系。

\begin{figure}[H]
\centering
\includegraphics[width=1.0\textwidth]{figures/swig.png}
\caption{Swig代码生成器}}
 \label{fig:swig}
\end{figure}

其中,\ascii{Bazel}的生成规则定义于\code{//tensorflow/python:pywrap\_tensorflow\_internal},如下代码所示。

\begin{leftbar}
\begin{python}
tf_py_wrap_cc(
    name = "pywrap_tensorflow_internal",
    srcs = ["tensorflow.i"],
    swig_includes = [
        "client/device_lib.i",
        "client/events_writer.i",
        "client/tf_session.i",
        "client/tf_sessionrun_wrapper.i",
        "framework/cpp_shape_inference.i",
        "framework/python_op_gen.i",
        "grappler/cost_analyzer.i",
        "grappler/model_analyzer.i",
        "grappler/tf_optimizer.i",
        "lib/core/py_func.i",
        "lib/core/strings.i",
        "lib/io/file_io.i",
        "lib/io/py_record_reader.i",
        "lib/io/py_record_writer.i",
        "platform/base.i",
        "pywrap_tfe.i",
        "training/quantize_training.i",
        "training/server_lib.i",
        "util/kernel_registry.i",
        "util/port.i",
        "util/py_checkpoint_reader.i",
        "util/stat_summarizer.i",
        "util/tfprof.i",
        "util/transform_graph.i",
    ]
)
\end{python}
\end{leftbar}

下文以客户端\ascii{Session}生命周期的实现为例,揭示前端\ascii{Python}与后端\cpp{}系统的实现通道。

\end{content}

\section{会话控制}

\begin{content}

严格意义上,\ascii{C API}并非是\ascii{Client}与\ascii{Master}的分界线。如\refig{tf-client-session}所示,\ascii{Client}存在部分\cpp{}实现,即\code{tensorflow::Session}。其中,\code{tf.Session}实例直接持有\code{tensorflow::Session}实例的句柄。在实际运行时环境中,\code{tensorflow::Session}可能存在多种实现。例如,\code{DirectSession}负责\emph{本地模式}的会话控制。而\code{GrpcSession}负责基于\ascii{gRPC}协议的\emph{分布式模式}的会话控制。一般地,用户使用的是\code{tf.Session}实施编程,而非\code{tensorflow::Session}。

\begin{figure}[!htbp]
\centering
\includegraphics[width=0.9\textwidth]{figures/tf-client-session.png}
\caption{客户端:tensorflow::Session实例创建过程}}
 \label{fig:tf-client-session}
\end{figure}

\end{content}

\section{会话生命周期}

\begin{content}

会话的生命周期包括会话的创建,创建计算图,扩展计算图,执行计算图,关闭会话,销毁会话的基本过程。在前端\ascii{Python}和后端\cpp{}表现为两套相兼容的接口实现。

\subsection{Python前端}

如\refig{py-session-lifecycle}所示,在\ascii{Python}前端,\code{Session}的生命周期主要体现在:

\begin{enum}
  \eitem{创建\code{Session(target)};}
  \eitem{迭代执行\code{Session.run(fetches, feed\_dict)};}
    \begin{enum}
      \eitem{\code{Session.\_extend\_graph(graph)};}
      \eitem{\code{Session.TF\_Run(feeds, fetches, targets)};}
    \end{enum}
  \eitem{关闭\code{Session};}
  \eitem{销毁\code{Session};}
\end{enum}

\begin{figure}[H]
\centering
\includegraphics[width=0.5\textwidth]{figures/py-session-lifecycle.png}
\caption{Python: Session生命周期}}
 \label{fig:py-session-lifecycle}
\end{figure}

例如,此处创建了本地模式的\code{Session}实例,并启动\code{mnist}的训练过程。

\begin{leftbar}
\begin{python}
sess = tf.Session()
for _ in range(1000):
  batch_xs, batch_ys = mnist.train.next_batch(100)
  sess.run(train_step, feed_dict={x: batch_xs, y_: batch_ys})
sess.close()
\end{python}
\end{leftbar}

\subsection{C++后端}

相应地,在\cpp{}后端,\code{Session}的生命周期主要体现在:

\begin{enum}
  \eitem{根据\code{target}多态创建\code{Session};}
  \eitem{\code{Session.Create(graph)}:有且仅有一次;}
  \eitem{\code{Session.Extend(graph)}:零次或多次;}
  \eitem{迭代执行\code{Session.Run(inputs, outputs, targets)};}
  \eitem{关闭\code{Session.Close};}
  \eitem{销毁\code{Session}对象。}
\end{enum}

\begin{figure}[H]
\centering
\includegraphics[width=0.9\textwidth]{figures/cc-session-lifecycle.png}
\caption{C++: Session生命周期}}
 \label{fig:cc-session-lifecycle}
\end{figure}

例如,此处创建了本地模式的\code{DirectSession}实例,并启动计算图的执行过程。

\begin{leftbar}
\begin{c++}
// create/load graph ...
tensorflow::GraphDef graph;

// local runtime, target is ""
tensorflow::SessionOptions options;

// create Session
std::unique_ptr<tensorflow::Session> 
sess(tensorflow::NewSession(options));

// create graph at initialization.
tensorflow::Status s = sess->Create(graph);
if (!s.ok()) { ... }

// run step
std::vector<tensorflow::Tensor> outputs;
s = session->Run(
  {},               // inputs is empty
  {"output:0"},     // outputs names
  {"update_state"}, // target names
  &outputs);        // output tensors
if (!s.ok()) { ... }

// close
session->Close();
\end{c++}
\end{leftbar}

\end{content}

\section{创建会话}

\begin{content}

下面介绍Session创建的详细过程,从\ascii{Python}前端为起点,通过\ascii{Swig}自动生成的\ascii{Python-C++}的包装器,并以此为媒介,实现了\ascii{Python}到\ascii{TensorFlow}的\ascii{C API}的调用。其中,\ascii{C API}是前端系统与后端系统的分水岭。

\begin{figure}[H]
\centering
\includegraphics[width=1.0\textwidth]{figures/py-create-session.png}
\caption{创建会话}}
 \label{fig:py-create-session}
\end{figure}

\subsection{编程接口}

当\ascii{Client}要启动计算图的执行过程时,先创建了一个\code{Session}实例,进而调用父类\code{BaseSession}的构造函数。

\begin{leftbar}
\begin{python}[caption={tensorflow/python/client/session.py}]
class Session(BaseSession):
  def __init__(self, target='', graph=None, config=None):
    super(Session, self).__init__(target, graph, config=config)
    self._default_graph_context_manager = None
    self._default_session_context_manager = None
\end{python}
\end{leftbar}

在\code{BaseSession}的构造函数中,将调用\code{pywrap\_tensorflow}模块中的函数。其中,\code{TF\_NewDeprecatedSession}是遗留下来的,已被废弃的接口实现。新的接口在性能提升方面做了很大的改进,但不幸的是,截止\ascii{r1.3}版本,新的接口实现还未被正式启用。

\begin{leftbar}
\begin{python}[caption={tensorflow/python/client/session.py}]
from tensorflow.python import pywrap_tensorflow as tf_session

class BaseSession(SessionInterface):
  def __init__(self, target='', graph=None, config=None):
    # ignoring implements...
    self._session = None
    opts = tf_session.TF_NewSessionOptions(target=self._target,
                                           config=config)
    try:
      with errors.raise_exception_on_not_ok_status() as status:
        if self._created_with_new_api:
          self._session = tf_session.TF_NewSession(
              self._graph._c_graph, opts, status)
        else:
          self._session = tf_session.TF_NewDeprecatedSession(opts, status)
    finally:
      tf_session.TF_DeleteSessionOptions(opts)
\end{python}
\end{leftbar}

\subsubsection{Python包装器}

在\code{pywrap\_tensorflow}模块中,通过\code{\_pywrap\_tensorflow\_internal}的转发,实现了从\ascii{Python}到动态连接库\code{\_pywrap\_tensorflow\_internal.so}的函数调用。

\begin{leftbar}
\begin{python}[caption={tensorflow/bazel-bin/tensorflow/python/pywrap\_tensorflow\_internal.py}]
def TF_NewDeprecatedSession(opts, status):
    return _pywrap_tensorflow_internal.TF_NewDeprecatedSession(opts, status)

def TF_NewSession(graph, opts, status):
  return _pywrap_tensorflow_internal.TF_NewSession(graph, opts, status)
\end{python}
\end{leftbar}

\subsubsection{C++包装器}

在\code{pywrap\_tensorflow\_internal.cc}的具体实现中,静态注册了函数调用的符号表,实现\ascii{Python}的函数名称到\cpp{}函数实现的具体映射。

\begin{leftbar}
\begin{c++}[caption={tensorflow/bazel-bin/tensorflow/python/pywrap\_tensorflow\_internal.cc}]
static PyMethodDef SwigMethods[] = {
   // ...
   { (char *)"TF_NewDeprecatedSession", 
     _wrap_TF_NewDeprecatedSession, METH_VARARGS, NULL},

   { (char *)"TF_NewSession", 
     _wrap_TF_NewSession, METH_VARARGS, NULL},
};
\end{c++}
\end{leftbar}

最终,\code{\_wrap\_TF\_NewSession/\_wrap\_TF\_NewDeprecatedSession}将分别调用\code{c\_api.h}对其开放的\ascii{API}接口:\code{TF\_NewSession/TF\_NewDeprecatedSession}。也就是说,自动生成的\code{pywrap\_tensorflow\_internal.cc}仅仅负责\ascii{Python}函数到\ascii{C/C++}函数调用的转发,最终将调用底层\ascii{C}系统向上提供的\ascii{API}接口。

\subsection{C API}

\code{c\_api.h}是\ascii{TensorFlow}的后端执行系统面向前端开放的公共\ascii{API}接口。其中,新的接口实现采用了引用计数的技术,实现图实例在多个\code{Session}实例中共享。

\begin{leftbar}
\begin{c++}[caption={tensorflow/c/c\_api.c}]
TF_Session* TF_NewSession(TF_Graph* graph, const TF_SessionOptions* opt,
                          TF_Status* status) {
  Session* session;
  status->status = NewSession(opt->options, &session);
  if (status->status.ok()) {
    if (graph != nullptr) {
      mutex_lock l(graph->mu);
      graph->num_sessions += 1;
    }
    return new TF_Session(session, graph);
  } else {
    return nullptr;
  }
}

TF_DeprecatedSession* TF_NewDeprecatedSession(const TF_SessionOptions* opt,
                                              TF_Status* status) {
  Session* session;
  status->status = NewSession(opt->options, &session);
  if (status->status.ok()) {
    return new TF_DeprecatedSession({session});
  } else {
    DCHECK_EQ(nullptr, session);
    return nullptr;
  }
}
\end{c++}
\end{leftbar}

\subsection{后端系统}

\code{NewSession}将根据前端传递的\code{target},使用\code{SessionFactory}多态创建不同类型的\code{tensorflow::Session}实例。

\begin{leftbar}
\begin{c++}[caption={tensorflow/c/c\_api.c}]
Status NewSession(const SessionOptions& options, Session** out_session) {
  SessionFactory* factory;
  const Status s = SessionFactory::GetFactory(options, &factory);
  if (!s.ok()) {
    *out_session = nullptr;
    return s;
  }
  *out_session = factory->NewSession(options);
  if (!*out_session) {
    return errors::Internal("Failed to create session.");
  }
  return Status::OK();
}
\end{c++}
\end{leftbar}

\subsubsection{工厂方法}

在后端\cpp{}实现中,\code{tensorflow::Session}的创建使用了抽象工厂方法。如果\code{SessionOptions}中的\code{target}为空字符串(默认的),则创建\code{DirectSession}实例,启动本地运行模式;如果\code{SessionOptions}中的\code{target}以\code{grpc://}开头,则创建\code{GrpcSession}实例,启动基于\code{RPC}的分布式运行模式。如\refig{cc-session-factory}所示。

\begin{figure}[H]
\centering
\includegraphics[width=1.0\textwidth]{figures/cc-session-factory.png}
\caption{tensorflow::Session创建:抽象工厂方法}}
 \label{fig:cc-session-factory}
\end{figure}

\end{content}

\section{创建/扩展图}

\begin{content}

随后,\ascii{Python}前端将迭代调用\code{Session.run}接口,将构造好的计算图,以\code{GraphDef}的形式发送给\cpp{}后端。其中,前端每次调用\code{Session.run}接口时,都会试图将新增节点的计算图发送给后端系统,以便将新增节点的计算图\ascii{Extend}到原来的计算图中。特殊地,在首次调用\code{Session.run}时,将发送整个计算图给后端系统。

后端系统首次调用\code{Session.Extend}时,转调(或等价实现)\code{Session.Create}。以后,后端系统每次调用\code{Session.Extend}时将真正执行\code{Extend}的语义,将新增的计算图的节点追加至原来的计算图中。

\begin{figure}[H]
\centering
\includegraphics[width=0.9\textwidth]{figures/py-session-create-graph.png}
\caption{创建图}}
 \label{fig:py-session-create-graph}
\end{figure}

\subsection{编程接口}

在既有的接口实现中,需要将图构造期构造好的图序列化,并传递给后端\ascii{C++}系统,这里通过\code{\_extend\_graph}实现。在新的接口实现中,无需实现图的创建或扩展,因为在创建\ascii{OP}时,后端\ascii{C++}直接创建相应的图实例,并将节点实时添加至后端系统的图实例中。

\begin{leftbar}
\begin{python}[caption={tensorflow/python/client/session.py}]
class Session(BaseSession):
  def run(self, fetch_list, feed_dict=None, options=None, run_metadata=None):
    # ignores implements...
    self._extend_graph()
    # ignores implements...

\end{python}
\end{leftbar}

在首次调用\code{self.\_extend\_graph}时,或者有新的节点被添加至计算图中时,对计算图\code{GraphDef}实施序列化操作,最终触发\code{tf\_session.TF\_ExtendGraph}的调用。

\begin{leftbar}
\begin{python}[caption={tensorflow/python/client/session.py}]
from tensorflow.python import pywrap_tensorflow as tf_session

class Session(BaseSession):
  def _extend_graph(self):
    if self._created_with_new_api: return

    with self._extend_lock:
      if self._graph.version > self._current_version:
        graph_def, self._current_version = self._graph._as_graph_def(
            from_version=self._current_version,
            add_shapes=self._add_shapes)

        with errors.raise_exception_on_not_ok_status() as status:
          tf_session.TF_ExtendGraph(
              self._session, graph_def.SerializeToString(), status)
\end{python}
\end{leftbar}

\subsubsection{Python包装器}

\begin{leftbar}
\begin{python}[caption={tensorflow/bazel-bin/tensorflow/python/pywrap\_tensorflow\_internal.py}]
def TF_ExtendGraph(sess, graph_def, status):
  return _pywrap_tensorflow.TF_ExtendGraph(sess, graph_def, status)
\end{python}
\end{leftbar}

\subsubsection{C++包装器}

\begin{leftbar}
\begin{c++}[caption={tensorflow/bazel-bin/tensorflow/python/pywrap\_tensorflow\_internal.cc}]
static PyMethodDef SwigMethods[] = {
  // ignore implements...
  { (char *)"TF_ExtendGraph", _wrap_TF_ExtendGraph, METH_VARARGS, NULL},
};
\end{c++}
\end{leftbar}

\subsection{C API}

\code{TF\_ExtendGraph}是\ascii{C API}对接上层编程环境的接口。首先,它完成计算图\code{GraphDef}的反序列化,最终调用\code{tensorflow::Session}的\code{Extend}接口。

\begin{leftbar}
\begin{c++}[caption={tensorflow/c/c\_api.c}]
void TF_ExtendGraph(TF_DeprecatedSession* sess, 
  const void* proto, size_t proto_len, TF_Status* status) {
  GraphDef g;
  if (!tensorflow::ParseProtoUnlimited(&g, proto, proto_len)) {
    status->status = InvalidArgument("Invalid GraphDef");
    return;
  }
  status->status = sess->session->Extend(g);
}
\end{c++}
\end{leftbar}

\subsection{后端系统}

\code{tensorflow::Session}在运行时根据\code{Session}的动态类型,将多态地调用相应子类的实现。

\begin{leftbar}
\begin{c++}[caption={tensorflow/core/common\_runtime/session.h}]
class Session {
public:
  virtual Status Create(const GraphDef& graph) = 0;
  virtual Status Extend(const GraphDef& graph) = 0;
};
\end{c++}
\end{leftbar}

其中,\code{Create}表示在当前的\code{tensorflow::Session}实例上注册计算图,如果要注册新的计算图,需要关闭该\code{tensorflow::Session}对象。\code{Extend}表示在\code{tensorflow::Session}实例上已注册的计算图上追加节点。\code{Extend}首次执行时,等价于\code{Create}的语义,因为首次\code{Extend}时,已注册的计算图为空。事实上,系统就是按照如上方案实现的,此处以\code{GrpcSession}实现为例。

\subsubsection{首次扩展图: GrpcSession}

如果判断引用\code{Master}的\code{handle}不为空,则执行\code{Extend};否则,执行\code{Create}的语义,建立与\code{Master}的连接,并持有\code{MasterSession}的\code{handle}。

\begin{leftbar}
\begin{c++}[caption={tensorflow/core/distributed\_runtime/rpc/grpc\_session.cc}]
Status GrpcSession::Extend(const GraphDef& graph) {
  CallOptions call_options;
  call_options.SetTimeout(options_.config.operation_timeout_in_ms());
  return ExtendImpl(&call_options, graph);
}

Status GrpcSession::ExtendImpl
  (CallOptions* call_options, const GraphDef& graph) {
  if (handle_is_empty()) {
    // Session was unitialized, 
    // so simply initialize the session with 'graph'.
    return Create(graph);
  }
  // ignore implements...  
}
\end{c++}
\end{leftbar}

\end{content}

\section{迭代运行}

\begin{content}

如\refig{py-session-run}所示,\ascii{Python}前端\code{Session.run}实现将\code{fetches, feed\_dict}传递给后端系统,后端系统调用\code{Session.Run}接口。后端系统的一次\code{Session.Run}执行常常被称为一次\ascii{Step}。其中,\ascii{Step}的执行过程是\ascii{TensorFlow}运行时的关键路径。

\begin{figure}[H]
\centering
\includegraphics[width=1.0\textwidth]{figures/py-session-run.png}
\caption{迭代执行}}
 \label{fig:py-session-run}
\end{figure}

\subsection{编程接口}

当\ascii{Client}调用\code{Session.run}时,最终会调用\code{pywrap\_tensorflow\_internal}模块中的函数。

\begin{leftbar}
\begin{python}[caption={tensorflow/python/client/session.py}]
from tensorflow.python import pywrap_tensorflow as tf_session

class Session(BaseSession):
  def run(self, fetch_list, feed_dict=None, options=None, run_metadata=None):
    # ignores other implements...
    self._extend_graph()
    with errors.raise_exception_on_not_ok_status() as status:
      if self._created_with_new_api:
        return tf_session.TF_SessionRun_wrapper(
            session, options, feed_dict, fetch_list, target_list,
            run_metadata, status)
      else:
        return tf_session.TF_Run(session, options,
                                 feed_dict, fetch_list, target_list,
                                 status, run_metadata)
\end{python}
\end{leftbar}


\subsubsection{Python包装器}

\begin{leftbar}
\begin{python}[caption={tensorflow/bazel-bin/tensorflow/python/pywrap\_tensorflow\_internal.py}]
def TF_SessionRun_wrapper(session, run_options, inputs, 
  outputs, targets, run_metadata, out_status):
  return _pywrap_tensorflow_internal.TF_SessionRun_wrapper(
    session, run_options, inputs, outputs, targets, run_metadata, out_status)

def TF_Run(sess, options, feeds, outputs, 
  targets, status, run_metadata):
  return _pywrap_tensorflow.TF_Run(
    sess, options, feeds, outputs, targets, status, run_metadata)
\end{python}
\end{leftbar}

\subsubsection{C++包装器}

\begin{leftbar}
\begin{c++}[caption={tensorflow/bazel-bin/tensorflow/python/pywrap\_tensorflow\_internal.cc}]
static PyMethodDef SwigMethods[] = {
  // ...
  { (char *)"TF_Run", 
    _wrap_TF_Run, METH_VARARGS, NULL},

  { (char *)"TF_SessionRun_wrapper", 
    _wrap_TF_SessionRun_wrapper, METH_VARARGS, NULL},
};
\end{c++}
\end{leftbar}

最终,\code{\_wrap\_TF\_Run/\_wrap\_TF\_SessionRun\_wrapper}将分别转调\ascii{C API}对应的\code{TF\_Run/TF\_SessionRun}接口函数。

\subsection{C API}

\code{TF\_Run}是\ascii{C API}对接上层编程环境的接口。首先,它完成输入数据从\ascii{C}到\cpp{}的格式转换,并启动后台的\code{tensorflow::Session}的执行过程。当执行完成后,再将\code{outputs}的输出数据从\cpp{}到\ascii{C}的格式转换。\code{TF\_SessionRun}与\code{TF\_Run}工作机制差不多,在此不再冗述。

\begin{leftbar}
\begin{c++}[caption={tensorflow/c/c\_api.c}]
void TF_Run(TF_DeprecatedSession* s, 
  // session options
  const TF_Buffer* run_options,
  // Input tensors
  const char** c_input_names, TF_Tensor** c_inputs, int ninputs,
  // Output tensors
  const char** c_output_names, TF_Tensor** c_outputs, int noutputs,
  // Target nodes
  const char** c_target_oper_names, int ntargets,
  // run\_metadata
  TF_Buffer* run_metadata, TF_Status* status) {
  // convert data format, ignore implements...
  s->session->Run(options_proto, input_names, output_names,
                  target_names, &outputs, &run_metadata); 
  // store results in c\_outputs...
}

void TF_SessionRun(TF_Session* session, 
  const TF_Buffer* run_options,
  // Input tensors
  const TF_Output* inputs, TF_Tensor* const * input_values, int ninputs, 
  // Output tensors
  const TF_Output* outputs, TF_Tensor** output_values, int noutputs,
  // Target nodes
  const TF_Operation* const* target_opers, int ntargets,
  // run\_metadata
  TF_Buffer* run_metadata, TF_Status* status) {
  // ignore implements.
}
\end{c++}
\end{leftbar}

\subsection{后端系统}

\code{tensorflow::Session}在运行时其动态类型,将多态地调用相应的子类实现。

\begin{leftbar}
\begin{c++}[caption={tensorflow/core/common\_runtime/session.h}]
class Session {
public:
  virtual Status Run(
    const RunOptions& options,
    const vector<pair<string, Tensor> >& inputs,
    const vector<string>& output_names,
    const vector<string>& target_names,
    vector<Tensor>* outputs, RunMetadata* run_metadata) {
      return errors::Unimplemented(
        "Run with options is not supported for this session.");
  }
};
\end{c++}
\end{leftbar}

输入包括:

\begin{enum}
  \eitem{\code{options}:\code{Session}的运行配置参数;}
  \eitem{\code{inputs}: 输入\code{Tensor}的名字列表;}
  \eitem{\code{output\_names}: 输出\code{Tensor}的名字列表;}
  \eitem{\code{targets}: 无输出,待执行的\ascii{OP}的名字列表。} 
\end{enum}

输出包括:

\begin{enum}
  \eitem{\code{outputs}: 输出的\code{Tensor}列表;}
  \eitem{\code{run\_metadata}: 运行时元数据的收集器。}
\end{enum}

其中,输出的\code{outputs}列表与输入的\code{output\_names}一一对应,如果运行时因并发执行,导致\code{outputs}乱序执行,最终返回时需要对照输入的\code{output\_names}名字列表,对\code{outputs}进行排序。

\end{content}

\section{关闭会话}

\begin{content}

当计算图执行完毕后,需要关闭\code{tf.Session},以便释放后端的系统资源,包括队列,\ascii{IO}等。会话关闭流程较为简单,如\refig{py-session-close}所示。。

\begin{figure}[H]
\centering
\includegraphics[width=0.9\textwidth]{figures/py-session-close.png}
\caption{关闭会话}}
 \label{fig:py-session-close}
\end{figure}

\subsection{编程接口}

当\ascii{Client}调用\code{Session.close}时,最终会调用\code{pywrap\_tensorflow}模块中的函数: \code{TF\_CloseDeprecatedSession}。

\begin{leftbar}
\begin{python}[caption={tensorflow/python/client/session.py}]
from tensorflow.python import pywrap_tensorflow as tf_session

class Session(BaseSession):
  def close(self):
    if self._created_with_new_api:
      if self._session and not self._closed:
        self._closed = True
        with errors.raise_exception_on_not_ok_status() as status:
          tf_session.TF_CloseSession(self._session, status)
    else:
      with self._extend_lock:
        if self._opened and not self._closed:
          self._closed = True
          with errors.raise_exception_on_not_ok_status() as status:
            tf_session.TF_CloseDeprecatedSession(self._session, status)
\end{python}
\end{leftbar}

\subsubsection{Python包装器}

\begin{leftbar}
\begin{python}[caption={tensorflow/bazel-bin/tensorflow/python/pywrap\_tensorflow\_internal.py}]
def TF_CloseSession(sess, status):
    return _pywrap_tensorflow_internal.TF_CloseSession(sess, status)

def TF_CloseDeprecatedSession(sess, status):
  return _pywrap_tensorflow.TF_CloseDeprecatedSession(sess, status)
\end{python}
\end{leftbar}

\subsubsection{C++包装器}

\begin{leftbar}
\begin{c++}[caption={tensorflow/bazel-bin/tensorflow/python/pywrap\_tensorflow\_internal.cc}]
static PyMethodDef SwigMethods[] = {
  // ...
  { (char *)"TF_CloseSession", 
    _wrap_TF_CloseSession, METH_VARARGS, NULL},

  { (char *)"TF_CloseDeprecatedSession", 
    _wrap_TF_CloseDeprecatedSession, METH_VARARGS, NULL},
};
\end{c++}
\end{leftbar}

最终,\code{\_wrap\_TF\_CloseSession/\_wrap\_TF\_CloseDeprecatedSession}将分别转调\ascii{C API}对应的\code{TF\_CloseSession/TF\_CloseDeprecatedSession}接口函数。

\subsection{C API}

\code{TF\_CloseSession/TF\_CloseDeprecatedSession}直接完成\code{tensorflow::Session}的关闭操作。

\begin{leftbar}
\begin{c++}[caption={tensorflow/c/c\_api.c}]
void TF_CloseSession(TF_Session* s, TF_Status* status) {
  status->status = s->session->Close();
}

void TF_CloseDeprecatedSession(TF_DeprecatedSession* s, TF_Status* status) {
  status->status = s->session->Close();
}
\end{c++}
\end{leftbar}

\subsection{后端系统}

\code{Session(C++)}在运行时其动态类型,将多态地调用相应的子类实现。

\begin{leftbar}
\begin{c++}[caption={tensorflow/core/common\_runtime/session.h}]
class Session {
public:
  virtual Status Close() = 0;
};
\end{c++}
\end{leftbar}

\end{content}

\section{销毁会话}

\begin{content}

当\code{tf.Session}不在被使用,由\ascii{Python}的\ascii{GC}释放。\code{Session.\_\_del\_\_}被调用后,将启动后台\code{tensorflow::Session}对象的析构过程。如\refig{py-delete-session}所示。

\begin{figure}[H]
\centering
\includegraphics[width=0.9\textwidth]{figures/py-delete-session.png}
\caption{销毁会话}
 \label{fig:py-delete-session}
\end{figure}

\subsection{编程接口}

当\ascii{Client}调用\code{Session.\_\_del\_\_}时,先启动\code{Session.close}的调用,最终会调用\code{pywrap\_tensorflow}模块中的函数\code{TF\_DeleteDeprecatedSession}。而对于新的接口实现,调用\code{TF\_DeleteSession}。

\begin{leftbar}
\begin{python}[caption={tensorflow/python/client/session.py}]
from tensorflow.python import pywrap_tensorflow as tf_session

class Session(BaseSession):
  def __del__(self):
    # 1. close session unconditionally.
    try:
      self.close()
    except Exception:
      pass
    # 2. delete session unconditionally.
    if self._session is not None:
      try:
        status = c_api_util.ScopedTFStatus()
        if self._created_with_new_api:
          tf_session.TF_DeleteSession(self._session, status)
        else:
          tf_session.TF_DeleteDeprecatedSession(self._session, status)
      except AttributeError:
        pass
      self._session = None
\end{python}
\end{leftbar}

\subsubsection{Python包装器}

\begin{leftbar}
\begin{python}[caption={tensorflow/bazel-bin/tensorflow/python/pywrap\_tensorflow\_internal.py}]
def TF_DeleteSession(sess, status):
    return _pywrap_tensorflow_internal.TF_DeleteSession(sess, status)

def TF_DeleteDeprecatedSession(sess, status):
  return _pywrap_tensorflow.TF_DeleteDeprecatedSession(sess, status)
\end{python}
\end{leftbar}

\subsubsection{C++包装器}

\begin{leftbar}
\begin{c++}[caption={tensorflow/bazel-bin/tensorflow/python/pywrap\_tensorflow\_internal.cc}]
static PyMethodDef SwigMethods[] = {
  // ...
  { (char*)"TF_DeleteSession", 
    _wrap_TF_DeleteSession, METH_VARARGS, NULL},

  { (char*)"TF_DeleteDeprecatedSession", 
    _wrap_TF_DeleteDeprecatedSession, METH_VARARGS, NULL},
};
\end{c++}
\end{leftbar}

最终,\code{\_wrap\_TF\_DeleteSession/\_wrap\_TF\_DeleteDeprecatedSession}将分别转调\ascii{C API}对应的\code{TF\_DeleteSession/TF\_DeleteDeprecatedSession}接口函数。

\subsection{C API}

\code{TF\_DeleteDeprecatedSession}直接完成\code{tensorflow::Session}对象的释放。而新的接口\code{TF\_DeleteSession}实现中,当需要删除\code{tensorflow::Session}实例时,相应的图实例的计数器减\ascii{1}。当计数器为\ascii{0}时,则删除该图实例;否则,不删除该图实例。

\begin{leftbar}
\begin{c++}[caption={tensorflow/c/c\_api.c}]
void TF_DeleteSession(TF_Session* s, TF_Status* status) {
  status->status = Status::OK();
  TF_Graph* const graph = s->graph;
  if (graph != nullptr) {
    graph->mu.lock();
    graph->num_sessions -= 1;
    const bool del = graph->delete_requested && graph->num_sessions == 0;
    graph->mu.unlock();
    if (del) delete graph;
  }
  delete s->session;
  delete s;
}

void TF_DeleteDeprecatedSession(TF_DeprecatedSession* s, TF_Status* status) {
  status->status = Status::OK();
  delete s->session;
  delete s;
}
\end{c++}
\end{leftbar}

\subsection{后端系统}

\code{tensorflow::Session}在运行时其动态类型,多态地调用相应子类实现的析构函数。

\begin{leftbar}
\begin{c++}[caption={tensorflow/core/common\_runtime/session.h}]
class Session {
public:
  virtual ~Session() {};
};
\end{c++}
\end{leftbar}

\end{content}

\section{性能调优}

\begin{content}

相比遗留的接口实现,新的接口实现存在若干优化技术提升系统的性能。虽然,截止本书撰写时,新的接口并未完全对外发布,但可以预期未来将删除既有已废弃的接口,替换为了新的接口实现。

\subsection{共享图实例}

如\refig{tf-graph-session-relation}所示,一个\code{Session}只能运行一个图实例。如果一个\code{Session}要运行其他的图实例,必须先关掉\code{Session},然后再将新的图实例注册到此\code{Session}中,最后启动新的计算图的执行过程。

但反过来,一个计算图可以运行在多个\code{Session}实例上。如果在\code{Graph}实例上维持\code{Session}的引用计数器,在\code{Session}创建时,在该图实例上增加\ascii{1};在\code{Session}销毁时(不是关闭\code{Session}),在该图实例上减少1。当计数器为\ascii{0}时,则自动删除图实例。在新的接口实现中,实现了该引用计数器的技术。

\begin{figure}[H]
\centering
\includegraphics[width=0.7\textwidth]{figures/tf-graph-session-relation.png}
\caption{计算图:会话引用计数器技术}}
 \label{fig:tf-graph-session-relation}
\end{figure}

\subsection{消除序列化}

如\refig{tf-old-session-interface}所示,在遗留的接口实现中,前端\ascii{Python}在图构造期,将图构造完成后,通过\code{Session::Create}或\code{Session::Extend}接口将其序列化,并传递给后端\ascii{C++}系统,后端反序列化获取相应的图实例。这本质是一个图实例的拷贝过程,具有很大的时延开销。

\begin{figure}[H]
\centering
\includegraphics[width=0.8\textwidth]{figures/tf-old-session-interface.png}
\caption{图实例:序列化/反序列化}}
 \label{fig:tf-old-session-interface}
\end{figure}

如\refig{tf-new-session-interface}所示,在新的接口实现中,可以去除\code{Session}的\code{Create/Extend}的语义。在图的构造器,前端\ascii{Python}在构造每个\ascii{OP}时,直接通过\ascii{C API}将其追加至后端\ascii{C++}的图实例中,从而避免了图实例在前后端的序列化和反序列化的开销。

\begin{figure}[H]
\centering
\includegraphics[width=0.8\textwidth]{figures/tf-new-session-interface.png}
\caption{图实例:实施注册ascii{OP}}
 \label{fig:tf-new-session-interface}
\end{figure}

\end{content}


\part{编程模型}
\begin{savequote}[45mm]
\ascii{Any fool can write code that a computer can understand. Good programmers write code that humans can understand.}
\qauthor{\ascii{- Martin Flower}}
\end{savequote}

\chapter{计算图} 
\label{ch:computation-graph}

\begin{content}

在\tf{}的计算图中,使用\ascii{OP}表示节点,根据\ascii{OP}之间计算和数据依赖关系,构造\ascii{OP}之间生产与消费的数据依赖关系,并通过有向边表示。

其中,有向边存在两种类型,一种承载数据,并使用\code{Tensor}表示;另一种不承载数据,仅表示计算依赖关系。

本章将阐述\tf{}中最重要的领域对象:计算图。为了全面阐述计算图的关键实现技术,将分别探讨前后端的系统设计和实现,并探究前后端系统间计算图转换的工作流原理。

\end{content}

\section{Python前端}

\begin{content}

在\ascii{Python}的前端系统中,并没有\code{Node, Edge}的概念,仅存在\code{Operation, Tensor}的概念。事实上,在前端\ascii{Python}系统中,\code{Operation}表示图中的\code{Node}实例,而\code{Tensor}表示图中的\code{Edge}实例。

\subsection{Operation}

\code{Operation}表示某种抽象计算,它以零个或多个\code{Tensor}作为输入,经过计算后,输出零个或多个\code{Tensor}。

如\refig{py-operation}所示。在计算图构造期间,通过\ascii{OP}构造器\ascii{(OP Constructor)},构造\code{Operation}实例,并将其注册至默认的图实例中;与此同时,\code{Operation}反过来通过\ascii{graph}直接持有该图实例。

\code{Operation}的元数据由\code{OpDef}与\code{NodeDef}持有,它们以\ascii{ProtoBuf}的格式存在,它描述了\code{Operation}最本质的东西。其中,\code{OpDef}描述了\code{OP}的静态属性信息,例如名称,输入输出的属性名等信息。而\code{NodeDef}描述了\ascii{OP}的动态属性值信息。

\code{Operation}根据上游节点的输出,经过计算输出到下游。其中,\code{Operation}的输入和输出以\code{Tensor}的形式存在。从而上下游产生了数据依赖关系。

此外,\code{Operation}可能持有上游的控制依赖边的集合,表示其潜在的计算依赖关系。

\begin{figure}[!htbp]
\centering
\includegraphics[width=0.9\textwidth]{figures/py-operation.png}
\caption{领域对象:Operation}
 \label{fig:py-operation}
\end{figure}

\subsection{Tensor}

一个\code{Tensor}表示\code{Operation}的某个输出的符号句柄,它并不持\code{Operation}输出的真实数据。可以通过\code{Session.run}计算得到\code{Tensor}所持有的真实数据。

如\refig{py-tensor}所示。\code{Tensor}是两个\code{Operation}数据交换的桥梁,它们之间构造了典型的「生产者-消费者」的关系。

\begin{figure}[!htbp]
\centering
\includegraphics[width=0.9\textwidth]{figures/py-tensor.png}
\caption{领域对象:Tensor}
 \label{fig:py-tensor}
\end{figure}

其中,\code{Tensor}通过\ascii{op}持有扮演生产者角色的\code{Operation},并且使用\code{index}表示该\code{Tensor}在该\code{Operation}输出列表中的索引。也就是说,可以使用\code{op:index}的二元组信息在图中唯一标识一个\code{Tensor}实例。

此外,\code{Tensor}持有\code{Operation}的消费者列表。计算图以\code{Tensor}为边,构建\code{Operation}之间的数据连接,从而实现了整个计算图的数据依赖构建。

\subsubsection{生产者与消费者}

如\refig{py-tensor-producter-consumer}所示。上游\code{Operation}作为生产者,经过某种抽象计算,生产了一个\code{Tensor},并以此作为该上游\code{Operation}的输出之一,并使用\code{index}标识。

该\code{Tensor}被传递给下游\code{Operation},并作为下游\code{Operation}的输入,下游\code{Operation}充当该\code{Tensor}的消费者。

\begin{figure}[!htbp]
\centering
\includegraphics[width=0.9\textwidth]{figures/py-tensor-producter-consumer.png}
\caption{Tensor: 生产者-消费者关系}
 \label{fig:py-tensor-producter-consumer}
\end{figure}

\subsubsection{建立关联}

最后,参看\code{Operation}与\code{Tensor}的部分实现,很容易找两者「生产者-消费者」的关联关系。当\code{Tensor}列表作为输入流入\code{Operation}时,此时建立了下游\code{Operation}与输入的\code{Tensor}列表之间的消费关系。

\begin{leftbar}
\begin{python}
class Operation(object):
  def __init__(self, node_def, graph, inputs=None, output_types=None):
    # \_inputs as consumers
    self._inputs = list(inputs)
    for a in self._inputs:
      a._add_consumer(self)

    # self as producer
    self._output_types = output_types
    self._outputs = [Tensor(self, i, output_type)
                     for i, output_type in enumerate(output_types)]
\end{python}
\end{leftbar}


同样地,\code{Tensor}在构造函数中持有作为上游的的生产者\code{Operation},及其它在该\code{Operation}的\code{outputs}列表中的索引。此外,当调用\code{\_add\_consumer},将该下游\code{Operation}追加至消费者列表之中。

\begin{leftbar}
\begin{python}
class Tensor(_TensorLike):
  def __init__(self, op, value_index, dtype):    
    # Index of the OP's endpoint that produces this tensor.
    self._op = op
    self._value_index = value_index
    
    # List of operations that use this Tensor as input.  
    # We maintain this list to easily navigate a computation graph.
    self._consumers = []

  def _add_consumer(self, consumer):
    if not isinstance(consumer, Operation):
      raise TypeError("Consumer must be an Operation: %s" % consumer)
    self._consumers.append(consumer)
\end{python}
\end{leftbar}

\subsection{Graph}

如\refig{py-graph}所示。一个\code{Graph}对象将包含一系列\code{Operation}对象,表示计算单元的集合;同时,它间接持有一系列\code{Tensor}对象,表示数据单元的集合。

\begin{figure}[!htbp]
\centering
\includegraphics[width=0.9\textwidth]{figures/py-graph.png}
\caption{领域对象:Graph}
 \label{fig:py-graph}
\end{figure}

\subsection{图构造}

在计算图的构造期间,不执行任何\ascii{OP}的计算。简单地说,图的构造过程就是根据\ascii{OP}构造器完成\code{Operation}实例的构造。而在\code{Operation}实例的构造之前,需要实现完成\code{OpDef}与\code{NodeDef}的构造过程。

\subsubsection{OpDef仓库}

\code{OpDef}仓库在系统首次访问时,实现了\code{OpDef}的延迟加载和注册。也就是说,对于某中类型的\code{OpDef}仓库,\code{\_InitOpDefLibrary}模块首次导入时,扫描\code{op\_list\_ascii}表示的所有\ascii{OP},并将其转换为\ascii{Protobuf}格式的\code{OpList}实例,最终将其注册到\code{OpDefLibrary}实例之中。

例如,模块\code{gen\_array\_ops}是构建版本时自动生成的,它主要完成所有\code{array\_ops}类型的\code{OpDef}的定义,并自动注册到\code{OpDefLibrary}的仓库实例中,并提供按名查找\code{OpDef}的服务接口。

\begin{leftbar}
\begin{python}
_op_def_lib = _InitOpDefLibrary()

def _InitOpDefLibrary():
  op_list = _op_def_pb2.OpList()
  _text_format.Merge(_InitOpDefLibrary.op_list_ascii, op_list)   
  op_def_lib = _op_def_library.OpDefLibrary()
  op_def_lib.add_op_list(op_list)
  return op_def_lib

_InitOpDefLibrary.op_list_ascii = """op {
  name: "ZerosLike"
  input_arg {
    name: "x"
    type_attr: "T"
  }
  output_arg {
    name: "y"
    type_attr: "T"
  }
  attr {
    name: "T"
    type: "type"
  }
}
# ignore others
"""
\end{python}
\end{leftbar}

\subsubsection{工厂方法}

如\refig{py-op-factory-and-repo}所示。当\ascii{Client}使用\ascii{OP}构造器创建一个\code{Operation}实例时,将最终调用\code{Graph.create\_op}方法,将该\code{Operation}实例注册到该图实例中。

也就是说,一方面,\code{Graph}充当\code{Operation}的工厂,负责\code{Operation}的创建职责;另一方面,\code{Graph}充当\code{Operation}的仓库,负责\code{Operation}的存储,检索,转换等操作。

这个过程常称为计算图的构造。在计算图的构造期间,并不会触发运行时的\ascii{OP}运算,它仅仅描述计算节点之间的依赖关系,并构建\ascii{DAG}图,对整个计算过程做整体规划。

\begin{figure}[!htbp]
\centering
\includegraphics[width=0.9\textwidth]{figures/py-op-factory-and-repo.png}
\caption{Graph: OP工厂 + OP仓库}
 \label{fig:py-op-factory-and-repo}
\end{figure}

\subsubsection{OP构造器}

如\refig{py-op-constructor}所示。在图构造期,\ascii{Client}使用\code{tf.zeros\_like}构造一个名为\code{ZerosLike}的\ascii{OP},该\ascii{OP}拥有一个输入,输出一个全\ascii{0}的\ascii{Tensor};其中,\code{tf.zeros\_like}常称为\ascii{OP}构造器。

然后,\ascii{OP}构造器调用一段自动生成的代码,进而转调\code{OpDefLibrary.apply\_op}方法。

\begin{figure}[!htbp]
\centering
\includegraphics[width=0.9\textwidth]{figures/py-op-constructor.png}
\caption{OP构造器与代码生成器}
 \label{fig:py-op-constructor}
\end{figure}

\subsubsection{构造OpDef与NodeDef}

然后,如\refig{py-graph-create-op}所示。\code{OpDefLibrary}根据\ascii{OP}的名字从\code{OpDefLibrary}中,找到对应\code{OpDef}实例;最终,通过\code{Graph.create\_op}的工厂方法,创建\code{NodeDef}实例,进而创建\code{Operation}实例,将其自身注册到图实例中。

\begin{figure}[!htbp]
\centering
\includegraphics[width=0.9\textwidth]{figures/py-graph-create-op.png}
\caption{创建Operation实例: 创建OpDef, NodeDef实例}
 \label{fig:py-graph-create-op}
\end{figure}

\end{content}

\section{后端C++}

\begin{content}

在\ascii{C++}后端,计算图是\ascii{TensorFlow}领域模型的核心。

\subsection{边}

\code{Edge}持有前驱节点与后驱节点,从而实现了计算图的连接。一个节点可以拥有零条或多条输入边,与可以有零条或多条输出边。一般地,计算图中存在两类边:

\begin{enum}
  \eitem{普通边:用于承载数据(以\code{Tensor}表示),表示节点间“生产者-消费者”的数据依赖关系,常用实线表示;}
  \eitem{控制依赖:不承载数据,用于表示节点间的执行依赖关系,常用虚线表示。} 
\end{enum}

\subsubsection{两个标识}

\ascii{Edge}持有两个重要的索引:

\begin{enum}
  \eitem{\code{src\_output}:表示该边为「前驱节点」的第\code{src\_output}条输出边;}
  \eitem{\code{dst\_input}:表示该边为「后驱节点」的第\code{dst\_input}条输入边。} 
\end{enum}


\begin{figure}[!htbp]
\centering
\includegraphics[width=0.9\textwidth]{figures/cc-edge-model.png}
\caption{领域对象:Edge}
 \label{fig:cc-edge-model}
\end{figure}

例如,存在两个前驱节点\code{s1, s2},都存在两条输出边;存在两个后驱节点\code{d1, d2},都存在两条输入边。

\begin{figure}[!htbp]
\centering
\includegraphics[width=0.9\textwidth]{figures/cc-edge-model-example.png}
\caption{边例子}
 \label{fig:cc-edge-model-example}
\end{figure}

\subsubsection{控制依赖}

对于控制依赖边,其\code{src\_output, dst\_input}都为\code{-1(Graph::kControlSlot)},暗喻控制依赖边不承载任何数据。

\begin{leftbar}
\begin{c++}
bool Edge::IsControlEdge() const {
   // or dst\_input\_ == Graph::kControlSlot;
   return src_output_ == Graph::kControlSlot;
}
\end{c++}
\end{leftbar}

\subsubsection{Tensor标识}

一般地,计算图的「普通边」承载\code{Tensor},并使用\code{TensorId}标识。\code{Tensor}标识由源节点的名字,及其所在边的\code{src\_output}唯一确定。

\begin{leftbar}
\begin{c++}
TensorId ::= node_name:src_output
\end{c++}
\end{leftbar}

缺省地,\code{src\_output}默认为\ascii{0};也就是说,\code{node\_name}与\code{node\_name:0}两者等价。特殊地,当\code{src\_output}等于\ascii{-1}时,表示该边为「控制依赖边」,\code{TensorId}可以标识为\code{\^node\_name},标识该边依赖于\code{node\_name}所在的节点。

\subsection{节点}

\code{Node}(节点)可以拥有零条或多条输入/输出的边,并使用\code{in\_edges, out\_edges}分别表示输入边和输出边的集合。另外,\code{Node}持有\code{NodeDef, OpDef}。其中,\code{NodeDef}包含设备分配信息,及其\ascii{OP}的属性值列表;\code{OpDef}持有\ascii{OP}的元数据,包括\ascii{OP}输入输出类型等信息。

\begin{figure}[!htbp]
\centering
\includegraphics[width=0.9\textwidth]{figures/cc-node-model.png}
\caption{领域对象:Node}
 \label{fig:cc-node-model}
\end{figure}

\subsubsection{输入边}

在输入边的集合中,可以按照索引\code{(dst\_input)}线性查找。当节点输入的边比较多时,可能会成为性能的瓶颈。依次类推,按照索引\code{(src\_output)}查找输出边,算法类同。

\begin{leftbar}
\begin{c++}
Status Node::input_edge(int idx, const Edge** e) const {
  for (auto edge : in_edges()) {
    if (edge->dst_input() == idx) {
      *e = edge;
      return Status::OK();
    }
  }
  return errors::NotFound("not found input edge ", idx);
}
\end{c++}
\end{leftbar}

\subsubsection{前驱节点}

首先通过\code{idx}索引找到输入边,然后通过输入边找到前驱节点。依次类推,按照索引查找后驱节点,算法类同。

\begin{leftbar}
\begin{c++}
Status Node::input_node(int idx, const Node** n) const {
  const Edge* e = nullptr;
  TF_RETURN_IF_ERROR(input_edge(idx, &e));
  *n = e == nullptr ? nullptr : e->src();
  return Status::OK();
}
\end{c++}
\end{leftbar}

\subsection{图}

\code{Graph}(计算图)就是节点与边的集合。计算图是一个\ascii{DAG}图,计算图的执行过程将按照\ascii{DAG}的拓扑排序,依次启动\ascii{OP}的运算。其中,如果存在多个入度为\ascii{0}的节点,\ascii{TensorFlow}运行时可以实现并发,同时执行多个\ascii{OP}的运算,提高执行效率。

\begin{figure}[!htbp]
\centering
\includegraphics[width=0.9\textwidth]{figures/cc-graph-model.png}
\caption{领域模型:图}
 \label{fig:cc-graph-model}
\end{figure}

\subsubsection{空图}

计算图的初始状态,并非是一个空图。实现添加了两个特殊的节点:\code{Source}与\code{Sink}节点,分别表示\ascii{DAG}图的起始节点与终止节点。其中,\code{Source}的\code{id}为\ascii{0},\code{Sink}的\code{id}为\ascii{1};依次论断,普通\ascii{OP}节点的\ascii{id}将大于\ascii{1}。

\code{Source}与\code{Sink}之间,通过连接「控制依赖」的边,保证计算图的执行始于\code{Source}节点,终于\code{Sink}节点。它们之前的控制依赖边,其\code{src\_output, dst\_input}值都为\ascii{-1}。

\begin{figure}[!htbp]
\centering
\includegraphics[width=0.9\textwidth]{figures/cc-empty-graph.png}
\caption{空图}
 \label{fig:cc-empty-graph}
\end{figure}

\code{Source}与\code{Sink}是两个内部实现保留的节点,其节点名称以下划线开头,分别使用\code{\_SOURCE}和\code{\_SINK}命名;并且,它们都是\code{NoOp},表示不执行任何计算。

\begin{leftbar}
\begin{c++}
Node* Graph::AddInternalNode(const char* name, int id) {
  NodeDef def;
  def.set_name(name);
  def.set_op("NoOp");

  Status status;
  Node* node = AddNode(def, &status);
  TF_CHECK_OK(status);
  CHECK_EQ(node->id(), id);
  return node;
}

Graph::Graph(const OpRegistryInterface* ops)
    : ops_(ops), arena_(8 << 10 /* 8kB */) {
  auto src  = AddInternalNode("_SOURCE", kSourceId);
  auto sink = AddInternalNode("_SINK",   kSinkId);
  AddControlEdge(src, sink);
}
\end{c++}
\end{leftbar}

习惯上,仅包含\code{Source}与\code{Sink}节点的计算图也常常称为空图。

\subsubsection{非空图}

在前端,用户使用\ascii{OP}构造器,将构造任意复杂度的计算图。对于运行时,实现将用户构造的计算图通过控制依赖的边与\code{Source/Sink}节点连接,保证计算图执行始于\code{Source}节点,终于\code{Sink}节点。

\begin{figure}[!htbp]
\centering
\includegraphics[width=0.9\textwidth]{figures/cc-non-empty-graph.png}
\caption{非空图}
 \label{fig:cc-non-empty-graph}
\end{figure}

\subsubsection{添加边}

计算图的构造过程非常简单,首先通过\code{Graph::AddNode}在图中放置节点,然后再通过\code{Graph::AddEdge}在图中放置边,实现节点之间的连接。

\begin{leftbar}
\begin{c++}
const Edge* Graph::AllocEdge() const {
  Edge* e = nullptr;
  if (free_edges_.empty()) {
    e = new (arena_.Alloc(sizeof(Edge))) Edge;
  } else {
    e = free_edges_.back();
    free_edges_.pop_back();
  }
  e->id_ = edges_.size();
  return e;
}

const Edge* Graph::AddEdge(Node* source, int x, Node* dest, int y) {
  auto e = AllocEdge();
  e->src_ = source;
  e->dst_ = dest;
  e->src_output_ = x;
  e->dst_input_ = y;

  CHECK(source->out_edges_.insert(e).second);
  CHECK(dest->in_edges_.insert(e).second);

  edges_.push_back(e);
  edge_set_.insert(e);
  return e;
}
\end{c++}
\end{leftbar}

\subsubsection{添加控制依赖边}

添加控制依赖边,则可以转发调用\code{Graph::AddEdge}实现;此时,\code{src\_output, dst\_input}都为\ascii{-1}。

\begin{leftbar}
\begin{c++}
const Edge* Graph::AddControlEdge(Node* src, Node* dst) {
  return AddEdge(src, kControlSlot, dst, kControlSlot);
}
\end{c++}
\end{leftbar}

\subsection{OpDef仓库}

同样地,\code{OpDef}仓库在\ascii{C++}系统\code{main}函数启动之前完成\code{OpDef}的加载和注册。它使用\ascii{REGISTER\_OP}宏完成\ascii{OpDef}的注册。

\begin{figure}[!htbp]
\centering
\includegraphics[width=0.9\textwidth]{figures/cc-op-repo.png}
\caption{OpDef注册:使用REGISTER\_OP}
 \label{fig:cc-op-repo}
\end{figure}

\end{content}

\section{图传递}

\begin{content}

\begin{figure}[!htbp]
\centering
\includegraphics[width=0.9\textwidth]{figures/py-graph-creation.png}
\caption{图的序列化与反序列化}
 \label{fig:py-graph-creation}
\end{figure}

\end{content}


\begin{savequote}[45mm]
\ascii{Any fool can write code that a computer can understand. Good programmers write code that humans can understand.}
\qauthor{\ascii{- Martin Flower}}
\end{savequote}

\chapter{设备} 
\label{ch:device}

\begin{content}

\end{content}

\section{设备规范}

\begin{content}

设备规范\ascii{(Device Specification)}用于描述\ascii{OP}存储或计算设备的具体位置。

\subsection{形式化}

一个设备规范可以形式化地描述为:

\begin{leftbar}
\begin{python}
DEVICE_SPEC ::= COLOCATED_NODE | PARTIAL_SPEC
COLOCATED_NODE ::= "@" NODE_NAME
PARTIAL_SPEC ::= ("/" CONSTRAINT) *
CONSTRAINT ::= ("job:" JOB_NAME)
             | ("replica:" [1-9][0-9]*)
             | ("task:" [1-9][0-9]*)
             | ( ("gpu" | "cpu") ":" ([1-9][0-9]* | "*") )
\end{python}
\end{leftbar}

\subsubsection{完整指定}

如下例所示,完整地描述某个\ascii{OP}被放置在\ascii{PS}作业,\ascii{0}号备份,\ascii{0}号任务,\ascii{GPU 0}号设备。

\begin{leftbar}
\begin{python}
/job:ps/replica:0/task:0/device:GPU:0
\end{python}
\end{leftbar}

\subsubsection{部分指定}

设备规范也可以部分指定,甚至为空。例如,下例仅描述了\code{GPU0}号设备。

\begin{leftbar}
\begin{python}
/device:GPU:0
\end{python}
\end{leftbar}

特殊地,当设备规范为空时,则表示对\ascii{OP}未实施设备约束,运行时自动选择设备放置\ascii{OP}。

\subsubsection{同位}

使用\code{COLOCATED\_NODE}指示该\ascii{OP}与指定的节点被同时放置在相同的设备上。例如,该节点与\code{other/node}放置在相同的设备上。

\begin{leftbar}
\begin{python}
@other/node  # colocate with "other/node"
\end{python}
\end{leftbar}

\subsubsection{DeviceSpec}

一个设备规范可以使用字符串,或者\code{DeviceSpec}表示。其中,\code{DeviceSpec}是一个值对象,使用如下\ascii{5}个标识确定设备规范。

\begin{enum}
  \eitem{作业名称}
  \eitem{备份索引}
  \eitem{任务索引}
  \eitem{设备类型}
  \eitem{设备索引}
\end{enum}

例如,使用\code{DeviceSpec}构造的设备规范。

\begin{leftbar}
\begin{python}
# '/job:ps/replica:0/task:0/device:CPU:0'
DeviceSpec(job="ps", replica=0, task=0, device_type="CPU", device_index=0)
\end{python}
\end{leftbar}

\subsection{上下文管理器}

常常使用上下文管理器\code{device(device\_spec)}指定\ascii{OP}设备规范,在该上下文的作用域内构造的\code{OP}在运行时将被放置在指定设备上运行。

\begin{leftbar}
\begin{python}
with g.device('/gpu:0'):
  # All OPs constructed here will be placed on GPU 0.
\end{python}
\end{leftbar}

其中,\code{device}是\code{Graph}的一个方法,它设计了一个栈式结构的上下文管理器,实现设备规范的闭包、合并、覆盖等特性。

\subsubsection{合并}

可以对两个不同范围的设备规范进行合并。

\begin{leftbar}
\begin{python}
with device("/job:ps"):
  # All OPs constructed here will be placed on PS.
  with device("/task:0/device:GPU:0"):
    # All OPs constructed here will be placed on
    # /job:ps/task:0/device:GPU:0
\end{python}
\end{leftbar}

\subsubsection{覆盖}

在合并两个相同范围的设备规范时,内部指定的设备规范具有高优先级,实现设备规范的覆盖。

\begin{leftbar}
\begin{python}
with device("/device:CPU:0"):
  # All OPs constructed here will be placed on CPU 0.
  with device("/job:ps/device:GPU:0"):
    # All OPs constructed here will be placed on
    # /job:ps/device:GPU:0
\end{python}
\end{leftbar}

\subsubsection{重置}

特殊地,当内部的设备规范置位为\code{None}时,将忽略外部所有设备规范的定义。

\begin{leftbar}
\begin{python}
with device("/device:GPU:0"):
  # All OPs constructed here will be placed on CPU 0.
  with device(None):
    # /device:GPU:0 will be ignored.
\end{python}
\end{leftbar}

\subsubsection{设备规范函数}

当指定设备规范时,常常使用字符串,或\code{DeviceSpec}进行描述。也可以使用更加灵活的\emph{设备规范函数},它提供了一种更加灵活的扩展方式指定设备规范。设备规范函数是一个回调函数,入参为\code{Operation},生成一个字符串格式的设备规范。

\begin{leftbar}
\begin{python}
def matmul_on_gpu(n):
 if n.type == "MatMul":
   return "/gpu:0"
 else:
   return "/cpu:0"

with g.device(matmul_on_gpu):
  # All OPs of type "MatMul" constructed in this context
  # will be placed on GPU 0; all other OPs will be placed
  # on CPU 0.
\end{python}
\end{leftbar}

\subsubsection{实现}

\code{Graph.device(spec)}实现了一个栈式结构的上下文管理器,它可以接受字符串格式的设备规范,或者\emph{设备规范函数}。事实上,当给\code{device}函数传递字符串,或者\code{DeviceSpec}时,首先会对该字符串,或\code{DeviceSpec}做一个简单的适配,统一转换为设备规范函数。

\begin{leftbar}
\begin{python}
class Graph(object):
  def device(self, device_name_or_func):
    def to_device_func():
      if (device_name_or_func is not None
          and not callable(device_name_or_func)):
        return pydev.merge_device(device_name_or_func)
      else:
        return device_name_or_func

    try:
      self._device_function_stack.append(to_device_func())
      yield
    finally:
      self._device_function_stack.pop()
\end{python}
\end{leftbar}

当用户使用\code{device}时,未显式地指定图实例,则隐式地使用全局唯一的默认图实例。也就是说,\code{tf.device(spec)}函数事实上是对\code{get\_default\_graph().device(spec)}的一个简单包装。

\begin{leftbar}
\begin{python}
# tensorflow/python/framework/ops.py
def device(device_name_or_function):
  return get_default_graph().device(device_name_or_function)
\end{python}
\end{leftbar}

\subsubsection{应用}

在\code{Graph.device}实现中,\code{pydev.merge\_device}生成一个设备规范函数。该设备函数以输入的\code{spec}创建新的副本,通过调用\code{copy\_spec.merge\_from(current\_device)},合并既有的\code{node\_def.device}设备规范,并且\code{node\_def.device}具有更高的优先级。实现较为隐晦,为什么\code{node\_def.device}具有较高优先级呢?这取决于\code{\_apply\_device\_functions}实现覆盖、合并、重置的需求。

\begin{leftbar}
\begin{python}
def merge_device(spec):
  # replace string to DeviceSpec
  if not isinstance(spec, DeviceSpec):
    spec = DeviceSpec.from_string(spec or "")

  # returns a device function that merges devices specifications
  def _device_function(node_def):
    current_device = DeviceSpec.from_string(node_def.device or "")
    copy_spec = copy.copy(spec)

    # IMPORTANT: `node\_def.device` takes precedence.
    copy_spec.merge_from(current_device)      
    return copy_spec
  return _device_function
\end{python}
\end{leftbar}

当\code{Graph.create\_op}时,将调用\code{\_apply\_device\_functions}设置\code{NodeDef}的设备规范。它将对\code{\_device\_function\_stack}依次实施出栈操作,并调用相应的设备规范函数,并将结果直接设置为\code{NodeDef}的设备规范。如此,内嵌指定的\code{device}具有较高优先级,实现了合并、覆盖、重置外围指定的\code{device}。

\begin{leftbar}
\begin{python}
class Graph(object):
  def _apply_device_functions(self, op):
    for device_function in reversed(self._device_function_stack):
      if device_function is None:
        break
      # IMPORTANT: `node\_def.device` takes precedence.  
      op._set_device(device_function(op)) 
\end{python}
\end{leftbar}

\end{content}

\begin{savequote}[45mm]
\ascii{Any fool can write code that a computer can understand. Good programmers write code that humans can understand.}
\qauthor{\ascii{- Martin Flower}}
\end{savequote}

\chapter{会话} 
\label{ch:session}

\begin{content}

客户端以\code{Session}为桥梁,与后台计算引擎建立连接,并启动计算图的执行过程。其中,通过调用\code{Session.run}将触发\ascii{TensorFlow}的一次计算\ascii{(Step)}。

事实上,\code{Session}建立了执行计算图的闭包环境,它封装了\ascii{OP}计算,及其\ascii{Tensor}求值的计算环境。

\end{content}

\section{资源管理}

\begin{content}

在\code{Session}的生命周期中,将根据计算图的计算需求,按需分配系统资源,包括变量,队列,读取器等。

\subsection{关闭会话}

当计算完成后,需要确保\code{Session}被安全地关闭,以便安全释放所管理的系统资源。

\begin{leftbar}
\begin{python}
sess = tf.Session()
sess.run(targets)
sess.close()
\end{python}
\end{leftbar}

\subsection{上下文管理器}

一般地,常常使用上下文管理器创建\code{Session},使得\code{Session}在计算完成后,能够自动关闭,确保资源安全性地被释放。

\begin{leftbar}
\begin{python}
with tf.Session() as sess:
  sess.run(targets)
\end{python}
\end{leftbar}

\subsection{图实例}

一个\code{Session}实例,只能运行一个图实例;但是,一个图实例,可以运行在多个\code{Session}实例中。如果尝试在同一个\code{Session}运行另外一个图实例,必须先关闭\code{Session}(不必销毁),再启动新图的计算过程。

虽然一个\code{Session}实例,只能运行一个图实例。但是,可以\code{Session}是一个线程安全的类,可以并发地执行该图实例上的不同子图。例如,一个典型的机器学习训练模型中,可以使用同一个\code{Session}实例,并发地运行输入子图,训练子图,及其\ascii{Checkpoint}子图。

\subsubsection{引用计数器}

为了提高效率,避免计算图频繁地创建与销毁,存在一种实现上的优化技术。在图实例中维护一个\code{Session}的引用计数器,当且仅当\code{Session}的数目为零时,才真正地销毁图实例。

\begin{figure}[!htbp]
\centering
\includegraphics[width=0.7\textwidth]{figures/tf-graph-session-relation.png}
\caption{优化技术:会话实例的引用计数器}
 \label{fig:tf-graph-session-relation}
\end{figure}

\subsubsection{数据结构}

此处,摘取\code{TF\_Graph}部分关于\code{Session}引用计数器技术的关键字段;其中,\code{TF\_Graph}结构体定义于\ascii{C API}的头文件。

\begin{leftbar}
\begin{c++}
struct TF_Graph {
  TF_Graph();

  tensorflow::mutex mu;
  tensorflow::Graph graph GUARDED_BY(mu);

  // TF\_Graph may only and must be deleted when
  // num\_sessions == 0 and delete\_requested == true

  // num\_sessions incremented by TF\_NewSession, 
  // and decremented by TF\_DeleteSession.
  int num_sessions GUARDED_BY(mu);
  bool delete_requested GUARDED_BY(mu);
};
\end{c++}
\end{leftbar}

同理,\code{TF\_Session}持有一个二元组:\code{<tensorflow::Sesssion, TF\_Graph>},它们之间是一对一的关系。其中,\code{tensorflow::Sesssion}是\ascii{C++}客户端侧的会话实例。

\begin{leftbar}
\begin{c++}
struct TF_Session {
  TF_Session(tensorflow::Session* s, TF_Graph* g)
      : session(s), graph(g), last_num_graph_nodes(0) {}
  tensorflow::Session* session;
  TF_Graph* graph;
  tensorflow::mutex mu;
  int last_num_graph_nodes;
};
\end{c++}
\end{leftbar}

\subsubsection{创建会话}

\begin{leftbar}
\begin{c++}
TF_Session* TF_NewSession(TF_Graph* graph, const TF_SessionOptions* opt,
                          TF_Status* status) {
  Session* session;
  status->status = NewSession(opt->options, &session);
  if (status->status.ok()) {
    if (graph != nullptr) {
      mutex_lock l(graph->mu);
      graph->num_sessions += 1;
    }
    return new TF_Session(session, graph);
  } else {
    DCHECK_EQ(nullptr, session);
    return nullptr;
  }
}
\end{c++}
\end{leftbar}

\subsubsection{销毁会话}

\begin{leftbar}
\begin{c++}
void TF_DeleteSession(TF_Session* s, TF_Status* status) {
  status->status = Status::OK();
  TF_Graph* const graph = s->graph;
  if (graph != nullptr) {
    graph->mu.lock();
    graph->num_sessions -= 1;
    const bool del = graph->delete_requested && graph->num_sessions == 0;
    graph->mu.unlock();
    if (del) delete graph;
  }
  delete s->session;
  delete s;
}
\end{c++}
\end{leftbar}

\section{默认会话}

\begin{content}

通过调用\ascii{Session.as\_default()},将该\code{Session}置为默认\code{Session},同时它返回了一个上下文管理器。在默认\code{Session}的上文中,可以直接实施\ascii{OP}的运算,或者\ascii{Tensor}的求值。

\begin{leftbar}
\begin{python}
hello = tf.constant('hello, world')

sess = tf.Session()  
with sess.as_default():
  print(hello.eval())
sess.close()
\end{python}
\end{leftbar}

但是,\code{Session.as\_default()}并不会自动关闭\code{Session},需要用户显式地调用\code{Session.close}方法。

\subsection{张量求值}

如上例代码,\code{hello.eval()}等价于\code{tf.get\_default\_session().run(hello)}。其中,\code{Tensor.eval}如下代码实现。

\begin{leftbar}
\begin{python}
class Tensor(_TensorLike):
  def eval(self, feed_dict=None, session=None):
    if session is None:
      session = get_default_session()
    return session.run(tensors, feed_dict)
\end{python}
\end{leftbar}

\subsection{OP运算}

同理,当用户未显式提供\code{Session},\code{Operation.run}将自动获取默认的\code{Session}实例,并按照当前\ascii{OP}的依赖关系,以某个特定的拓扑排序执行该计算子图。

\begin{leftbar}
\begin{python}
class Operation(object):
  def run(self, feed_dict=None, session=None):
    if session is None:
      session = tf.get_default_session()
    session.run(self, feed_dict)
\end{python}
\end{leftbar}

\subsection{线程相关}

默认会话仅仅对当前线程有效,以便在当前线程追踪Session的调用栈。如果在新的线程中使用默认会话,需要在线程函数中通过调用\code{as\_default}将\code{Session}置为默认会话。

事实上,在\ascii{TensorFlow}运行时维护了一个\code{Session}的本地线程栈,实现默认\code{Session}的自动管理。

\begin{leftbar}
\begin{python}
_default_session_stack = _DefaultStack()

def get_default_session(session):
  return _default_session_stack.get_default(session)
\end{python}
\end{leftbar}

其中,\code{\_DefaultStack}表示栈的数据结构。

\begin{leftbar}
\begin{python}
class _DefaultStack(threading.local):
  def __init__(self):
    super(_DefaultStack, self).__init__()
    self.stack = []

  def get_default(self):
    return self.stack[-1] if len(self.stack) >= 1 else None

  @contextlib.contextmanager
  def get_controller(self, default):
    try:
      self.stack.append(default)
      yield default
    finally:
      self.stack.remove(default)
\end{python}
\end{leftbar}

\end{content}

\section{会话类型}

\begin{content}

一般地,存在两种基本的会话类型:\code{Session}与\code{InteractiveSession}。后者常常用于交互式环境,它在构造期间将其自身置为默认,简化默认会话的管理过程。

此外,两者在运行时的配置也存在差异。例如,\code{InteractiveSession}将\code{GPUOptions.allow\_growth}置为\code{True},避免在实验环境中独占整个GPU的存储资源。

\begin{figure}[!htbp]
\centering
\includegraphics[width=0.7\textwidth]{figures/py-session-hierarchy.png}
\caption{Session:类层次结构}
 \label{fig:py-session-hierarchy}
\end{figure}

\subsection{Session}

\code{Session}继承\code{BaseSession},并增加了默认图与默认会话的上下文管理器的功能,保证系统资源的安全释放。

一般地,使用\code{with}进入会话的上下文管理器,并自动切换默认图与默认会话的上下文;退出\code{with}语句时,将自动关闭默认图与默认会话的上下文,并自动关闭会话。

\begin{leftbar}
\begin{python}
class Session(BaseSession):
  def __init__(self, target='', graph=None, config=None):
    super(Session, self).__init__(target, graph, config=config)
    self._default_graph_context_manager = None
    self._default_session_context_manager = None

  def __enter__(self):
    self._default_graph_context_manager = self.graph.as_default()
    self._default_session_context_manager = self.as_default()

    self._default_graph_context_manager.__enter__()
    return self._default_session_context_manager.__enter__()

  def __exit__(self, exec_type, exec_value, exec_tb):
    self._default_session_context_manager.__exit__(
        exec_type, exec_value, exec_tb)
    self._default_graph_context_manager.__exit__(
        exec_type, exec_value, exec_tb)

    self._default_session_context_manager = None
    self._default_graph_context_manager = None

    self.close()
\end{python}
\end{leftbar}

\subsection{InteractiveSession}

与\code{Session}不同,\code{InteractiveSession}在构造期间将其自身置为默认,并实现默认图与默认会话的自动切换。与此相反,\code{Session}必须借助于\code{with}语句才能完成该功能。在交互式环境中,\code{InteractiveSession}简化了用户管理默认图和默认会话的过程。

同理,\code{InteractiveSession}在计算完成后需要显式地关闭,以便安全地释放其所占用的系统资源。

\begin{leftbar}
\begin{python}
class InteractiveSession(BaseSession):
  def __init__(self, target='', graph=None, config=None):
    super(InteractiveSession, self).__init__(target, graph, config)

    self._default_session_context_manager = self.as_default()
    self._default_session_context_manager.__enter__()

    self._default_graph_context_manager = graph.as_default()
    self._default_graph_context_manager.__enter__()

  def close(self):
    super(InteractiveSession, self).close()
    self._default_graph.__exit__(None, None, None)
    self._default_session.__exit__(None, None, None)
\end{python}
\end{leftbar}

\subsection{BaseSession}

\code{BaseSession}是两者的基类,它主要实现会话的创建,关闭,执行,销毁等管理生命周期的操作;它与后台计算引擎相连接,实现前后端计算的交互。

\subsubsection{创建会话}

通过调用\ascii{C API}的接口,\code{self.\_session}直接持有后台计算引擎的会话句柄,后期执行计算图,关闭会话等操作都以此句柄为标识。

\begin{leftbar}
\begin{python}
class BaseSession(SessionInterface):
  def __init__(self, target='', graph=None, config=None):
    # ignore implements...
    with errors.raise_exception_on_not_ok_status() as status:
      self._session = 
        tf_session.TF_NewDeprecatedSession(opts, status)
\end{python}
\end{leftbar}

\subsubsection{执行计算图}

通过调用\code{run}接口,实现计算图的一次计算。它首先通过\code{tf\_session.TF\_ExtendGraph}将图注册给后台计算引擎,然后再通过调用\code{tf\_session.TF\_Run}启动计算图的执行。

\begin{leftbar}
\begin{python}
class BaseSession(SessionInterface):
  def run(self, 
    fetches, feed_dict=None, options=None, run_metadata=None):
    self._extend_graph()
    with errors.raise_exception_on_not_ok_status() as status:
      return tf_session.TF_Run(session, 
        options, feed_dict, fetch_list, 
        target_list, status, run_metadata)
  
  def _extend_graph(self):
    with errors.raise_exception_on_not_ok_status() as status:
      tf_session.TF_ExtendGraph(self._session,
        graph_def.SerializeToString(), status)  
\end{python}
\end{leftbar}

\subsubsection{关闭会话}

\begin{leftbar}
\begin{python}
class BaseSession(SessionInterface):
  def close(self):
    with errors.raise_exception_on_not_ok_status() as status:
      tf_session.TF_CloseDeprecatedSession(self._session, status)
\end{python}
\end{leftbar}

\subsubsection{销毁会话}

\begin{leftbar}
\begin{python}
class BaseSession(SessionInterface):
  def __del__(self):
    try:
      status = tf_session.TF_NewStatus()
      tf_session.TF_DeleteDeprecatedSession(self._session, status)
    finally:
      tf_session.TF_DeleteStatus(status)
\end{python}
\end{leftbar}

\begin{content}
\begin{savequote}[45mm]
\ascii{Any fool can write code that a computer can understand. Good programmers write code that humans can understand.}
\qauthor{\ascii{- Martin Flower}}
\end{savequote}

\chapter{变量} 
\label{ch:variable}

\begin{content}

\ascii{Variable}是一个特殊的\ascii{OP},它拥有状态\ascii{(Stateful)}。从实现技术探究,\ascii{Variable}的\ascii{Kernel}实现直接持有一个\ascii{Tensor}实例,其生命周期与变量一致。相对于普通的\ascii{Tensor}实例,其生命周期仅对本次迭代\ascii{(Step)}有效;而\ascii{Variable}对多个迭代都有效,甚至可以存储到文件系统,或从文件系统中恢复。

\end{content}

\section{实战:线性模型}

\begin{content}

以一个简单的线性模型为例(为了简化问题,此处省略了训练子图)。首先,使用\code{tf.placeholder}定义模型的输入,然后定义了两个全局变量,同时它们都是训练参数,最后定义了一个简单的线性模型。

\begin{leftbar}
\begin{python}
x  = tf.placeholder(tf.float32, [None, 784])
W = tf.Variable(tf.zeros([784,10]), name='W')
b = tf.Variable(tf.zeros([10]), name='b') 
y = tf.matmul(x, W) + b
\end{python}
\end{leftbar}

在使用变量之前,必须对变量进行初始化。按照习惯用法,使用\code{tf.global\_variables}
\code{\_initializer()}将所有全局变量的初始化器汇总,并对其进行初始化。

\begin{leftbar}
\begin{python}
init = tf.global_variables_initializer()

with tf.Session() as sess:
  sess.run(init)
\end{python}
\end{leftbar}

按照既有经验,其计算图大致如\refig{tf-linear-model}所示。

\begin{figure}[!h]
\centering
\includegraphics[width=0.6\textwidth]{figures/tf-linear-model.png}
\caption{计算图: 线性加权和}
 \label{fig:tf-linear-model}
\end{figure}

事实上,正如\refig{tf-real-linear-model}所示,实际的计算图要复杂得多,让我们从头说起。

\begin{figure}[!h]
\centering
\includegraphics[width=1.0\textwidth]{figures/tf-real-linear-model.png}
\caption{计算图: 线性加权和}
 \label{fig:tf-real-linear-model}
\end{figure}

\end{content}

\section{初始化模型}

\begin{content}

\ascii{Variable}是一个特殊的OP,它拥有状态\ascii{(Stateful)}。如果从实现技术探究,\ascii{Variable}的K\ascii{ernel}实现直接持有一个\ascii{Tensor}实例,其生命周期与\ascii{Variable}一致。相对于普通的\ascii{Tensor}实例,其生命周期仅对本次迭代\ascii{(Step)}有效;而\ascii{Variable}对多个迭代\ascii{(Step)}都有效,甚至可以持久化到文件系统,或从文件系统中恢复。

\subsection{操作变量}

存在几个操作\ascii{Variable}的特殊OP用于修改变量的值,例如\ascii{Assign, AssignAdd}等。\ascii{Variable}所持有的\ascii{Tensor}以引用的方式输入到\ascii{Assign}中,\ascii{Assign}根据初始值\ascii{(Initial Value)}或新值,就地修改\ascii{Tensor}内部的值,最后以引用的方式输出该\ascii{Tensor}。

从设计角度看,\ascii{Variable}可以看做\ascii{Tensor}的包装器,\ascii{Tensor}所支持的所有操作都被\ascii{Variable}重载实现。也就是说,\ascii{Variable}可以出现在\ascii{Tensor}的所有地方。例如,

\begin{leftbar}
\begin{python}
# Create a variable
W = tf.Variable(tf.zeros([784,10]), name='W')

# Use the variable in the graph like any Tensor.
y = tf.matmul(x, W)

# The overloaded operators are available too.
z = tf.sigmoid(w + y)

# Assign a new value to the variable with assign/assign\_add.
w.assign(w + 1.0)
w.assign_add(1.0)
\end{python}
\end{leftbar}

\subsection{初始值}

一般地,在使用变量之前,必须对变量进行初始化。事实上,\ascii{TensorFlow}设计了一个精巧的变量初始化模型。\ascii{Variable}根据初始值\ascii{(Initial Value)}推演\ascii{Variable}的数据类型,并确定\ascii{Tensor}的形状\ascii{(Shape)}。

例如,\code{tf.zeros}称为\ascii{Variable}的初始值,它确定了\ascii{Variable}的类型为\code{int32},且\ascii{Shape}为\code{[784, 10]}。

\begin{leftbar}
\begin{python}
# Create a variable.
W = tf.Variable(tf.zeros([784,10]), name='W')
\end{python}
\end{leftbar}

如下表所示,构造变量初始值的常见\ascii{OP}包括:

\subsection{初始化器}

另外,变量通过初始化器\ascii{(Initializer)}在初始化期间,将初始化值赋予\ascii{Variable}内部所持有\ascii{Tensor},完成\ascii{Variable}的就地修改。

在变量使用之前,必须保证变量被初始化器已初始化。事实上,变量初始化过程,即运行变量的初始化器。

证如上例\code{W}的定义,可以如下完成\code{W}的初始化。此处,\code{W.initializer}实际上为\ascii{Assign}的\ascii{OP},这是\ascii{Variable}默认的初始化器。

\begin{leftbar}
\begin{python}
# Run the variable initializer.
with tf.Session() as sess:
  sess.run(W.initializer)
\end{python}
\end{leftbar}

一旦完成\ascii{Variable}的初始化,其类型与值得以确定。随后可以使用\ascii{Assign}族的\ascii{OP}(例如\ascii{Assign, AssignAdd}等)修改\ascii{Variable}的值。

需要注意的是,在\ascii{TensorBoard}中展示\ascii{Assign}的输入,其边使用特殊的\ascii{ref}标识。数据流向与之刚好相反,否则计算图必然出现环,显然违反了\ascii{DAG}(有向无环图)的基本需求。

\subsection{快照}

如果要读取变量的值,则通过\ascii{Identity}恒等变化,直接输出变量所持有的\ascii{Tensor}。\ascii{Identity}去除了\ascii{Variable}的引用标识,同时也避免了内存拷贝。

\ascii{Identity}操作\ascii{Variable}常称为一个快照\ascii{(Snapshot)},表示\ascii{Variable}当前的值。

事实上,通过\ascii{Identity}将\ascii{Variable}转变为普通的\ascii{Tensor},使得它能够兼容所有\ascii{Tensor}的操作。


\subsection{变量子图}

例如,变量\ascii{W}的定义如下。

\begin{leftbar}
\begin{python}
W = tf.Variable(tf.zeros([784,10]), name='W')
\end{python}
\end{leftbar}

\code{tf.zeros([784,10])}常称为初始值,它通过初始化器\ascii{Assign},将\code{W}内部持有的\ascii{Tensor}以引用的形式就地修改为该初始值;同时,\ascii{Identity}去除了\ascii{Variable}的引用标识,实现了\ascii{Variable}的读取。

\begin{figure}[!h]
\centering
\includegraphics[width=0.8\textwidth]{figures/variable-initialization-model.png}
\caption{变量子图}
 \label{fig:variable-initialization-model}
\end{figure}


\subsection{初始化过程}

更为常见的是,通过调用\code{tf.global\_variables\_initializer()}将所有变量的初始化器进行汇总,然后启动\ascii{Session}运行该\ascii{OP}。

\begin{leftbar}
\begin{python}
init = tf.global_variables_initializer()
\end{python}
\end{leftbar}

事实上,搜集所有全局变量的初始化器的\ascii{OP}是一个\ascii{NoOp},即不存在输入,也不存在输出。所有变量的初始化器通过控制依赖边与该\ascii{NoOp}相连,保证所有的全局变量被初始化。

\begin{figure}[!h]
\centering
\includegraphics[width=0.8\textwidth]{figures/variable-initialization-no-op.png}
\caption{初始化OP}
 \label{fig:variable-initialization-no-op}
\end{figure}

\subsection{同位关系}

同位关系是一种特殊的设备约束关系。显而易见,\ascii{Assign, Identity}这两个\ascii{OP}与\ascii{Variable}关系极其紧密,分别实现了变量的修改与读取。因此,它们必须与\ascii{Variable}在同一个设备上执行;这样的关系,常称为同位关系\ascii{(Colocation)}。

可以在\ascii{Assign/Identity}节点上指定\code{\_class}属性值:\code{[s: "loc:@W"]},它表示这两个\ascii{OP}与\code{W}放在同一个设备上运行。

例如,以\code{W/read}节点为例,该节点增加了\code{\_class}属性,指示与\code{W}的同位关系。

\begin{leftbar}
\begin{python}
node {
  name: "W/read"
  op: "Identity"
  input: "W"
  attr {
    key: "T"
    value {
      type: DT_FLOAT
    }
  }
  attr {
    key: "_class"
    value {
      list {
        s: "loc:@W"
      }
    }
  }
}
\end{python}
\end{leftbar}

\subsection{初始化依赖}

如果一个变量初始化需要依赖于另外一个变量的初始值,则需要特殊地处理。例如,变量\code{V}的初始值依赖于\code{W}的初始值,可以通过\code{W.initialized\_value()}指定。

\begin{leftbar}
\begin{python}
W = tf.Variable(tf.zeros([784,10]), name='W')
V = tf.Variable(W.initialized_value(), name='V')
\end{python}
\end{leftbar}

事实上,两者通过\ascii{Identity}衔接,并显式地添加了依赖控制边,保证\code{W}在\code{V}之前初始化。此处,存在两个\ascii{Identity}的\ascii{OP},但职责不一样,它们分别完成初始化依赖和变量读取。


\begin{figure}[!h]
\centering
\includegraphics[width=0.8\textwidth]{figures/variable-initialization-dependency-1.png}
\caption{初始化依赖}
 \label{fig:variable-initialization-dependency-1}
\end{figure}

同样地,可以通过调用\code{tf.global\_variables\_initializer()}将变量的所有初始化器进行汇总,然后启动\ascii{Session}完成所有变量的初始化。

\begin{leftbar}
\begin{python}
init = tf.global_variables_initializer()
\end{python}
\end{leftbar}

按照依赖关系,因为增加了\code{W/Assign}与\code{Identity}之间的控制依赖边,从而巧妙地实现了\code{W}在\code{V}之前完成初始化,并通过\code{W}当前的初始化值,最终完成\code{V}的初始化。

\begin{figure}[!h]
\centering
\includegraphics[width=0.8\textwidth]{figures/variable-initialization-dependency-2.png}
\caption{初始化OP}
 \label{fig:variable-initialization-dependency-2}
\end{figure}

\subsection{初始化器列表}

可以使用\code{variables\_initializer}构建变量列表的初始化器列表。其中,\code{group}将构造一个仅控制依赖于\code{\_initialier\_list()}的\ascii{NoOP}。

\begin{leftbar}
\begin{python}
def variables_initializer(var_list, name="init"):
  def _initialier_list():
    return *[v.initializer for v in var_list]
  return control_flow_ops.group(_initialier_list(), name=name)
\end{python}
\end{leftbar}

例如,全局变量列表的初始化器列表可以如下构造。

\begin{leftbar}
\begin{python}
def global_variables_initializer():
  return variables_initializer(global_variables())
\end{python}
\end{leftbar}

\end{content}

\section{变量分组}

\begin{content}

默认地,\ascii{Variable}被划分在全局变量和训练变量的集合中。正如上例,\code{W, V}自动划分至全局变量和训练变量的集合中。

\subsection{全局变量}

可以通过\code{tf.global\_variables()}方便地检索全局变量的集合。在分布式环境中,全局变量能在不同的进程间实现参数共享。

\begin{leftbar}
\begin{python}
def global_variables():
  return ops.get_collection(ops.GraphKeys.GLOBAL_VARIABLES)
\end{python}
\end{leftbar}

\subsection{本地变量}

可以通过\code{tf.local\_variables()}方便地检索本地变量的集合。

\begin{leftbar}
\begin{python}
def local_variables():
  return ops.get_collection(ops.GraphKeys.LOCAL_VARIABLES)
\end{python}
\end{leftbar}

可以使用\code{local\_variable}的语法糖,构建一个本地变量。

\begin{leftbar}
\begin{python}
def local_variable(initial_value, validate_shape=True, name=None):
  return variables.Variable(
      initial_value, trainable=False,
      collections=[ops.GraphKeys.LOCAL_VARIABLES],
      validate_shape=validate_shape, name=name)
\end{python}
\end{leftbar}

本地变量表示进程内的共享变量,它通常不需要做断点恢复\ascii{(Checkpoint)},仅用于临时的计数器的用途。例如,在分布式环境中,使用本地变量记录该进程已读数据的\ascii{Epoch}数目。

\subsection{训练变量}

可以通过\code{tf.trainable\_variables()}检索训练变量的集合。在机器学习中,训练变量表示模型参数。

\begin{leftbar}
\begin{python}
def trainable_variables():
  return ops.get_collection(ops.GraphKeys.TRAINABLE_VARIABLES)
\end{python}
\end{leftbar}

\subsection{global\_step}

\code{global\_step}是一个特殊的\ascii{Variable},它不是训练变量,但它是一个全局变量。在分布式环境中,\code{global\_step}常用于追踪已运行\ascii{step}的次数,并在不同进程间实现数据的同步。

创建一个\code{global\_step}可以使用如下函数:

\begin{leftbar}
\begin{python}
def create_global_step(graph=None):
  graph = ops.get_default_graph() if graph is None else graph
  with graph.as_default() as g, g.name_scope(None):
    collections = [GLOBAL_VARIABLES, GLOBAL_STEP]
    return variable(
        GLOBAL_STEP,
        shape=[],
        dtype=dtypes.int64,
        initializer=init_ops.zeros_initializer(),
        trainable=False,
        collections=collections)
\end{python}
\end{leftbar}

\end{content}

\section{源码分析:构造变量}

\begin{content}

为了简化代码实现,此处对\ascii{Variable}做了简单的重构。

\begin{leftbar}
\begin{python}
class Variable(object):
  def __init__(self, initial_value=None, trainable=True,
    collections=None, name=None, dtype=None):
    with ops.name_scope(name, "Variable", [initial_value]) as name:
      self._cons_initial_value(initial_value, dtype)
      self._cons_variable(name)
      self._cons_initializer()
      self._cons_snapshot()
    self._cons_collections(trainable, collections)
\end{python}
\end{leftbar}

构造\ascii{Variable}实例,基本包括如下几个步骤:

\subsection{构造初始值}

\begin{leftbar}
\begin{python}
  def _cons_initial_value(self, initial_value, dtype):
    self._initial_value = ops.convert_to_tensor(
        initial_value, name="initial_value", dtype=dtype)
\end{python}
\end{leftbar}

\subsection{构造变量OP}

\ascii{Variable}根据初始值的类型和大小完成自动推演。

\begin{leftbar}
\begin{python}
  def _cons_variable(self, name):
    self._variable = state_ops.variable_op_v2(
      self._initial_value.get_shape(),
      self._initial_value.dtype.base_dtype,
      name=name)
\end{python}
\end{leftbar}

\subsection{构造初始化器}

\ascii{Variable}的初始化器本质上是一个\ascii{Assign},它持有\ascii{Variable}的引用,并使用初始值就地修改变量本身。

\begin{leftbar}
\begin{python}
  def _cons_initializer(self):
    self._initializer_op = state_ops.assign(
      self._variable,
      self._initial_value).op
\end{python}
\end{leftbar}

\subsection{构造快照}

\ascii{Variable}的快照本质上是一个\ascii{Identity},表示\ascii{Variable}的当前值。

\begin{leftbar}
\begin{python}
  def _cons_snapshot(self):
    with ops.colocate_with(self._variable.op):
      self._snapshot = array_ops.identity(
        self._variable, name="read")
\end{python}
\end{leftbar}

\subsection{变量分组}

默认地,\ascii{Variable}被划分在全局变量的集合中;如果\code{trainable}为真,则表示该变量为训练参数,并将其划分到训练变量的集合中。

\begin{leftbar}
\begin{python}
  def _cons_collections(self, trainable, collections)
    if collections is None:
      collections = [GLOBAL_VARIABLES]
    if trainable and TRAINABLE_VARIABLES not in collections:
      collections = list(collections) + [TRAINABLE_VARIABLES]
    ops.add_to_collections(collections, self)
\end{python}
\end{leftbar}

\end{content}
\begin{savequote}[45mm]
\ascii{Any fool can write code that a computer can understand. Good programmers write code that humans can understand.}
\qauthor{\ascii{- Martin Flower}}
\end{savequote}

\chapter{队列} 
\label{ch:queue}

\begin{content}

\ascii{TensorFlow}的\code{Session}是线程安全的。也就是说,多个线程可以使用同一个\code{Session}实例,并发地执行同一个图实例的不同\ascii{OP};\ascii{TensorFlow}执行引擎会根据输入与输出对图实施剪枝,得到一个最小依赖的子图。

因此,通过多线程并使用同一个\code{Session}实例,并发地执行同一个图实例的不同\ascii{OP},最终实现的效果是子图之间的并发执行。

对于典型的模型训练,可以充分发挥\code{Session}多线程的并发能力,提升训练的性能。例如,输入子图运行在一个单独的线程中,用于准备样本数据;而训练子图则运行在另外一个单独的线程中,并按照\code{batch\_size}的大小一个批次取走训练样本,并启动迭代的训练过程。

本文将讲解上述并发模型中的基础设施,包括队列,多线程的协调器,及其控制\code{Enqueue OP}执行的\ascii{QueueRunner}。

\end{content}

\section{队列}

\begin{content}

在\ascii{TensorFlow}的执行引擎中,\ascii{Queue}是一种控制异步计算的强大工具。特殊地,\ascii{Queue}是一种特殊的\ascii{OP},与\ascii{Variable}类似,它是一类有状态的\ascii{OP}。

与之类似,\ascii{Variable}拥有关联的\ascii{Assign}等修改其状态的\ascii{OP},\ascii{Queue}也有与之关联的\ascii{OP},例如\code{Enqueue,Dequeue,EnqueueMany,DequeueMany}等\ascii{OP},它们都能直接修改\ascii{Queue}的状态。

\subsection{FIFOQueue}

举一个简单例子。首先,构造了一个\code{FIFOQueue}队列;然后,在计算图中添加了一个\code{EnqueueMany},该\ascii{OP}用于在队列头部追加\ascii{1}个或多个元素;其次,再添加一个出队的\code{Dequeue};最后,将出队元素的值增加\ascii{1},再将其结果入队。在启动计算图执行之前,计算图的构造如下图所示。

\begin{figure}[!h]
\centering
\includegraphics[width=0.9\textwidth]{figures/py-queue-example-1.png}
\caption{图构造期}
 \label{fig:py-queue-example-1}
\end{figure}

执行\code{EnqueueMany}操作后,计算图的状态如下图所示。

\begin{figure}[!h]
\centering
\includegraphics[width=0.9\textwidth]{figures/py-queue-example-2.png}
\caption{图执行期:执行一次EnqueueMany}
 \label{fig:py-queue-example-2}
\end{figure}


执行第一步\code{Enqueue}后,计算图的状态如下图所示。

\begin{figure}[!h]
\centering
\includegraphics[width=0.9\textwidth]{figures/py-queue-example-3.png}
\caption{图执行期:执行一次Enqueue}
 \label{fig:py-queue-example-3}
\end{figure}

\subsection{用途}

队列在模型训练中扮演重要角色,后文将讲述数据加载的\ascii{Pipeline},训练模型常常使用\code{RandomShuffleQueue}为其准备样本数据。为了提高\ascii{IO}的吞吐率,可以使用多线程,并发地将样本数据追加到样本队列中;与此同时,训练模型的线程迭代执行\code{train\_op}时,一次获取\code{batch\_size}大小的批次样本数据。

显而易见,队列在\ascii{Pipeline}过程中扮演了异步协调和数据交换的功能,这给\ascii{Pipeline}的设计和实现带来很大的弹性空间。

需要注意的是,为了使得队列在多线程最大化发挥作用,需要解决两个棘手的问题:

\begin{enum}
  \eitem{如何同时停止所有的线程,及其处理异常报告?}
  \eitem{如何并发地向队列中追加样本数据?} 
\end{enum}

因此,\ascii{TensorFlow}设计了\code{tf.train.Coordinator}和\code{tf.train.QueueRunner}两个类,分别解决上述两个问题。

这两个类相辅相成,\code{Coordinator}协调多个线程同时停止运行,并且向等待停止通知的主程序报告异常;而\code{QueueRunner}创建了一组线程,并协作多个入队\ascii{OP}(例如\code{Enqueue,EnqueueMany})的执行。

\end{content}

\section{协调器}

\begin{content}

\code{Coordinator}提供了一种同时停止一组线程执行的简单机制。它拥有\ascii{3}个重要的方法:

\begin{enum}
\eitem{\code{should\_stop}: 判断当前线程是否应该退出}
\eitem{\code{request\_stop}: 请求所有线程停止执行}
\eitem{\code{join}: 等待所有线程停止执行}
\end{enum}

\subsection{使用方法}

一般地,主程序常常使用如下模式使用\code{Coordinator}。

\begin{leftbar}
\begin{python}
# Create a coordinator.
coord = tf.train.Coordinator()

# Create 10 threads that run 'MyLoop()'
threads = [threading.Thread(target=MyLoop, args=(coord,)) 
          for i in xrange(10)]

# Start the threads.
for t in threads:
  t.start()
  
# wait for all of them to stop
coord.join(threads)
\end{python}
\end{leftbar}

任何子线程,都可以通过调用\code{coord.request\_stop},通知其他线程停止执行。因此,每个线程的迭代执行中,都要事先检查\code{coord.should\_stop()}。一旦\code{coord.request\_stop}被调用,其他线程的\code{coord.request\_stop()}将立即返回\code{True}。

一般地,一个子线程的迭代执行方法遵循如下实现模式。

\begin{leftbar}
\begin{python}
def MyLoop(coord):
  try
    while not coord.should_stop():
      # ...do something...
  except Exception as e:
    coord.request_stop(e)
\end{python}
\end{leftbar}

\subsection{异常处理}

当某个线程发生了异常,则可以通过\code{coord.request\_stop(e)}报告异常的发生。

\begin{leftbar}
\begin{python}
try:
  while not coord.should_stop():
    # ...do some work...
except Exception as e:
  coord.request_stop(e)
\end{python}
\end{leftbar}

为了消除异常代码处理的重复代码,可以使用\code{coord.stop\_on\_exception()}的上下文管理器。

\begin{leftbar}
\begin{python}
with coord.stop_on_exception():
  while not coord.should_stop():
    # ...do some work...
\end{python}
\end{leftbar}

其中,该异常也会在\code{coord.join}中被重新抛出。因此,在主程序也需要合理地处理异常。

\begin{leftbar}
\begin{python}
try:
  # Create a coordinator.
  coord = tf.train.Coordinator()

  # Create 10 threads that run 'MyLoop()'
  threads = [threading.Thread(target=MyLoop, args=(coord,)) 
            for i in xrange(10)]

  # Start the threads.
  for t in threads:
    t.start()

  # wait for all of them to stop
  coord.join(threads)
except Exception as e:
  # ...exception that was passed to coord.request\_stop(e)
\end{python}
\end{leftbar}

\subsection{实战:LoopThread}

\end{content}

\section{QueueRunner}

\begin{content}

一个\code{QueueRunner}实例持有一个或多个\code{Enqueue}的入队\ascii{OP},它为每个\code{Enqueue OP}启动一个线程。

\begin{figure}[!htbp]
\centering
\includegraphics[width=0.9\textwidth]{figures/tf-queue-runner-model.png}
\caption{TensorFlow系统架构}
 \label{fig:tf-queue-runner-model}
\end{figure}

\subsection{注册QueueRunner}

可以调用\code{tf.train.add\_queue\_runner}往计算图中注册\code{QueueRunner}实例,并且将其添加到\code{GraphKeys.QUEUE\_RUNNERS}集合中。

\begin{leftbar}
\begin{python}
def add_queue_runner(qr, collection=ops.GraphKeys.QUEUE_RUNNERS):
  ops.add_to_collection(collection, qr)
\end{python}
\end{leftbar}

\subsection{执行QueueRunner}

可以调用\code{tf.train.start\_queue\_runners}时,它会从计算图中找到所有\code{QueueRunner}实例,并从\code{QueueRunner}实例中取出所有\code{Enqueue OP},为每个\ascii{OP}启动一个线程。

\begin{leftbar}
\begin{python}
def start_queue_runners(sess, coord, daemon=True, start=True,
                        collection=ops.GraphKeys.QUEUE_RUNNERS):
  with sess.graph.as_default():
    threads = []
    for qr in ops.get_collection(collection):
      threads.extend(qr.create_threads(
          sess, coord=coord, daemon=daemon, start=start))
  return threads
\end{python}
\end{leftbar}

在\code{QueueRunner.create\_threads}方法中,为其包含的每个\code{Enqueue}类型的\ascii{OP}启动单独的线程。

\begin{leftbar}
\begin{python}
class QueueRunner(object):
  def create_threads(self, sess, coord, daemon, start):
    """Create threads to run the enqueue ops.
    """
    threads = [threading.Thread(
        target=self._run, args=(sess, op, coord))
        for op in self._enqueue_ops]
    if coord:
      threads.append(threading.Thread(
          target=self._close_on_stop, 
          args=(sess, self._cancel_op, coord)))
    for t in threads:
      if coord:
        coord.register_thread(t)
      if daemon:
        t.daemon = daemon
      if start:
        t.start()
    return threads
\end{python}
\end{leftbar}

\subsubsection{迭代执行Enqueue}

每个\code{Enqueue}子线程将迭代执行\code{Enqueue OP}。当发生\code{OutOfRangeError}异常时,将自动关闭队列,并退出子线程;但是,如果发生其他类型的异常,会主动通知\code{Coordinator}停止所有线程的运行,并退出子线程。

\begin{leftbar}
\begin{python}
class QueueRunner(object):
  def _run(self, sess, enqueue_op, coord):
    try:
      enqueue_callable = sess.make_callable(enqueue_op)
      while True:
        if coord.should_stop():
          break
        try:
          enqueue_callable()
        except errors.OutOfRangeError:  
          sess.run(self._close_op)
          return
    except Exception as e:
      coord.request_stop(e)
\end{python}
\end{leftbar}

\subsubsection{监听队列关闭}

另外,如果给定\code{Coordinator}实例,\code{QueueRunner}还会额外启动一个线程;当\code{Coordinator}实例被触发调用\code{request\_stop}方法后,该线程将会自动关闭队列。

\begin{leftbar}
\begin{python}
class QueueRunner(object):
  def _close_on_stop(self, sess, cancel_op, coord):
    """Close the queue, and cancel pending enqueue ops
       when the Coordinator requests stop.
    """
    coord.wait_for_stop()
    try:
      sess.run(cancel_op)
    except Exception:
      pass
\end{python}
\end{leftbar}

其中,\code{Queue}的\code{Cancel OP}与\code{Close OP}都会关闭队列,但是\code{Cancel OP}会撤销已缓存的\code{Enqueue OP}列表,但\code{Close OP}则保留已缓存的\code{Enqueue OP}列表。

\subsection{关闭队列}

当队列被关闭后,对于任何尝试\code{Enqueue}将会产生错误。但是,对于任何尝试\code{Dequeue}依然是成功的,只要队列中遗留元素;否则,\code{Dequeue}将立即失败,抛出\code{OutOfRangeError}异常,而不会阻塞等待更多元素被入队。

\end{content}
\begin{savequote}[45mm]
\ascii{Any fool can write code that a computer can understand. Good programmers write code that humans can understand.}
\qauthor{\ascii{- Martin Flower}}
\end{savequote}

\chapter{OP本质论} 
\label{ch:essential-op}

\begin{content}

\end{content}

\section{OP的注册}

\begin{content}

在\cpp{}后端系统,系统初始化时,系统完成所有\ascii{OP}的注册。\ascii{OP}的注册是通过\code{REGISTER\_OP}宏完成的。

\subsection{REGISTER\_OP}

实施上,\code{REGISTER\_OP}定义了一套精致的内部\ascii{DSL},系统自动完成字符串表示的翻译表达,并将其转换为\code{OpDef}的内部表示,最后保存在\code{OpDef}的仓库中。

\begin{figure}[!h]
\centering
\includegraphics[width=0.9\textwidth]{figures/cc-op-repo.png}
\caption{REGISTER\_OP:注册OP的实用宏}
 \label{fig:cc-op-repo}
\end{figure}

\subsection{查询接口}

\begin{leftbar}
\begin{c++}
struct OpRegistryInterface {
  virtual ~OpRegistryInterface() {}

  virtual Status LookUp(
    const string& op_name,
    const OpRegistrationData** op_reg_data) const = 0;
  Status LookUpOpDef(const string& op_name, const OpDef** op_def) const;
};
\end{c++}
\end{leftbar}

其中,\code{OpRegistrationData}描述了\ascii{OP}两种基本的元数据:\code{OpDef}与\code{OpShapeInferenceFn};前者用于描述\ascii{OP}的输入/输出参数信息,属性名列表,及其约束关系。后者用于描述\ascii{OP}的\ascii{Shape}的推演规则。

\begin{leftbar}
\begin{c++}
struct OpRegistrationData {
  OpDef op_def;
  OpShapeInferenceFn shape_inference_fn;
};

using OpRegistrationDataFactory = 
  std::function<Status(OpRegistrationData*)>;
\end{c++}
\end{leftbar}

\subsection{OpDef仓库}

实现中,采用延迟初始化的技术。为了简化问题的描述,此处做了简单的代码重构,以便帮助大家理解\code{OpRegistry}的工作原理。

\begin{leftbar}
\begin{c++}
struct OpRegistry : OpRegistryInterface {  
  OpRegistry();
  ~OpRegistry() override;

  void Register(const OpRegistrationDataFactory& factory);

 private:
  Status LookUp(
     const string& op_name,
     const OpRegistrationData** op_reg_data) const override;

 private:
  using Registry = 
    std::unordered_map<string, OpRegistrationData*>;

  mutex mu_;
  Registry registry_;
};
\end{c++}
\end{leftbar}

\begin{leftbar}
\begin{c++}
Status OpRegistry::Register(
  const OpRegistrationDataFactory& factory) {
  auto op_reg_data(std::make_unique<OpRegistrationData>());
  Status s = factory(op_reg_data.get());
  if (s.ok()) {
    gtl::InsertIfNotPresent(&registry_, 
      op_reg_data->op_def.name(),
      op_reg_data.get())
  }
  if (s.ok()) {
    op_reg_data.release();
  } else {
    op_reg_data.reset();
  }
  return watcher_status;
}
\end{c++}
\end{leftbar}

\end{content}

\part{运行模型}
\begin{savequote}[45mm]
\ascii{Any fool can write code that a computer can understand. Good programmers write code that humans can understand.}
\qauthor{\ascii{- Martin Flower}}
\end{savequote}

\chapter{本地执行} 
\label{ch:local}

\begin{content}

\tf{}可以独立地运行在一个进程内,完成计算图的执行过程。本章将重点介绍本地运行时的基本架构与运行机制;重点讨论计算图剪枝、分裂、优化、执行等实现技术细节;并且详细探究在本地模式下,跨设备间\ascii{OP}之间数据交互的工作机制,及其\ascii{OP}在设备集上的编排(\ascii{placement})算法。

\end{content}

\section{本地模式}
\label{sec:local-runtime}

\begin{content}

如\refig{local}所示,在本地模式下,\ascii{Client, Master, Worker}部署在同一台机器同一进程内,并由\code{DirectSession}同时扮演这三个角色。\code{DirectSession}运行在单独的进程内,各服务实体之间是函数调用关系。

\begin{figure}[H]
\centering
\includegraphics[width=0.7\textwidth]{figures/local.png}
\caption{本地模式}
 \label{fig:local}
\end{figure}

\ascii{Client}负责计算图的构造,通过调用\code{Session.run},启动计算图的执行过程。如\refig{local-runtime}所示,在\code{run\_step}执行过程之中,涉及计算图的剪枝、分裂、执行三个重要阶段。

\begin{figure}[H]
\centering
\includegraphics[width=0.8\textwidth]{figures/local-runtime.png}
\caption{本地模式:图操作}
 \label{fig:local-runtime}
\end{figure}

\subsection{部分执行}

\ascii{Master}收到计算图执行命令后,启动计算图的剪枝操作。它根据计算图的输入输出反向遍历图,寻找一个最小依赖的子图,常称为\code{ClientGraph}。

也就是说,每次执行\code{run\_step}时,并不会执行整个计算图(\code{FullGraph}),而是执行部分的子图。剪枝体现了\tf{}部分执行的设计理念。

\subsection{并发执行}

然后,运行时按照当前设备集完成图的分裂,生成了很多子图,每个子图称为\code{PartitionGraph};然后触发各个\ascii{Worker}并发地执行每个\code{PartitionGraph};对于每一个\ascii{PartitionGraph},运行时将启动一个\ascii{Executor},按照其拓扑排序完成\code{PartitionGraph}的执行。

也就是说,分裂和执行体现了\tf{}并发执行的设计理念。

\end{content}

\section{会话控制}

\begin{content}

在本地模式下,其运行时由\code{DirectSession}控制。一般地,\code{DirectSession}执行计算图时,各组件之间都是函数调用关系。但是,\code{DirectSession}也存在清晰的生命周期管理机制,如\refig{local-direct-session-lifecycle}所示。

\begin{figure}[H]
\centering
\includegraphics[width=0.6\textwidth]{figures/local-direct-session-lifecycle.png}
\caption{DirectSession生命周期}
 \label{fig:local-direct-session-lifecycle}
\end{figure}

\subsection{领域模型}

如\refig{local-direct-session-model}所示,\code{DirectSession}持有\code{SimpleGraphExecutionState}实例,后者负责计算图的剪枝,生成\code{ClientGraph}实例。

\code{DirectSession}同时持有一组线程池,但是没次\code{DirectSession.run}时,根据外部配置的索引,从线程池组里选择其一为其提供服务。因为\code{DirectSession}是线程安全的,支持多个并发执行的\code{DirectSession.run},即可以同时运行多个线程池实例。

\begin{figure}[H]
\centering
\includegraphics[width=0.9\textwidth]{figures/local-direct-session-model.png}
\caption{DirectSession领域模型}
 \label{fig:local-direct-session-model}
\end{figure}

\subsection{创建会话}

如\refig{local-direct-session-factory}所示,\code{DirectSession}由\code{DirectSessionFactory}多态创建。其中,\code{DeviceFactory::AddDevices}将创建本地设备集。

其中,\code{DirectSession}中主要完成线程池组的创建。

\begin{figure}[H]
\centering
\includegraphics[width=0.6\textwidth]{figures/local-direct-session-factory.png}
\caption{多态创建DirectSession}
 \label{fig:local-direct-session-factory}
\end{figure}

\begin{leftbar}
\begin{c++}
struct DirectSessionFactory : SessionFactory {
  bool AcceptsOptions(const SessionOptions& options) override {
    return options.target.empty();
  }

  Session* NewSession(const SessionOptions& options) override {
    std::vector<Device*> devices;
    DeviceFactory::AddDevices(
        options, "/job:localhost/replica:0/task:0", &devices);
    return new DirectSession(options, new DeviceMgr(devices));
  }
};
\end{c++}
\end{leftbar}

其中,\code{DirectSessionFactory::NewSession}由\ascii{C API}调用。

\begin{leftbar}
\begin{c++}
Status NewSession(const SessionOptions& options, Session** out_session) {
  SessionFactory* factory;
  Status s = SessionFactory::GetFactory(options, &factory);
  if (!s.ok()) {
    *out_session = nullptr;
    return s;
  }
  *out_session = factory->NewSession(options);
  if (!*out_session) {
    return errors::Internal("Failed to create session.");
  }
  return Status::OK();
}

TF_DeprecatedSession* TF_NewDeprecatedSession(
  const TF_SessionOptions* opt, TF_Status* status) {
  Session* session;
  status->status = NewSession(opt->options, &session);
  if (status->status.ok()) {
    return new TF_DeprecatedSession({session});
  } else {
    return nullptr;
  }
}
\end{c++}
\end{leftbar}

在\code{DirectSession}的构造函数中,主要负责其领域模型的初始化,包括线程池的创建,构建\code{CancellationManager}实例。

\begin{leftbar}
\begin{c++}
DirectSession::DirectSession(
    const SessionOptions& options,
    const DeviceMgr* device_mgr)
    : options_(options),
      device_mgr_(device_mgr),
      cancellation_manager_(new CancellationManager()) {
  // thread\_pools\_ = ... 
}
\end{c++}
\end{leftbar}

\subsection{销毁会话}

由\ascii{SessionFactory}所\code{new}出来的\code{DirectSession},由\ascii{C API}负责\code{delete}掉。

\begin{leftbar}
\begin{c++}
void TF_DeleteDeprecatedSession(TF_DeprecatedSession* s, TF_Status* status) {
  status->status = Status::OK();
  delete s->session;  // delete DirectSession
  delete s;
}
\end{c++}
\end{leftbar}

随后,\code{DirectSession}的析构函数被调用,它负责清理其负责管理的系统资源。主要包括\code{Executor}列表,\code{ThreadPool}列表,\code{CancellationManager}实例。

\begin{leftbar}
\begin{c++}
DirectSession::~DirectSession() {
  for (auto& it : partial_runs_) {
    it.second.reset(nullptr);
  }
  
  for (auto& it : executors_) {
    it.second.reset();
  }
  
  for (auto d : device_mgr_->ListDevices()) {
    d->op_segment()->RemoveHold(session_handle_);
  }
  
  delete cancellation_manager_;
  
  for (const auto& p_and_owned : thread_pools_) {
    if (p_and_owned.second) delete p_and_owned.first;
  }

  execution_state_.reset(nullptr);
  flib_def_.reset(nullptr);
}
\end{c++}
\end{leftbar}

\subsection{创建/扩展图}

首次扩展图,等价于创建图。扩展图就是在原有计算图的基础上,追加新的子图。当然,追加的子图中所包含的节点,在原有的计算图中不应该存在。


\begin{leftbar}
\begin{c++}
Status DirectSession::Create(const GraphDef& graph) {
  if (graph.node_size() > 0) {
    mutex_lock l(graph_def_lock_);
    return ExtendLocked(graph);
  }
  return Status::OK();
}

Status DirectSession::Extend(const GraphDef& graph) {
  mutex_lock l(graph_def_lock_);
  return ExtendLocked(graph);
}
\end{c++}
\end{leftbar}

当创建计算图时,\code{DirectSession}主要完成\code{SimpleGraphExecutionState}实例的创建。如\refig{local-simple-graph-execution-state-model}所示,\code{SimpleGraphExecutionState}实例持有\code{FullGraph}两种格式的实例:\code{Graph}与\code{GraphDef},并由它负责管理和维护\code{FullGraph}的生命周期。

\begin{figure}[H]
\centering
\includegraphics[width=0.5\textwidth]{figures/local-simple-graph-execution-state-model.png}
\caption{创建SimpleGraphExecutionState实例}
 \label{fig:local-simple-graph-execution-state-model}
\end{figure}

其中,\code{SimpleGraphExecutionState}的主要职责包括:

\begin{enum}
  \eitem{构造\code{FullGraph}}:发生在\code{DirectSession.Create};
  \eitem{执行简单的\ascii{OP}编排算法}:发生在\code{DirectSession.Create};
  \eitem{执行图的剪枝操作}:发生在\code{DirectSession.Run}。
\end{enum}

当执行\code{DirectSession::Create}时,将创建\code{SimpleGraphExecutionState}实例,并完成\code{FullGraph}实例的构建和初始化。

\begin{leftbar}
\begin{c++}
Status SimpleGraphExecutionState::MakeForBaseGraph(
    GraphDef* graph_def, const SimpleGraphExecutionStateOptions& opts,
    std::unique_ptr<SimpleGraphExecutionState>* out_state) {
  auto ret = std::make_unique<SimpleGraphExecutionState>(graph_def, opts));

  AddDefaultAttrsToGraphDef(&ret->original_graph_def_, *ret->flib_def_, 0));
  if (!ret->session_options_->config.graph_options().place_pruned_graph()) {
    ret->InitBaseGraph();
  }
  *out_state = std::move(ret);
  return Status::OK();
}
\end{c++}
\end{leftbar}

其中,\code{SimpleGraphExecutionState::InitBaseGraph}完成\code{FullGraph}从\code{GraphDef}到\code{Graph}的格式转换,并启动\code{SimplePlacer}的\ascii{OP}编排算法。

\begin{leftbar}
\begin{c++}
Status SimpleGraphExecutionState::InitBaseGraph() {
  auto ng = std::make_unique<Graph>(OpRegistry::Global());

  GraphConstructorOptions opts;
  ConvertGraphDefToGraph(opts, *original_graph_def_, ng.get());

  SimplePlacer placer(ng.get(), device_set_, session_options_);
  placer.Run();

  this->graph_ = ng.release();
  return Status::OK();
}
\end{c++}
\end{leftbar}

\subsubsection{图构造:GraphDef -> Graph}

刚开始,\code{SimpleGraphExecutionState}得到的是\code{GraphDef},这是最原始的图结构。它由\ascii{Client}将序列化后传递到后端\ascii{C++},然后由后端反序列化得到的图结构。

如\refig{local-graph-def-to-graph}所示,通过调用\code{ConvertGraphDefToGraph}将\code{GraphDef}实例变换为等价的\code{Graph}实例;同理,可以调用\code{Graph.ToGraphDef}将\code{Graph}实例变换为等价的\code{GraphDef}实例。

其中,\code{GraphDef}是使用\ascii{protobuf}格式存在的图结构,它包含了图所有元数据;而\code{Graph}则是运行时系统中用于描述图结构的领域对象,它不仅仅持有\code{GraphDef}的元数据,并包含其它图结构的其它信息。

\begin{figure}[H]
\centering
\includegraphics[width=0.6\textwidth]{figures/local-graph-def-to-graph.png}
\caption{\code{GraphDef}与\code{Graph}之间的格式转换}
 \label{fig:local-graph-def-to-graph}
\end{figure}

\subsubsection{OP编排:SimplePlacer}

\ascii{OP}的编排(\ascii{placement})指的是,将计算图中包含的\ascii{OP}以最高效的方式置放在合适的计算设备上运算,以最大化计算资源的利用率,可以形式化地描述为\refig{local-cost-model}。

\begin{figure}[H]
\centering
\includegraphics[width=0.6\textwidth]{figures/local-cost-model.png}
\caption{费用模型}
 \label{fig:local-cost-model}
\end{figure}

求取最优的编排方案,我猜想这是一个\ascii{NP}问题。该问题取决于计算图的特征,网络拓扑与带宽,样本数目等多个复杂的因素,该问题也是社区中最活跃的问题之一。

\subsection{迭代执行}

\code{DirectSession.Run}是\tf{}运行时的关键路径,它负责完成一次迭代计算。首先,\code{DirectSession}根据输入/输出对\code{FullGraph}实施剪枝,生成\code{ClientGraph};然后,根据所持有本地设备集,将\code{ClientGraph}分裂为多个\code{PartitionGraph};运行时为其每个\code{PartitionGraph}启动一个\code{Executor}实例,后者执行\code{PartitionGraph}的拓扑排序算法,完成计算图的执行。

具体实现,请参照\refsec{graph-operation-prune},\refsec{graph-operation-split},\refsec{graph-operation-exec}。

\subsubsection{图操作}

如\refig{local-graph-transformation}所示,在本地模式下,计算图经历三个形态的变换,最终被分解至各个计算设备上,以便实现在各个计算设备上并发执行子图。

\begin{figure}[H]
\centering
\includegraphics[width=0.9\textwidth]{figures/local-graph-transformation.png}
\caption{图变换}
 \label{fig:local-graph-transformation}
\end{figure}

\begin{itemize}
  \item \code{FullGraph}: \ascii{Client}负责构造的完整的计算图,常称为\code{FullGraph};但是,一次\code{Session.run}并不会执行整个计算图;
  \item \code{ClientGraph}: \ascii{Master}根据\code{Session.run}传递\code{feeds, fetches}输入输出列表,对\ascii{FullGraph}实施剪枝操作,计算得到本地迭代执行的最小依赖子图,常称为\code{ClientGraph};
  \item \code{PartitionGraph}: \ascii{Master}根据当前计算设备集,及其\ascii{OP}的设备约束规范,将\code{ClientGraph}分裂为多个\code{PartitionGraph};其中,每个计算设备对应一个\code{PartitionGraph},计算设备负责\code{PartitionGraph}的执行。
\end{itemize}

但是,\code{FullGraph, ClientGraph, PartitionGraph}的数据结构相同,它们都是\code{Graph}三种不同表现形式,仅仅大小和范畴存在差异。

\subsubsection{形式化}

在真实的系统实现中,本地模式的运行时是使用\ascii{C++}实现。其中,\tf{}运行时的关键路径为\code{run\_step}。因为真实系统实现中涉及过多的细节,不易发现算法的主干和逻辑。为了简化问题的描述,将形式化地描述\code{run\_step}的实现过程。

\begin{leftbar}
\begin{python}
def do_run_partitions(executors_and_partitions):
  barrier = ExecutorBarrier(executors_and_partitions.size())
  for (executor, partition) in executors_and_partitions:
    executor.run(partition, barrier)  
  barrier.wait()

def run_partitions(executors_and_partitions, inputs, outputs):
  frame = FunctionCallFrame()
  frame.set_args(inputs)
  do_run_partitions(executors_and_partitions)
  frame.get_ret_vals(outputs)

def run_step(devices, full_graph, inputs, outputs):
  client_graph = prune(full_graph, inputs, outputs)
  executors_and_partitions = split(client_graph, devices)
  run_partitions(executors_and_partitions, inputs, outputs)
\end{python}
\end{leftbar}

其中,在每个计算设备上,启动一个\code{Executor}执行分配给它的\code{PartitionGraph}。当某一个计算设备执行完所分配的\code{PartitionGraph}之后,\code{ExecutorBarrier}的计数器加\ascii{1},直至所有设备完成\code{PartitionGraph}列表的执行,\code{barrier.wait()}阻塞操作退出。

跨设备的\code{PartitionGraph}之间可能存在数据依赖关系,它们之间通过插入\code{Send/Recv}节点完成交互。事实上,在本地模式中,\code{Send/Recv}通过\code{Rendezvous}完成数据交换的。\code{Send}将数据放在\code{Rendezvous}上,而\code{Recv}则根据标识从\code{Rendezvous}取走。其中,\code{Send}不阻塞,而\code{Recv}是阻塞的。

\subsection{关闭会话}

\begin{leftbar}
\begin{c++}
Status DirectSession::Close() {
  cancellation_manager_->StartCancel();
  {
    mutex_lock l(closed_lock_);
    if (closed_) return Status::OK();
    closed_ = true;
  }
  return Status::OK();
}
\end{c++}
\end{leftbar}

如\refig{local-cancellation-manager}所示,将\ascii{Step}注册给\code{DirectSession}的\code{CancellationManager}之中。当\code{DirectSession}被关闭时,\code{DirectSession}的\code{CancellationManager},将取消这次\ascii{step}的执行过程。

\begin{figure}[H]
\centering
\includegraphics[width=0.9\textwidth]{figures/local-cancellation-manager.png}
\caption{CancellationManager工作原理}
 \label{fig:local-cancellation-manager}
\end{figure}

\begin{leftbar}
\begin{c++}
Status DirectSession::Run(
   const NamedTensorList& inputs,
   const std::vector<string>& output_names,
   const std::vector<string>& target_nodes,
   std::vector<Tensor>* outputs) {
  // step\_cancellation\_manager is passed to `OpKernelContext`
  CancellationManager step_cancellation_manager;

  // Register this step with session's cancellation manager, so that
  // `Session::Close()` will cancel the step.
  CancellationToken cancellation_token =
      cancellation_manager_->get_cancellation_token();
  bool already_cancelled = !cancellation_manager_->RegisterCallback(
      cancellation_token, [&step_cancellation_manager]() {
        step_cancellation_manager.StartCancel();
      });
  // ignore others...
}
\end{c++}
\end{leftbar}

当前\ascii{Step}的\code{CancellationManager}最终会传递给\code{OpKernelContext}。\ascii{Kernel}实现计算时,如果保存了中间状态,可以向其注册相应的回调钩子。其中,每个回调钩子都有唯一的\code{token}标识。

当\ascii{Step}被取消时,回调钩子被调用,该\ascii{Kernel}可以取消该\ascii{OP}的计算。例如,\code{FIFOQueue}实现\code{TryEnqueue}时,便往本次\ascii{Step}的\code{CancellationManager}注册了回调钩子,用于取消该\ascii{Kernel}中间的状态信息。

\begin{leftbar}
\begin{c++}
void FIFOQueue::TryEnqueue(const Tuple& tuple, OpKernelContext* ctx,
                           DoneCallback callback) {
  CancellationManager* cm = ctx->cancellation_manager();
  CancellationToken token = cm->get_cancellation_token();
  bool already_cancelled;
  {
    mutex_lock l(mu_);
    already_cancelled = !cm->RegisterCallback(
        token, [this, cm, token]() { Cancel(kEnqueue, cm, token); });
  }
  // ignore others...
}
\end{c++}
\end{leftbar}

\end{content}

\section{剪枝}
\label{sec:graph-operation-prune}

\begin{content}

\code{DirectSession::Run}执行时,首先完成\code{ClientGraph}的构造。事实上,\code{ClientGraph}的构造过程,主要完成\code{FullGraph}的剪枝算法,并生成\code{ClientGraph}。

\subsection{构建ClientGraph}

如\refig{local-simple-graph-execution-state}所示,\code{SimpleGraphExecutionState}实例持有\code{FullGraph}实例,并根据输入/输出列表,生成\code{ClientGraph}。

\begin{figure}[H]
\centering
\includegraphics[width=0.55\textwidth]{figures/local-simple-graph-execution-state.png}
\caption{生成\code{ClientGraph}}
 \label{fig:local-simple-graph-execution-state}
\end{figure}

其中,\code{BuildGraphOptions}包含了输入/输出列表,调用\code{SimpleGraphExecutionState::BuildGraph}生成\code{ClientGraph}实例。

\begin{leftbar}
\begin{c++}
namespace {
  BuildGraphOptions build_graph_options(
    const NamedTensorList& inputs,
    const std::vector<string>& outputs,
    const std::vector<string>& targets) {
    // sort inputs/outputs/targets
    std::vector<string> inputs_sorted(inputs.begin(), inputs.end());
    std::sort(inputs_sorted.begin(), inputs_sorted.end());

    std::vector<string> outputs_sorted(outputs.begin(), outputs.end());
    std::sort(outputs_sorted.begin(), outputs_sorted.end());

    std::vector<string> tn_sorted(targets.begin(), targets.end());
    std::sort(tn_sorted.begin(), tn_sorted.end());

    // build graph options
    BuildGraphOptions options;
    options.feed_endpoints = inputs_sorted;
    options.fetch_endpoints = outputs_sorted;
    options.target_nodes = tn_sorted;
    options.use_function_convention = !run_state_args->is_partial_run;
    return options;
  }
}

Status DirectSession::Run(
  const RunOptions& run_options,
  const NamedTensorList& inputs,
  const std::vector<string>& output_names,
  const std::vector<string>& target_nodes,
  std::vector<Tensor>* outputs,
  RunMetadata* run_metadata) {

  // 1. prune graph
  // client\_graph = prune(full\_graph, inputs, outputs)
  std::unique_ptr<SimpleClientGraph> client_graph;
  execution_state_->BuildGraph(
    build_graph_options(inputs, output_names, target_nodes), 
    &client_graph);
   
  // 2. split graph into partition by devices 
  // executors\_and\_partitions = split(client\_graph, devices)
  
  // 3. lauch executor per partition
  // def run\_partitions(executors\_and\_partitions, inputs, outputs):
  // \ \ frame = FunctionCallFrame()
  // \ \ frame.set\_args(inputs)
  // \ \ for (executor, partition) in executors\_and\_partitions: 
  // \ \ \ \ exec.run(part)
  // \ \ frame.get\_ret\_vals(outputs)

  return Status::OK();
}
\end{c++}
\end{leftbar}

\code{ClientGraph}初始来自原始的\code{FullGraph},调用\code{RewriteGraphForExecution}函数,将根据输入/输出,对\code{ClientGraph}实施改写操作,包括增加节点,或删除节点,最终生成\code{SimpleClientGraph}实例。

\begin{leftbar}
\begin{c++}
const DeviceAttributes& 
SimpleGraphExecutionState::local_device_attr() const {
  return device_set_->client_device()->attributes();
}

Status SimpleGraphExecutionState::BuildGraph(
  const BuildGraphOptions& options, 
  std::unique_ptr<SimpleClientGraph>* out) {
  // 1. create new\_graph from origin graph, 
  // which is client graph.
  std::unique_ptr<Graph> ng;
  ng.reset(new Graph(flib_def_.get()));
  CopyGraph(*graph_, ng.get());

  // 2. prune the client graph
  subgraph::RewriteGraphForExecution(
    ng.get(), options.feed_endpoints, options.fetch_endpoints,
    options.target_nodes, local_device_attr(),
    options.use_function_convention);
  }

  // 3. create SimpleClientGraph, and return it.
  std::unique_ptr<SimpleClientGraph> dense_copy(
      new SimpleClientGraph(std::move(flib)));
  CopyGraph(*ng, &dense_copy->graph);
  *out = std::move(dense_copy);

  return Status::OK();
}
\end{c++}
\end{leftbar}

因此,构建\code{ClientGraph}过程,其关键路径为\code{RewriteGraphForExecution},即剪枝算法。剪枝算法根据输入/输出列表,反向遍历\ascii{FullGraph},找到最小的依赖子图\code{ClientGraph}。

一般地,对于\code{ClientGraph}输入节点,扮演了起始节点;而输出节点,扮演了终止节点。因此,关于输入和输出,存在两个比较棘手的问题:

\begin{enum}
  \eitem{输入:当\code{ClientGraph}计算开始前,外部的运行时如何传递\code{Tensor}给输入节点};
  \eitem{输出:当\code{ClientGraph}计算完成后,外部的运行时又如何从输出节点获取\code{Tensor}}。
\end{enum}

存在两种媒介:\code{FunctionCallFrame}和\code{Rendezvous},外部运行时与输入/输出节点可以使用其中一种媒介交换数据。

\code{FunctionCallFrame}用于\code{Arg/RetVal}函数调用的\ascii{OP},用于函数调用时传递函数参数值,及其返回函数值。但是,它们仅适用于单进程的运行时环境。

\code{Rendezvous}用于\code{Send/Recv}消息发送的\ascii{OP},这是一种更为通用的通信方式,适用于分布式的运行时环境。

\subsection{基于Rendezvous}

如\refig{client-prune-graph}所示,根据\code{fetches}列表,反向搜索依赖的节点,直至\code{feeds},计算得到最小依赖的子图。

对于\code{Feed}的边实施剪枝,例如剪枝\code{ina:0})边,并在此处插入节点\code{Recv},并按照输入边的名字命名该节点,例如\code{\_recv\_ina\_0}。

同理,对于\code{Fetch}的边也实施剪枝,例如剪枝\code{f:0}边,并在此处插入节点\code{Send}节点,并按照输出边的名字命名该节点,例如\code{\_send\_f\_0}。

最终,通过插入\code{Source/Sink}节点,将剪枝后得到各个联通的子图进行汇总,形成一个完整的\ascii{DAG}图。

\begin{figure}[H]
\centering
\includegraphics[width=0.9\textwidth]{figures/client-prune-graph.png}
\caption{图剪枝:插入Send/Recv节点}
 \label{fig:client-prune-graph}
\end{figure}

\subsection{基于FunctionCallFrame}

但是,输入/输出通过\code{Rendezvous}交换数据可能存在性能上的瓶颈。因为待发送的\code{Tensor}需要携带发送设备,接收设备,\code{TensorId},共同组成了唯一的字符串标识,数据发送和接收需要花费很长的字符串解析的时间开销。

特殊地,对于本地模式,因为在同一进程内,使用\code{Rendezvous}交换数据存在不必要的性能损耗。可以使用基于\code{FunctionCallFrame}函数调用替代之。

因此,在本地模式下,可以使用\code{Arg/RetVal}分别替代\code{Send/Recv}节点,从而实现了函数调用交换数据的方式,替代原有基于\code{Rendezvous}交互数据的方式。

如\refig{client-prune-graph-function-ops}所示。对于\code{Feed}的边实施剪枝,例如剪枝\code{ina:0})边,并在此处插入节点\code{Arg},并按照输入边的名字命名该节点,例如\code{\_arg\_ina\_0}。

同理,对于\code{Fetch}的边也实施剪枝,例如剪枝\code{f:0}边,并在此处插入节点\code{RetVal}节点,并按照输出边的名字命名该节点,例如\code{\_retval\_f\_0}。

最终,通过插入\code{Source/Sink}节点,将剪枝后得到各个联通的子图进行汇总,形成一个完整的\ascii{DAG}图。

\begin{figure}[H]
\centering
\includegraphics[width=0.9\textwidth]{figures/client-prune-graph-function-ops.png}
\caption{图剪枝:插入Arg/RetVal节点}
 \label{fig:client-prune-graph-function-ops}
\end{figure}

\subsection{剪枝算法实现}

剪枝算法主要由\code{RewriteGraphForExecution}完成,主要包括\ascii{3}个子过程。

\begin{enum}
  \eitem{追加输入节点}
  \eitem{追加输出节点} 
  \eitem{反向剪枝}
\end{enum}

\begin{leftbar}
\begin{c++}
void RewriteGraphForExecution(Graph* g, bool use_function, 
    const ArraySlice<string>& fed_outputs,
    const ArraySlice<string>& fetch_outputs,
    const ArraySlice<string>& target_node_names,
    const DeviceAttributes& device_info) {
  FeedInputs(g, use_function, device_info, fed_outputs);

  std::vector<Node*> fetch_nodes;
  FetchOutputs(g, use_function, device_info, 
    fetch_outputs, &fetch_nodes);

  PruneForTargets(g, fetch_nodes, target_node_names);
}
\end{c++}
\end{leftbar}

\subsubsection{追加输入节点}

如\refig{local-prune-feed}所示,对于任意一条输入边实施剪枝时,插入相应的\code{Arg}或\code{Recv}节点,删除既有的边,并重新连接相应的边。

在计算图中,一条边唯一地由\code{TensorId}标识,它由\code{op:src\_output}二元组构成。前者表示边的上游节点,后者表示给边为上游节点的第几条边。

示例代码删除了部分不重要的逻辑,并搬迁了部分函数的职责,并在局部尝试部分函数提取,以便更好地还原算法的逻辑。其中,假设\code{Graph}可以按照\code{TensorId}索引节点和边。

\begin{figure}[H]
\centering
\includegraphics[width=0.6\textwidth]{figures/local-prune-feed.png}
\caption{剪枝:输入边}
 \label{fig:local-prune-feed}
\end{figure}

\begin{leftbar}
\begin{c++}
namespace {
  DataType data_type(Graph& g, const TensorId& tensor_id) {
    Node* upstream_node = g.upstream_node(tensor_id);
    return BaseType(upstream_node->output_type(tensor_id.src_output()));
  }

  Node* AppendRecvNode(Graph& g, 
    const TensorId& tensor_id, const DeviceAttributes& device_info) {
      Node* recv_node;
      NodeBuilder(strings::StrCat(
        "_recv_", tensor_id.op(), "_", tensor_id.src_output()), "_Recv")
        .Attr("tensor_type", data_type(g, tensor_id))
        .Attr("tensor_name", tensor_id.name())
        .Attr("send_device", device_info.name())
        .Attr("recv_device", device_info.name())
        .Attr("send_device_incarnation", device_info.incarnation())
        .Attr("client_terminated", true)
        .Finalize(g, &recv_node);
      return recv_node;
  }

  Node* AppendArgNode(Graph& g, size_t index, 
    const TensorId& tensor_id, const DeviceAttributes& device_info) {
    Node* arg_node;
    NodeBuilder(strings::StrCat(
      "_arg_", tensor_id.op(), "_", tensor_id.src_output()), "_Arg")
      .Attr("T", data_type(g, tensor_id))
      .Attr("index", index)
      .Finalize(g, &arg_node);
    return arg_node;
  }

  // 1. append arg/recv node
  Node* AppendNewNode(Graph& g, bool use_function, size_t index, 
    const TensorId& tensor_id,const DeviceAttributes& device_info) {
    if (use_function) {
      return AppendArgNode(g, index, tensor_id, device_info);
    } else {
      return AppendRecvNode(g, tensor_id, device_info);
    }
  }

  void AppendNewEdges(Graph& g, 
    Node* new_node, const TensorId& tensor_id) {
    // 2. add control edge between source node and new node.
    g.AddControlEdge(g.source_node(), new_node);

    Edge* old_edge = g.edge(tensor_id);
    
    // 3. add edge between new node and downstream node.
    g.AddEdge(new_node, 0, old_edge->dst(), old_edge->dst_input());
    
    // 4. remove old edge.
    g.RemoveEdge(old_edge);
  }
}

void FeedInputs(Graph& g, bool use_function,
  const DeviceAttributes& device_info,
  const ArraySlice<TensorId>& feeds) {
  for (size_t i = 0; i < feeds.size(); ++i) {
    Node* new_node = AppendNewNode(use_function, i, feeds[i]);
    AppendNewEdges(g, new_node, feeds[i]);
  }
}
\end{c++}
\end{leftbar}

\subsubsection{追加输出节点}

对于任意一条输出边实施剪枝时,插入相应的\code{RetVal}或\code{Send}节点,并将其与\code{Sink}节点通过控制依赖边连接。

如\refig{local-prune-fetch}所示,对输出边实施剪枝操作。新节点与上游节点的连接关系,在构造新节点时,通过\code{Input}已经指定。另外,函数直接返回了新节点(\code{RetVal/Send})为终止节点,因此没必要删除原来的边,其算法与输入边的处理存在微妙的差异。

\begin{figure}[H]
\centering
\includegraphics[width=0.6\textwidth]{figures/local-prune-feed.png}
\caption{剪枝:输出边}
 \label{fig:local-prune-fetch}
\end{figure}

\begin{leftbar}
\begin{c++}
namespace {
  Node* AppendSendNode(Graph& g, 
    const TensorId& tensor_id, const DeviceAttributes& device_info) {
    Node* send_node;
    NodeBuilder(strings::StrCat(
      "_send_", tensor_id.op(), "_", id.src_output()), "_Send")
      // 2. add edge between upstream node and send node.
      .Input(g.upstream_node(tensor_id), tensor_id.src_output())
      .Attr("tensor_name", tensor_id.name())
      .Attr("send_device", device_info.name())
      .Attr("recv_device", device_info.name())
      .Attr("send_device_incarnation",
            device_info.incarnation())
      .Attr("client_terminated", true)
      .Finalize(g, &send_node);
    return send_node;
  }

  Node* AppendRetvalNode(Graph& g, size_t index, 
    const TensorId& tensor_id, const DeviceAttributes& device_info) {
    Node* retval_node;
    NodeBuilder(strings::StrCat(
      "_retval_", tensor_id.op(), "_", tensor_id.src_output(), "_", index), 
      "_Retval")
      // 2. add edge between upstream node and retval node.
      .Input(g.upstream_node(tensor_id), tensor_id.src_output())
      .Attr("T", data_type(g, tensor_id))
      .Attr("index", index)
      .Finalize(g, &retval_node))
    return retval_node;
  }

  // 1. append retval/send node
  Node* AppendNewNode(Graph& g, bool use_function, size_t index, 
    const TensorId& tensor_id,const DeviceAttributes& device_info) {
    if (use_function) {
      return AppendRetvalNode(g, index, tensor_id, device_info);
    } else {
      return AppendSendNode(g, tensor_id, device_info);
    }
  }
}

void FetchOutputs(Graph& g, bool use_function,
  const DeviceAttributes& device_info,
  const ArraySlice<TensorId>& fetches,
  std::vector<Node*>& fetch_nodes) {
  for (size_t i = 0; i < fetches.size(); ++i) {
    Node* new_node = AppendNewNode(use_function, i, fetches[i]);
    
    // 3. add control edge between new node and sink node. 
    g->AddControlEdge(new_node, g->sink_node());

    fetch_nodes.push_back(new_node);
  }
}
\end{c++}
\end{leftbar}

\subsubsection{反向剪枝}

剪枝操作,其本质就是\ascii{DAG}反向的宽度优先遍历算法。首先,创建了一个队列,及其一个\code{visited}数组,后者用于记录已经遍历过的节点。初始化时,队列仅包含输出节点和输入节点(\code{targets})。当图遍历完毕后,不再\code{visited}里面的节点,表示本此执行不依赖于它,应从图中删除该节点,及其相关联的边。

经过剪枝后,将形成若干\ascii{DAG}子图。将入度为\code{0}的节点,与\code{Source}节点通过控制依赖边相连接;出度为\ascii{0}的节点,与\code{Sink}节点通过控制依赖边相连接,最终形成一个完整的\ascii{DAG}图。

\begin{leftbar}
\begin{c++}
namespace {
  void ReverseBFS(
    Graph* g, std::unordered_set<const Node*>& visited) {
    std::deque<const Node*> queue(visited.begin(), visited.end());
    while (!queue.empty()) {
      const Node* n = queue.front();
      queue.pop_front();
      for (const Node* in : n->in_nodes()) {
        if (visited.insert(in).second) {
          queue.push_back(in);
        }
      }
    }
  }

  void RemoveUnvisitedNodes(
    Graph* g, std::unordered_set<const Node*>& visited) {
    for (Node* n : g->nodes()) {
      if (visited.count(n) == 0 && !n->IsSource() && !n->IsSink()) {
        g->RemoveNode(n);
      }
    }
  }

  void PruneForReverseReachability(
    Graph* g, std::unordered_set<const Node*>& visited) {
    ReverseBFS(g, visited);
    RemoveUnvisitedNodes(g, visited);
  }

  void FixupSourceEdges(Graph* g, Node* n) {
    if (!n->IsSource() && n->in_edges().empty()) {
      g->AddControlEdge(g->source_node(), n);
    }  
  }

  void FixupSinkEdges(Graph* g, Node* n) {
    if (!n->IsSink() && n->out_edges().empty()) {
      g->AddControlEdge(n, g->sink_node());
    }  
  }

  void FixupSourceAndSinkEdges(Graph* g) {
    for (Node* n : g->nodes()) {
      FixupSourceEdges(g, n);
      FixupSinkEdges(g, n);
    }
  }

  void AppendTargetNodes(Graph& g, 
    const ArraySlice<string>& target_names,
    std::unordered_set<const Node*>& targets) {
    for (auto name : target_names) {
      Node* target = g.GetNodeBy(name);
      targets.insert(target);
    }
  }  
}

void PruneForTargets(Graph* g, 
  std::vector<Node*>& fetch_nodes,
  const ArraySlice<string>& target_names) {
  std::unordered_set<const Node*> targets(
    begin(fetch_nodes), end(fetch_nodes));

  AppendTargetNodes(g, target_names, targets);
  PruneForReverseReachability(g, targets);
  FixupSourceAndSinkEdges(g);
}
\end{c++}
\end{leftbar}

\end{content}

\section{分裂}
\label{sec:graph-operation-split}

如\refig{local-graph-split-by-device}所示,假如\code{d}节点放置在\ascii{GPU0}上执行,而其他节点放置在\ascii{CPU0}上执行。其中,节点\code{a}与\code{b}通过\code{Arg}输入数据;节点\code{f}将其结果输出到\code{RetVal}节点上。

\begin{figure}[H]
\centering
\includegraphics[width=0.9\textwidth]{figures/local-graph-split-by-device.png}
\caption{按本地设备集执行图分裂}
 \label{fig:local-graph-split-by-device}
\end{figure}

因此,计算图中存在若干条边跨越设备。对于跨越设备的边,运行时将其分裂,并就地插入\code{Send/Recv}边,分别用于原设备上发送数据,并在目标设备上接受数据,完成设备间的数据交换。如\refig{local-graph-split-insert-send-recv}所示。

其中,\code{Arg/RetVal}节点通过媒介\code{FunctionCallFrame}交换数据;\code{Send/Recv}节点通过媒介\code{Rendezvous}交换数据。

\begin{figure}[H]
\centering
\includegraphics[width=1.0\textwidth]{figures/local-graph-split-insert-send-recv.png}
\caption{跨设备OP之间插入Send/Recv节点}
 \label{fig:local-graph-split-insert-send-recv}
\end{figure}

\subsection{情况1}

最简单的情况下,\code{src}与\code{dst}在同一个\code{Partition}内。因此,直接将其划归在同一个\code{Partition}即可。

\begin{figure}[H]
\centering
\includegraphics[width=0.6\textwidth]{figures/split-graph-1.png}
\caption{情况1:src与dst在同一个Partition内}
 \label{fig:split-graph-1}
\end{figure}

\subsection{情况2}

如果\code{src}与\code{dst}不在同一个\code{Partition}内,但两者之间原来是通过普通边连接在一起的。因此,仅需要在它们中间增加\code{Send}与\code{Recv}节点,将其划归在两个不同的\code{Partition}内。

\begin{figure}[H]
\centering
\includegraphics[width=0.7\textwidth]{figures/split-graph-2.png}
\caption{情况2:src与dst不在同一个Partition内,但两者之间是普通边}
 \label{fig:split-graph-2}
\end{figure}

\subsection{情况3}

如果\code{src}与\code{dst}不在同一个\code{Partition}内,但两者之间原来是通过控制依赖边连接在一起的。

此时,需要在\code{src}侧增加一个\code{Const}的\code{DummyNode},并作为\code{src}的下游通过控制依赖边相连;最终,在通过\code{Send}将其值发送到对端。

在\code{dst}侧,\code{Recv}收到该值,使用\code{Identity}将其消费掉;最终,再将\code{Identity}与\code{dst}连接控制依赖边。

在这里,\code{Const}扮演生产者,\code{Identity}扮演消费者角色。既满足了跨设备间通信的需求,又满足原来\code{src}与\code{dst}之间的控制依赖的关系。但是,其缺点就是存在微妙的性能开销。

\begin{figure}[H]
\centering
\includegraphics[width=0.8\textwidth]{figures/split-graph-3.png}
\caption{情况3:src与dst不在同一个Partition内,但两者之间是控制依赖边}
 \label{fig:split-graph-3}
\end{figure}

\subsection{分裂算法实现}

分裂算法也是一个反向遍历图的算法。对于当前遍历的节点,将其标记为\code{dst};然后再寻找\code{dst}的所有输入边;遍历所有输入边,从而找到与改边相连的源节点,将其标记为\code{src}。

因此,更具上述讨论的三种情况,迭代实现\code{src}与\code{dst}两者之前的\code{Partition}划分算法。

\begin{leftbar}
\begin{c++}
namespace {
  
  using Edges = std::vector<const Edge*>;
  using Partitions = std::unordered_map<string, GraphDef>;

  void AddInput(NodeDef* dst, StringPiece src_name, int src_slot) {
    if (src_slot == Graph::kControlSlot) {
      dst->add_input(strings::StrCat("^", src_name));
    } else if (src_slot == 0) {
      dst->add_input(src_name.data(), src_name.size());
    } else {
      dst->add_input(strings::StrCat(src_name, ":", src_slot));
    }
  }

  Edges InputsOf(const Node* dst) {
    Edges inputs(dst->num_inputs(), nullptr);
    for (auto edge : dst.in_edges()) {
      if (edge->IsControlEdge()) {
        inputs.push_back(e);
      } else {
        inputs[edge->dst_input()] = edge;
      }
    }
    return inputs;
  }

  NodeDef* InitDstNodeDef(const Node& dst, NodeDef* dst_def) {
    dst_def = dst.def();
    dst_def->set_device(dst.assigned_device_name());
    dst_def->clear_input();
    return dst_def;  
  }

  NodeDef* AddDummyConst(const PartitionOptions& opts, GraphDef* gdef,
                         const Edge* edge, Status* status) {
    const Node* src = edge->src();
    Tensor tensor(DT_FLOAT, TensorShape({0}));
    NodeDef* result = gdef->add_node();
    *status = NodeDefBuilder(opts.new_name(src->name()), "Const")
                  .Device(src->assigned_device_name())
                  .Attr("dtype", DT_FLOAT)
                  .Attr("value", tensor)
                  .Finalize(result);
    return result;
  }

  NodeDefBuilder::NodeOut BuildSendFrom(
      const PartitionOptions& opts,
      GraphDef* src_graph,
      const Edge* edge,
      NodeDefBuilder::NodeOut& send_from) {
    if (edge->IsControlEdge()) {
      // Case 3: DummyNode(Const) -ctrl-> src -> send  
      NodeDef* dummy = AddDummyConst(opts, src_graph, edge);
      AddInput(dummy, edge->src()->name(), Graph::kControlSlot);
      send_from.Reset(dummy->name(), 0, DT_FLOAT);
    } else {
      // Case 2: src -> send  
      send_from.Reset(edge->src()->name(),
                      edge->src_output(), 
                      EdgeType(edge));
    }
  }

  void SetSendRecvAttrs(
      const PartitionOptions& opts, 
      const Edge* edge,
      NodeDefBuilder* builder) {
    builder->Attr("tensor_name",
                  strings::StrCat("edge_", edge->id(), "_", edge->src()->name()));
    builder->Attr("send_device", edge->src()->assigned_device_name());
    builder->Attr("send_device_incarnation",
                  static_cast<int64>(
                      opts.get_incarnation(edge->src()->assigned_device_name())));
    builder->Attr("recv_device", edge->dst()->assigned_device_name());
    builder->Attr("client_terminated", false);
  }

  NodeDef* AddSend(
      const PartitionOptions& opts, 
      GraphDef* gdef, 
      const Edge* edge,
      NodeDefBuilder::NodeOut send_from) {
    NodeDef* send = gdef->add_node();
    NodeDefBuilder builder(opts.new_name(edge->src()->name()), "_Send");
    SetSendRecvAttrs(opts, edge, &builder);
    builder.Device(edge->src()->assigned_device_name())
           .Input(send_from)
           .Finalize(send);
    return send;
  }

  NodeDef* AddRecv(const PartitionOptions& opts, const GraphInfo& g_info,
                   GraphDef* gdef, const Edge* edge, NodeDef** real_recv,
                   Status* status) {
    NodeDef* recv = gdef->add_node();
    NodeDefBuilder builder(opts.new_name(src->name()), "_Recv");
    SetSendRecvAttrs(opts, edge, &builder);
    builder.Device(dst->assigned_device_name())
           .Attr("tensor_type", EdgeType(edge))
           .Finalize(recv);
    return recv;

    if (edge->IsControlEdge()) {
      // Case 3: Recv -> Identity -contrl-> dst
      NodeDef* id = gdef->add_node();
      NodeDefBuilder(opts.new_name(src->name()), "Identity")
          .Device(dst->assigned_device_name())
          .Input(recv->name(), 0, cast_dtype)
          .Finalize(id);
      return id;
    } else {
      return recv;
    }
  }

  void InsertSendRecv(
      const PartitionOptions& opts,
      GraphDef* src_graph, 
      Edge* edge, 
      GraphDef* dst_graph, 
      NodeDef* dst_def) {
    NodeDefBuilder::NodeOut send_from;
    BuildSendFrom(opts, src_graph, edge, send_from);

    NodeDef* send = AddSend(opts, src_graph, edge, send_from);
    NodeDef* recv = AddRecv(opts, dst_graph, edge);

    if (edge->IsControlEdge()) {
      // Case 3: In fact, recv is identity.
      AddInput(dst_def, recv->name(), Graph::kControlSlot);
    } else {
      AddInput(dst_def, recv->name(), 0);
    }
  }
}

Status Partition(const PartitionOptions& opts, 
                 Partitions& partitions, Graph& client_graph) {
  for (const Node* dst : client_graph.op_nodes()) {
    // 1. find dst node
    GraphDef* dst_graph = &partitions[opts.node_to_loc(dst)];
    NodeDef* dst_def = InitDstNodeDef(*dst, dst_graph->add_node());
    
    // 2. search all input edges.
    for (const Edge* edge : InputsOf(dst)) {
      // 3. find src node: edge->src()
      GraphDef* src_graph = &partitions[opts.node_to_loc(src)];

      // skip sink/source nodes.
      if (!edge->src()->IsOp()) 
        continue;  

      // Case 1: same partition
      if (src_graph == dst_graph) {
        AddInput(dst_def, src->name(), edge->src_output());
        continue;
      }

      // Case 2-3: different partition
      InsertSendRecv(opts, src_graph, edge, dst_graph, dst_def);
    }
  }
}
\end{c++}
\end{leftbar}

\subsection{回调函数}

在\code{PartitionOptions}中,存在两个重要的回调函数。\code{NodeToLocFunc}用于图分裂;\code{NewNameFunc}用于给新增加的节点命名,例如\code{Send/Recv}。

\begin{leftbar}
\begin{c++}
struct PartitionOptions {
  typedef std::function<string(const Node*)> NodeToLocFunc;
  NodeToLocFunc node_to_loc = nullptr;

  typedef std::function<string(const string&)> NewNameFunc;
  NewNameFunc new_name = nullptr;

  // ignore others...
};
\end{c++}
\end{leftbar}

对于图分裂,存在两种最基本的分裂方法。

\begin{leftbar}
\begin{c++}
string SplitByDevice(const Node* node) {
  return node->assigned_device_name();
}

string SplitByWorker(const Node* node) {
  string task, device;
  DeviceNameUtils::SplitDeviceName(
      node->assigned_device_name(), &task, &device);
  return task;
}
\end{c++}
\end{leftbar}

在本地模式下,\code{NodeToLocFunc}被配置为\code{SplitByDevice}。如图\code{intraprocess-splity-by-device}所示。

\begin{figure}[H]
\centering
\includegraphics[width=0.7\textwidth]{figures/intraprocess-splity-by-device.png}
\caption{本地模式:SplitByDevice}
 \label{fig:intraprocess-splity-by-device}
\end{figure}


在分布式模式下,\code{Master}的\code{NodeToLocFunc}被配置为\code{SplitByWorker};而\code{Worker}
的\code{NodeToLocFunc}被配置为\code{SplitByDevice}。

因此,在分布式模式下,图分裂经历了两级分离。第一级按照\code{SplitByWorker}分裂,将图划分到各个\code{Worker}上去;第二级按照\code{SplitByDevice},再将图划分到各个计算设备上去。

\begin{figure}[H]
\centering
\includegraphics[width=0.7\textwidth]{figures/interprocess-splity-by-worker.png}
\caption{分布式模式:两级分裂}
 \label{fig:interprocess-splity-by-worker}
\end{figure}

\section{执行}
\label{sec:graph-operation-exec}

接下来,运行时将并发执行各个\code{PartitionGraph}。如\refig{local-graph-execution}所示,每个\code{PartitionGraph}启动一个\code{Executor},实现并发执行图的计算。

每个\code{Executor}将执行\code{PartitionGraph}的拓扑排序算法,将入度为\ascii{0}的\ascii{OP}追加到\code{ready\_queue}之中,并将其关联的\ascii{OP}的入度减\ascii{1}。调度器调度\code{ready\_queue}之中\ascii{OP
},并将其放入\code{ThreadPool}中执行对应的\ascii{Kernel}实现。

在所有\code{Partition}开始并发执行之前,需要外部将其输入传递给相应的\code{Arg}节点;当所有\code{Partition}完成计算后,外部再从\code{RetVal}节点中取走数据。其中,\code{Arg/RetVal}节点之间的数据时通过\code{FunctionCallFrame}完成交互的。

如果\code{PartitionGraph}之间需要跨设备交换数据,生产者将其放在\code{Send}节点,消费者通过\code{Recv}节点获取数据。其中,发送方不阻塞;接收方如果数据未到,则发生阻塞直至超时。此外,\code{Send/Recv}节点之间的数据是通过\code{Rendezvous}完成交互的。

\begin{figure}[H]
\centering
\includegraphics[width=1.0\textwidth]{figures/local-graph-execution.png}
\caption{执行图}
 \label{fig:local-graph-execution}
\end{figure}

因此,执行图计算需要解决如下\ascii{3}个核心问题:

\begin{enum}
  \eitem{输入/输出处理}
  \eitem{设备间数据交换} 
  \eitem{执行\code{PartitionGraph}}
\end{enum}

\subsection{输入}

在某个设备上,\code{PartitionGraph}的起始节点为\code{Arg}节点,结束节点为\code{RetVal}节点。整个过程可以看成函数调用过程,\code{Arg}用于传递函数参数,\code{RetVal}用于返回函数值。

更确切地说,\code{Arg}完成\code{PartitionGraph}的输入,\code{RetVal}完成
\code{PartitionGraph}的输出。对于\code{Arg}节点,其调用时序为:\code{set\_arg -> get\_arg}。其中,前者由\code{DirectSession}在启动\code{Executor}列表之前,通过调用\code{FunctionCallFrame.SetArgs(feeds)},传递输入参数列表的值;后者由\code{Arg}的\ascii{Kernel}实现调用。

\begin{leftbar}
\begin{c++}
Status DirectSession::Run(
  const RunOptions& run_options,
  const NamedTensorList& inputs,
  const std::vector<string>& output_names,
  const std::vector<string>& target_nodes,
  std::vector<Tensor>* outputs,
  RunMetadata* run_metadata) {

  // 1. prune graph
  // client\_graph = prune(full\_graph, inputs, outputs)
   
  // 2. split graph into partition by devices 
  // executors\_and\_partitions = split(client\_graph, devices)
  ExecutorsAndKeys* executors_and_keys = ... // ignore implements...
  
  // 3. lauch executor per partition
  // def run\_partitions(executors\_and\_partitions, inputs, outputs):
  // \ \ frame = FunctionCallFrame()
  // \ \ frame.set\_args(inputs)
  // \ \ for (executor, partition) in executors\_and\_partitions: 
  // \ \ \ \ exec.run(part)
  // \ \ frame.get\_ret\_vals(outputs)

  // 3.1 construct FunctionCallFrame
  FunctionCallFrame call_frame(
    executors_and_keys->input_types,
    executors_and_keys->output_types);
  
  // 3.2 frame.set\_args(inputs)
  // 3.2.1 construct feeds list
  gtl::InlinedVector<Tensor, 4> feed_args(inputs.size());
  for (const auto& it : inputs) {
    // (first, second) => (tensor\_name, tensor)
    feed_args[executors_and_keys->input_name_to_index[it.first]] = it.second;
  }

  // 3.2.2 frame.set\_args(feeds)
  call_frame.SetArgs(feed_args);
  
  // 3.3 concurent execution
  // for (executor, partition) in executors\_and\_partitions:
  // \ \ executor.run(partition) 

  // 3.4 fetch outputs.
}
\end{c++}
\end{leftbar}

而\code{frame.get\_arg}则有\code{Arg}来获取,并且\code{Arg}将其输出到\code{PartitionGraph}中的第一个计算节点。

\begin{leftbar}
\begin{c++}
struct ArgOp : OpKernel {
  explicit ArgOp(OpKernelConstruction* ctx) : OpKernel(ctx) {
    ctx->GetAttr("T", &dtype_);
    ctx->GetAttr("index", &index_);
  }

  void Compute(OpKernelContext* ctx) override {
    auto frame = ctx->call_frame();

    Tensor val;
    frame->GetArg(index_, &val);

    // put it into downsteram op's input.
    ctx->set_output(0, val); 
  }

 private:
  int index_;
  DataType dtype_;
};
\end{c++}
\end{leftbar}

\subsection{并发执行}

经过图分裂后,运行为每个\code{Partition}启动一个\code{Executor}。为了监听所有\code{Executor}是否全部完成,创建了一个\code{ExecutorBarrier}。并且在启动所有\code{Executor}之后,调用\code{executors\_done.Wait()}阻塞,等待所有\code{Executor}完成执行。

当完成一个\code{Executor}中完成,\code{ExecutorBarrier}的计算器减\ascii{1}(初始值为\code{num\_executors}),直至为\ascii{0},将调用其完成钩子,最终触发\code{executors\_done.Notify()}。

\begin{leftbar}
\begin{c++}
Status DirectSession::Run(
  const RunOptions& run_options,
  const NamedTensorList& inputs,
  const std::vector<string>& output_names,
  const std::vector<string>& target_nodes,
  std::vector<Tensor>* outputs,
  RunMetadata* run_metadata) {

  // 1. prune graph
  // client\_graph = prune(full\_graph, inputs, outputs)
   
  // 2. split graph into partition by devices 
  // executors\_and\_partitions = split(client\_graph, devices)
  ExecutorsAndKeys* executors_and_keys = ... // ignore implements...
  
  // 3. lauch executor per partition
  // def run\_partitions(executors\_and\_partitions, inputs, outputs):
  // \ \ frame = FunctionCallFrame()
  // \ \ frame.set\_args(inputs)
  // \ \ for (executor, partition) in executors\_and\_partitions: 
  // \ \ \ \ exec.run(part)
  // \ \ frame.get\_ret\_vals(outputs)

  // 3.1 construct FunctionCallFrame
  FunctionCallFrame call_frame(
    executors_and_keys->input_types,
    executors_and_keys->output_types);
  
  // 3.2 frame.set\_args(inputs)
  // 3.2.1 construct feeds list
  gtl::InlinedVector<Tensor, 4> feed_args(inputs.size());
  for (const auto& it : inputs) {
    // (first, second) => (tensor\_name, tensor)
    feed_args[executors_and_keys->input_name_to_index[it.first]] = it.second;
  }

  // 3.2.2 frame.set\_args(feeds)
  call_frame.SetArgs(feed_args);
  
  // 3.3 concurent execution
  // barrier = ExecutorBarrier(executors\_and\_partitions.size())
  // for (executor, partition) in executors\_and\_partitions:
  // \ \ executor.run(partition) 
  // barrier.wait()
  RunState run_state(&devices_);
  run_state.rendez = new IntraProcessRendezvous(device_mgr_.get());
  
  // 3.3.1 notify when finished.
  size_t num_executors = executors_and_keys->items.size();
  ExecutorBarrier* barrier = new ExecutorBarrier(
      num_executors, run_state.rendez, [&run_state](const Status& ret) {
        {
          mutex_lock l(run_state.mu_);
          run_state.status.Update(ret);
        }
        run_state.executors_done.Notify();
      });

  Executor::Args args;
  args.call_frame = &call_frame;
  args.rendezvous = run_state.rendez;
  args.runner = [this, pool](Executor::Args::Closure c) {
    SchedClosure(pool, std::move(c));
  };

  // 3.3.2 lauch all executors.
  for (const auto& item : executors_and_keys->items) {
    item.executor->RunAsync(args, barrier->Get());
  }

  // 3.3.3 wait until all executors finished.
  WaitForNotification(&run_state, 
      &step_cancellation_manager,
      GetTimeoutInMs(run_options));

  // 3.4 fetch outputs.
}
\end{c++}
\end{leftbar}

\subsection{输出}

同理,对于\code{RetVal}节点,其调用时序为:\code{set\_ret\_val -> get\_ret\_val}。前者由\code{RetVal}完成,后者由\code{DirectSession}完成。

\begin{leftbar}
\begin{c++}
struct RetvalOp : OpKernel {
  explicit RetvalOp(OpKernelConstruction* ctx) : OpKernel(ctx) {
    ctx->GetAttr("T", &dtype_);
    ctx->GetAttr("index", &index_);
  }

  void Compute(OpKernelContext* ctx) override {
    // get upstream op's output.
    const Tensor& val = ctx->input(0); 

    auto frame = ctx->call_frame();
    frame->SetRetval(index_, val);
  }

 private:
  int index_;
  DataType dtype_;
};
\end{c++}
\end{leftbar}

等所有\code{Executor}运行结束后,\code{DirectSession}便可以从\code{FunctionCallFrame}中取出所有输出值,并将其放置在\code{outputs},并返回\ascii{Client}。

\begin{leftbar}
\begin{c++}
Status DirectSession::Run(
  const RunOptions& run_options,
  const NamedTensorList& inputs,
  const std::vector<string>& output_names,
  const std::vector<string>& target_nodes,
  std::vector<Tensor>* outputs,
  RunMetadata* run_metadata) {
  
  // 1. prune graph
  // client\_graph = prune(full\_graph, inputs, outputs)
   
  // 2. split graph into partition by devices 
  // executors\_and\_partitions = split(client\_graph, devices)
  executors_and_keys = ... // ignore implements...
  
  // 3. lauch executor per partition
  // def run\_partitions(executors\_and\_partitions, inputs, outputs):
  // \ \ frame = FunctionCallFrame()
  // \ \ frame.set\_args(inputs)
  // \ \ for (executor, partition) in executors\_and\_partitions: 
  // \ \ \ \ exec.run(part)
  // \ \ frame.get\_ret\_vals(outputs)

  // 3.1 construct FunctionCallFrame
  FunctionCallFrame call_frame(
    executors_and_keys->input_types,
    executors_and_keys->output_types);
  
  // 3.2 frame.set\_args(inputs)
  // 3.2.1 construct feeds list
  gtl::InlinedVector<Tensor, 4> feed_args(inputs.size());
  for (const auto& it : inputs) {
    // (first, second) => (tensor\_name, tensor)
    feed_args[executors_and_keys->input_name_to_index[it.first]] = it.second;
  }

  // 3.2.2 frame.set\_args(feeds)
  call_frame.SetArgs(feed_args);
  
  // 3.3 concurent execution
  // barrier = ExecutorBarrier(executors\_and\_partitions.size())
  // for (executor, partition) in executors\_and\_partitions:
  // \ \ executor.run(partition) 
  // barrier.wait()
  RunState run_state(&devices_);
  run_state.rendez = new IntraProcessRendezvous(device_mgr_.get());
  
  // 3.3.1 notify when finished.
  size_t num_executors = executors_and_keys->items.size();
  ExecutorBarrier* barrier = new ExecutorBarrier(
      num_executors, run_state.rendez, [&run_state](const Status& ret) {
        {
          mutex_lock l(run_state.mu_);
          run_state.status.Update(ret);
        }
        run_state.executors_done.Notify();
      });

  Executor::Args args;
  args.call_frame = &call_frame;
  args.rendezvous = run_state.rendez;
  args.runner = [this, pool](Executor::Args::Closure c) {
    SchedClosure(pool, std::move(c));
  };

  // 3.3.2 lauch all executors.
  for (const auto& item : executors_and_keys->items) {
    item.executor->RunAsync(args, barrier->Get());
  }

  // 3.3.3 wait until all executors finished.
  WaitForNotification(&run_state, 
      &step_cancellation_manager,
      GetTimeoutInMs(run_options)); 

  // 3.4 fetch outputs. 
  // 3.4.1 frame.get\_get\_ret\_vals
  std::vector<Tensor> sorted_outputs;
  Status s = call_frame.ConsumeRetvals(&sorted_outputs);

  // 3.4.2 emplace to outputs, and return to client.
  outputs->reserve(sorted_outputs.size());
  for (int i = 0; i < output_names.size(); ++i) {
    const string& output_name = output_names[i];
    outputs->emplace_back(
      std::move(sorted_outputs[
        executors_and_keys->output_name_to_index[output_name]]));
  }
}
\end{c++}
\end{leftbar}

至此,整个\code{DirectSession.Run}解读完毕。但是,\code{Partition}中节点如何被调度执行的,\code{Partition}之间的\code{Send/Recv}是如何工作的呢?

因此,在最后一公里,还要探究三件事情。

\begin{enum}
  \eitem{\code{SendOp}与\code{RecvOp}的工作原理}
  \eitem{\code{IntraProcessRendezvous}的工作原理} 
  \eitem{\code{Executor}的调度算法}
\end{enum}

\section{设备间通信}

\code{SendOp/RecvOp}通过\code{Rendezvous}交换数据的;它实现了消息发送/接受,与具体消息传递相解耦。例如,在单进程内,\code{SendOp/RecvOp}基于\code{IntraProcessRendezvous}传递数据的;而在多进程环境中,\code{SendOp/RecvOp}则可以基于\code{GrpcRendezvous}传递数据。

首先,探究这两个\ascii{OP}的工作原理;然后,再探究本地模式下,\code{IntraProcessRendezvous}的工作原理。

\subsection{SendOp实现}

如\refig{local-send-recv-ops}所示,进程内的\code{Send/Recv}通过唯一的标识\code{ParsedKey}实现数据的交换。

\begin{figure}[H]
\centering
\includegraphics[width=0.8\textwidth]{figures/local-send-recv-ops.png}
\caption{进程内\code{SendOp}与\code{RecvOp}的数据交换}
 \label{fig:local-send-recv-ops}
\end{figure}

参考\code{SendOp}的\ascii{Kernel}实现,看起来非常复杂,但是它实际上就做了一件事情。首先,它构造设备间通信的关键字\code{ParsedKey},然后调用\code{Rendezvous.Send}操作,将上游\ascii{OP}输入到\code{SendOp}的\code{Tensor}发送到\code{Rendezvous}缓存之中,该操作是非阻塞的。

其中,\code{ParsedKey}包括:发送设备,接受设备,设备全局标识,及其待发送\code{Tensor}的标识(\code{src:output\_index})组成。

\begin{leftbar}
\begin{c++}
struct SendOp : OpKernel {
  explicit SendOp(OpKernelConstruction* ctx) : OpKernel(ctx) {
    string send_device;
    ctx->GetAttr("send_device", &send_device);

    string recv_device;
    ctx->GetAttr("recv_device", &recv_device);

    uint64 send_device_incarnation;
    ctx->GetAttr(
        "send_device_incarnation",
        reinterpret_cast<int64*>(&send_device_incarnation));

    string tensor_name;
    ctx->GetAttr("tensor_name", &tensor_name);

    key_prefix_ = GetRendezvousKeyPrefix(
        send_device, recv_device,
        send_device_incarnation, tensor_name);

    GetRendezvousKey(key_prefix_, {0, 0}, &parsed_key_.buf_);
    Rendezvous::ParseKey(parsed_key_.buf_, &parsed_key_);

    if (!ctx->GetAttr("_hostmem_sendrecv", &hostmem_sendrecv_).ok()) {
      hostmem_sendrecv_ = false;
    }
  }

  void Compute(OpKernelContext* ctx) override {
    Rendezvous::Args args;
    args.device_context = ctx->op_device_context();
    args.alloc_attrs = ctx->input_alloc_attr(0);
    
    // get it from upstream op's output, and as this op's input.
    ctx->rendezvous()->Send(
        CreateParsedkey(ctx), args, ctx->input(0),
        ctx->is_input_dead());
  }
 
 private:
  Rendezvous::ParsedKey CreateParsedkey(OpKernelContext* ctx) {
    FrameAndIter frame_iter = GetFrameAndIter(ctx, hostmem_sendrecv_);
    if (frame_iter == FrameAndIter(0, 0)) {
      return parsed_key_;
    } else {
      Rendezvous::ParsedKey in_loop_parsed;
      GetRendezvousKey(key_prefix_, frame_iter, &in_loop_parsed.buf_);
      Rendezvous::ParseKey(in_loop_parsed.buf_, &in_loop_parsed);
      return in_loop_parsed;
    }  
  }

 private:
  string key_prefix_;
  Rendezvous::ParsedKey parsed_key_;
  bool hostmem_sendrecv_;
};
\end{c++}
\end{leftbar}

\subsection{RecvOp实现}

同理地,可以猜测\code{Recv}的\ascii{Kernel}的实现了。它首先构造\code{Rendezvous}的\code{ParsedKey},然后调用\code{Rendezvous.RecvAsync}操作,从\code{Rendezvous}取出相应的\code{Tensor}。

这是一个异步操作,当\code{Rendezvous}中数据可获取,便开始执行回调函数\code{done\_cb},它将其得到的\code{Tensor}输出到下游\ascii{OP}。

\begin{leftbar}
\begin{c++}
struct RecvOp : AsyncOpKernel {
  explicit RecvOp(OpKernelConstruction* ctx) : AsyncOpKernel(ctx) {
    string send_device;
    ctx->GetAttr("send_device", &send_device);
  
    string recv_device;
    ctx->GetAttr("recv_device", &recv_device);

    uint64 send_device_incarnation;
    ctx->GetAttr(
        "send_device_incarnation",
        reinterpret_cast<int64*>(&send_device_incarnation));
  
    string tensor_name;
    ctx->GetAttr("tensor_name", &tensor_name);

    key_prefix_ = GetRendezvousKeyPrefix(
        send_device, recv_device,
        send_device_incarnation, tensor_name);
  
    GetRendezvousKey(key_prefix_, {0, 0}, &parsed_key_.buf_);
    Rendezvous::ParseKey(parsed_key_.buf_, &parsed_key_));
    if (!ctx->GetAttr("_hostmem_sendrecv", &hostmem_sendrecv_).ok()) {
      hostmem_sendrecv_ = false;
    }
  }

  void ComputeAsync(OpKernelContext* ctx, DoneCallback done) override {
    Rendezvous::Args args;
    args.device_context = ctx->op_device_context();
    args.alloc_attrs = ctx->output_alloc_attr(0);

    ctx->rendezvous()->RecvAsync(
      CreateParsedKey(ctx), args, CreateDoneCallback(ctx));
  }

 private:
  Rendezvous::ParsedKey CreateParsedKey(OpKernelContext* ctx) {
    FrameAndIter frame_iter = GetFrameAndIter(ctx, hostmem_sendrecv_);
    if (frame_iter == FrameAndIter(0, 0)) {
      return parsed_key_;
    } else {
      Rendezvous::ParsedKey in_loop_parsed;
      GetRendezvousKey(key_prefix_, frame_iter, &in_loop_parsed.buf_);
      Rendezvous::ParseKey(in_loop_parsed.buf_, &in_loop_parsed);
      return in_loop_parsed;
    }  
  }

  Rendezvous::DoneCallback CreateDoneCallback(OpKernelContext* ctx) {
    using namespace std::placeholders;
    return std::bind([ctx](DoneCallback done, const Status& s, 
        const Rendezvous::Args&, const Rendezvous::Args&, 
        const Tensor& val, bool is_dead) {
          ctx->SetStatus(s);
          if (s.ok()) {
            if (!is_dead) {
              // put it into downstream op's input.
              ctx->set_output(0, val);  
            }
            *ctx->is_output_dead() = is_dead;
          }
          done();
        },
        std::move(done), _1, _2, _3, _4, _5);  
  }

 private:
  string key_prefix_;
  Rendezvous::ParsedKey parsed_key_;
  bool hostmem_sendrecv_;
};
\end{c++}
\end{leftbar}

\begin{savequote}[45mm]
\ascii{Any fool can write code that a computer can understand. Good programmers write code that humans can understand.}
\qauthor{\ascii{- Martin Flower}}
\end{savequote}

\chapter{分布式TensorFlow} 
\label{ch:distributed}

\begin{content}

\tf{}可以运行在分布式环境中,完成计算图的执行过程。本章将重点介绍 分布式运行时的基本架构与运行机制;重点讨论各个服务进程之间的交互关系;并且深入剖析在分布式环境中图操作,及其会话生命周期控制的关键技术;

\end{content}

\section{分布式模式}

\begin{content}

在分布式模式中,\ascii{Client}负责计算图的构造,然后通过调用\code{Session.run},启动计算图的执行过程。

\ascii{Master}进程收到计算图执行的消息后,启动计算图的剪枝,分裂,优化等操作;最终将子图分发注册到各个\ascii{Worker}进程上,然后触发各个\ascii{Worker}进程并发执行子图。

\ascii{Worker}进程收到子图注册的消息后,根据本地计算设备资源,再将计算子图实施二次分裂,将子图分配在各个计算设备上,最后启动各个计算设备并发地执行子图;如果\ascii{Worker}之间存在数据交换,可以通过进程间通信完成交互。

\begin{figure}[H]
\centering
\includegraphics[width=0.8\textwidth]{figures/distributed.png}
\caption{分布式模式}
 \label{fig:distributed}
\end{figure}

\subsection{图操作}

如\refig{dist-runtime}所示,在\code{run\_step}执行过程之中,涉及计算图的剪枝、分裂、执行三个重要的图操作。其中,在分布式运行时,图分裂经历了两级分裂过程。

\begin{enum}
  \eitem{一级分裂:由\code{MasterSession}完成,按照\code{SplitByWorker}或\code{SplitByTask}完成图分裂过程;}
  \eitem{二级分裂:由\code{WorkerSession}完成,按照\code{SplitByDevice}完成图分裂过程。}
\end{enum}

在分布式模式中,图剪枝也体现了\tf{}部分执行的设计理念;而图分裂和执行也体现了\tf{}并发执行的设计理念。其中,图剪枝仅发生在\ascii{Master}上,不发生在\ascii{Worker}上;而图分裂发生在\ascii{Master}和\ascii{Worker}上;图执行仅仅发生在\ascii{Worker}上,不发生在\ascii{Master}上。

\begin{figure}[H]
\centering
\includegraphics[width=1.0\textwidth]{figures/dist-runtime.png}
\caption{分布式:图操作}
 \label{fig:dist-runtime}
\end{figure}

\subsubsection{图分裂}

为了更好地理解分布式运行时的工作原理,以一个简单的例子阐述图操作的具体过程。如\refig{dist-exp-1}所示,假如存在一个简单的计算图,并且\code{f, c, a}部署在\code{/job:ps/task:0}上,且分别被编排到\code{CPU0, CPU1, CPU2}上;\code{g, h}部署在\code{/job:worker/task:0}上,且同时被编排到\code{GPU0}上;\code{b, d, e}部署在\code{/job:worker/task:1}上,且\code{d, e}被编排到\code{GPU0}上,而\code{b}被编排到\code{GPU1}上。

\begin{figure}[H]
\centering
\includegraphics[width=1.0\textwidth]{figures/dist-exp-1.png}
\caption{分布式:图分裂}
 \label{fig:dist-exp-1}
\end{figure}

\subsubsection{数据交换}

如\refig{dist-exp-2}所示,对于跨设备的边,运行时自动实施边的分裂,分别在发送端和接收端插入\code{Send}和\code{Recv}两个末端节点。

进程间的\code{Send}和\code{Recv}节点,通过\code{GrpcRemoteRendezvous}实现数据交换。例如,\code{/job:ps/task:0}与\code{/job:worker/task:0},\code{/job:ps/task:0}与\code{/job:worker/task:1},或\code{/job:worker/task:0}与\code{/job:worker/task:1}之间是通过\code{GrpcRemoteRendezvous}完成数据交换的。

而进程内的\code{Send}和\code{Recv}节点,则通过\code{IntraProcessRendezvous}实现数据交换。例如,\code{/job:worker/task:1}内存在两个\code{GPU},它们之间采用\code{IntraProcessRendezvous}实现数据交换。

关于\code{Rendezvous}的具体实现过程,下文将还会重点讲述。

\begin{figure}[H]
\centering
\includegraphics[width=0.8\textwidth]{figures/dist-exp-2.png}
\caption{分布式:数据交换}
 \label{fig:dist-exp-2}
\end{figure}

\subsection{领域模型}

如\refig{cc-dist-model}所示,在\tf{}分布式运行时,存在一个精巧的领域模型。

\begin{figure}[H]
\centering
\includegraphics[width=0.7\textwidth]{figures/cc-dist-model.png}
\caption{分布式:领域模型}
 \label{fig:cc-dist-model}
\end{figure}

\subsubsection{Cluster}

\ascii{Cluster}使用\ascii{ClusterSpec}进行描述,它可以划分为一个或多个\ascii{Job},一个\ascii{Job}包含一个或多个\ascii{Task}。也就是说,\ascii{TensorFlow}集群是由执行计算图的任务集\ascii{(Task Set)}组成的。

每个\ascii{Task}可以独立运行在单独的机器上,也可以在一台机器上运行多个\ascii{Task}(例如,单机多\ascii{CPU},或单机多\ascii{GPU})。

\subsubsection{Job}

将目的相同的\ascii{Task}划归在同一个\ascii{Job}中。每个\ascii{Job}使用\code{job\_id}唯一标识。

一般地,在分布式深度学习的模型训练过程中,存在两种基本的\ascii{Job}类型:

\begin{enum}
  \eitem{\ascii{ps}:负责模型参数的存储和更新;}
  \eitem{\ascii{worker}:负责计算密集型的模型训练和推理。}
\end{enum}

\begin{figure}[H]
\centering
\includegraphics[width=0.5\textwidth]{figures/py-dist-ps-worker.png}
\caption{分布式模型训练:PS与Worker之间的交互}
 \label{fig:py-dist-ps-worker}
\end{figure}

\subsubsection{Task}

一般地,在分布式运行时中,\ascii{Task}运行在独立的进程中,并在其上运行一个\code{tf.train.Server}实例。其中,\ascii{Task}使用\code{job\_id:task\_index}的二元组唯一标识。

\subsubsection{Server}

\ascii{Server}表示\ascii{Task}的服务进程,它对外提供\code{MasterService}和\code{WorkerService}服务。也就是说,\ascii{Server}可以同时扮演\ascii{Master}和\ascii{Worker}两种角色。

\subsection{组建集群}

在分布式的\tf{}运行时中,每个\ascii{Task}启动了一个\ascii{Server},并对外提供\code{MasterService}服务和\code{WorkerService}服务。其中,组建\ascii{TensorFlow}集群包括两个基本步骤:

\begin{enum}
  \eitem{创建\code{tf.train.ClusterSpec},描述集群中\ascii{Task}的部署信息,并以\ascii{Job}的方式组织;}
  \eitem{对于每一个\ascii{Task},启动一个\code{tf.train.Server}实例。}
\end{enum}

\subsubsection{集群配置}

\code{ClusterSpec}描述了集群中\ascii{Task}的部署信息,并以\ascii{Job}的方式组织。一般地,在分布式的执行模式中,为每个\ascii{Task}启动一个进程。因此,\code{ClusterSpec}同时也描述了\ascii{TensorFlow}分布式运行时的进程分布情况。

例如,存在一个\ascii{TensorFlow}集群,它由\code{ps}和\code{worker}两个\ascii{Job}组成。其中,\code{ps}部署在\code{ps0:2222, ps1:2222}上;\code{worker}部署在\code{worker0:2222, worker1:2222, worker2:2222}上。

\begin{leftbar}
\begin{python}
tf.train.ClusterSpec({
  "worker": [
    "worker0:2222",   # /job:worker/task:0
    "worker1:2222",   # /job:worker/task:1
    "worker2:2222"    # /job:worker/task:2
  ],  
  "ps": [
    "ps0:2222",       # /job:ps/task:0
    "ps1:2222"        # /job:ps/task:0
  ]})
\end{python}
\end{leftbar}

在此例中,未显式地指定\ascii{Task}的索引。默认地,一个\ascii{Job}的\ascii{Task}集合中,\ascii{Task}索引从\ascii{0}开始按序自增的。

\subsubsection{Protobuf描述}

\begin{leftbar}
\begin{python}
message JobDef {
  string name = 1;
  map<int32, string> tasks = 2;
}

message ClusterDef {
  repeated JobDef job = 1;
}
\end{python}
\end{leftbar}

其中,\code{tasks}的关键字表示\code{task\_index},值表示\code{host:port}。

\end{content}

\section{服务器}

\begin{content}

\code{Server}是一个基于\ascii{gRPC}的服务器,负责管理本地设备集。它对外提供\code{MasterService}服务和\code{WorkerService}服务,具有同时扮演\ascii{Master}和\ascii{Worker}的角色。

\subsection{领域模型}

如\refig{cc-server-model}所示,\code{GrpcServer}扮演\ascii{Master}的角色时,对外提供\code{MasterService}服务;其中,它为每一个接入的\ascii{Client}启动一个\code{MasterSession}实例,并使用全局唯一的\code{session\_handle}标识它。也就是说,\ascii{Master}可以接入多个\ascii{Client},而一个\ascii{Client}则只能接入一个特定的\ascii{Master}。

\code{GrpcServer}扮演\ascii{Worker}的角色时,对外提供\code{WorkerService}服务;其中,每个\ascii{Worker}可以为多个\ascii{Master}提供计算服务,它为每个向它请求计算服务的\code{MasterSession}生成一个相应的\code{WorkerSession}实例,等待相应的\code{MasterSession}下发计算图的\emph{注册}和\emph{执行}命令。

整个\code{GrpcServer}实例承载于\code{grpc::Server}进程之上,它监听特定端口的消息,当消息到达时自动派发到\code{MasterService}或\code{WorkerService}中相应的消息处理的回调函数。

\begin{figure}[H]
\centering
\includegraphics[width=1.0\textwidth]{figures/cc-server-model.png}
\caption{Server领域模型}
 \label{fig:cc-server-model}
\end{figure}

\subsubsection{Protobuf描述}

当\code{protocol}为\code{grpc}时,系统运行时将启用基于\ascii{gRPC}实现的\code{GrpcServer}实例。此外,可以通过\code{ConfigProto}实现运行时参数的配置。也就是说,\tf{}的架构是对外开放的。例如,通过扩展\code{protocol}支持新的通信协议,实现基于新协议的\code{Server}实例。

\begin{leftbar}
\begin{python}
message ServerDef {
  ClusterDef cluster = 1;
  
  string job_name = 2;
  int32 task_index = 3;

  ConfigProto default_session_config = 4;
  string protocol = 5;
}
\end{python}
\end{leftbar}

\subsubsection{服务互联}

如\refig{cc-server-interact}所示,一个\ascii{Server}实例通过\code{tf.train.ClusterSpec}与集群中的其他\ascii{Server}实例实现互联。

\begin{figure}[H]
\centering
\includegraphics[width=0.5\textwidth]{figures/cc-server-interact.png}
\caption{服务互联}
 \label{fig:cc-server-interact}
\end{figure}

如\refig{cc-server-interact-1}所示,当\ascii{Client}接入其中一个\ascii{Server},此时它扮演了\ascii{Master}的角色,其他\ascii{Server}则扮演了\ascii{Worker}的角色。特殊地,\ascii{Client}接入的\ascii{Server}也扮演了\ascii{Worker}的角色。

\begin{figure}[H]
\centering
\includegraphics[width=0.5\textwidth]{figures/cc-server-interact-1.png}
\caption{单Client接入集群}
 \label{fig:cc-server-interact-1}
\end{figure}

如\refig{cc-server-interact-2}所示,可能存在多个\ascii{Client}分别接入不同的\ascii{Server}实例。此时,\ascii{Client}接入的\ascii{Server}实例扮演了\ascii{Master}角色。但是,该\ascii{Server}实例,相对于集群中另外的\ascii{Server}实例,则扮演了\ascii{Worker}角色。

\begin{figure}[H]
\centering
\includegraphics[width=0.5\textwidth]{figures/cc-server-interact-2.png}
\caption{多Client接入集群}
 \label{fig:cc-server-interact-2}
\end{figure}

特殊地,\ascii{Client}与\ascii{Master}可以部署在同一个进程内。此时,\ascii{Client}与\ascii{Master}之间的交互更加简单,两者直接使用函数调用,避免了\ascii{gRPC}交互的额外开销。依次类推,在同一个\ascii{Server}内,\ascii{Master}与\ascii{Worker}可以部署在同一进程内。此时,\ascii{Master}与\ascii{Worker}之间直接使用函数调用。

\subsection{状态机}

如\refig{dist-grpc-server-state-machine}所示,\code{GrpcServer}是一个基于\code{grpc::Server}的服务器,它管理和维护了一个简单的状态机。

\code{GrpcServer}在\code{New}状态上启动了\code{grpc::Server}服务,但对外并没有提供服务;而在\code{Started}状态上启动服务,对外提供\code{MasterService}和\code{WorkerService}的\code{RPC}消息服务;最终,在\code{Stopped}状态下停止\code{MasterService}和\code{WorkerService}服务。

\begin{figure}[H]
\centering
\includegraphics[width=0.7\textwidth]{figures/dist-grpc-server-state-machine.png}
\caption{GrpcServer状态机}
 \label{fig:dist-grpc-server-state-machine}
\end{figure}

\subsubsection{创建服务}

\begin{figure}[H]
\centering
\includegraphics[width=0.7\textwidth]{figures/dist-grpc-server-factory.png}
\caption{多态创建Server实例}
 \label{fig:dist-grpc-server-factory}
\end{figure}

\begin{leftbar}
\begin{c++}
struct GrpcServerFactory : ServerFactory {
  bool AcceptsOptions(const ServerDef& server_def) override {
    return server_def.protocol() == "grpc";
  }

  Status NewServer(const ServerDef& server_def,
      std::unique_ptr<ServerInterface>* out_server) override {
    GrpcServer::Create(server_def, Env::Default(), out_server);
    return Status::OK();
  }
};
\end{c++}
\end{leftbar}

\begin{leftbar}
\begin{c++}
void GrpcServer::Create(
    const ServerDef& server_def, Env* env,
    std::unique_ptr<ServerInterface>* out_server) {
  auto ret = std::make_unique<GrpcServer>(server_def, env);
  ret->Init();
  *out_server = std::move(ret);
}
\end{c++}
\end{leftbar}

如\refig{cc-server-model}所示,\code{GrpcServer::Init}将完成\code{GrpcServer}领域对象的初始化,主要包括如下\ascii{3}个基本过程。

\begin{enum}
  \eitem{初始化\code{MasterEnv}实例;}  
  \eitem{初始化\code{WorkerEnv}实例;}  
  \eitem{创建并启动\code{grpc::Server}}    
    \begin{enum}
    \eitem{初始化\code{MasterService}}      
    \begin{nitemize}
      \eitem{创建\code{Master}实例;}  
      \eitem{创建\code{MasterService}实例;}
    \end{nitemize}
    \eitem{初始化\code{WorkerService}}          
    \begin{nitemize}          
      \eitem{创建\code{Worker}实例;}  
      \eitem{创建\code{WorkerService}实例。}
    \end{nitemize}      
    \end{enum}
\end{enum}

为了更好地理解整个\code{GrpcServer}实例的初始化过程,此处做了局部的重构。首先,它初始化\code{MasterEnv, WorkerEnv}实例;然后,创建并启动\code{grpc::Server}服务器。其中,初始化\code{MasterEnv/WorkerEnv},就是完成\code{MasterEnv/WorkerEnv}对象的构造过程,在此不再冗述。重点探究一下\code{grpc::Server}实例的构建和启动过程。

\begin{leftbar}
\begin{c++}
void GrpcServer::Init() {
  InitMasterEnv();
  InitWorkerEnv();
  StartGrpcServer();
}
\end{c++}
\end{leftbar}

实现使用构建器创建\code{grpc::Server}实例。首先,配置\code{grpc::Server}的服务选项;然后,分别构建\code{MasterService}实例和\code{WorkerService}实例。最后,调用\code{builder.BuildAndStart}方法启动\code{grpc::Server}服务器。

需要注意的是,\code{grpc::Server}启动时,\code{GrpcServer}依然处于\code{New}状态,
\code{grpc::Server}暂时还未对外提供\code{MasterService}服务和\code{WorkerService}服务。直至\code{GrpcServer}迁移至\code{Started}状态位置,\code{grpc::Server}才真正对外提供\code{MasterService}服务和\code{WorkerService}服务。

\begin{leftbar}
\begin{c++}
void InitServerBuilder(::grpc::ServerBuilder& builder) {
  builder.AddListeningPort(
    strings::StrCat("0.0.0.0:", GetRequestedPort()),
    GetServerCredentials(server_def_), &bound_port_);
  builder.SetMaxMessageSize(std::numeric_limits<int32>::max());
  builder.SetOption(
      std::unique_ptr<::grpc::ServerBuilderOption>(new NoReusePortOption));
}

void GrpcServer::StartGrpcServer() {
  ::grpc::ServerBuilder builder;

  InitServerBuilder(builder);
  InitMasterService(builder);
  InitWorkerService(builder);

  server_ = builder.BuildAndStart();  
}
\end{c++}
\end{leftbar}

很容易发现,\code{grpc::Server}对外提供\code{MasterService}服务的实体是\code{GrpcMasterService}实例。当消息到达时,将自动回调\code{GrpcMasterService}实例中相应的消息处理函数。其中,在消息处理函数中,其业务逻辑的处理完全依托于\code{Master}的领域对象。

\begin{leftbar}
\begin{c++}
std::unique_ptr<Master> GrpcServer::CreateMaster(
    MasterEnv* master_env) {
  return std::make_unique<Master>(master_env);
}

AsyncServiceInterface* NewGrpcMasterService(
    Master* master, ::grpc::ServerBuilder* builder) {
  return new GrpcMasterService(master, builder);
}

void GrpcServer::InitMasterService() {
  master_impl_ = CreateMaster(&master_env_);
  master_service_ = NewGrpcMasterService(
      master_impl_.get(), &builder);  
}
\end{c++}
\end{leftbar}

依次类推,\code{grpc::Server}对外提供\code{WorkerService}服务的实体是\code{GrpcWorkerService}实例。当消息到达时,将自动回调\code{GrpcWorkerService}实例中相应的消息处理函数。其中,在消息处理函数中,其业务逻辑的处理完全依托于\code{GrpcWorker}的领域对象。

\begin{leftbar}
\begin{c++}
std::unique_ptr<GrpcWorker> NewGrpcWorker(WorkerEnv* env) {
  return std::unique_ptr<GrpcWorker>(new GrpcWorker(env));
}

AsyncServiceInterface* NewGrpcWorkerService(
    GrpcWorker* worker, ::grpc::ServerBuilder* builder) {
  return new GrpcWorkerService(worker, builder);
}

void GrpcServer::InitWorkerService(::grpc::ServerBuilder& builder) {
  worker_impl_ = NewGrpcWorker(&worker_env_);
  worker_service_ = NewGrpcWorkerService(
    worker_impl_.get(), &builder);
}
\end{c++}
\end{leftbar}

\subsubsection{启动服务}

在\code{New}状态,\code{grpc::Server}已经启动,但暂时没有对外提供\code{MasterService}服务和\code{WorkerService}服务。通过调用\code{GrpcServer::Start}方法后,\code{GrpcServer}的状态从\code{New}迁移\code{Started}状态,并启动了两个独立的线程,分别启动\code{MasterService}和\code{WorkerService}的消息处理器。此时,\code{GrpcServer}正式对外提供\code{MasterService}和\code{WorkerService}。

\begin{leftbar}
\begin{c++}
Status GrpcServer::Start() {
  mutex_lock l(mu_);
  switch (state_) {
    case NEW: {
      master_thread_.reset(
          env_->StartThread(ThreadOptions(), "TF_master_service",
                            [this] { master_service_->HandleRPCsLoop(); }));
      worker_thread_.reset(
          env_->StartThread(ThreadOptions(), "TF_worker_service",
                            [this] { worker_service_->HandleRPCsLoop(); }));
      state_ = STARTED;
      return Status::OK();
    }
    case STARTED:
      LOG(INFO) << "Server already started(" << target() << ")";    
      return Status::OK();
    case STOPPED:
    default:
      CHECK(false);
  }
}
\end{c++}
\end{leftbar}

\subsubsection{等待终止服务}

为了持久地对外提供\code{MasterService}服务和\code{WorkerService}服务,需要分别对线程\code{TF\_master\_service}和\code{TF\_worker\_service}实施\code{join}操作,使得主线程挂起,直至这两个线程终止。

通过调用\code{GrpcServer::Join}方法,当\code{GrpcServer}处于\code{Started}或\code{Stoped}状态时,它将自动调用\code{Thread}的析构函数。

\begin{leftbar}
\begin{c++}
Status GrpcServer::Join() {
  mutex_lock l(mu_);
  switch (state_) {
    case NEW:
      // Prevent the server from being started subsequently.
      state_ = STOPPED;
      return Status::OK();
    case STARTED:
    case STOPPED:
      master_thread_.reset();
      worker_thread_.reset();
      return Status::OK();
    default:
      CHECK(false);
  }
}
\end{c++}
\end{leftbar}

例如,基于\code{C++}标准库实现的\code{StdThread}中,其析构函数将调用\code{std::thread}的\code{join}方法。

\begin{leftbar}
\begin{c++}
struct StdThread : Thread {
  StdThread(const ThreadOptions&, const string&, 
      std::function<void()> fn)
    : thread_(fn) {
  }

  ~StdThread() override { 
    thread_.join(); 
  }

 private:
  std::thread thread_;
};
\end{c++}
\end{leftbar}

\subsubsection{终止服务}

遗憾的是,目前\code{GrpcServer}并不能优雅地退出。因此,在工程实践环境中,\tf{}的分布式运行时常常需要借助于\code{Kubernetes},实现\code{GrpcServer}服务的自动管理。

\begin{leftbar}
\begin{c++}
Status GrpcServer::Stop() {
  mutex_lock l(mu_);
  switch (state_) {
    case NEW:
      state_ = STOPPED;
      return Status::OK();
    case STARTED:
      return errors::Unimplemented(
          "Clean shutdown is not currently implemented");
    case STOPPED:
      LOG(INFO) << "Server already stopped(" << target() << ")";
      return Status::OK();
    default:
      CHECK(false);
  }
}
\end{c++}
\end{leftbar}

\section{Master服务}

\begin{content}

\code{MasterService}是一个\ascii{RPC}服务。当\ascii{Client}根据\code{target}接入\ascii{Server}实例后,\ascii{Server}扮演了\ascii{Master}的角色,对外提供\code{MasterService}服务。

其中,\ascii{Client}与\ascii{Master}之间的交互遵循\code{MasterService}定义的接口规范。也就是说,\code{MasterService}定义了\ascii{Client}接入\ascii{Master}的公共契约,负责协调和控制多个\code{WorkerService}的执行过程。

\subsection{接口定义}

在\code{master\_service.proto}文件中,定义了\code{MasterService}的所有接口;而在\code{master.proto}文件中,定义了各个接口的消息体。

\begin{leftbar}
\begin{c++}
service MasterService {
  rpc CreateSession(CreateSessionRequest) 
      returns (CreateSessionResponse);
  
  rpc ExtendSession(ExtendSessionRequest) 
      returns (ExtendSessionResponse);

  rpc PartialRunSetup(PartialRunSetupRequest) 
      returns (PartialRunSetupResponse);

  rpc RunStep(RunStepRequest) 
      returns (RunStepResponse);
  
  rpc CloseSession(CloseSessionRequest) 
      returns (CloseSessionResponse);
  
  rpc ListDevices(ListDevicesRequest) 
      returns (ListDevicesResponse);

  rpc Reset(ResetRequest) 
      returns (ResetResponse);
}
\end{c++}
\end{leftbar}

\subsection{访问服务}

一般地,\ascii{Client}使用接口\code{MasterInterface}获取远端\code{MasterService}的服务。特殊地,\code{MasterInterface}的所有接口都是同步接口,使得\ascii{Client}访问远端\code{MasterService}服务犹如调用本地函数一般。

需要注意的是,因为\code{RunStepRequest/RunStepResponse}消息中可能包含较大的\code{Tensor}实例。为了避免不必要的对象拷贝,实现特化实现了消息包装器。

\begin{leftbar}
\begin{c++}
// Abstract interface for communicating with the TensorFlow Master service.
//
// This interface supports both RPC-based master implementations, and
// in-process master implementations that do not require an RPC roundtrip.
struct MasterInterface {
  virtual ~MasterInterface() {}
  
  virtual Status CreateSession(
      CallOptions* call_options,
      const CreateSessionRequest* request,
      CreateSessionResponse* response) = 0;

  virtual Status ExtendSession(
      CallOptions* call_options,
      const ExtendSessionRequest* request,
      ExtendSessionResponse* response) = 0;

  virtual Status PartialRunSetup(
      CallOptions* call_options,
      const PartialRunSetupRequest* request,
      PartialRunSetupResponse* response) {
    return errors::Unimplemented(
      "Partial run not implemented for master");
  }

  virtual Status RunStep(
      CallOptions* call_options,
      RunStepRequestWrapper* request,
      MutableRunStepResponseWrapper* response) = 0;

  // Wrapper classes for the `MasterService.RunStep` message.
  //
  // The `RunStepRequest/RunStepResponse` message can contain 
  // potentially large tensor data as part of its `feed/fetch` 
  // submessages.
  virtual Status RunStep(
    CallOptions* call_options,
    const RunStepRequest* request,
    RunStepResponse* response) {
    std::unique_ptr<RunStepRequestWrapper> wrapped_request(
        new ProtoRunStepRequest(request));
    std::unique_ptr<MutableRunStepResponseWrapper> wrapped_response(
        new NonOwnedProtoRunStepResponse(response));
    return RunStep(call_options, 
        wrapped_request.get(), 
        wrapped_response.get());
  }

  // Returns a request object for use in calls to
  // `RunStep()`. Ownership is transferred to the caller.
  virtual MutableRunStepRequestWrapper* CreateRunStepRequest() {
    return new MutableProtoRunStepRequest;
  }

  // Returns a response object for use in calls to
  // `RunStep()`. Ownership is transferred to the caller.
  virtual MutableRunStepResponseWrapper* CreateRunStepResponse() {
    return new OwnedProtoRunStepResponse;
  }

  virtual Status CloseSession(
    CallOptions* call_options,
    const CloseSessionRequest* request,
    CloseSessionResponse* response) = 0;

  virtual Status ListDevices(
    CallOptions* call_options,
    const ListDevicesRequest* request,
    ListDevicesResponse* response) = 0;

  virtual Status Reset(
    CallOptions* call_options, const ResetRequest* request,
    ResetResponse* response) = 0;
};
\end{c++}
\end{leftbar}

如\refig{dist-master-interface}所示,\code{MasterInterface}存在两种基本实现。

\begin{enum}
  \eitem{分布式:基于\ascii{gRPC}的\code{GrpcRemoteMaster}实现,\ascii{Client}与\ascii{Master}分别部署在两个不同的进程;}
  \eitem{本地模式:基于函数调用的\code{LocalMaster}实现,\ascii{Client}与\ascii{Master}在同一个进程内。}
\end{enum}

\begin{figure}[H]
\centering
\includegraphics[width=0.6\textwidth]{figures/dist-master-interface.png}
\caption{\code{MasterInterface}}
 \label{fig:dist-master-interface}
\end{figure}

在分布式模式中,\code{GrpcRemoteMaster}使用如下类似的伪代码,并通过\ascii{gRPC}获取远端\code{MasterService}服务。

\begin{leftbar}
\begin{c++}
stub = NewStub("/job:worker/replica:0/task:0")
handle = stub->CreateSession({graph_def})
do {
  stub->RunStep(handle, feeds, fetches);
} while (!should_stop());
stub->CloseSession({handle})
\end{c++}
\end{leftbar}

\subsection{RPC过程}

如\refig{dist-client-master-interaction}所示,\ascii{Client}通过\code{MasterInterface}获取远端\ascii{MasterService}的服务。

\begin{figure}[H]
\centering
\includegraphics[width=0.8\textwidth]{figures/dist-client-master-interaction.png}
\caption{Client获取MasterService的原理}
 \label{fig:dist-client-master-interaction}
\end{figure}

其中,\code{GrpcRemoteMaster}是\ascii{gRPC}客户端的一种实现,它最终通过\code{Stub}获取远端\ascii{Master}上的\code{GrpcMasterService}服务,使得其行为表现得犹如本地函数调用一般。其中,\code{GrpcMasterService}实现了\code{MasterService}定义的所有服务接口,它是\code{MasterService}真正的服务实体。

\begin{remark}
从严格意义上讲,\script{GrpcSession, ClientMaster, GrpcRemoteMaster}都是\ascii{Client}实现的一部分。而不是通常理解的那样,\ascii{Python}前端系统是完整的\ascii{Client}实现,后端\ascii{C++}后端系统不包括\ascii{Client}的任何实现。
\end{remark}

\subsection{消息定义}

接下来,将详细看看各个接口的消息定义。其中,最重要的就是识别出各个服务的标识。例如,\ascii{Master}可以供多个\ascii{Client}接入,并为每个\ascii{Client}生成对应的\code{MasterSession}实例。因此\code{GrpcSession}持有\code{MasterSession}句柄,实现\ascii{Client}获取\ascii{Master}的服务。

\subsubsection{CreateSession}

如\refig{dist-ms-create-sess-req}所示,\code{CreateSessionRequest}消息中携带初始的计算图,并与\code{target}指定的\ascii{Master}建立连接。当\ascii{Master}收到请求消息后,建立一个相对应的\code{MasterSession}实例,并使用\code{session\_handle}唯一地标识该\code{MasterSession}实例。

待\ascii{Master}逻辑处理完成后,通过返回消息\code{CreateSessionResponse}给\ascii{Client}。其中,\code{CreateSessionResponse}消息中携带\code{session\_handle},通过它\ascii{Client}端的\code{GrpcSession}与\ascii{Master}端的\code{MasterSession}建立关联关系。随后,\ascii{Client}与\ascii{Master}的所有交互中,在请求消息中通过携带\code{session\_handle},\ascii{Master}通过它索引与之相对应的\ascii{MasterSession}实例。

此外,\code{CreateSessionResponse}也携带了初始的\code{graph\_version},用于后续发起\code{ExtendSession}操作,往原始的计算图中追加新的节点。

\begin{figure}[H]
\centering
\includegraphics[width=0.6\textwidth]{figures/dist-ms-create-sess-req.png}
\caption{\code{CreateSession}}
 \label{fig:dist-ms-create-sess-req}
\end{figure}

\begin{leftbar}
\begin{c++}
message CreateSessionRequest {
  GraphDef graph_def = 1;
  ConfigProto config = 2;
  string target = 3;
}

message CreateSessionResponse {
  string session_handle = 1;
  int64 graph_version = 2;
}
\end{c++}
\end{leftbar}

\subsubsection{ExtendSession}

当\ascii{CreateSession}成功后,后续\ascii{Client}可以通过\code{ExtendSession},携带待扩展的子图给\ascii{Master},增加原有计算图的规模(只能追加子图,不能修改或删除节点)。

如\refig{dist-ms-extend-sess-req}所示,在请求消息中需要携带\code{current\_graph\_version},\ascii{Master}端进行版本匹配验证;待\code{ExtendSession}的逻辑处理完成后,在响应消息中携带\code{new\_graph\_version},用于下一此\code{ExtendSession}操作。其中,初始的\code{graph\_version}由\code{CreateSessionResponse}携带给\ascii{Client}的。

\begin{figure}[H]
\centering
\includegraphics[width=0.7\textwidth]{figures/dist-ms-extend-sess-req.png}
\caption{\code{ExtendSession}}
 \label{fig:dist-ms-extend-sess-req}
\end{figure}

\begin{leftbar}
\begin{c++}
message ExtendSessionRequest {
  string session_handle = 1;

  // REQUIRED: The nodes to be added to the session's graph. 
  // If any node has the same name as an existing node, 
  // the operation will fail with ILLEGAL\_ARGUMENT.
  GraphDef graph_def = 2;

  // REQUIRED: The version number of the graph to be extended. 
  // This will be tested against the current server-side version 
  // number, and the operation will fail with FAILED\_PRECONDITION 
  // if they do not match.
  int64 current_graph_version = 3;
}

message ExtendSessionResponse {
  // The new version number for the extended graph, 
  // to be used in the next call to ExtendSession.
  int64 new_graph_version = 4;
}
\end{c++}
\end{leftbar}

\subsubsection{RunStep}

一般地,在客户端迭代地执行\code{RunStep}。如\refig{dist-ms-run-step-req}所示,在每一次\code{RunStep}执行过程中,\ascii{Client}在请求消息中携带\code{feed, fetch, target},分别表示输入的\ascii{NamedTensor}列表,待输出\ascii{Tensor}的名称列表,待执行\ascii{OP}的名称列表;在响应消息中携带\code{tensor},表示对应于\code{fetch}的名字列表,输出的\ascii{Tensor}列表。

\begin{figure}[H]
\centering
\includegraphics[width=0.7\textwidth]{figures/dist-ms-run-step-req.png}
\caption{\code{RunStep}}
 \label{fig:dist-ms-run-step-req}
\end{figure}

\begin{leftbar}
\begin{c++}
message RunStepRequest {
  string session_handle = 1;

  repeated NamedTensorProto feed = 2;
  repeated string fetch = 3;
  repeated string target = 4;

  RunOptions options = 5;
  string partial_run_handle = 6;
}

message RunStepResponse {
  repeated NamedTensorProto tensor = 1;
  RunMetadata metadata = 2;
}
\end{c++}
\end{leftbar}

\subsubsection{CloseSession}

当计算完成后,需要关闭会话,释放系统计算资源。如\refig{dist-ms-closs-sess}所示,\ascii{Client}通过发送\code{CloseSession}给\ascii{Master},启动计算资源的释放过程。

\begin{figure}[H]
\centering
\includegraphics[width=0.5\textwidth]{figures/dist-ms-closs-sess.png}
\caption{\code{CloseSession}}
 \label{fig:dist-ms-closs-sess}
\end{figure}

\begin{leftbar}
\begin{c++}
message CloseSessionRequest {
  string session_handle = 1;
}

message CloseSessionResponse {
}
\end{c++}
\end{leftbar}

\section{Worker服务}

\begin{content}

\code{WorkerService}也是一个\ascii{gRPC}服务,负责调度本地设备集执行本地子图。它定义了接入\ascii{Worker}的接口规范,即\code{master\_service.proto}中定义的接口。

\ascii{Master}根据\code{ClusterSpec}信息,找到集群中其他的\ascii{Server}实例,此时这些\ascii{Server}实例将扮演\ascii{Worker}的角色。\ascii{Master}将子图分发给各个\ascii{Worker}节点,并启动各个\ascii{Worker}节点的子图计算的执行过程。

如果\ascii{Worker}之间存在数据依赖,则通过进程间通信完成交互。其中,\ascii{Master}与\ascii{Worker}之间,\ascii{Worker}与\ascii{Worker}之间的交互遵循\code{WorkerService}定义的接口规范。

\subsection{接口定义}

在\code{worker\_service.proto}文件中,定义了\code{WorkerService}的所有接口;而在\code{worker.proto}文件中,定义了各个接口的消息体。

\begin{leftbar}
\begin{c++}
service WorkerService {
  rpc GetStatus(GetStatusRequest) 
      returns (GetStatusResponse);

  rpc CreateWorkerSession(CreateWorkerSessionRequest)
      returns (CreateWorkerSessionResponse);

  rpc RegisterGraph(RegisterGraphRequest) 
      returns (RegisterGraphResponse);

  rpc DeregisterGraph(DeregisterGraphRequest) 
      returns (DeregisterGraphResponse);

  rpc RunGraph(RunGraphRequest) 
      returns (RunGraphResponse);

  rpc CleanupGraph(CleanupGraphRequest) 
      returns (CleanupGraphResponse);

  rpc CleanupAll(CleanupAllRequest) 
      returns (CleanupAllResponse);

  rpc RecvTensor(RecvTensorRequest) 
      returns (RecvTensorResponse) {
  }

  rpc Logging(LoggingRequest) 
      returns (LoggingResponse);

  rpc Tracing(TracingRequest) 
      returns (TracingResponse);
}
\end{c++}
\end{leftbar}

\subsection{访问服务}

一般地,\ascii{Master/Worker}使用接口\code{WorkerInterface}获取远端\code{WorkerService}的服务。其中,\code{WorkerInterface}定义了异步访问\code{WorkerService}的接口;与\code{MasterInterface}类似,因为\code{RunGraphRequest/RunGraphResponse}中可能含有较大的\code{Tensor},为了避免不必要的对象拷贝,特化了实现了消息的包装器。

\begin{leftbar}
\begin{c++}
struct WorkerInterface {
  // async interfaces.
  virtual void GetStatusAsync(
      const GetStatusRequest* request,
      GetStatusResponse* response,
      StatusCallback done) = 0;

  virtual void CreateWorkerSessionAsync(
      const CreateWorkerSessionRequest* request,
      CreateWorkerSessionResponse* response, 
      StatusCallback done) = 0;

  virtual void RegisterGraphAsync(
      const RegisterGraphRequest* request,
      RegisterGraphResponse* response,
      StatusCallback done) = 0;

  virtual void DeregisterGraphAsync(
      const DeregisterGraphRequest* request,
      DeregisterGraphResponse* response,
      StatusCallback done) = 0;

  virtual void RunGraphAsync(
      CallOptions* opts, 
      RunGraphRequestWrapper* request,
      MutableRunGraphResponseWrapper* repsonse,
      StatusCallback done) = 0;

  // Wrapper classes for the `WorkerService.RunGraph` message.
  //
  // The `RunGraphRequest/RunGraphResponse` message can contain 
  // potentially large tensor data as part of its `send/response`
  // submessages.
  virtual void RunGraphAsync(
      CallOptions* opts, 
      const RunGraphRequest* request,
      RunGraphResponse* response, 
      StatusCallback done) {
    RunGraphRequestWrapper* wrapped_request = 
        new ProtoRunGraphRequest(request);
    MutableRunGraphResponseWrapper* wrapped_response =
        new NonOwnedProtoRunGraphResponse(response);
    RunGraphAsync(opts, wrapped_request, wrapped_response,
        [wrapped_request, wrapped_response, done](const Status& s) {
            done(s);
            delete wrapped_request;
            delete wrapped_response;
        });
  }

  // Returns a request object for use in calls to
  // `RunGraphAsync()`. Ownership is transferred to the caller.
  virtual MutableRunGraphRequestWrapper* CreateRunGraphRequest() {
    return new MutableProtoRunGraphRequest;
  }

  // Returns a response object for use in calls to
  // `RunGraphAsync()`. Ownership is transferred to the caller.
  virtual MutableRunGraphResponseWrapper* CreateRunGraphResponse() {
    return new OwnedProtoRunGraphResponse;
  }

  virtual void CleanupGraphAsync(
      const CleanupGraphRequest* request,
      CleanupGraphResponse* response,
      StatusCallback done) = 0;

  virtual void CleanupAllAsync(
      const CleanupAllRequest* request,
      CleanupAllResponse* response,
      StatusCallback done) = 0;

  virtual void RecvTensorAsync(
      CallOptions* opts,
      const RecvTensorRequest* request,
      TensorResponse* response,
      StatusCallback done) = 0;

  virtual void LoggingAsync(
      const LoggingRequest* request,
      LoggingResponse* response, 
      StatusCallback done) = 0;

  virtual void TracingAsync(
      const TracingRequest* request,
      TracingResponse* response, 
      StatusCallback done) = 0;
};
\end{c++}
\end{leftbar}


\code{WorkerInterface}同时也定义了同步访问接口。同步接口通过\code{CallAndWait}的适配器,间接实现于异步接口之上。特殊地,同步接口使得\ascii{Master/Worker}调用远端\code{WorkerService}具有犹如调用本地函数一般。

\begin{leftbar}
\begin{c++}
struct WorkerInterface {
  // sync interfaces.
  Status GetStatus(
      const GetStatusRequest* request,
      GetStatusResponse* response) {
    return CallAndWait(&ME::GetStatusAsync, request, response);
  }

  Status CreateWorkerSession(
      const CreateWorkerSessionRequest* request,
      CreateWorkerSessionResponse* response) {
    return CallAndWait(&ME::CreateWorkerSessionAsync, request, response);
  }

  Status RegisterGraph(
      const RegisterGraphRequest* request,
      RegisterGraphResponse* response) {
    return CallAndWait(&ME::RegisterGraphAsync, request, response);
  }

  Status DeregisterGraph(
      const DeregisterGraphRequest* request,
      DeregisterGraphResponse* response) {
    return CallAndWait(&ME::DeregisterGraphAsync, request, response);
  }

  Status CleanupGraph(
      const CleanupGraphRequest* request,
      CleanupGraphResponse* response) {
    return CallAndWait(&ME::CleanupGraphAsync, request, response);
  }

  Status CleanupAll(
      const CleanupAllRequest* request,
      CleanupAllResponse* response) {
    return CallAndWait(&ME::CleanupAllAsync, request, response);
  }

  Status Logging(
      const LoggingRequest* request, 
      LoggingResponse* response) {
    return CallAndWait(&ME::LoggingAsync, request, response);
  }

  Status Tracing(
      const TracingRequest* request, 
      TracingResponse* response) {
    return CallAndWait(&ME::TracingAsync, request, response);
  }
 
 private:
  typedef WorkerInterface ME;

  template <typename Method, typename Req, typename Resp>
  Status CallAndWait(Method func, const Req* req, Resp* resp) {
    Status ret;
    Notification n;
    (this->*func)(req, resp, [&ret, &n](const Status& s) {
      ret = s;
      n.Notify();
    });
    n.WaitForNotification();
    return ret;
  }
};
\end{c++}
\end{leftbar}

特殊地,\code{WorkerInterface}生成的实例由\code{WorkerCacheInterface::ReleaseWorker}负责删除。因此,此处为了避免外部非法删除\code{WorkerInterface}实例,限制\code{WorkerInterface}的析构函数为\code{protected},并且声明\code{WorkerCacheInterface}为友元。

\begin{leftbar}
\begin{c++}
struct WorkerInterface {
 protected:
  virtual ~WorkerInterface() {}
  friend class WorkerCacheInterface;
};
\end{c++}
\end{leftbar}

如\refig{dist-worker-interface}所示,\code{WorkerService}存在两种实现。其中,在本地模式中,直接使用\code{GrpcWorker};在分布式模式中,\ascii{Worker}部署在另一个不同的进程内,使用\code{GrpcRemoteWorker}。

\begin{figure}[H]
\centering
\includegraphics[width=0.6\textwidth]{figures/dist-worker-interface.png}
\caption{\code{WorkerInterface}接口}
 \label{fig:dist-worker-interface}
\end{figure}

\subsection{RPC过程}

如\refig{dist-worker-interaction}所示,在分布式模式中,\code{GrpcRemoteWorker}是\ascii{gRPC}客户端的一种实现,它最终通过\code{Stub}获取远端\ascii{Worker}上的\code{GrpcWorkerService}服务,使得其行为表现得犹如本地函数调用一般。其中,\code{GrpcWorkerService}实现了\code{WorkerService}定义的所有服务接口。

\begin{remark}
从严格意义上讲,\script{GrpcRemoteWorker}是\ascii{Master}或者对端\ascii{Worker}实现的一部分。
\end{remark}

而在本地模式中,通过\code{GrpcWorker}的函数调用,直接获取到了\ascii{WorkerService}的服务,避免了额外的网络传输开销。

\begin{figure}[H]
\centering
\includegraphics[width=0.7\textwidth]{figures/dist-worker-interaction.png}
\caption{获取\code{MasterService}的RPC过程}
 \label{fig:dist-worker-interaction}
\end{figure}

\subsection{消息定义}

接下来,将详细看看\code{WorkerService}各个接口的消息定义。其中,最重要的就是识别出各个服务的标识。当创建\code{WorkerSession}时,\code{MasterSession}的标识传递给\ascii{Worker},实现了\code{MasterSession}统一管理多个隶属的\code{WorkerSession}实例。

当\code{Worker}首次完成\code{RegisterGraph}后,向\code{Master}返回唯一的\code{graph\_handle},以此标识该图实例。因此,在集群内可以使用\code{(session\_handle, graph\_handle)}二元组唯一标识该图实例。

当\ascii{Master}广播通知各个\ascii{Worker}并发地\code{RunGraph}。为了区分不同\code{step},\ascii{Master}生成全局唯一的\code{step\_id},并通过\code{RunGraph}传递给各个\ascii{Worker}。

\begin{enum}
  \eitem{\code{session\_handle}: 创建\code{MasterSession}实例时自动生成,通过\code{CreateSessionResponse}携带给\ascii{Client};通过\code{CreateWorkerSessionRequest}携带给\ascii{Worker};}  
  \eitem{\code{graph\_id}: 首次\code{RegisterGraph}时由\code{Worker}生成,通过\code{RegisterGraphResponse}携带给\code{Master};}
  \eitem{\code{step\_id}: 每次\code{RunStep}时,由\code{Master}生成唯一的标识,通过\code{RunGraphRequest}携带给\ascii{Worker}。} 
\end{enum}

\subsubsection{CreateWorkerSession}

如\refig{dist-worker-create-worker-sess}所示,\code{CreateWorkerSessionRequest}消息中携带\code{MasterSession}分配的\code{session\_handle}。当\ascii{Worker}收到请求消息后,生成一个\code{WorkerSession}实例,并使用\code{session\_handle}在该\ascii{Worker}内唯一地标识该实例。

在同一个集群中,对于一个\code{MasterSession}实例,其他\ascii{Worker}收到相同的\code{session\_handle}。如此,该\code{MasterSession}实例便能统一管理隶属于它的所有\code{WorkerSession}实例。

\begin{figure}[H]
\centering
\includegraphics[width=0.6\textwidth]{figures/dist-worker-create-worker-sess.png}
\caption{\code{CreateWorkerSession}}
 \label{fig:dist-worker-create-worker-sess}
\end{figure}

\begin{leftbar}
\begin{c++}
message CreateWorkerSessionRequest {
  string session_handle = 1;
  ServerDef server_def = 2;
}

message CreateWorkerSessionResponse {
}
\end{c++}
\end{leftbar}

\subsubsection{RegisterGraph}

如\refig{dist-worker-register-graph}所示,\code{RegisterGraphRequest}消息中携带\code{MasterSession}分配的\code{session\_handle},及其子图实例\ascii{graph\_def}。当\code{Worker}完成子图注册及其初始化后,向\ascii{Master}返回该子图的\code{graph\_handle}。

需要注意的是,\code{Master}只会在执行一次\code{RegisterGraph},除非计算图的节点被重新编排,或者\code{Master}进程被重启。

\begin{figure}[H]
\centering
\includegraphics[width=0.6\textwidth]{figures/dist-worker-register-graph.png}
\caption{\code{RegisterGraph}}
 \label{fig:dist-worker-register-graph}
\end{figure}

\begin{leftbar}
\begin{c++}
message RegisterGraphRequest {
  string session_handle = 1;

  GraphDef graph_def = 2;
  bool has_control_flow = 3 [deprecated = true];

  GraphOptions graph_options = 4;
  DebugOptions debug_options = 5;
}

message RegisterGraphResponse {
  string graph_handle = 1;
}
\end{c++}
\end{leftbar}


\subsubsection{DeregisterGraph}

如\refig{dist-worker-deregister-graph}所示,当\code{Worker}节点上的子图不再需要时(例如,计算图被重新调度,图中节点被重新编排),此时\code{Master}向\ascii{Worker}发送\code{DeregisterGraph}消息,以便\code{Worker}注销掉该子图实例。

\begin{figure}[H]
\centering
\includegraphics[width=0.6\textwidth]{figures/dist-worker-deregister-graph.png}
\caption{\code{DeregisterGraph}}
 \label{fig:dist-worker-deregister-graph}
\end{figure}

\begin{leftbar}
\begin{c++}
message DeregisterGraphRequest {
  string session_handle = 2;
  string graph_handle = 1;
}

message DeregisterGraphResponse {
}
\end{c++}
\end{leftbar}

\subsubsection{RunGraph}

执行\ascii{Worker}节点上注册的子图时,为了区分不同\code{step},\ascii{Master}生成唯一\code{step\_id}并传递给各个\ascii{Worker},各个\ascii{Worker}通过\code{step\_id}实现数据的协同。

此外,\code{RunGraphRequest}携带了\code{send, recv\_key},分别表示子图输入的\code{Tensor}标识和数据,及其子图输出的\code{Tensor}的标识。\code{RunGraphResponse}返回\code{recv\_key}相对应的\code{Tensor}列表。

\begin{figure}[H]
\centering
\includegraphics[width=0.6\textwidth]{figures/dist-worker-run-graph.png}
\caption{\code{RunGraph}}
 \label{fig:dist-worker-run-graph}
\end{figure}

\begin{leftbar}
\begin{c++}
message RunGraphRequest {
  string session_handle = 8;
  string graph_handle = 1;
  int64 step_id = 2;

  ExecutorOpts exec_opts = 5;

  repeated NamedTensorProto send = 3;
  repeated string recv_key = 4;

  bool is_partial = 6;
  bool is_last_partial_run = 7;
}

message RunGraphResponse {
  repeated NamedTensorProto recv = 1;

  // execution stats
  StepStats step_stats = 2;
  CostGraphDef cost_graph = 3;
  repeated GraphDef partition_graph = 4;
}
\end{c++}
\end{leftbar}

\subsubsection{RecvTensor}

当执行某一次\code{step}中,如果两个\ascii{Worker}需要交互数据,消费者向生产者发送\code{RecvTensorRequest}消息,通过携带\code{(step\_id, rendezvous\_key)}二元组,请求对端\ascii{Worker}相应的\code{Tensor}数据,并通过\code{RecvTensorResponse}返回。

\begin{figure}[H]
\centering
\includegraphics[width=0.6\textwidth]{figures/dist-worker-recv-tensor.png}
\caption{\code{RecvTensor}}
 \label{fig:dist-worker-recv-tensor}
\end{figure}

\begin{leftbar}
\begin{c++}
message RecvTensorRequest {
  int64 step_id = 1;
  string rendezvous_key = 2;

  // If true, use an out-of-band DMA mechanism to transfer the
  // received tensor.
  bool dma_ok = 3;

  // Optional information on client-side device locality.
  DeviceLocality client_locality = 4;

  // Optional information on server-side device locality.
  DeviceLocality server_locality = 5;

  // Optional information needed by the RPC subsystem.
  google.protobuf.Any transport_options = 6;
}

message RecvTensorResponse {
  // The tensor as a proto.
  TensorProto tensor = 1;

  // If true, this tensor was the output of a dead node, and the
  // content is invalid.
  bool is_dead = 2;

  // The time at which tensor was available and started to be returned.
  int64 send_start_micros = 3;

  // Optional additional information about how to receive the tensor,
  // e.g. in the event that `RecvTensorRequest.dma\_ok` was true.
  google.protobuf.Any transport_options = 4;
}
\end{c++}
\end{leftbar}

\end{content}

\section{会话控制}

\emph{会话控制}是\tf{}分布式运行时的核心,也是整个\tf{}执行引擎的的关键路径。为了理顺会话控制的脉络,接下来的文章将重点讲述整个会话控制的详细过程。

\begin{content}

\subsection{会话协同}

如\refig{dist-session-overview}所示,在分布式模式中,会话控制通过\code{GrpcSession, MasterSession, WorkerSession}之间的协同实现的,它们分别驻留在\code{Client, Master, Worker}上,使用同一个\code{session\_handle}实现协同工作的。

其中,\code{tf.Session}使用\ascii{Python}实现,是\tf{}对外提供的\ascii{API}。它与\code{GrpcSession}在同一个进程内,并且直接持有\code{GrpcSession}的句柄(或指针)实现的。

\begin{figure}[H]
\centering
\includegraphics[width=0.6\textwidth]{figures/dist-session-overview-1.png}
\caption{会话协同}
 \label{fig:dist-session-overview}
\end{figure}

如\refig{dist-multi-client-conn}所示,在分布式模式中,可能存在多个\ascii{Client}同时接入一个\ascii{Master},\ascii{Master}为其每个接入的\ascii{Client}创建一个\code{MasterSession}实例。\ascii{Worker}也可能同时为多个\ascii{Master}提供计算服务,\ascii{Worker}为其每个请求计算的\ascii{Master}创建一个\code{WorkerSession}实例。为了区分不同的\ascii{Client}的计算服务,使用不同的\code{session\_handle}区分。

\begin{figure}[H]
\centering
\includegraphics[width=0.9\textwidth]{figures/dist-multi-client-conn.png}
\caption{会话控制:领域模型}
 \label{fig:dist-multi-client-conn}
\end{figure}

\subsection{生命周期}

\code{GrpcSession}控制\ascii{Client}的会话生命周期,\code{MasterSession}控制\ascii{Master}的会话生命周期,\code{WorkerSession}控制\ascii{Worker}的会话生命周期,它们之间通过\code{session\_handle}实现协同。

\subsubsection{GrpcSession生命周期}

在分布式模式下,\code{Client}的运行时由\code{GrpcSession}控制,\code{GrpcSession}的生命周期过程如\refig{dist-grpc-session-life-cycle}所示。

\begin{figure}[H]
\centering
\includegraphics[width=0.8\textwidth]{figures/dist-grpc-session-life-cycle.png}
\caption{\code{GrpcSession}生命周期}
 \label{fig:dist-grpc-session-life-cycle}
\end{figure}

\subsubsection{MasterSession生命周期}

在分布式模式下,\code{Master}的运行时由\code{MasterSession}控制,\code{MasterSession}生命周期过程如\refig{dist-master-session-life-cycle}所示。

\begin{figure}[H]
\centering
\includegraphics[width=0.8\textwidth]{figures/dist-master-session-life-cycle.png}
\caption{\code{MasterSession}生命周期}
 \label{fig:dist-master-session-life-cycle}
\end{figure}

\subsubsection{WorkerSession生命周期}

在分布式模式下,\code{Worker}的运行时由\code{WorkerSession}控制,\code{WorkerSession}生命周期过程如\refig{dist-worker-session-life-cycle}所示。

\begin{figure}[H]
\centering
\includegraphics[width=0.8\textwidth]{figures/dist-worker-session-life-cycle.png}
\caption{\code{WorkerSession}生命周期}
 \label{fig:dist-worker-session-life-cycle}
\end{figure}

\subsection{会话过程}

在用户编程环境中,\ascii{Client}从\code{tf.Session(target)}为起点,通过\code{Session.run}启动迭代执行,最终计算完成后调用\code{Session.close}关闭会话。但是,在分布式执行引擎的实现中,其过程要复杂得多。

\begin{nitemize}
  \eitem{创建会话}    
    \begin{enum}
      \eitem{创建\code{GrpcSession};}  
      \eitem{获取远端设备集;} 
      \eitem{创建\code{MasterSession};}
      \eitem{创建\code{WorkerSession};}      
    \end{enum}
  \eitem{迭代执行}          
    \begin{enum}
      \eitem{启动执行;}  
      \eitem{图剪枝;}  
      \eitem{图分裂;}        
      \eitem{注册子图;}              
      \eitem{运行子图;}                         
    \end{enum}      
  \eitem{关闭会话}          
    \begin{enum}          
      \eitem{关闭\code{GrpcSession};}  
      \eitem{关闭\code{MasterSession};}
      \eitem{关闭\code{WorkerSession};}      
    \end{enum}  
\end{nitemize}

\end{content}

\section{创建会话}

\subsection{创建GrpcSession}

\begin{content}

如\refig{dist-grpc-session-factory}所示,\code{GrpcSession}由\code{GrpcSessionFactory}多态创建。

\begin{figure}[H]
\centering
\includegraphics[width=0.6\textwidth]{figures/dist-grpc-session-factory.png}
\caption{多态创建GrpcSession}
 \label{fig:dist-grpc-session-factory}
\end{figure}

\begin{leftbar}
\begin{c++}
const char* kSchemePrefix = "grpc://";

struct GrpcSessionFactory : SessionFactory {
  bool AcceptsOptions(const SessionOptions& options) override {
    return StringPiece(options.target).starts_with(kSchemePrefix);
  }

  Session* NewSession(const SessionOptions& options) override {
    std::unique_ptr<GrpcSession> ret;
    Status s = GrpcSession::Create(options, &ret);
    if (s.ok()) {
      return ret.release();
    } else {
      return nullptr;
    }
  }
};
\end{c++}
\end{leftbar}

当\code{target}以\code{grpc://}开头,则\code{SessionFactory::GetFactory}返回\code{GrpcSessionFactory}实例,而\code{GrpcSessionFactory}工厂方法委托\code{GrpcSession::Create}的静态工厂方法负责创建\code{GrpcSession}实例。

而\code{GrpcSessionFactory::NewSession}由\ascii{C API}调用。其中,\ascii{C API}是\tf{}后端系统对外提供的标准接口,对外供多种编程语言调用。

\begin{leftbar}
\begin{c++}
Status NewSession(const SessionOptions& options, Session** out_session) {
  SessionFactory* factory;
  Status s = SessionFactory::GetFactory(options, &factory);
  if (!s.ok()) {
    *out_session = nullptr;
    return s;
  }
  *out_session = factory->NewSession(options);
  if (!*out_session) {
    return errors::Internal("Failed to create session.");
  }
  return Status::OK();
}

TF_DeprecatedSession* TF_NewDeprecatedSession(
  const TF_SessionOptions* opt, TF_Status* status) {
  Session* session;
  status->status = NewSession(opt->options, &session);
  if (status->status.ok()) {
    return new TF_DeprecatedSession({session});
  } else {
    return nullptr;
  }
}
\end{c++}
\end{leftbar}

在\code{GrpcSession::Create}静态工厂方法中,它主要负责创建\code{GrpcSession}实例,并完成相应的初始化工作。在初始化过程中,最重要的就是构建\code{MasterInterface}实例。其中,\code{MasterInterface}存在两种子类实现,分别对应两种不同应用场景:

\begin{enum}
  \eitem{\code{LocalMaster}:\ascii{Client}与\ascii{Master}在同一进程内,调用\code{LocalMaster::Lookup}直接获取\code{LocalMaster}实例;}
  \eitem{\code{GrpcRemoteMaster}:\ascii{Client}与\ascii{Master}不在同一进程内,调用工厂方法\code{NewGrpcMaster}生成\code{GrpcRemoteMaster}实例。}
\end{enum}

\begin{leftbar}
\begin{c++}
Status GrpcSession::Create(
    const SessionOptions& options,
    std::unique_ptr<GrpcSession>* out_session) {
  std::unique_ptr<GrpcSession> session(new GrpcSession(options));
  std::unique_ptr<MasterInterface> master;
  // intra-process between client and master.
  if (!options.config.rpc_options().use_rpc_for_inprocess_master()) {
    master = LocalMaster::Lookup(options.target);
  }
  // inter-process between client and master.
  if (!master) {
    SharedGrpcChannelPtr master_channel;
    TF_RETURN_IF_ERROR(NewHostPortGrpcChannel(
        options.target.substr(strlen(kSchemePrefix)), &master_channel));
    master.reset(NewGrpcMaster(master_channel));
  }
  session->SetRemoteMaster(std::move(master));
  *out_session = std::move(session);
  return Status::OK();
}
\end{c++}
\end{leftbar}

\end{content}

\subsection{创建MasterSession}

\begin{content}

如\refig{dist-create-master-session}所示,当调用\code{GprcSession::Create}时,将初始的计算图通过\code{CreateSessionRequst}消息发送给\ascii{Master}。

当\ascii{Master}收到\code{CreateSessionRequst}消息后,生成相对应的\code{MasterSession}实例,并使用全局唯一的\code{handle}标识,最终通过\code{CreateSessionResponse}消息带回给\code{GrpcSession}。

\begin{figure}[H]
\centering
\includegraphics[width=1.0\textwidth]{figures/dist-create-master-session.png}
\caption{创建\code{MasterSession}}
 \label{fig:dist-create-master-session}
\end{figure}

\subsubsection{GrpcSesion::Create(graph\_def)}

\code{GrpcSession::Create(graph\_def)}方法主要用于\code{Client}请求\code{Master}创建\code{MasterSession}实例。首先,\code{GrpcSession::Create}方法完成构造\code{CreateSessionRequst}消息,然后通过\code{GrpcRemoteMaster}将其发送给\ascii{Master}。

接着,等\code{GrpcSession}收到\code{CreateSessionResponse}消息后,保存\code{MasterSession}的\code{handle},及其新的计算图版本号\code{graph\_version}。其中,\code{handle}用于标识\ascii{Master}侧的\code{MasterSession}实例,\code{graph\_version}用于后续扩展计算图使用。

\begin{leftbar}
\begin{c++}
void GrpcSession::BuildCreateSessionReq(
    const GraphDef& graph,
    CreateSessionRequest& req) {
  *req.mutable_config() = options_.config;
  *req.mutable_graph_def() = graph;
  req.set_target(options_.target);
}

void GrpcSession::SaveCreateSessionRsp(
    CreateSessionResponse& rsp) {
  mutex_lock l(mu_);
  swap(handle_, *(resp.mutable_session_handle()));
  current_graph_version_ = resp.graph_version();
}

Status GrpcSession::CreateImpl(CallOptions* call_options,
                               const GraphDef& graph) {
  CreateSessionRequest req;
  CreateSessionResponse resp;

  BuildCreateSessionReq(graph, req);
  Status s = master_->CreateSession(call_options, &req, &resp);
  if (s.ok()) {
    SaveCreateSessionRsp(resp);
  }
  return s;
}

Status GrpcSession::Create(const RunOptions& run_options,
                           const GraphDef& graph) {
  CallOptions call_options;
  call_options.SetTimeout(run_options.timeout_in_ms());
  return CreateImpl(&call_options, graph);
}

Status GrpcSession::Create(const GraphDef& graph) {
  CallOptions call_options;
  call_options.SetTimeout(options_.config.operation_timeout_in_ms());
  return CreateImpl(&call_options, graph);
}
\end{c++}
\end{leftbar}

\subsubsection{GrpcRemoteMaster::CreateSession}

\code{GrpcRemoteMaster}是一个\ascii{gRPC}的客户端实现。它的实现非常简单,通过\ascii{gRPC}的一个\code{stub}调用远端\ascii{Master}相应的服务。

\begin{leftbar}
\begin{c++}
namespace {
  void SetClientContext(
      const CallOptions& call_options,
      ::grpc::ClientContext& ctx) {
    ctx.set_fail_fast(false);
    SetDeadline(&ctx, call_options.GetTimeout());
  }
}

Status GrpcRemoteMaster::CreateSession(
    CallOptions* call_options,
    const CreateSessionRequest* request,
    CreateSessionResponse* response) override {
  ::grpc::ClientContext ctx;
  SetClientContext(*call_options, ctx);
  return FromGrpcStatus(stub_->CreateSession(&ctx, *request, response));
}
\end{c++}
\end{leftbar}

\subsubsection{GrpcMasterService::CreateSessionHandler}

\code{GrpcMasterService}是一个\ascii{gRPC}服务,它实现了\code{MasterService}的\ascii{RPC}服务接口。

当收到\code{CreateSession}消息后,将由\code{GrpcMasterService::CreateSessionHandler}处理该消息。它将委托\code{Master}处理该消息。

\code{Master}处理完成后,将回调完成时的\ascii{lambda}表达式,向\ascii{Client}返回\code{CreateSessionResponse}的响应消息。

\begin{leftbar}
\begin{c++}
// RPC handler for creating a session.
void GrpcMasterService::CreateSessionHandler(
  MasterCall<CreateSessionRequest, CreateSessionResponse>* call) {
  master_impl_->CreateSession(
    &call->request, &call->response,
    [call](const Status& status) {
        call->SendResponse(ToGrpcStatus(status));
    });
  ENQUEUE_REQUEST(CreateSession, true);
}
\end{c++}
\end{leftbar}

\subsubsection{Master::CreateSession}

然后,\code{Master::CreateSession}将会在线程池里启动一个线程,并在线程内按照\code{cluster\_spec}信息,寻找所有的\ascii{Worker},并收集本地设备,及其远端设备的信息。最后,该线程创建了一个\code{MasterSession};并且当创建成功后,\ascii{Master}会保存\code{(handle, master\_session)}的二元组信息,以便后续\ascii{Master}能够通过\code{handle}索引相应的\code{MasterSession}实例。

\begin{leftbar}
\begin{c++}
using RemoveDevices = unique_ptr<vector<unique_ptr<Device>>>;

void Master::CreateSession(const CreateSessionRequest* req,
                           CreateSessionResponse* resp, MyClosure done) {
  SchedClosure([this, req, resp, done]() {
    // 1. Find all workers.  
    // ignore implements...
    std::unique_ptr<WorkerCacheInterface> worker_cache_ptr;

    // 2. Find all remote devices. 
    // ignore implements...
    RemoveDevices remote_devices(new vector<unique_ptr<Device>>());

    // 3. Construct DeviceSet.
    // ignore implements...
    std::unique_ptr<DeviceSet> device_set;

    // 4. Create MasterSession
    SessionOptions options;
    options.config = req->config();
    
    MasterSession* session = env_->master_session_factory(
        options, env_, std::move(remote_devices), 
        std::move(worker_cache_ptr), std::move(device_set));

    GraphDef* gdef =
        const_cast<CreateSessionRequest*>(req)->mutable_graph_def();
    
    // 5. Create WorkerSession foreach worker.
    status = session->Create(gdef, worker_cache_factory_options);
    resp->set_session_handle(session->handle());
    
    // 6. Store <handle, master\_session> pair.
    {
      mutex_lock l(mu_);
      CHECK(sessions_.insert({session->handle(), session}).second);
    }
  });
}
\end{c++}
\end{leftbar}

其中,在创建完\code{MasterSession},并保存\code{(handle, master\_session)}的二元组信息之间,将启动\code{MasterSession::Create(graph\_def)}过程。

\subsubsection{MasterSession::Create(graph\_def)}

\code{MasterSession::Create(graph\_def)}主要完成两件事情。

\begin{enum}
  \eitem{初始化计算图,并生成\code{SimpleGraphExecutionState}实例;}
  \eitem{广播所有\ascii{Worker}创建相应的\code{WorkerSession}实例。}
\end{enum}

\begin{leftbar}
\begin{c++}
Status MasterSession::Create(
    GraphDef* graph_def,
    const WorkerCacheFactoryOptions& options) {
  SimpleGraphExecutionStateOptions execution_options;
  execution_options.device_set = devices_.get();
  execution_options.session_options = &session_opts_;
  {
    mutex_lock l(mu_);
    TF_RETURN_IF_ERROR(SimpleGraphExecutionState::MakeForBaseGraph(
        graph_def, execution_options, &execution_state_));
  }
  if (options.cluster_def != nullptr) {
    return CreateWorkerSessions(options);
  }
  return Status::OK();
}
\end{c++}
\end{leftbar}

\subsection{创建\code{WorkerSession}}

\ascii{Master}会广播所有\ascii{Worker}创建相应的\code{WorkerSession}实例。这些\code{WorkerSession}隶属于此\code{MasterSession}实例,因为它们使用与\code{MasterSession}实例相同的\code{handle}标识。

\code{MasterSession}为了收齐所有\ascii{Worker}返回的\code{CreateWorkerSessionResponse}消息,引入了\code{BlockingCounter}计数器。

\begin{leftbar}
\begin{c++}
struct MasterSession::Worker {
  Worker(MasterSession* sess, const string& name,
         const DeviceNameUtils::ParsedName& parsed_name,
         const WorkerCacheFactoryOptions& opts)
      : sess(sess),
        name(&name),
        worker(GetOrCreateWorker()) {
    BuildRequest(parsed_name, opts);
  }

  void CreateWorkerSession(BlockingCounter& done, Status& status) {
    auto cb = [&status, &done](const Status& s) {
      status.Update(s);
      done.DecrementCount();
    };
    worker->CreateWorkerSessionAsync(&request, &response, cb);
  }

  void Release() {
    if (worker != nullptr) {
      sess->worker_cache_->ReleaseWorker(*name, worker);
    }
  }

 private:
  WorkerInterface* GetOrCreateWorker() {
    return sess->worker_cache_->CreateWorker(*name);
  }

  void BuildRequest(const DeviceNameUtils::ParsedName& parsed_name,
                    const WorkerCacheFactoryOptions& opts) {
    request.set_session_handle(sess->handle_);
    BuildServerDef(parsed_name, opts, request.mutable_server_def());
  }

  void BuildServerDef(const DeviceNameUtils::ParsedName& parsed_name,
                      const WorkerCacheFactoryOptions& opts,
                      ServerDef* server_def) {
    *server_def->mutable_cluster() = *opts.cluster_def;
    server_def->set_protocol(*opts.protocol);
    server_def->set_job_name(parsed_name.job);
    server_def->set_task_index(parsed_name.task);
  }

 private:
  MasterSession* sess;

  // The worker name. (Not owned.)
  const string* name;

  // The worker referenced by name. (Not owned.)
  WorkerInterface* worker = nullptr;

  // Request and responses used for a given worker.
  CreateWorkerSessionRequest request;
  CreateWorkerSessionResponse response;
};

struct MasterSession::WorkerGroup {
  WorkerGroup(MasterSession* sess)
      : sess(sess) {
  }

  Status CreateWorkerSessions(const WorkerCacheFactoryOptions& opts) {
    TF_RETURN_IF_ERROR(CreateWorkers(opts));
    TF_RETURN_IF_ERROR(BroadcastWorkers());
    return Status::OK();
  }

  void ReleaseWorkers() {
    for (auto& worker : workers) {
      worker.Release();
    }
  }

 private:
  Status CreateWorkers(const WorkerCacheFactoryOptions& opts) {
    sess->worker_cache_->ListWorkers(&worker_names);
    for (size_t i = 0; i < worker_names.size(); ++i) {
      TF_RETURN_IF_ERROR(AppendWorker(worker_names[i], opts));
    }
    return Status::OK();
  }

  Status BroadcastWorkers() {
    Status status = Status::OK();
    BlockingCounter done(workers.size());
    for (size_t i = 0; i < workers.size(); ++i) {
      workers[i].CreateWorkerSession(done, status);
    }
    done.Wait();
    return status;
  }

  Status AppendWorker(const string& worker_name,
                    const WorkerCacheFactoryOptions& opts) {
    DeviceNameUtils::ParsedName parsed_name;
    TF_RETURN_IF_ERROR(ParseWorkerName(worker_name, &parsed_name));
    workers.emplace_back(Worker(sess, worker_name, parsed_name, opts));
    return Status::OK();
  }

  Status ParseWorkerName(const string& worker_name,
                         DeviceNameUtils::ParsedName* parsed_name) {
    if (!DeviceNameUtils::ParseFullName(worker_name, parsed_name)) {
      return errors::Internal("Could not parse name ", worker_name);
    }
    if (!parsed_name->has_job || !parsed_name->has_task) {
      return errors::Internal("Incomplete worker name ", worker_name);
    }
    return Status::OK();
  }

 private:
  MasterSession* sess;
  std::vector<string> worker_names;
  std::vector<Worker> workers;
};

Status MasterSession::CreateWorkerSessions(
    const WorkerCacheFactoryOptions& options) {
  CHECK(worker_cache_) << "CreateWorkerSessions should be called only with "
                       << "dynamic cluster membership.";

  WorkerGroup worker_group(this);

  // Release the workers.
  auto cleanup = gtl::MakeCleanup([&worker_group] {
    worker_group.ReleaseWorkers();
  });

  return worker_group.CreateWorkerSessions(options);
}
\end{c++}
\end{leftbar}

\subsubsection{GrpcRemoteWorker}

\begin{leftbar}
\begin{c++}
class GrpcRemoteWorker : public WorkerInterface {
  void CreateWorkerSessionAsync(const CreateWorkerSessionRequest* request,
                                CreateWorkerSessionResponse* response,
                                StatusCallback done) override {
    IssueRequest(request, response, createworkersession_, std::move(done));
  }

  void IssueRequest(const protobuf::Message* request,
                    protobuf::Message* response, const ::grpc::string& method,
                    StatusCallback done, CallOptions* call_opts = nullptr) {
    new RPCState<protobuf::Message>(counter_, &stub_, cq_, method, *request,
                                    response, std::move(done), call_opts);
  }
};
\end{c++}
\end{leftbar}

\subsection{Worker端}

\subsubsection{GrpcWorkerService}

\begin{leftbar}
\begin{c++}
class GrpcWorkerService : public AsyncServiceInterface {
  void CreateWorkerSessionHandler(
      WorkerCall<CreateWorkerSessionRequest, CreateWorkerSessionResponse>*
          call) {
    Schedule([this, call]() {
      Status s = worker_->CreateWorkerSession(&call->request, &call->response);
      call->SendResponse(ToGrpcStatus(s));
    });
    ENQUEUE_REQUEST(CreateWorkerSession, false);
  }
};
\end{c++}
\end{leftbar}

\subsubsection{创建WorkerSession}

\begin{leftbar}
\begin{c++}
void Worker::CreateWorkerSessionAsync(const CreateWorkerSessionRequest* request,
                                      CreateWorkerSessionResponse* response,
                                      StatusCallback done) {
  Status s = env_->session_mgr->CreateSession(request->session_handle(),
                                              request->server_def());
  done(s);
}
\end{c++}
\end{leftbar}

\begin{leftbar}
\begin{c++}
Status SessionMgr::CreateSession(const string& session,
                                 const ServerDef& server_def) {
  mutex_lock l(mu_);
  if (session.empty()) {
    return errors::InvalidArgument("Session must be non-empty.");
  }

  const string worker_name = WorkerNameFromServerDef(server_def);

  WorkerCacheInterface* worker_cache = nullptr;
  TF_RETURN_IF_ERROR(worker_cache_factory_(server_def, &worker_cache));

  std::vector<Device*> renamed_devices;
  for (Device* d : worker_env_->local_devices) {
    renamed_devices.push_back(
        RenamedDevice::NewRenamedDevice(worker_name, d, false));
  }
  std::unique_ptr<DeviceMgr> device_mgr(new DeviceMgr(renamed_devices));

  std::unique_ptr<GraphMgr> graph_mgr(
      new GraphMgr(worker_env_, device_mgr.get()));

  std::unique_ptr<WorkerSession> worker_session(new WorkerSession(
      worker_name, std::unique_ptr<WorkerCacheInterface>(worker_cache),
      std::move(device_mgr), std::move(graph_mgr)));

  sessions_.insert(std::make_pair(session, std::move(worker_session)));
  return Status::OK();
}
\end{c++}
\end{leftbar}

\section{迭代执行}

\subsection{Client端}

\subsubsection{GrpcSession}

\begin{leftbar}
\begin{c++}
namespace {
  using TensorIndex = std::unordered_map<string, int>;

  void BuildReqOptions(const SessionOptions& sess_options,
      const RunOptions& run_options, 
      RunOptions& options) {
    options = run_options;
    if (run_options.timeout_in_ms() == 0) {
      options.set_timeout_in_ms(
          sess_options.config.operation_timeout_in_ms());
    }    
  }

  void BuildReqFeeds(const vector<pair<string, Tensor>>& inputs,
      MutableRunStepRequestWrapper* req) {
    for (auto& it : inputs) {
      req->add_feed(it.first, it.second);
    }
  }

  void BuildReqFetches(const std::vector<string>& output_names,
      MutableRunStepRequestWrapper* req) {
    for (int i = 0; i < output_names.size(); ++i) {
      req->add_fetch(output_names[i]);
  }

  void BuildReqTargets(const std::vector<string>& target_names,
      MutableRunStepRequestWrapper* req) {
    for (string& target : target_names) {
      req->add_target(target);
    }
  }

  void BuildRunStepReq(
      const SessionOptions& sess_options,
      const RunOptions& run_options,
      const vector<pair<string, Tensor>>& inputs,
      const std::vector<string>& output_names,
      const std::vector<string>& target_names,
      MutableRunStepRequestWrapper* req) {
    BuildReqOptions(sess_options, run_options, 
        req->mutable_options());
    BuildReqFeeds(inputs, req);
    BuildReqFetches(output_names, req);
    BuildReqTargets(target_names, req); 
  }

  void BuildOuputNamesIndex(
      const std::vector<string>& output_names,
      TensorIndex& tensor_index) {
    for (int i = 0; i < output_names.size(); ++i) {
      const string& name = output_names[i];
      tensor_index.insert(make_pair(name, i));
    }
  }

  void BuildCallOptions(const RunOptions& options, 
      CallOptions& call_options) {
    call_options.SetTimeout(options.timeout_in_ms());
  }

  Status DoSaveOutputs(const TensorIndex& tensor_index,
      const std::vector<string>& output_names,
      MutableRunStepResponseWrapper* resp,
      std::vector<Tensor>* outputs) {
    for (size_t i = 0; i < resp->num_tensors(); ++i) {
      auto fetch_it = tensor_index.find(resp->tensor_name(i));
      if (fetch_it == tensor_index.end()) {
        return errors::Internal(
           "unrequested fetch: ", resp->tensor_name(i));
      }

      Tensor output;
      TF_RETURN_IF_ERROR(resp->TensorValue(i, &output));
      (*outputs)[fetch_it->second] = output;
    }  
  }

  Status SaveOutputs(const TensorIndex& tensor_index,
      const std::vector<string>& output_names,
      MutableRunStepResponseWrapper* resp,
      std::vector<Tensor>* outputs) {
    if (!output_names.empty()) {
      outputs->resize(output_names.size());
    }
    return DoSaveOutputs(tensor_index, 
        output_names, rsep, outputs);
  }

  void SaveRunMetaData(MutableRunStepResponseWrapper* resp,
      RunMetadata* run_metadata) {
    if (run_metadata) {
      run_metadata->Swap(resp->mutable_metadata());
    }
  }
  
  Status SaveRspToOutputs(const TensorIndex& tensor_index,
      const std::vector<string>& output_names,
      MutableRunStepResponseWrapper* resp,
      std::vector<Tensor>* outputs,
      RunMetadata* run_metadata) {
    SaveRunMetaData(resp, run_metadata);
    return SaveOutputs(tensor_index, output_names, rsep, outputs);
  }
}

Status GrpcSession::Run(
    const RunOptions& run_options,
    const vector<pair<string, Tensor>>& inputs,
    const vector<string>& output_names,
    const vector<string>& target_names,
    std::vector<Tensor>* outputs,
    RunMetadata* run_metadata) {
  unique_ptr<MutableRunStepRequestWrapper> req(
      master_->CreateRunStepRequest());

  unique_ptr<MutableRunStepResponseWrapper> resp(
      master_->CreateRunStepResponse());

  BuildRunStepReq(options_, run_options, inputs, 
      output_names, target_names, req.get());

  TensorIndex tensor_index;
  BuildOuputNamesIndex(output_names, tensor_index);

  CallOptions call_options;
  BuildCallOptions(req->options(), call_options)

  TF_RETURN_IF_ERROR(RunProto(&call_options, 
      req.get(), resp.get()));

  return SaveRspToOutputs(tensor_index, output_names, 
      resp.get(), outputs, run_metadata);
}
\end{c++}
\end{leftbar}

\begin{leftbar}
\begin{c++}
Status GrpcSession::RunProto(
    CallOptions* call_options,
    MutableRunStepRequestWrapper* req,
    MutableRunStepResponseWrapper* resp) {
  {
    mutex_lock l(mu_);
    req->set_session_handle(handle_);
  }
  return master_->RunStep(call_options, req, resp);
}
\end{c++}
\end{leftbar}

\subsubsection{GrpcRemoteMaster}

\begin{leftbar}
\begin{c++}
struct GrpcRemoteMaster : MasterInterface {
  Status RunStep(CallOptions* call_options, RunStepRequestWrapper* request,
                 MutableRunStepResponseWrapper* response) override {
    ::grpc::ClientContext ctx;
    ctx.set_fail_fast(false);
    SetDeadline(&ctx, call_options->GetTimeout());
    return FromGrpcStatus(stub_->RunStep(&ctx, request->ToProto(),
                                         get_proto_from_wrapper(response)));
  }
};
\end{c++}
\end{leftbar}

\subsection{Master端}

\subsubsection{GrpcRemoteMaster}

\begin{leftbar}
\begin{c++}
struct GrpcMasterService : AsyncServiceInterface {
  using RunStepCall = MasterCall<RunStepRequest, RunStepResponse>;
 
  void RunStepHandler(RunStepCall* call) {
    CallOptions* call_opts = CreateCallOptions(call);

    RunStepRequestWrapper* wrapped_request =
        new ProtoRunStepRequest(&call->request);

    MutableRunStepResponseWrapper* wrapped_response =
        new NonOwnedProtoRunStepResponse(&call->response);
  
    call->SetCancelCallback([call_opts]() { 
        call_opts->StartCancel(); 
    });

    master_impl_->RunStep(call_opts, wrapped_request, wrapped_response,
      [call, call_opts, wrapped_request, wrapped_response](
          const Status& status) {
        call->ClearCancelCallback();
        delete call_opts;
        delete wrapped_request;
        call->SendResponse(ToGrpcStatus(status));
      });
    ENQUEUE_REQUEST(RunStep, true);
  }

 private:
  CallOptions* CreateCallOptions(RunStepCall* call) {
    CallOptions* call_opts = new CallOptions;
    if (call->request.options().timeout_in_ms() > 0) {
      call_opts->SetTimeout(call->request.options().timeout_in_ms());
    } else {
      call_opts->SetTimeout(default_timeout_in_ms_);
    }
    return call_opts; 
  }
};
\end{c++}
\end{leftbar}

\subsubsection{Master}

\begin{leftbar}
\begin{c++}
void Master::RunStep(CallOptions* opts, 
    const RunStepRequestWrapper* req,
    MutableRunStepResponseWrapper* resp, 
    DoneClosure done) {
  auto session = FindMasterSession(req->session_handle());
  SchedClosure([this, session, opts, req, resp, done]() {
    Status status = session->Run(opts, *req, resp);
    session->Unref();
    done(status);
  });
}
\end{c++}
\end{leftbar}

\subsubsection{MasterSession:RunStep}

\begin{leftbar}
\begin{c++}
Status MasterSession::Run(
    CallOptions* opts, 
    const RunStepRequestWrapper& req,
    MutableRunStepResponseWrapper* resp) {
  Status status;
  if (!req.partial_run_handle().empty()) {
    status = DoPartialRun(opts, req, resp);
  } else {
    status = DoRunWithLocalExecution(opts, req, resp);
  }
  return status;
}
\end{c++}
\end{leftbar}

\begin{leftbar}
\begin{c++}
Status MasterSession::DoRunWithLocalExecution(
    CallOptions* opts, const RunStepRequestWrapper& req,
    MutableRunStepResponseWrapper* resp) {

  // 1. Prune: build ReffedClientGraph. 
  BuildGraphOptions bgopts;
  BuildBuildGraphOptions(req, &bgopts);
  
  ReffedClientGraph* rcg = nullptr;
  int64 count = 0;
  TF_RETURN_IF_ERROR(StartStep(bgopts, &count, &rcg, false));

  // 2. Register Partitions: build Partitions and register to workers. 
  core::ScopedUnref unref(rcg);
  TF_RETURN_IF_ERROR(BuildAndRegisterPartitions(rcg));

  // 3. Run Partitions: notify all of workers to run partitions.
  uint64 step_id = (random::New64() & ((1uLL << 56) - 1)) | (1uLL << 56);
  Status s = rcg->RunPartitions(env_, step_id, count, &pss, opts, req, resp,
                                &cancellation_manager_, false);
  // 4. Cleaup Partitions: notify all of workers to clearup partitions.
  Ref();
  rcg->Ref();
  rcg->CleanupPartitionsAsync(step_id, [this, rcg](const Status& s) {
    rcg->Unref();
    Unref();
  });
  return s;
}
\end{c++}
\end{leftbar}

\subsubsection{MasterSession:图分裂}

\begin{leftbar}
\begin{c++}
Status MasterSession::BuildAndRegisterPartitions(ReffedClientGraph* rcg) {
  PartitionOptions popts;
  popts.node_to_loc = SplitByWorker;
  popts.flib_def = rcg->client_graph()->flib_def.get();
  popts.control_flow_added = false;

  popts.new_name = [this](const string& prefix) {
    mutex_lock l(mu_);
    return strings::StrCat(prefix, "_S", next_node_id_++);
  };

  popts.get_incarnation = [this](const string& name) -> int64 {
    Device* d = devices_->FindDeviceByName(name);
    if (d == nullptr) {
      return PartitionOptions::kIllegalIncarnation;
    } else {
      return d->attributes().incarnation();
    }
  };

  TF_RETURN_IF_ERROR(rcg->RegisterPartitions(popts));
  return Status::OK();
}
\end{c++}
\end{leftbar}

\subsubsection{ReffedClientGraph:图分裂}

\begin{leftbar}
\begin{c++}
Status ReffedClientGraph::RegisterPartitions(
    const PartitionOptions& popts) {
  { 
    mu_.lock();
    if (!init_started_) {
      init_started_ = true;
      mu_.unlock();

      std::unordered_map<string, GraphDef> graph_defs;
      Status s = DoBuildPartitions(popts, &graph_defs);
      if (s.ok()) {
        s = DoRegisterPartitions(popts, std::move(graph_defs));
      }

      mu_.lock();
      init_result_ = s;
      init_done_.Notify();
    } else {
      mu_.unlock();
      init_done_.WaitForNotification();
      mu_.lock();
    }
    Status result = init_result_;
    mu_.unlock();
    return result;
  }
}
\end{c++}
\end{leftbar}

\begin{leftbar}
\begin{c++}
Status ReffedClientGraph::DoRegisterPartitions(
    const PartitionOptions& popts,
    std::unordered_map<string, GraphDef> graph_partitions) {
  partitions_.reserve(graph_partitions.size());
  Status s;
  for (auto& name_def : graph_partitions) {
    partitions_.resize(partitions_.size() + 1);
    Part* part = &partitions_.back();
    part->name = name_def.first;
    TrackFeedsAndFetches(part, name_def.second, popts);
    part->worker = worker_cache_->CreateWorker(part->name);
  }

  struct Call {
    RegisterGraphRequest req;
    RegisterGraphResponse resp;
    Status status;
  };

  const int num = partitions_.size();
  gtl::InlinedVector<Call, 4> calls(num);

  BlockingCounter done(num);
  for (int i = 0; i < num; ++i) {
    const Part& part = partitions_[i];
    Call* c = &calls[i];
    
    c->req.set_session_handle(session_handle_);
    c->req.mutable_graph_def()->Swap(&graph_partitions[part.name]);
    *c->req.mutable_graph_options() = session_opts_.config.graph_options();
    *c->req.mutable_debug_options() = debug_opts_;

    auto cb = [c, &done](const Status& s) {
      c->status = s;
      done.DecrementCount();
    };
    part.worker->RegisterGraphAsync(&c->req, &c->resp, cb);
  }
  done.Wait();

  for (int i = 0; i < num; ++i) {
    Call* c = &calls[i];
    s.Update(c->status);
    partitions_[i].graph_handle = c->resp.graph_handle();
  }
  return s;
}
\end{c++}
\end{leftbar}

\subsubsection{GrpcRemoteWorker}

\begin{leftbar}
\begin{c++}
class GrpcRemoteWorker : public WorkerInterface {
  void RegisterGraphAsync(const RegisterGraphRequest* request,
                          RegisterGraphResponse* response,
                          StatusCallback done) override {
    IssueRequest(request, response, registergraph_, std::move(done));
  }

  void IssueRequest(const protobuf::Message* request,
                    protobuf::Message* response, const ::grpc::string& method,
                    StatusCallback done, CallOptions* call_opts = nullptr) {
    new RPCState<protobuf::Message>(counter_, &stub_, cq_, method, *request,
                                    response, std::move(done), call_opts);
  }
};
\end{c++}
\end{leftbar}

\subsubsection{GrpcRemoteWorker}

\begin{leftbar}
\begin{c++}
class GrpcWorkerService : public AsyncServiceInterface {
  void RegisterGraphHandler(
      WorkerCall<RegisterGraphRequest, RegisterGraphResponse>* call) {
    Schedule([this, call]() {
      Status s = worker_->RegisterGraph(&call->request, &call->response);
      call->SendResponse(ToGrpcStatus(s));
    });
    ENQUEUE_REQUEST(RegisterGraph, false);
  }
};
\end{c++}
\end{leftbar}

\subsubsection{Worker}

\begin{leftbar}
\begin{c++}
void Worker::RegisterGraphAsync(const RegisterGraphRequest* request,
                                RegisterGraphResponse* response,
                                StatusCallback done) {
  WorkerSession* session =
      env_->session_mgr->WorkerSessionForSession(request->session_handle());
  Status s = session->graph_mgr->Register(
      request->session_handle(), request->graph_def(), request->graph_options(),
      request->debug_options(), response->mutable_graph_handle());
  done(s);
}
\end{c++}
\end{leftbar}

\subsubsection{GraphMgr}

\begin{leftbar}
\begin{c++}
Status GraphMgr::Register(const string& session, const GraphDef& gdef,
                          const GraphOptions& graph_options,
                          const DebugOptions& debug_options, string* handle) {
  Item* item = new Item;
  Status s = InitItem(session, gdef, graph_options, debug_options, item);
  if (!s.ok()) {
    item->Unref();
    return s;
  }

  {
    mutex_lock l(mu_);
    *handle = strings::Printf("%016llx", ++next_id_);
    item->handle = *handle;
    CHECK(table_.insert({*handle, item}).second);
  }
  return Status::OK();
}
\end{c++}
\end{leftbar}

\subsection{关闭会话}

\subsubsection{GrpcSession}

\begin{leftbar}
\begin{c++}
Status GrpcSession::Close() {
  CloseSessionRequest req;
  {
    mutex_lock l(mu_);
    if (handle_.empty()) {
      return errors::InvalidArgument("A session is not created yet....");
    }
    req.set_session_handle(handle_);
    handle_.clear();
  }
  CloseSessionResponse resp;
  CallOptions call_options;
  call_options.SetTimeout(options_.config.operation_timeout_in_ms());
  return master_->CloseSession(&call_options, &req, &resp);
}
\end{c++}
\end{leftbar}

\subsubsection{GrpcRemoteMaster}

\begin{leftbar}
\begin{c++}
struct GrpcRemoteMaster : MasterInterface {
  Status CloseSession(CallOptions* call_options,
                      const CloseSessionRequest* request,
                      CloseSessionResponse* response) override {
    ::grpc::ClientContext ctx;
    ctx.set_fail_fast(false);
    SetDeadline(&ctx, call_options->GetTimeout());
    return FromGrpcStatus(stub_->CloseSession(&ctx, *request, response));
  }
};
\end{c++}
\end{leftbar}

\subsubsection{GrpcMasterService}

\begin{leftbar}
\begin{c++}
struct GrpcMasterService : AsyncServiceInterface {
  void CloseSessionHandler(
      MasterCall<CloseSessionRequest, CloseSessionResponse>* call) {
    master_impl_->CloseSession(&call->request, &call->response,
                               [call](const Status& status) {
                                 call->SendResponse(ToGrpcStatus(status));
                               });
    ENQUEUE_REQUEST(CloseSession, false);
  }
};
\end{c++}
\end{leftbar}

\subsubsection{Master}

\begin{leftbar}
\begin{c++}
void Master::CloseSession(const CloseSessionRequest* req,
                          CloseSessionResponse* resp, MyClosure done) {
  MasterSession* session = nullptr;
  {
    mu_.lock();
    auto iter = sessions_.find(req->session_handle());
    if (iter == sessions_.end()) {
      mu_.unlock();
      done(errors::Aborted(
          "Session ", req->session_handle(),
          " is not found. Possibly, this master has restarted."));
      return;
    }
    session = iter->second;
    sessions_.erase(iter);
    mu_.unlock();
  }

  // Session Close() blocks on thread shutdown. Therefore, we need to
  // delete it in non-critical thread.
  SchedClosure([session, done]() {
    Status s = session->Close();
    session->Unref();
    done(s);
  });
}
\end{c++}
\end{leftbar}

\subsubsection{MasterSession}

\begin{leftbar}
\begin{c++}
Status MasterSession::Close() {
  {
    mutex_lock l(mu_);
    closed_ = true;  // All subsequent calls to Run() or Extend() will fail.
  }
  cancellation_manager_.StartCancel();
  std::vector<ReffedClientGraph*> to_unref;
  {
    mutex_lock l(mu_);
    while (num_running_ != 0) {
      num_running_is_zero_.wait(l);
    }
    ClearRunsTable(&to_unref, &run_graphs_);
    ClearRunsTable(&to_unref, &partial_run_graphs_);
  }
  for (ReffedClientGraph* rcg : to_unref) rcg->Unref();
  return Status::OK();
}
\end{c++}
\end{leftbar}

\subsubsection{ReffedClientGraph}

\begin{leftbar}
\begin{c++}
ReffedClientGraph::~ReffedClientGraph() { 
  DeregisterPartitions(); 
}
\end{c++}
\end{leftbar}

\begin{leftbar}
\begin{c++}
void ReffedClientGraph::DeregisterPartitions() {
  struct Call {
    DeregisterGraphRequest req;
    DeregisterGraphResponse resp;
  };
  for (Part& part : partitions_) {
    if (!part.graph_handle.empty()) {
      Call* c = new Call;
      c->req.set_session_handle(session_handle_);
      c->req.set_graph_handle(part.graph_handle);

      WorkerCacheInterface* worker_cache = worker_cache_;
      const string name = part.name;
      WorkerInterface* w = part.worker;

      auto cb = [worker_cache, c, name, w](const Status& s) {
        if (!s.ok()) {
          // This error is potentially benign, so we don't log at the
          // error level.
          LOG(INFO) << "DeregisterGraph error: " << s;
        }
        delete c;
        worker_cache->ReleaseWorker(name, w);
      };
      w->DeregisterGraphAsync(&c->req, &c->resp, cb);
    }
  }
}
\end{c++}
\end{leftbar}

\subsubsection{GrpcWorkerService}

\begin{leftbar}
\begin{c++}
struct GrpcWorkerService : AsyncServiceInterface {
  void CreateWorkerSessionHandler(
      WorkerCall<CreateWorkerSessionRequest, CreateWorkerSessionResponse>*
          call) {
    Schedule([this, call]() {
      Status s = worker_->CreateWorkerSession(&call->request, &call->response);
      call->SendResponse(ToGrpcStatus(s));
    });
    ENQUEUE_REQUEST(CreateWorkerSession, false);
  }
};
\end{c++}
\end{leftbar}

\subsubsection{Worker}

\begin{leftbar}
\begin{c++}
void Worker::DeregisterGraphAsync(const DeregisterGraphRequest* request,
                                  DeregisterGraphResponse* response,
                                  StatusCallback done) {
  WorkerSession* session =
      env_->session_mgr->WorkerSessionForSession(request->session_handle());
  Status s = session->graph_mgr->Deregister(request->graph_handle());

  done(s);
}
\end{c++}
\end{leftbar}

\subsubsection{GraphMgr}

\begin{leftbar}
\begin{c++}
Status GraphMgr::Deregister(const string& handle) {
  Item* item = nullptr;
  {
    mutex_lock l(mu_);
    auto iter = table_.find(handle);
    if (iter == table_.end()) {
      return errors::Aborted("Graph handle is not found: ", handle,
                             ". Possibly, this worker just restarted.");
    }
    item = iter->second;
    table_.erase(iter);
  }
  item->Unref();
  return Status::OK();
}
\end{c++}
\end{leftbar}

\begin{leftbar}
\begin{c++}
GraphMgr::Item::~Item() {
  for (const auto& unit : this->units) {
    delete unit.root;
    unit.device->op_segment()->RemoveHold(this->session);
  }
}
\end{c++}
\end{leftbar}

\end{content}





\part{模型训练}
\begin{savequote}[45mm]
\ascii{Any fool can write code that a computer can understand. Good programmers write code that humans can understand.}
\qauthor{\ascii{- Martin Flower}}
\end{savequote}

\chapter{BP算法} 
\label{ch:bp}

\begin{content}

\end{content}

\section{TensorFlow实现}

\begin{content}

\tf{}是一个实现了自动微分的软件系统。首先,它构造正向的计算图,实现计算图的前向计算。当调用\code{Optimizer.minimize}方法时,使用\code{compute\_gradients}方法,实现反向计算图的构造;使用\code{apply\_gradients}方法,实现参数更新的子图构造。

\begin{leftbar}
\begin{python}
class Optimizer(object):
  def minimize(self, loss, var_list=None, global_step=None):
    """Add operations to minimize loss by updating var_list.
    """
    grads_and_vars = self.compute_gradients(
      loss, var_list=var_list)
    return self.apply_gradients(
      grads_and_vars, 
      global_step=global_step)
\end{python}
\end{leftbar}

\subsection{计算梯度}

\code{compute\_gradients}将根据\code{loss}的值,求解\code{var\_list=[v1, v2, ..., vn]}的梯度,最终返回的结果为:\code{[(grad\_v1, v1), (grad\_v2, v2), ..., (grad\_vn, vn)]}。其中,\code{compute\_gradients}将调用\code{gradients}方法,构造反向传播的子图。

以一个简单实例,讲解反向子图的构造过程。首先,构造前向的计算图。

\begin{leftbar}
\begin{python}
X = tf.placeholder("float", name="X")
Y = tf.placeholder("float", name="Y")
w = tf.Variable(0.0, name="w")
b = tf.Variable(0.0, name="b")
loss = tf.square(Y - X*w - b)
global_step = tf.Variable(0, trainable=False, collections=[])
\end{python}
\end{leftbar}

使用\code{compute\_gradients}构造反向传播的子图。

\begin{leftbar}
\begin{python}
sgd = tf.train.GradientDescentOptimizer(0.01)
grads_and_vars = sgd.compute_gradients(loss)
\end{python}
\end{leftbar}

\subsubsection{构造算法}

反向子图的构建算法可以形式化地描述为:

\begin{leftbar}
\begin{python}
def gradients(loss, grad=I):
  vrg = build_virtual_reversed_graph(loss)
  for op in vrg.topological_sort():
    grad_fn = ops.get_gradient_function(op)
    grad = grad_fn(op, grad)
\end{python}
\end{leftbar}

首先,根据正向子图的拓扑图,构造一个虚拟的反向子图。之所以称为虚拟的,是因为真实的反向子图要比它复杂得多;更准确的说,虚拟的反向子图中的一个节点,对应于真实的反向子图中的一个局部子图。

同时,正向子图输出的最后一个节点,其输出梯度全为\ascii{1}的一个\code{Tensor},作为反向子图的初始的梯度值,常常记为\code{I}。

\begin{figure}[!htbp]
\centering
\includegraphics[width=0.7\textwidth]{figures/bp-back-graph-construction.png}
\caption{构造反向传播子图}
 \label{fig:bp-back-graph-construction}
\end{figure}

接下来,根据虚拟的反向子图构造真实的反向子图。首先,根据该反向的虚拟子图执行拓扑排序算法,得到该虚拟的反向子图的一个拓扑排序;然后,按照该拓扑排序,对每个正向子图中的\ascii{OP}寻找其「梯度函数」;最后,调用该梯度函数,该梯度函数将构造该\ascii{OP}对应的反向的局部子图。

综上述,正向的一个\ascii{OP}对应反向的一个局部子图,并由该\ascii{OP}的梯度函数负责构造。当整个拓扑排序算法完成后,正向子图中的每个\ascii{OP}在反向子图中都能找到对应的局部子图。

例如,在上例中,正向图中最后一个\ascii{OP}:求取平方的函数为例,讲述梯度函数的工作原理。

\subsubsection{梯度函数原型}

一般地,梯度函数满足如下原型:

\begin{leftbar}
\begin{python}
@ops.RegisterGradient("op_name")
def op_grad_func(op, grad):
\end{python}
\end{leftbar}

其中,梯度函数由\code{ops.RegisterGradient}完成注册,并放在保存梯度函数的仓库中。以后,便可以根据正向\ascii{OP}的名字,索引对应的梯度函数了。

对于一个梯度函数,第一个参数\code{op}表示正向计算的\ascii{OP},根据它可以获取正向计算时\ascii{OP}的输入和输出;第二个参数\code{grad},是反向子图中上游节点传递过来的梯度,它是一个已经计算好的梯度值(初始梯度值全为1)。

\subsubsection{实战:平方函数}

举个简单的例子,仅使用输入计算梯度。\code{y=square(x)},用于求取\code{x}的平方。首先,构造正向计算图:

\begin{figure}[!h]
\centering
\includegraphics[width=0.5\textwidth]{figures/bp-square-forward-graph.png}
\caption{Square函数:正向传播子图}
 \label{fig:bp-square-forward-graph}
\end{figure}

然后,反向构造虚拟的反向子图。根据该虚拟的反向子图的拓扑排序,构造真正的反向计算子图。假如,当前节点为\code{Square},根据其OP名称,从仓库中找到对应的梯度函数\code{SquareGrad}。

\begin{figure}[!htbp]
\centering
\includegraphics[width=0.5\textwidth]{figures/bp-square-backward-graph.png}
\caption{Square函数:反向传播子图}
 \label{fig:bp-square-backward-graph}
\end{figure}

因为,\code{y=Square(x)}的导数为\code{y'=2*x}。因此,其梯度函数\code{SquareGrad}的实现为:

\begin{leftbar}
\begin{python}
@ops.RegisterGradient("Square")
def SquareGrad(op, grad):
  x = op.inputs[0]
  with ops.control_dependencies([grad.op]):
    x = math_ops.conj(x)
    return grad * (2.0 * x)
\end{python}
\end{leftbar}

调用该梯度函数后,将得到正向\code{Square}的\ascii{OP},对应的反向子图\code{SquareGrad}。它需要使用\code{Square}的输入,完成相应的梯度计算。

\begin{figure}[!h]
\centering
\includegraphics[width=0.5\textwidth]{figures/bp-square-backward-graph-2.png}
\caption{Square函数:反向传播子图}
 \label{fig:bp-square-backward-graph-2}
\end{figure}

一般地,正向子图中的一个\ascii{OP},对应反向子图中的一个局部子图。因为,正向\ascii{OP}的梯度函数实现,可能需要多个\ascii{OP}才能完成相应的梯度计算。例如,\code{Square}的\ascii{OP},对应梯度函数构造了包含两个\ascii{2}个乘法\ascii{OP}。

\subsubsection{实战:指数函数}

再举个简单的例子,仅使用输出计算梯度。\code{y=exp(x)},指数函数;其导数为\code{y'=exp(x)},即\code{y'=y}。因此,其梯度函数实现为:

\begin{leftbar}
\begin{python}
@ops.RegisterGradient("Exp")
def _ExpGrad(op, grad):
  """Returns grad * exp(x)."""
  y = op.outputs[0]
  with ops.control_dependencies([grad.op]):
    y = math_ops.conj(y)
    return grad * y
\end{python}
\end{leftbar}

如下图所示,正向子图中该\ascii{OP}的输出,用于对应的反向的局部子图的梯度运算。而且,该局部子图进包含一个节点。

\begin{figure}[!h]
\centering
\includegraphics[width=0.5\textwidth]{figures/bp-exp-backward-graph.png}
\caption{Exp函数:反向传播子图}
 \label{fig:bp-exp-backward-graph}
\end{figure}

\subsection{应用梯度}

再做一个简单的总结,当调用\code{Optimizer.minimize}方法时,使用\code{compute\_gradients}方法,实现反向计算图的构造;使用\code{apply\_gradients}方法,实现参数更新的子图构造。

\subsubsection{构造算法}

首先,\code{compute\_gradients}在运行时将根据\code{loss}的值,求解\code{var\_list=[v1, v2, ..., vn]}的梯度,最终返回的结果为:\code{vars\_and\_grads = [(grad\_v1, v1), (grad\_v2, v2), ..., (grad\_vn, vn)]}。

然后,\code{apply\_gradients}迭代\code{grads\_and\_vars},对于每个\code{(grad\_vi, vi)},构造一个更新\code{vi}的子图。其中,算法可以形式化地描述为:

\begin{leftbar}
\begin{python}
def apply_gradients(grads_and_vars, learning_rate):
  for (grad, var) in grads_and_vars:
    apply_gradient_descent(learning_rate, grad, var)
\end{python}
\end{leftbar}

其中,\code{apply\_gradient\_descent}将构造一个使用梯度下降算法更新参数的计算子图。将\code{(grad, var)}的二元组,及其\code{learning\_rate}的\code{Const OP}作为\code{ApplyGradientDescent}的输入。

\code{ApplyGradientDescent}将应用\code{var <- var - learning*grad}的运算规则,实现\code{var}的就地更新。

\begin{figure}[!h]
\centering
\includegraphics[width=0.6\textwidth]{figures/bp-update-w.png}
\caption{参数更新子图}
 \label{fig:bp-update-w}
\end{figure}

\subsubsection{参数更新汇总}

如果存在多个训练的Variable,最终生成多个更新参数的局部子图。它们通过一个名为\code{update}的\code{NoOp},使用控制依赖边汇总在一起。因为各个\code{Variable}之间相互独立,可以实现最大化的并发。

\begin{figure}[!h]
\centering
\includegraphics[width=0.9\textwidth]{figures/bp-update-all-params.png}
\caption{参数更新汇总}
 \label{fig:bp-update-all-params}
\end{figure}

\subsubsection{探秘train\_op}

经过一轮\ascii{Step}运算,根据梯度,完成参数的更新,最终完成\code{global\_step}加1。而实现\code{global\_step}加\ascii{1}的\ascii{OP}为\code{AssignAdd},并标记为\code{train\_op};它持有\code{global\_step}变量的引用,然后完成就地修改,使其值加\ascii{1}。

\begin{figure}[!h]
\centering
\includegraphics[width=0.9\textwidth]{figures/bp-train-op.png}
\caption{train\_op}
 \label{fig:bp-train-op}
\end{figure}

\subsubsection{工作流}

如\refig{bp-train-pipeline}所示,整个训练过程,一次\ascii{Step}的训练过程由前向计算,反向梯度计算,参数更新,及其\code{global\_step}计数四个基本过程组成。

其中,每一轮\ascii{Step}从开始\code{Session.run}执行开始。通过前向子图的计算,得到每个\ascii{OP}的输出,并作为下游\ascii{OP}的输入。

当前向子图完成计算后,再以初始梯度向量$ I $为输入,反向计算各个训练参数的梯度,最终得到各个训练参数的梯度列表,并以\code{grads\_and\_vars = [(grad\_v1, v1), ..., (grad\_vn, vn)]}的二元组列表表示。

随后,参数更新子图以\code{grads\_and\_vars}为输入,执行梯度下降的更新算法;最后,通过\code{train\_op}完成\code{global\_step}值加\ascii{1},至此一轮\ascii{Step}执行完成。

\begin{figure}[!h]
\centering
\includegraphics[width=0.9\textwidth]{figures/bp-train-pipeline.png}
\caption{模型训练的工作流}
 \label{fig:bp-train-pipeline}
\end{figure}

\end{content}

\begin{savequote}[45mm]
\ascii{Any fool can write code that a computer can understand. Good programmers write code that humans can understand.}
\qauthor{\ascii{- Martin Flower}}
\end{savequote}

\chapter{数据加载} 
\label{ch:input-pipeline}

\begin{content}

一般地,\ascii{TensorFlow}输入样本数据到训练/推理子图中执行运算,存在三种读取样本数据的方法:

\begin{enum}
  \eitem{数据注入:通过字典\code{feed\_dict}将数据传递给\code{Session.run},以替代\code{Placeholder}的输出\code{Tensor}的值;}
  \eitem{数据管道:通过构造输入子图,并发地从文件中读取样本数据;}
  \eitem{数据预加载:对于小数据集,使用\code{Const}或\code{Variable}直接持有数据。}
\end{enum}

基于大型数据集的训练或推理任务,样本数据的输入常常使用数据的管道模式,确保高的吞吐率,提高训练/推理的执行效率。该过程使用队列实现输入子图与训练/推理子图之间的数据交互与异步控制。

本章将重点论述数据加载的\ascii{Pipeline}的工作机制,并深入了解\ascii{TensorFlow}并发执行的协调机制,及其队列在并发执行中扮演的角色。

\end{content}

\section{数据注入}

\begin{content}

数据注入是最为常见的数据加载的方法,它通过字典\code{feed\_dict}将样本数据传递给\code{Session.run},或者\code{Tensor.eval}方法;其中,字典的关键字为\code{Tensor}的名字,值为样本数据。

\ascii{TensorFlow}将按照字典中\code{Tensor}的名字,将样本数据替换该\code{Tensor}的值。

\begin{leftbar}
\begin{python}
x = tf.placeholder(tf.float32, [None, 784])
y_ = tf.placeholder(tf.float32, [None, 10])

with tf.Session():
  batch_xs, batch_ys = mnist.train.next_batch(100)
  sess.run(train_step, feed_dict={x: batch_xs, y_: batch_ys})
\end{python}
\end{leftbar}

一般地,\code{feed\_dict}可以替代任何\code{Tensor}的值。但是,常常使用\code{Placeholder}表示其输出\code{Tensor}的值未确定,待使用\code{feed\_dict}替代。

\end{content}

\section{数据预加载}

\begin{content}

可以使用\code{Const}或\code{Variable}直接持有数据,将数据预加载至内存中,提升执行效率。该方法仅适用于小数据集,当样本数据集比较大时,内存资源消耗非常可观。这里以\ascii{mnist}数据集为例,讲解数据预加载的使用方法。

\begin{leftbar}
\begin{python}
from tensorflow.examples.tutorials.mnist import input_data

data_sets = input_data.read_data_sets('/tmp/mnist/data')
\end{python}
\end{leftbar}

\subsection{使用Const}

由于\code{Const OP}输出\code{Tensor}的值是直接内联在计算图中。如果该\code{Const OP}在图中被使用多次,可能造成重复的冗余数据,白白浪费了不必要的内存资源。

\begin{leftbar}
\begin{python}
with tf.name_scope('input'):
  input_images = tf.constant(data_sets.train.images)
  input_labels = tf.constant(data_sets.train.labels)
\end{python}
\end{leftbar}

\subsection{使用Variable}

可以使用不可变、非训练的\code{Variable}替代\code{Const}。一旦初始化了该类型的\code{Variable},便不能改变其值,从而具备\code{Const}的属性。

用于数据预加载的\code{Variable}与用于训练的\code{Variable}之间存在差异,它将置位\code{trainable=False},系统不会将其归类于\code{GraphKeys.TRAINABLE\_VARIABLES}集合中。在训练过程中,系统不会对其实施更新操作。

另外,在构造该类型的\code{Variable}时,还将设置\code{collections=[]},系统不会将其归类于\code{GraphKeys.GLOBAL\_VARIABLES}集合中。在训练过程中,系统不会对其实施\ascii{Checkpoint}操作。

为了创建不可变、非训练的\code{Variable},此处写了一个简单的工厂方法。

\begin{leftbar}
\begin{python}
def immutable_variable(initial_value):
  initializer = tf.placeholder(
    dtype=initial_value.dtype,
    shape=initial_value.shape)
  return tf.Variable(initializer, trainable=False, collections=[])
\end{python}
\end{leftbar}

\code{immutable\_variable}使用传递进来的\code{initial\_value}构造\code{Placeholder}的类型与形状信息,并以此作为\code{Variable}的初始值。可以使用\code{immutable\_variable}创建不可变的,用于数据预加载的\code{Variable}。

\begin{leftbar}
\begin{python}
with tf.name_scope('input'):
  input_images = immutable_variable(data_sets.train.images)
  input_labels = immutable_variable(data_sets.train.labels)
\end{python}
\end{leftbar}

\subsection{批次预加载}

可以构建\ascii{Pipeline},结合数据预加载机制,实现样本的批式加载。首先,使用\code{tf.train.slice\_input\_producer}在每个\ascii{epoch}开始时将整个样本空间随机化,每次从样本集合中随机采样获取一个训练样本。

\begin{leftbar}
\begin{python}
def one(input_xs, input_ys, num_epochs)
  return tf.train.slice_input_producer(
    [input_xs, input_ys], num_epochs=num_epochs)
\end{python}
\end{leftbar}

然后,使用\code{tf.train.batch}每次得到一个批次的样本数据。

\begin{leftbar}
\begin{python}
def batch(x, y, batch_size)
  return tf.train.batch(
    [x, y], batch_size=batch_size)
\end{python}
\end{leftbar}

对于使用\code{Variable}预加载数据,可以如下方式获取一个批次的样本数据。

\begin{leftbar}
\begin{python}
with tf.name_scope('input'):
  input_images = immutable_variable(data_sets.train.images)
  input_labels = immutable_variable(data_sets.train.labels)

  image, label = one(input_images, input_labels, epoch=1)
  batch_images, batch_labels = batch(image, label, batch_size=100)
\end{python}
\end{leftbar}

事实上,\code{tf.train.slice\_input\_producer}将构造样本队列,通过\code{QueueRunner}并发地通过执行\code{Enqueue}操作,将训练样本逐一加入到样本队列中去。在每次迭代训练启动时,通过调用\code{DequeueMany}一次性获取\code{batch\_size}个的批次样本数据到训练子图中去。

\end{content}

\section{数据管道}

\begin{content}

一个典型的数据加载的\ascii{Pipeline(Input Pipeline)},包括如下几个重要数据处理实体:

\begin{enum}
  \eitem{文件名称队列:将文件名称的列表加入到该队列中;}
  \eitem{读取器:从文件名称队列中读取文件名(出队);并根据数据格式选择相应的文件读取器,解析文件的记录;}
  \eitem{解码器:解码文件记录,并转换为数据样本;}
  \eitem{预处理器:对数据样本进行预处理,包括正则化,白化等;}
  \eitem{样本队列:将处理后的样本数据加入到样本队列中。}
\end{enum}

以\ascii{mnist}数据集为例,假如数据格式为\code{TFRecord}。首先,使用\code{tf.train.string\_input\_producer}构造了一个持有文件名列表的\code{FIFOQueue}队列(通过执行\code{EnqueueMany OP}),并且在每个\ascii{epoch}周期内实现文件名列表的随机化。

\subsection{构建文件名队列}

\begin{leftbar}
\begin{python}
def input_producer(num_epochs):
  return tf.train.string_input_producer(
    ['/tmp/mnist/train.tfrecords'], num_epochs=num_epochs)
\end{python}
\end{leftbar}

构造好了文件名队列之后,使用\code{tf.TFRecordReader}从文件名队列中获取文件名(出队,通过调用执行\code{Dequeue OP}),并从文件中读取样本记录(\ascii{Record})。然后,使用\code{tf.parse\_single\_example}解析出样本数据。

\subsection{读取器}

\begin{leftbar}
\begin{python}
def parse_record(filename_queue):
  reader = tf.TFRecordReader()
  _, serialized_example = reader.read(filename_queue)
  features = tf.parse_single_example(
      serialized_example,
      features={
          'image_raw': tf.FixedLenFeature([], tf.string),
          'label': tf.FixedLenFeature([], tf.int64),
      })
  return features
\end{python}
\end{leftbar}

\subsection{解码器}

接着对样本数据进行解码,及其可选的预处理过程,最终得到训练样本。

\begin{leftbar}
\begin{python}
def decode_image(features):
  image = tf.decode_raw(features['image_raw'], tf.uint8)
  image.set_shape([28*28])

  # Convert from [0, 255] -> [-0.5, 0.5] floats.
  image = tf.cast(image, tf.float32) * (1. / 255) - 0.5
  return image

def decode_label(features):
  label = tf.cast(features['label'], tf.int32)
  return label

def one_example(features):
  return decode_image(features), decode_label(features)
\end{python}
\end{leftbar}

\subsection{构建样本队列}

可以使用\code{tf.train.shuffle\_batch}构建一个\code{RandomShuffleQueue}队列,将解析后的训练样本追加在该队列中(通过执行\code{Enqueue OP});当迭代执行启动时,将批次获取\code{batch\_size}个样本数据(通过执行\code{DequeueMany OP})。

\begin{leftbar}
\begin{python}
def shuffle_batch(image, label, batch_size):
    # Shuffle the examples and collect them into batch\_size
    # batches.(Uses a RandomShuffleQueue)
    images, labels = tf.train.shuffle_batch(
      [image, label], batch_size=batch_size, num_threads=2,
      capacity=1000 + 3 * batch_size,
      # Ensures a minimum amount of shuffling of examples.
      min_after_dequeue=1000)
    return images, labels
\end{python}
\end{leftbar}

\subsection{输入子图}

最后,将整个程序传接起来便构造了一个输入子图。

\begin{leftbar}
\begin{python}
def inputs(num_epochs, batch_size):
  with tf.name_scope('input'):
    filename_queue = input_producer(num_epochs)
    features = parse_record(filename_queue)
    image, label = one_example(features)
    return shuffle_batch(image, label, batch_size)
\end{python}
\end{leftbar}

\end{content}

\section{数据协同}

\begin{content}

事实上,数据加载的\ascii{Pipeline}其本质是构造一个输入子图,实现并发\ascii{IO}操作,使得训练过程不会因操作\ascii{IO}而阻塞,从而实现\ascii{GPU}的利用率的提升。

对于输入子图,数据流的处理划分为若干阶段(\ascii{Stage}),每个阶段完成特定的数据处理功能;各阶段之间以队列为媒介,完成数据的协同和交互。

如下图所示,描述了一个典型的神经网络的训练模式。整个流水线由两个队列为媒介,将其划分为3个阶段。

\begin{figure}[!htbp]
\centering
\includegraphics[width=0.9\textwidth]{figures/tf-input-pipeline.png}
\caption{模型训练工作流}
 \label{fig:tf-input-pipeline}
\end{figure}

\subsection{阶段1}

\code{string\_input\_producer}构造了一个\code{FIFOQueue}的队列,它是一个有状态的\ascii{OP}。根据\code{shuffle}选项,在每个\ascii{epoch}开始时,随机生成文件列表,并将其一同追加至队列之中。

\begin{figure}[!htbp]
\centering
\includegraphics[width=0.7\textwidth]{figures/tf-input-pipeline-stage-1.png}
\caption{阶段1:模型训练工作流}
 \label{fig:tf-input-pipeline-stage-1}
\end{figure}

\subsubsection{随机化}

首先,执行名为\code{filenames}的\code{Const OP},再经过\code{RandomShuffle}将文件名称列表随机化。

\subsubsection{Epoch控制}

为了实现\ascii{epoch}的计数,实现巧妙地设计了一个名为\code{epochs}的本地变量。其中,本地变量仅对本进程的多轮Step之间共享数据,并且不会被训练子图实施更新。

在\code{Session.run}之前,系统会执行本地变量列表的初始化,将名为\code{epochs}的\code{Variable}实施零初始化。

\ascii{epoch}的计数功能由\code{CountUpTo}完成,它的工作原理类似于\ascii{C++}的\code{i++}。它持有\code{Variable}的引用,及其上限参数\code{limit}。每经过一轮\ascii{epoch},使其\code{Variable}自增1,直至达到\code{num\_epochs}数目。

其中,当\ascii{epoch}数到达\code{num\_epochs}时,\code{CountUpTo}将自动抛出\code{OutOfRangeError}异常。详细实现可以参考\code{CountUpToOp}的\ascii{Kernel}实现。

\begin{leftbar}
\begin{c++}
template <class T>
struct CountUpToOp : OpKernel {
  explicit CountUpToOp(OpKernelConstruction* ctxt)
    : OpKernel(ctxt) {
    OP_REQUIRES_OK(ctxt, ctxt->GetAttr("limit", &limit_));
  }

  void Compute(OpKernelContext* ctxt) override {
    T before_increment;
    {
      mutex_lock l(*ctxt->input_ref_mutex(0));
      
      // Fetch the old tensor
      Tensor tensor = ctxt->mutable_input(0, true);
      T* ptr = &tensor.scalar<T>()();      
      before_increment = *ptr;
      
      // throw OutOfRangeError if exceed limit
      if (*ptr >= limit_) {
        ctxt->SetStatus(errors::OutOfRange(
            "Reached limit of ", limit_));
        return;
      }
      // otherwise increase 1
      ++*ptr;
    }
    // Output if no error.
    Tensor* out_tensor;
    OP_REQUIRES_OK(ctxt, ctxt->allocate_output(
        "output", TensorShape({}), &out_tensor));
    out_tensor->scalar<T>()() = before_increment;
  }

private:
  T limit_;
};
\end{c++}
\end{leftbar}

\subsubsection{入队操作}

事实上,将文件名列表追加到队列中,执行的是\code{EnqueueMany},类似于\code{Assign}修改\code{Variable}的值,\code{EnqueueMany}也是一个有状态的\ascii{OP},它持有队列的句柄,直接完成队列的状态更新。

在此处,\code{EnqueueMany}将被\code{Session.run}执行,系统反向遍历,找到依赖的\code{Identity},发现控制依赖于\code{CountUpTo},此时会启动一次\ascii{epoch}计数,直至到达\code{num\_epoch}数目抛出\code{OutOfRangeError}异常。同时,\code{Identity}依赖于\code{RandomShuffle},以便得到随机化了的文件名列表。

\subsubsection{QueueRunner}

另外,在调用\code{tf.train.string\_input\_producer}时,将往计算图中注册一个特殊的\ascii{OP}:\code{QueueRunner},并且将其添加到\code{GraphKeys.QUEUE\_RUNNERS}集合中。并且,一个\code{QueueRunner}持有一个或多个\code{Enqueue, EnqueueMany}类型的\ascii{OP}。

\subsection{阶段2}

\code{Reader}从文件名队列中按照\ascii{FIFO}的顺序获取文件名,并按照文件名读取文件记录,成功后对该记录进行解码和预处理,将其转换为数据样本,最后将其追加至样本队列中。

\subsubsection{读取器}

事实上,实现构造了一个\code{ReaderRead}的\ascii{OP},它持有文件名队列的句柄,从队列中按照\ascii{FIFO}的顺序获取文件名。

因为文件的格式为\code{TFRecord},\code{ReaderRead}将委托调用\code{TFRecordReader}的\ascii{OP},执行文件的读取。最终,经过\code{ReaderRead}的运算,将得到一个序列化了的样本。

\subsubsection{解码器}

得到序列化了的样本后,将使用合适的解码器实施解码,从而得到一个期望的样本数据。可选地,可以对样本实施预处理,例如\code{reshape}等操作。

\subsubsection{入队操作}

得到样本数据后,将启动\code{QueueEnqueue}的运算,将样本追加至样本队列中去。其中,\code{QueueEnqueue}是一个有状态的\ascii{OP},它持有样本队列的句柄,直接完成队列的更新操作。

实施上,样本队列是一个\code{RandomShuffleQueue},使用出队操作实现随机采样。

\subsubsection{并发执行}

为了提高\ascii{IO}的吞吐率,可以启动多路并发的\code{Reader}与\code{Decoder}的工作流,并发地将样本追加至样本队列中去。其中,\code{RandomShuffleQueue}是线程安全的,支持并发的入队或出队操作。

\subsection{阶段3}

当数据样本累计至一个\code{batch\_size}时,训练/推理子图将取走该批次的样本数据,启动一次迭代计算(常称为一次\ascii{Step})。

\subsubsection{出队操作}

事实上,训练子图使用\code{DequeueMany}获取一个批次的样本数据。

\subsubsection{迭代执行}

一般地,一次迭代运行,包括两个基本过程:前向计算与反向梯度传递。\ascii{Worker}任务使用\ascii{PS}任务更新到本地的当前值,执行前向计算得到本次迭代的损失。

然后,根据本次迭代的损失,反向计算各个\ascii{Variable}的梯度,并更新到\ascii{PS}任务中;\ascii{PS}任务更新各个\code{Variable}的值,并将当前值广播到各个\ascii{Worker}任务上去。

\subsubsection{Checkpoint}

\ascii{PS}任务根据容错策略,周期性地实施\ascii{Checkpoint}。将当前所有\code{Variable}的数据,及其图的元数据,包括静态的图结构信息,持久化到外部存储设备上,以便后续恢复计算图,及其所有\code{Variable}的数据。

\subsection{Pipeline节拍}

例如,往\code{FIFOQueue}的队列中添加文件名称列表,此时调用\code{EnqueueMany}起始的子图计算,其中包括执行所依赖的\code{CountUpTo}。当\code{CountUpTo}达到\code{limit}上限时,将自动抛出\code{OutOfRangeError}异常。

扮演主程序的\code{QueueRunner},捕获\code{coord.join}重新抛出的\code{OutOfRangeError}异常,随后立即关闭相应的队列,并且退出该线程的执行。队列被关闭之后,入队操作将变为非法;而出队操作则依然合法,除非队列元素为空。

同样的道理,下游\ascii{OP}从队列(文件名队列)中出队元素,一旦该队列元素为空,则自动抛出\code{OutOfRangeError}异常。该阶段对应的\code{QueueRunner}将感知该异常的发生,然后捕获异常并关闭下游的队列(样本队列),退出线程的执行。

在\ascii{Pipeline}的最后阶段,\code{train\_op}从样本队列中出队批次训练样本时,队列为空,并且队列被关闭了,则抛出\code{OutOfRangeError}异常,最终停止整个训练任务。

\begin{savequote}[45mm]
\ascii{Any fool can write code that a computer can understand. Good programmers write code that humans can understand.}
\qauthor{\ascii{- Martin Flower}}
\end{savequote}

\chapter{Saver} 
\label{ch:saver}

\section{Saver}

\begin{content}

在长期的训练任务过程中,为了实现任务的高可用性,\tf{}会周期性地执行断点检查(\ascii{Checkpoint})。

\code{Saver}是实现断点检查功能的基础设施,它会将所有的训练参数持久化在文件系统中;当需要恢复训练时,可以载从文件系统中恢复计算图,及其训练参数的值。也就是说,\code{Saver}承担如下两个方面的职责:

\begin{enum}
  \eitem{\code{save}: 将训练参数的当前值持久化到断点文件中;}
  \eitem{\code{restore}: 从断点文件中恢复训练参数的值。} 
\end{enum}

\subsection{使用方法}

例如,存在一个简单的计算图,包含两个训练参数。首先,执行初始化后,将其结果持久化到文件系统中。

\begin{leftbar}
\begin{python}
# construct graph
v1 = tf.Variable([0], name='v1')
v2 = tf.Variable([0], name='v2')

# run graph
with tf.Session() as sess:
  sess.run(tf.global_variables_initializer())
  saver = tf.train.Saver()
  saver.save(sess, 'ckp')
\end{python}
\end{leftbar}

随后,可以根据断点文件存储的位置恢复模型。

\begin{leftbar}
\begin{python}
with tf.Session() as sess:
  saver = tf.import_meta_graph('ckp.meta')
  saver.restore(sess, 'ckp')
\end{python}
\end{leftbar}

\subsection{文件功能}

当执行\code{Saver.save}操作之后,在文件系统中生成如下文件:

\begin{leftbar}
\begin{python}
├── checkpoint
├── ckp.data-00000-of-00001
├── ckp.index
├── ckp.meta
\end{python}
\end{leftbar}

\subsubsection{索引文件}

索引(\ascii{index})文件保存了一个不可变表(\code{tensorflow::table::Table})的数据;其中,关键字为\ascii{Tensor}的名称,其值描述该\ascii{Tensor}的元数据信息,包括该\ascii{Tensor}存储在哪个数据(\ascii{data})文件中,及其在该数据文件中的偏移,及其校验和等信息。

\subsubsection{数据文件}

数据(\ascii{data})文件记录了所有变量\ascii{(Variable)}的值。当\code{restore}某个变量时,首先从索引文件中找到相应变量在哪个数据文件,然后根据索引直接获取变量的值,从而实现变量数据的恢复。

\subsubsection{元文件}

元文件(\ascii{meta})中保存了\code{MetaGraphDef}的持久化数据,它包括\code{GraphDef, SaverDef}等元数据。

将描述计算图的元数据与存储变量值的数据文件相分离,实现了静态的图结构与动态的数据表示的分离。因此,在恢复\ascii{(Restore)}时,先调用\code{tf.import\_meta\_graph}先将\code{GraphDef}恢复出来,然后再恢复\code{SaverDef},从而恢复了描述静态图结构的\code{Graph}对象,及其用于恢复变量值的\code{Saver}对象,最后使用\code{Saver.restore}恢复所有变量的值。

这也是在上例中,在调用\code{Saver.restore}之前,得先调用\code{tf.import\_meta\_graph}的真正原因;否则,缺失计算图的实例,就无法谈及恢复数据到图实例中了。

\subsubsection{状态文件}

\ascii{Checkpoint}文件会记录最近一次的断点文件(\ascii{Checkpoint File})的前缀,根据前缀可以找对对应的索引和数据文件。当调用\code{tf.train.latest\_checkpoint},可以快速找到最近一次的断点文件。


此外,\ascii{Checkpoint}文件也记录了所有的断点文件列表,并且文件列表按照由旧至新的时间依次排序。当训练任务时间周期非常长,断点检查将持续进行,必将导致磁盘空间被耗尽。为了避免这个问题,存在两种基本的方法:

\begin{enum}
  \eitem{\code{max\_to\_keep}: 配置最近有效文件的最大数目,当新的断点文件生成时,且文件数目超过\code{max\_to\_keep},则删除最旧的断点文件;其中,\code{max\_to\_keep}默认值为\ascii{5};}
  \eitem{\code{keep\_checkpoint\_every\_n\_hours}: 在训练过程中每\code{n}小时做一次断点检查,保证只有一个断点文件;其中,该选项默认是关闭的。} 
\end{enum}

由于\ascii{Checkpoint}文件也记录了断点文件列表,并且文件列表按照由旧至新的时间依次排序。根据上述策略删除陈旧的断点文件将变得极其简单有效。

\subsection{模型}

\subsubsection{持久化模型}

为了实现持久化的功能,\code{Saver}在构造时在计算图中插入\code{SaveV2},及其关联的\ascii{OP}。其中,\code{file\_name}为一个\ascii{Const}的\ascii{OP},指定断点文件的名称;\code{tensor\_names}也是一个\ascii{Const}的\ascii{OP},用于指定训练参数的\ascii{Tensor}名称列表。

\begin{figure}[!htbp]
\centering
\includegraphics[width=0.5\textwidth]{figures/py-saver-save-model.png}
\caption{Saver:持久化模型}
 \label{fig:py-saver-save-model}
\end{figure}

\subsubsection{恢复模型}

同样地,为了实现恢复功能,\code{Saver}在构造期,为每个训练参数,插入了一个\code{RestoreV2},及其关联的\ascii{OP}。其中,包括从断点文件中恢复参数默认值的初始化器\ascii{(Initializer)},其本质是一个\code{Assign}的\ascii{OP}。 

另外,\code{file\_name}为一个\ascii{Const}的\ascii{OP},指定断点文件的名称;\code{tensor\_names}也是一个\ascii{Const}的\ascii{OP},用于指定训练参数的\ascii{Tensor}名称列表,其长度为\ascii{1}。

\begin{figure}[!htbp]
\centering
\includegraphics[width=0.9\textwidth]{figures/py-saver-restore-model.png}
\caption{Saver:恢复模型}
 \label{fig:py-saver-restore-model}
\end{figure}

\end{content}

\begin{savequote}[45mm]
\ascii{Any fool can write code that a computer can understand. Good programmers write code that humans can understand.}
\qauthor{\ascii{- Martin Flower}}
\end{savequote}

\chapter{MonitoredSession} 
\label{ch:monitored-session}

\begin{content}

训练一个简单的模型,可以通过运行\code{train\_op}数次直至模型收敛,最终将训练参数实施\ascii{Checkpoint},持久化训练模型。对于小规模的学习模型,这个过程至多需要花费数小时的时间。

但是,对于大规模的学习模型,需要花费数天时间;而且可能需要使用多份复本\ascii{(replica)},此时需要更加健壮的训练过程支持模型的训练。因此,需要解决三个基本问题:

\begin{enum}
  \eitem{当训练过程异常关闭,或程序崩溃,能够合理地处理异常;}
  \eitem{当异常关闭,或程序崩溃之后,能够恢复训练过程;} 
  \eitem{能够通过\ascii{TensorBoard}监控整个训练过程。}   
\end{enum}

当训练被异常关闭或程序崩溃之后,为了能够恢复训练过程,必须周期性实施\ascii{Checkpoint}。当训练过程重启后,可以通过寻找最近一次的\ascii{Checkpoint}文件,恢复训练过程。

为了能够使用\ascii{TensorBoard}监控训练过程,可以通过周期性运行一些\ascii{Summary}的\ascii{OP},并将结果追加到事件文件中。\ascii{TensorBoard}能够监控和解析事件文件的数据,可视化整个训练过程,包括展示计算图的结构。

\end{content}

\section{引入MonitoredSession}

\begin{content}

\code{tf.train.MonitoredSession},它可以定制化\code{Hook},用于监听整个\code{Session}的生命周期;内置\code{Coordinator}对象,用于协调所有运行中的线程同时停止,并监听,上报和处理异常;当发生\code{AbortedError}或\code{UnavailableError}异常时,可以重启\code{Session}。

\subsection{使用方法}

一般地,首先使用\code{ChiefSessionCreator}创建\code{Session}实例,并且注册三个最基本的\code{tf.train.SessionRunHook}:

\begin{enum}
  \eitem{\code{CheckpointSaverHook}:周期性地\ascii{Checkpoint};}
  \eitem{\code{SummarySaverHook}:周期性地运行\ascii{Summary};} 
  \eitem{\code{StepCounterHook}:周期性地统计每秒运行的\ascii{Step}数目。}   
\end{enum}

为了能够安全处理异常,并且能够关闭\code{MonitoredSession},常常使用\code{with}的上下文管理器。

\begin{leftbar}
\begin{python}
session_creator = tf.train.ChiefSessionCreator(
  checkpoint_dir=checkpoint_dir,
  master=master,
  config=config)

hooks = [
  tf.train.CheckpointSaverHook(
    checkpoint_dir=checkpoint_dir,
    save_secs=save_checkpoint_secs),
  tf.train.SummarySaverHook(
    save_secs=save_summaries_secs,
    output_dir=checkpoint_dir),
  tf.train.StepCounterHook(
    output_dir=checkpoint_dir, 
    every_n_steps=log_step_count_steps)
]

with tf.train.MonitoredSession(
  session_creator=session_creator,
  hooks=hooks) as sess:
  if not sess.should_stop():
    sess.run(train_op)
\end{python}
\end{leftbar}

\subsection{使用工厂}

使用\code{MonitoredTrainingSession}的工厂方法,可以简化\code{MonitoredSession}的创建过程。

\begin{figure}[!htbp]
\centering
\includegraphics[width=0.4\textwidth]{figures/py-train-monitored-training-session.png}
\caption{MonitoredTrainingSession:工厂方法}
 \label{fig:py-train-monitored-training-session}
\end{figure}

\begin{leftbar}
\begin{python}
with MonitoredTrainingSession(
  master=master,
  is_chief=is_chief,
  checkpoint_dir=checkpoint_dir
  config=config) as sess:
  if not sess.should_stop():
    sess.run(train_op)
\end{python}
\end{leftbar}

\subsection{装饰器}

为了得到复合功能的\code{MonitoredSession},可以将完成子功能的\code{WrappedSession}进行组合拼装。

\begin{enum}
  \eitem{\code{RecoverableSession}:当发生\code{AbortedError}或\code{UnavailableError}异常时,可以恢复和重建\code{Session};}
  \eitem{\code{CoordinatedSession}:内置\code{Coordinator}对象,用于协调所有运行中的线程同时停止,并监听,上报和处理异常;} 
  \eitem{\code{HookedSession}:可以定制化\code{Hook},用于监听整个\code{Session}的生命周期。}   
\end{enum}

\begin{figure}[!htbp]
\centering
\includegraphics[width=0.9\textwidth]{figures/py-train-monitored-session-decorator.png}
\caption{MonitoredSession:装饰器}
 \label{fig:py-train-monitored-session-decorator}
\end{figure}

最终,可以组合三者的特性,构建得到\code{MonitoredSession}(伪代码实现,详情请查阅\code{MonitoredSession}的具体实现)。

\begin{leftbar}
\begin{python}
MonitoredSession(
  RecoverableSession(
    CoordinatedSession(
      HookedSession(
        tf.Session(target, config)))))
\end{python}
\end{leftbar}

\end{content}

\section{生命周期}

\begin{content}

\code{MonitoredSession}具有\code{Session}的生命周期特征(但并非\ascii{IS-A}关系,而是\ascii{Like-A}关系,这是一种典型的按照鸭子编程的风格)。

在生命周期过程中,插入了\code{SessionRunHook}的回调钩子,用于监听\code{MonitoredSession}的生命周期过程。

\subsection{初始化}

在初始化阶段,\code{MonitoredSession}主要完成如下过程:

\begin{enum}
  \eitem{运行所有回调钩子的\code{begin}方法;}
  \eitem{通过调用\code{scaffold.finalize()}冻结计算图;} 
  \eitem{创建会话:使用\code{SessionCreator}多态创建\code{Session}}   
  \eitem{运行所有回调钩子的\code{after\_create\_session}方法}
\end{enum}

其中,使用\code{SessionCreator}多态创建\ascii{Session}的过程,存在两种类型。

\begin{enum}
  \eitem{\code{ChiefSessionCreator}:调用\code{SessionManager.prepare\_session},通过从最近的\ascii{Checkpoing}恢复模型,或运行\code{init\_op},完成模型的初始化;然后,启动所有\code{QueueRunner}实例;}
  \eitem{\code{WorkerSessionCreator}:调用\code{SessionManager.wait\_for\_session},等待\code{Chief}完成模型的初始化。} 
\end{enum}

\begin{figure}[!htbp]
\centering
\includegraphics[width=0.9\textwidth]{figures/py-train-monitored-session-initialization.png}
\caption{MonitoredSession:初始化}
 \label{fig:py-train-monitored-session-initialization}
\end{figure}

\subsection{执行}

在执行阶段,在运行\code{Session.run}前后分别回调钩子的\code{before\_run}和\code{after\_run}方法。如果在运行过程发生了\code{AbortedError}或\code{UnavailableError}异常,则重启会话服务。

\begin{figure}[!htbp]
\centering
\includegraphics[width=0.9\textwidth]{figures/py-train-monitored-session-execution.png}
\caption{MonitoredSession:执行}
 \label{fig:py-train-monitored-session-execution}
\end{figure}

\subsection{关闭}

当训练过程结束后,通过调用\code{close}方法,关闭\code{MonitoredSession},释放系统的计算资源。

此时,将回调钩子的\code{end}方法,并且会通过调用\code{Coordinator.request\_stop}方法,停止所有\code{QueueRunner}实例。最终,听过调用\code{tf.Session.close}方法,释放系统的资源。

另外,如果发生\code{OutOfRangeError}异常,\code{MonitoredSession}认为训练过程正常终止,并忽略该异常。

\begin{figure}[!htbp]
\centering
\includegraphics[width=0.9\textwidth]{figures/py-train-monitored-session-close.png}
\caption{MonitoredSession:关闭}
 \label{fig:py-train-monitored-session-close}
\end{figure}

\end{content}

\section{模型初始化}

\begin{content}

\code{MonitoredSession}在初始化时,使用\code{SessionCreator}完成会话的创建和模型的初始化。

一般地,在分布式环境下,存在两种类型的\ascii{Worker}:

\begin{enum}
  \eitem{\ascii{Chief}: 负责模型的初始化;}
  \eitem{\ascii{Non-Chief}: 等待\ascii{Chief}完成模型的初始化。} 
\end{enum}

两者之间,通过一个简单的协调协议共同完成模型的初始化。

\subsection{协调协议}

对于\ascii{Chief},它会尝试从\ascii{Checkpoint}文件中恢复模型;如果没有成功,则会通过执行\code{init\_op}全新地初始化模型;其初始化算法,可以形式化描述为:

\begin{leftbar}
\begin{python}
def prepare_session(master, init_op, saver, ckp_dir):
  if is_chief():
    sess = tf.Session(master)
    sess.run(init_op) if not saver.restore(sess, ckp_dir)
\end{python}
\end{leftbar}

对于\ascii{Non-Chief},它会周期性地通过运行\ascii{ready\_op},查看\ascii{Chief}是否已经完成模型的初始化。

\begin{leftbar}
\begin{python}
def wait_for_session(master, ready_op, recovery_wait_secs):
  while True:
    sess = tf.Session(master)
    if sess.run(ready_op):
      return sess
    else:
      sess.close()
      time.sleep(recovery_wait_secs)   
\end{python}
\end{leftbar}

\subsection{SessionManager}

事实上,上述算法主要由\code{SessionManager}实现,它主要负责从\ascii{Checkpoint}文件中完成模型的恢复,或直接通过运行\code{init\_op}完成模型的初始化,最终创建可工作的\code{Session}实例。

\begin{enum}
  \eitem{对于\ascii{Chief},通过调用\code{prepare\_session}方法,完成模型的初始化;}
  \eitem{对于\ascii{Non-Chief},通过调用\code{wait\_for\_session}方法,等待\ascii{Chief}完成模型的初始化。} 
\end{enum}

详情可以参考\code{SessionManager}的具体实现。

\subsection{引入工厂}

使用工厂方法,分别使用\code{ChiefSessionCreator}和\code{WorkerSessionCreator}分别完成上述算法。

\begin{figure}[!htbp]
\centering
\includegraphics[width=0.9\textwidth]{figures/py-train-session-creator.png}
\caption{SessionManager}
 \label{fig:py-train-session-creator}
\end{figure}

\subsection{Scaffold}

当要构建一个模型训练,需要\code{init\_op}初始化变量;需要\code{Saver}周期性实施\ascii{Checkpoint};需要\code{ready\_op}查看一个模型是否已经初始化完毕;需要\code{summary\_op}搜集所有\ascii{Summary},用于训练过程的可视化。

一般地,在计算图中通过\code{GraphKey}标识了这些特殊的OP或对象,以便可以从计算图中检索出这些特殊的\ascii{OP}或对象。

在训练模型的特殊领域中,提供了一个基础工具库:\code{Scaffold},用于创建这些\ascii{OP}或对象的默认值,并添加到计算图的集合中,并且\code{Scaffold}提供了查询接口可以方便地获取到这些\ascii{OP}或对象。

可以通过调用\code{Scaffold.finalize}方法,如果对应的\ascii{OP}或对象为\code{None},则默认创建该类型的实例。最终冻结计算图,之后禁止再往图中增加节点。

\begin{leftbar}
\begin{python}
class Scaffold(object):
  def finalize(self):
    """Creates operations if needed and finalizes the graph."""
    
    # create init\_op
    if self._init_op is None:
      def default_init_op():
        return control_flow_ops.group(
            variables.global_variables_initializer(),
            resources.initialize_resources(
              resources.shared_resources()))
      self._init_op = Scaffold.get_or_default(
          'init_op',
          ops.GraphKeys.INIT_OP,
          default_init_op)

    # create ready\_op
    if self._ready_op is None:
      def default_ready_op():
        return array_ops.concat([
            variables.report_uninitialized_variables(),
            resources.report_uninitialized_resources()
        ], 0)
      self._ready_op = Scaffold.get_or_default(
          'ready_op', 
          ops.GraphKeys.READY_OP,
          default_ready_op)
    
    # create ready\_for\_local\_init\_op
    if self._ready_for_local_init_op is None:
      def default_ready_for_local_init_op():
        return variables.report_uninitialized_variables(
            variables.global_variables())
      self._ready_for_local_init_op = Scaffold.get_or_default(
          'ready_for_local_init_op',
          ops.GraphKeys.READY_FOR_LOCAL_INIT_OP,
          default_ready_for_local_init_op)
    
    # create local\_init\_op
    if self._local_init_op is None:
      def _default_local_init_op():
        return control_flow_ops.group(
            variables.local_variables_initializer(),
            lookup_ops.tables_initializer())
      self._local_init_op = Scaffold.get_or_default(
          'local_init_op',
          ops.GraphKeys.LOCAL_INIT_OP,
          _default_local_init_op)
    
    # create summary\_op
    if self._summary_op is None:
      self._summary_op = Scaffold.get_or_default(
          'summary_op',
          ops.GraphKeys.SUMMARY_OP,
          summary.merge_all)
    
    # create Saver
    if self._saver is None:
      self._saver = training_saver._get_saver_or_default()
    self._saver.build()

    ops.get_default_graph().finalize()
    return self
\end{python}
\end{leftbar}

从\code{finalize}的实现可以看出,以下\ascii{OP}完成的功能为:

\begin{enum}
  \eitem{\code{init\_op}: 完成所有全局变量和全局资源的初始化;}
  \eitem{\code{local\_init\_op}: 完成所有本地变量和表格的初始化;} 
  \eitem{\code{ready\_op}: 查看所有的全局变量和全局资源是否已经初始化了;否则报告未初始化的全局变量和全局资源的列表;}   
  \eitem{\code{ready\_for\_local\_init\_op}: 查看所有的本地变量和表格是否已经初始化了;否则报告未初始化的本地变量和表格的列表;}   
  \eitem{\code{summary\_op}: 汇总所有\ascii{Summary}的输出;}       
\end{enum}

其中,本地变量不能持久化到\ascii{Checkpoint}文件中;当然,也就不能从\ascii{Checkpoint}文件中恢复本地变量的值。

\subsection{初始化算法}

通过观测上面的\ascii{OP}的定义,理解\code{prepare\_session}模型初始化的完整语义便不是那么困难了。

\begin{leftbar}
\begin{python}
class SessionManager(object):
  def prepare_session(self,
                      master,
                      saver=None
                      checkpoint_filename=None
                      init_op=None,
                      init_feed_dict=None,
                      init_fn=None):
    """Creates a Session. Makes sure the model is ready."""

    def _restore_checkpoint():
      sess = session.Session(master)
      if not saver or not checkpoint_filename):
        return sess, False
      else:
        saver.restore(sess, checkpoint_filename)
        return sess, True

    def _try_run_init_op(sess):
      if init_op is not None:
        sess.run(init_op, feed_dict=init_feed_dict)
      if init_fn:
        init_fn(sess)
    
    sess, is_succ = self._restore_checkpoint()
    if not is_succ:
      _try_run_init_op(sess)
    self._try_run_local_init_op(sess)
    self._model_ready(sess)
    return sess
\end{python}
\end{leftbar}

其初始化算法非常简单。首先,尝试从\ascii{Checkpoint}文件中恢复(此处为了简化问题,省略了部分实现);如果失败,则调用\code{init\_op}和\code{init\_fn}完成全局变量和资源的初始化;然后,才能实施本地变量和表格的初始化;最后,验证所有全局变量和资源是否已经初始化了。

\begin{figure}[!htbp]
\centering
\includegraphics[width=0.9\textwidth]{figures/py-train-session-initialization-algo.png}
\caption{模型初始化算法}
 \label{fig:py-train-session-initialization-algo}
\end{figure}

\subsection{本地变量初始化}

对于非空的\code{local\_init\_op},必须等所有全局变量已经初始化完毕后才能进行初始化(通过调用\code{\_ready\_for\_local\_init\_op});否则,报告未初始化的全局变量列表到\code{msg}字段中。

也就是说,本地变量初始化在全局变量初始化之后,且本地变量不会持久化到\ascii{Checkpoint}文件中。

\begin{leftbar}
\begin{python}
class SessionManager(object):
  def _ready_for_local_init(self, sess):
    """Checks if the model is ready to run local_init_op.
    """
    return _ready(self._ready_for_local_init_op, sess,
                  "Model not ready for local init")

  def _try_run_local_init_op(self, sess):
    """Tries to run _local_init_op, if not None, 
       and is ready for local init.
    """
    if not self._local_init_op:
      return True, None:
    
    is_ready, msg = self._ready_for_local_init(sess)
    if is_ready:
      sess.run(self._local_init_op)
      return True, None
    else:
      return False, msg
\end{python}
\end{leftbar}

\subsection{验证模型}

最后,通过执行\code{\_ready\_op},查看所有全局变量和全局资源是否都已经初始化了;否则,报告未初始化的变量列表到\code{msg}字段中。

\begin{leftbar}
\begin{python}
class SessionManager(object):
  def _model_ready(self, sess):
    """Checks if the model is ready or not.
    """
    return _ready(self._ready_op, sess, "Model not ready")
\end{python}
\end{leftbar}

其中,\code{\_ready}使用函数,用于运行相应的\code{ready\_op},查看相应的变量或资源是否完成初始化。

\begin{leftbar}
\begin{python}
def _ready(op, sess, msg):
  """Checks if the model is ready or not, as determined by op.
  """
  if op is None:
    return True, None

  ready_value = sess.run(op)
  if (ready_value.size == 0):
    return True, None
  else:
    uninitialized_vars = ", ".join(
        [i.decode("utf-8") for i in ready_value])
    return False, "initialized vars: " + uninitialized_vars
\end{python}
\end{leftbar}

\end{content}

\section{异常安全}

\begin{content}

一般地,常常使用\code{with}的上下文管理器,实现\code{MonitoredSession}的异常安全和资源安全释放。

\subsection{上下文管理器}

当退出\code{with}语句后,将停止运行所有\code{QueueRunner}实例,并实现\code{tf.Session}的安全关闭。

\begin{leftbar}
\begin{python}
class _MonitoredSession(object):
  def __exit__(self, exception_type, exception_value, traceback):
    if exception_type in [errors.OutOfRangeError, StopIteration]:
      exception_type = None
    self._close_internal(exception_type)
    return exception_type is None
  
  def _close_internal(self, exception_type=None):
    try:
      if not exception_type:
        for h in self._hooks:
          h.end(self.tf_sess)
    finally:
      try:
        self._sess.close()
      finally:
        self._sess = None
        self.tf_sess = None
        self.coord = None  
\end{python}
\end{leftbar}

特殊地,当发生\code{OutOfRangeError}或\code{StopIteration},则认为正常终止,忽视该异常。如果抛出了其它类型的异常,则不会调用\code{end}的回调钩子。

\subsection{停止QueueRunner}

另外,当执行\code{self.\_sess.close()},最终将调用\code{\_CoordinatedSession}的\code{close}方法。通过调用\code{coord.request\_stop}通知所有\code{QueueRunner}实例停止运行,并且通过调用\code{coord.join}方法等待所有\code{QueueRunner}实例运行完毕。

\begin{leftbar}
\begin{python}
class _CoordinatedSession(_WrappedSession):
  def close(self):
    self._coord.request_stop()
    try:
      self._coord.join()
    finally:
      try:
        _WrappedSession.close(self)
      except Exception:
        pass
\end{python}
\end{leftbar}

\end{content}

\section{回调钩子}

\begin{content}

可以通过定制\code{SessionRunHook},实现对\code{MonitorSession}生命周期过程的监听和管理。

\begin{leftbar}
\begin{python}
class SessionRunHook(object):
  def begin(self):
    pass

  def after_create_session(self, session, coord):
    pass

  def before_run(self, run_context):
    return None

  def after_run(self, run_context, run_values):
    pass

  def end(self, session):
    pass
\end{python}
\end{leftbar}

其中,最常见的\code{Hook}包括:

\begin{enum}
  \eitem{\code{CheckpointSaverHook}:周期性地\ascii{Checkpoint};}
  \eitem{\code{SummarySaverHook}:周期性地运行\ascii{Summary};} 
  \eitem{\code{StepCounterHook}:周期性地统计每秒运行的\ascii{Step}数目。}   
\end{enum}

\begin{figure}[!htbp]
\centering
\includegraphics[width=0.9\textwidth]{figures/py-train-session-run-hook.png}
\caption{SessionRunHook}
 \label{fig:py-train-session-run-hook}
\end{figure}

\end{content}








%%%%%%%%%%%%%%%%%%%%%
\appendix

\part{附录}
\chapter{代码阅读} 
\label{ch:code-reading}

\begin{content}

在程序员的日常工作之中,绝大多数时间都是在\emph{阅读代码},而不是在写代码。但是,阅读代码往往是一件很枯燥的事情,尤其当遇到了一个不漂亮的设计,反抗的心理往往更加强烈。事实上,变换一下习惯、思路和方法,代码阅读其实是一个很享受的过程。

阅读代码的模式,实践和习惯,集大成者莫过于希腊作者\ascii{Diomidis Spinellis}的经典之作:\ascii{Code Reading, The Open Source Perspective}。本文从另外一个视角出发,谈谈我阅读代码的一些习惯,期待找到更多知音的共鸣。

\end{content}

\section{工欲善其事,必先利其器}

\begin{content}

首先,阅读代码之前先准备好一个称心如意的工具箱,包括\ascii{IDE}, \ascii{UML},脑图等工具。我主要使用的编程语言包括\ascii{C++, Scala, Java, Ruby, Python};我更偏向使用\ascii{JetBrains}公司的产品,其很多习惯用法对程序员都很贴心。

其次,高效地使用快捷键,这是一个良好的代码阅读习惯,它极大地提高了代码阅读的效率和质量。例如,查看类层次关系,函数调用链,方法引用点等等。

\begin{remark}
拔掉鼠标,减低对鼠标的依赖。当发现没有鼠标而导致工作无法进行下去时,尝试寻找对应的快捷键。通过日常的点滴积累,工作效率必然能够得到成倍的提高。
\end{remark}

\end{content}

\section{力行而后知之真}

\begin{content}

阅读代码一种常见的反模式就是通过\ascii{Debug}的方式来阅读代码。作者不推荐这种代码阅读的方式,其一,因为运行时线程间的切换很容易导致方向的迷失;其二,了解代码调用栈对于理解系统行为并非见得有效,因为其包含太多实现细节,不易发现问题的本质。

但在阅读代码之前,有几件事情是必须做的。其一,手动地构建一次工程,并运行测试用例;其二,亲自动手写几个\ascii{Demo}感受一下。

先将工程跑起来,目的不是为了\ascii{Debug}代码,而是在于了解工程构建的方式,及其认识系统的基本结构,并体会系统的使用方式。

如果条件允许,可以尝试使用\ascii{ATDD}的方式,发现和挖掘系统的行为。通过这个过程,将自己当成一个客户,思考系统的行为,这是理解系统最重要的基石。

\end{content}

\section{发现领域模型}

\begin{content}

阅读代码,不是为了了解每个类,每个函数干什么,而是为了挖掘更本质,更不易变化的知识。事实上,发现\emph{领域模型}是阅读代码最重要的一个目标,因为领域模型是系统的灵魂所在。通过代码阅读,找到系统本质的知识,并通过自己的模式表达出来,才能真正地抓住系统的脉络,否则一切都是空谈。

例如,在阅读\tf{}的\ascii{Python}实现的客户端代码时,理顺计算图的领域模型,对于理解\ascii{TensorFlow}的编程模型,及其系统运行时的行为极其重要。

\begin{figure}[!htbp]
\centering
\includegraphics[width=0.9\textwidth]{figures/py-graph.png}
\caption{领域对象:Graph}
 \label{fig:py-graph}
\end{figure}

\end{content}

\section{挖掘系统架构}

\begin{content}

阅读代码犹如在大海中航行,系统架构图就是航海图。阅读代码不能没有整体的系统概念,否则收效不佳,阅读质量大大折扣。必须拥有系统思维,并明确目标,才不至于迷失方向。

首要的任务,就是找到系统的边界,并能够以抽象的思维思考外部系统的行为特征。其次,理清系统中各组件之间的交互,关联关系,及其职责,对于理解整个系统的行为极为重要。

例如,对于\ascii{TensorFlow},\ascii{C API}是衔接前后端系统的桥梁。理解\ascii{C API}的设计,基本能够猜测前后端系统的行为。

\begin{figure}[!h]
\centering
\includegraphics[width=0.9\textwidth]{figures/tf-architecture-simple.png}
\caption{TensorFlow系统架构}
 \label{fig:tf-architecture-simple}
\end{figure}

\end{content}

\section{细节是魔鬼}

\begin{content}

纠结于细节,将导致代码阅读代码的效率和质量大大折扣。例如,日志打印,解决某个\ascii{Bug}的补丁实现,某版本分支的兼容方案,某些变态需求的锤子代码实现等等。

阅读代码的一个常见的反模式就是「给代码做批注」。这是一个高耗低效,投入产出比极低的实践。一般地,越是优雅的系统,注释越少;越是复杂的系统,再多的注释也是于事无补。

我有一个代码阅读的习惯,为代码阅读建立一个单独的\ascii{code-reading}分支,一边阅读代码,一边删除这些无关的代码。

\begin{leftbar}
\begin{scala}
$ git checkout -b code-reading
\end{scala}
\end{leftbar}

删除这些噪声后,你会发现系统根本没有想象之中那么复杂。现实中,系统的复杂性,往往都是不成熟的设计和实现导致的额外复杂度。随着对系统的深入理解,很多细节都会自然地浮出水面,所有神秘的面纱都将被揭开而公示天下。

\end{content}

\section{适可而止}

\begin{content}

阅读代码的一个常见的反模式就是「一根筋走到底,不到黄河绝不死心」。程序员都拥有一颗好奇心,总是对不清楚的事情感兴趣。例如,消息是怎么发送出去的?任务调度工作原理是什么?数据存储怎么做到的?;虽然这种勇气值得赞扬,但在代码阅读时绝对不值得鼓励。

还有另外一个常见的反模式就是「追踪函数调用栈」。这是一个极度枯燥的过程,常常导致思维的僵化;因为你永远活在作者的阴影下,完全没有自我。

我个人阅读代码的时候,函数调用栈深度绝不超过\ascii{3},然后使用抽象的思维方式思考底层的调用。因为我发现,随着年龄的增长,曾今值得骄傲的记忆力,现在逐渐地变成自己的短板。当我尝试追踪过深的调用栈之后,之前的阅读信息完全地消失记忆了。

也就是说,我更习惯于「广度遍历」,而不习惯于「深度遍历」的阅读方式。这样,我才能找到系统隐晦存在的「分层概念」,并理顺系统的层次结构。

\end{content}

\section{发现她的美}

\begin{content}

三人行,必有我师焉。在代码阅读代码时,当发现好的设计,包括实现模式,习惯用法等,千万不要错过;否则过上一段时间,这次代码阅读对你来说就没有什么价值了。

当我发现一个好的设计时,我会尝试使用类图,状态机,序列图等方式来表达设计;如果发现潜在的不足,将自己的想法补充进去,将更加完美。

例如,当我阅读\ascii{Hamcrest}时,尝试画画类图,并体会它们之间关系,感受一下设计的美感,也是受益颇多的。

\begin{figure}[!h]
\centering
\includegraphics[width=0.9\textwidth]{figures/hamcrest.png}
\caption{组合式设计}
 \label{fig:hamcrest}
\end{figure}

\end{content}

\section{尝试重构}

\begin{content}

因为这是一次代码阅读的过程,不会因为重构带来潜在风险的问题。在一些复杂的逻辑,通过重构的等价变换可以将其变得更加明晰,直观。

对于一个巨函数,我常常会提取出一个抽象的代码层次,以便发现它潜在的本质逻辑。例如,这是一个使用\ascii{Scala}实现的\ascii{ArrayBuffer},当需要在尾部添加一个元素时,既有的设计是这样子的。

\begin{leftbar}
\begin{python}
def +=(elem: A): this.type = {
  if (size + 1 > array.length) {
    var newSize: Long = array.length
    while (n > newSize)
      newSize *= 2
    newSize = math.min(newSize, Int.MaxValue).toInt
  
    val newArray = new Array[AnyRef](newSize)
    System.arraycopy(array, 0, newArray, 0, size)
    array = newArray
  }
  array(size) = elem.asInstanceOf[AnyRef]
  size += 1
  this
}
\end{python}
\end{leftbar}

这段代码给阅读造成了极大的障碍,我会尝试通过快速的函数提取,发现逻辑的主干。

\begin{leftbar}
\begin{python}
def +=(elem: A): this.type = {
  if (atCapacity)
    grow()
  addElement(elem)
}
\end{python}
\end{leftbar}

至于\code{atCapacity, grow, addElement}是怎么实现的,压根不用关心,因为我已经达到阅读代码的效果了。

\end{content}

\section{形式化}

\begin{content}

当阅读代码时,有部分人习惯画程序的「流程图」。相反,我几乎从来不会画「流程图」,因为流程图反映了太多的实现细节,而不能深刻地反映算法的本质。

我更倾向于使用「形式化」的方式来描述问题。它拥有数学的美感,简洁的表达方式,及其高度抽象的思维,对挖掘问题本质极其关键。

例如,对于\ascii{FizzBuzzWhizz}的问题,相对于冗长的文字描述,或流程图,形式化的方式将更加简单,并富有表达力。以\ascii{3, 5, 7}为输入,形式化后描述后,可清晰地挖掘出问题的本质所在。

\begin{leftbar}
\begin{python}
r1: times(3) => Fizz || 
    times(5) => Buzz ||
    times(7) => Whizz

r2: times(3) && times(5) && times(7) => FizzBuzzWhizz ||
    times(3) && times(5) => FizzBuzz  ||
    times(3) && times(7) => FizzWhizz ||
    times(5) && times(7) => BuzzWhizz

r3: contains(3) => Fizz

rd: others => string of others

spec: r3 || r2 || r1 || rd
\end{python}
\end{leftbar}

\end{content}

\section{实例化}

\begin{content}

实例化是认识问题的一种重要方法,当逻辑非常复杂时,一个简单例子往往使自己豁然开朗。在理想的情况下,实例化可以做成自动化的测试用例,并以此描述系统的行为。

如果存在某个算法和实现都相当复杂时,也可以通过实例化探究算法的工作原理,这对于理解问题本身大有益处。

以\ascii{Spark}中划分\ascii{DAG}算法为例。以\ascii{G}为起始节点,从后往前按照\ascii{RDD}的依赖关系,依次识别出各个\ascii{Stage}的边界。

\begin{figure}[!htbp]
\centering
\includegraphics[width=0.9\textwidth]{figures/spark-stage-dag.png}
\caption{Spark:Stage划分算法}
 \label{fig:spark-stage-dag}
\end{figure}

\begin{itemize}
 \item \ascii{Stage 3}的划分
   \begin{enum}
     \eitem{\ascii{G}与\ascii{B}之间是窄依赖,规约为同一\ascii{Stage(3)};}
     \eitem{\ascii{B}与\ascii{A}之间是宽依赖,\ascii{A}为新的起始\ascii{RDD},递归调用此过程;}
     \eitem{\ascii{G}与\ascii{F}之间是宽依赖,\ascii{F}为新的起始\ascii{RDD},递归调用此过程。} 
   \end{enum}

 \item \ascii{Stage 1}的划分
   \begin{enum}
	   \eitem{\ascii{A}没有父亲\ascii{RDD},\ascii{Stage(1)}划分结束。特殊地\ascii{Stage(1)}仅包含\ascii{RDD A}。}
   \end{enum}
 \item \ascii{Stage 2}的划分
   \begin{enum}
     \eitem{因\ascii{RDD}之间的关系都为窄依赖,规约为同一个\ascii{Stage(2)};}
     \eitem{直至\ascii{RDD C, E},因没有父亲\ascii{RDD},\ascii{Stage(2)}划分结束。}
   \end{enum} 
\end{itemize}

最终,形成了\ascii{Stage}的依赖关系,依次提交\ascii{TaskSet}至\ascii{TaskScheduler}进行调度执行。

\end{content}

\section{独乐乐,不如众乐乐}

\begin{content}

与他人分享你的经验,也许可以找到更多的启发;尤其对于熟知该领域的人沟通,如果是\ascii{Owner}就更好了,肯定能得到意外的惊喜和收获。

也可以通过各种渠道,收集他人的经验,并结合自己的思考,推敲出自己的理解,如此才能将知识放入自己的囊中。

阅读代码,不是一个人的世界;应该走出去,多参加一些社区活动,了解生态圈中主流的研究方向,技术动态,产业发展,对于理解业务是极其有帮助的。


\end{content}


\chapter{持续学习} 
\label{ch:learning}


\section{说文解字}

\begin{content}

\begin{remark}
读书有三到,谓心到,眼到,口到。- 朱熹《训学斋规》
\end{remark}

我出生时,父亲为我取名\emph{刘光云},承\quo{光}辈,单字\quo{云}。但自上学之后,便不知所云了。有一天语文老师说文解字道:\quo{聪者,耳到,眼到,口到,心到也}。判若相识恨晚的感觉,我对\quo{聪}字情有独钟,随将自己的名字改为\emph{刘光聪}。

说来也巧,自那之后,妈妈再也没有担心过我的学习了。

\subsection{选择}

耳到,择其善者而从之,择不善者而改之。有人习惯于巨函数,大逻辑,问究为何如此,美其名曰:都是为了效\ascii{(hai)}率\ascii{(zi)};而我更偏爱具有层次感的代码风格,短小精干,意图明确。

他人的经验固然重要,但需要我们自己选择性地接收,而不是一味的听取。不要为大师所迷,大师有时也会犯错。关键在于自我思考,善于辨别。尤其在这个浮躁的世间里,能在地上跑的都喊自己是\quo{大师}。

\subsection{抽象}

眼到,扫除外物,直觅本来也。一眼便能看到的都是假象,看不到,摸不着的往往才是本质。有人习惯平铺直叙的的逻辑,任其重复;而我更加偏爱抽象,并将揭示本质过程当成一种享受。

抽象,固然存在复杂度。但这样的复杂度是存在上下文的,如果大家具有类似的经验,抽象自然就变成了模式。那是一种美,一种沟通的媒介。

如果对方缺乏上下文,抽象自然是困难的。所谓简单,是问题本质的揭示,并为此付出最小的代价;而不是平铺直叙,简单是那些门外汉永远也感受不到的美感。

过而不及,盲目抽象,必然增加不必要的复杂度。犹如大规模的预先设计,畅谈客户的各种需求,畅谈软件设计中各种变化,盲目抽象。

\subsection{分享}

口到,传道,授业,解惑也。分享是一种生活的信念,明白了分享的同时,自然明白了存在的意义。我喜欢分享自己的知识,并将其当成一种学习动力,督促自己透彻理解问题的本质。

因为能够分享,所以知识自然变成了自己的东西。每日的\ascii{Code Review},我常常鼓励团队成员积极分享,一则为了促就无差异的团队,二则协助分享者透彻问题的本质。

要让别人信服你的观点,关键是要给别人带来信服的理由。分享的同时,能够帮助锻炼自己的表达能力,这需要长时间的\emph{刻意练习}。

\subsection{领悟}

心到,学而思之,思则得之,不思则不得也。只有通过自己独立思考,归纳总结的知识,才是真正属于自己的。

我偏爱使用图表来总结知识,一方面图的表达力远远大于文字;另外,通过画图也逼迫自己能够透彻问题的本质。

\end{content}

\section{成长之路}

\begin{content}

\subsection{消除重复}

代码需要消除重复,工作的习惯也要消除重复。不要拘于固有的工作状态,重复的工作状态往往使人陷入舒服的假象,陷入\emph{三年效应}的危机。

\subsection{提炼知识}

首先我们学习的不是信息,而是知识。知识是有价值的,而信息则没有价值。只有通过自己的筛选,提炼,总结才可能将信息转变为知识。

\subsection{成为习惯}

知识是容易忘记的,只有将知识付诸于行动,并将其融汇到自己的工作状态中去,才能永久性地成为自己的财产。

例如,快捷键的使用,不要刻意地去记忆,而是变成自己的一种工作习惯;不要去重复地劳动,使用\ascii{Shell}提供自动化程度,让\ascii{Shell}成为工作效率提升的利器,并将成为一种工作习惯。

\subsection{更新知识}

我们需要常常更新既有的知识体系,尤其我们处在一个知识大爆炸的时代。我痛恨那些信守教条的信徒,举个简单的例子,陈旧的代码规范常常要求\code{if (NULL != p)}这样的\code{YODA Notation}习惯用法。但是这样的表达编译器是高兴了,但对程序员是非常不友好的。

\ascii{\quo{if you are at least 18 years old}}明显比\ascii{\quo{if 18 years is less than or equal to your age}}更加符合英语表达习惯。

有人驳论此这个习惯用法,但是现代编译器对此类误用通常报告警告;而且保持\ascii{TDD}开发节奏,小步前进,此类低级错误很难逃出测试的法网。

\subsection{重构自我}

学,然后知不足;教,然后知困。不要停留在原点,应该时刻重构自己的知识体系。

在刚入门\ascii{OO}设计的时候,我无处不用设计模式;因为我看到的所有书籍,都是在讲设计模式如何如何地好。直至后来看到了演进式设计,简单设计和过度设计的一些观点后,让我重新回归到理性。

\subsection{专攻术业}

人的精力是有限的,一个人不可能掌握住世界上所有的知识。与其在程序设计语言的抉择上犹豫不决,不如透彻理解方法论的内在本质;与其在众多框架中悬而未决,不如付出实际,着眼于问题本身。

总之,博而不精,不可不防。

\end{content}

%%%%%%%%%%%%%%%%%%%%
\backmatter
\listoffigures
\myclearpage

\listoftables
\myclearpage

\bibliographystyle{alpha}
\renewcommand\bibname{参考文献}
\begin{thebibliography}{20}

\ascii{

\bibitem{tf-white-paper} M.\ Abadi, A.\ Agarwal.
  \newblock \emph{Tensorflow, Large-scale machine learning on heterogeneous distributed systems}.
  \newblock arXiv preprint, 1603.04467, 2016. \\
  \url{https://arxiv.org/abs/1603.04467}.

\bibitem{theano-framework} R.\ Al-Rfou, G.\ Alain, 
  \newblock \emph{Theano: A Python framework for fast computation of mathematical expressions}. 
  \newblock arXiv preprint, 1605.02688, 2016. \\
  \url{https://arxiv.org/abs/1605.02688}.

\bibitem{mxnet-framework} T.\ Chen, M.\ Li. 
  \newblock \emph{MXNet: A flexible and efficient machine learning library for heterogeneous distributed systems}. 
  \newblock In Proceedings of LearningSys, 2015. \\
  \url{www.cs.cmu.edu/̃muli/file/mxnet-learning-sys.pdf}.  

\bibitem{distributed-dnn} J. Dean, G. S. Corrado. 
  \newblock \emph{Large scale distributed deep networks}.
  \newblock In Proceedings of NIPS, pages 1232–1240, 2012. \\
  \url{http://research.google.com/archive/large_deep_networks_nips2012.pdf}.

\bibitem{eigen} Ga\"{e}l Guennebaud.
  \newblock \emph{Eigen: a c++ linear algebra library}. \\
  \url{http://downloads.tuxfamily.org/eigen/eigen_CGLibs_Giugno_Pisa_2013.pdf}.

}

\end{thebibliography}

\endinput

\end{document}
%%%% 正文部分结束
%%%%%%%%------------------------------------------------------------------------
